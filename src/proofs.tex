\subsection{Homotopy invariance of unstable $\K_2$}

\subsection{Early stability}
We start by recalling the so-called Zariski excision property of Steinberg groups.
\begin{lemma} \label{lem:zariski-glueing}
Let $\Phi$ be any simply-laced root system of rank $\geq 3$.
Let $A$ be a commutative domain and $a, b \in A$ be a pair of coprime elements.
\begin{enumerate}
    \item Let $\delta$ be an element of $\St(\Phi, A_{ab})$.
    Then $\delta$ can be presented as $\lambda_b(\alpha) \cdot \lambda_a(\beta)$ for some
    $\alpha \in \St(\Phi, A_a)$ and $\beta \in \St(\Phi, A_b)$.
    \item  Let $\alpha \in \St(\Phi, A_a)$ and $\beta \in \St(\Phi, A_b)$ be such that the equality $\lambda_b(\alpha) = \lambda_a(\beta)$ holds in $\St(\Phi, A_{ab})$.
    Then there exists $\gamma \in \St(\Phi, A)$ such that $\alpha = \lambda_a(\gamma)$, $\beta = \lambda_b(\gamma)$.
\end{enumerate}

\end{lemma}
\begin{proof}
    This is a special case of Nisnevich excision for domains, see~\cite[Proposition~4.5]{LSV2}
    (cf. also the proof of~\cite[Lemma~2.6]{LSV2}).
\end{proof}

For the rest of this subsection $\Psi$ is an arbitrary simply-laced root system of rank $\geq 3$
 embedded into another simply-laced root system $\Phi$.
We denote by $j$ the corresponding homomorphism of Steinberg groups $\St(\Psi, R) \to \St(\Phi, R)$.

\begin{lemma} \label{lem:horrocks-b}
 Let $A$ be a commutative domain and $f \in A[X]$ be a unitary polynomial.
 Let $\alpha$ be an element of $\K_2(\Phi, A[X], XA[X])$ such that $\lambda_f(\alpha)$ belongs to the image of the stabilization map
 $\St(\Psi, A[X]_f) \to \St(\Phi, A[X]_f)$.
 Then $\alpha$ lies in the image of $\St(\Psi, A[X]) \to \St(\Phi, A[X])$.
\end{lemma}
\begin{proof}
    Suppose that $f(X) = X^n + a_1 X^{n-1} \ldots + a_n$.
    Set $g(X\inv) = 1 + a_1 X\inv + \ldots + a_{n} X^{-n}$.
    It is clear that $A[X, X\inv]_f = A[X\inv]_{X\inv g}$ and, moreover, that $X\inv$ and $g$ are not zero divisors and together generate the unit ideal of $A$.

    Consider the following diagram:
    %! suppress = EscapeAmpersand
    \[ \xymatrix{\St(\Phi, A[X]) \ar[r]^{\lambda_X} \ar[d]_{\lambda_f} & \St(\Phi, A[X, X\inv]) \ar[d]_{\lambda_f}  & \St(\Phi, A[X\inv]) \ar[l]_{\lambda_{X\inv}} \ar[d]_{\lambda_g}  \\
                 \St(\Phi, A[X]_f) \ar[r] & \St(\Phi, A[X, X\inv]_f) & \St(\Phi, A[X\inv]_g) \ar[l] \\
                 \St(\Psi, A[X]_f) \ar[r] \ar[u]_j & \St(\Psi, A[X, X\inv]_f) \ar[u]_j & \St(\Psi, A[X\inv]_g) \ar[l] \ar[u]_j \\
                                   & \St(\Psi, A[X, X\inv]) \ar[u]_{\lambda_f^\Psi}   & \St(\Psi, A[X\inv]). \ar[l] \ar[u]_{\lambda_g^\Psi}}\]

    By assumption there exists $\widetilde{\alpha} \in \St(\Psi, A[X]_f)$ such that $j(\widetilde{\alpha}) = \lambda_f(\alpha)$.
    By the first part of~\cref{lem:zariski-glueing} one can write $\lambda_X(\widetilde{\alpha}) = \lambda_f^\Psi(\gamma) \cdot \lambda_{X\inv}(\beta)$
     for some $\beta \in \St(\Psi, A[X\inv]_g)$, $\gamma \in \St(\Psi, A[X, X\inv]).$
    Consequently, one has $\lambda_f(j(\gamma)\inv \cdot \lambda_X(\alpha)) = \lambda_{X\inv}(j(\beta))$.
    By the second part of~\cref{lem:zariski-glueing} there exists $\beta' \in \St(\Phi, A[X\inv])$ such that $\lambda_X(\alpha) = j(\gamma) \cdot \lambda_{X\inv}(\beta').$
    The assertion now follows from~\cref{lem:horrocks--ingredient} and the local--global principle for $\K_2$, see~\cite{LS17}. %TODO
\end{proof}

We now come to the main theorem of this section.
\begin{thm}\label{thm:early-stability}
Let $\Phi$ be a root system of type $\rA_{\geq 4}$, $\rD_{\geq 5}$ or $\rE_{6,7,8}$ and let $A$ be an arbitrary noetherian commutative ring of Krull dimension $\leq 1$.
Then for any $n \geq 0$ the obvious inclusion $\rA_4 \subseteq \Phi$ induces a surjection
\[\K_2(\rA_4, A[X_1,\ldots, X_n]) \to \K_2(\Phi, A[X_1, \ldots X_n]).\]
\end{thm}
\begin{proof}[Proof of Theorem 1]
    The proof is modeled after the proof of~\cite[Theorem~5.3]{Tu83}.
    We proceed by induction on $n$.
    Our assumption on the dimension of $A$ guarantees that it satisfies the condition $\mathrm{SR}_3$ in the sense of~\cite{St78}.
    Thus, from Corollary~3.2 and Theorem~4.1 of~\cite{St78} we conclude that the composite arrow in the following diagram is a surjection:
    \[\K_2(\rA_2, A) \to \K_2(\rA_4, A) \to \K_2(\Phi, A).\]
    Consequently, we obtain that the right arrow is a surjection, which yields the induction base.

    Now let us verify the induction step.
    Set $C = A[X_2, \ldots , X_n]$ and $B = C[X_1]$.
    We need to show that $\K_2(\rA_4, B) \to \K_2(\Phi, B)$ is surjective.
    Every element $\alpha \in \K_2(\Phi, B)$ can be decomposed as a product $\alpha = \alpha_0 \cdot \alpha_1$,
      where $\alpha_0 \in \K_2(\Phi, C)$ and $\alpha_1 \in \K_2(\Phi, B, X_1 B)$.
    By inductive assumption $\K_2(\rA_4, C)$ surjects onto $\K_2(\Phi, C)$, so it remains to show that $\alpha_1$ lies in the image of $\K_2(\rA_4, B)$.

    Denote by $S$ the multiplicative system $S \subseteq B$ consisting of polynomials $f$ such that for sufficiently large $m$
    the polynomial $f$ becomes unitary in $Y_1$ after substitutions $X_1 \coloneqq Y_1,$ $X_2 \coloneqq Y_2 + Y_1^m, \ldots X_n \coloneqq Y_n + Y_1^{m^n}$.
    Recall from~\cite[\S~6]{Su77} that $\dim(S^{-1}B) \leq \dim(A) = 1$.
    By induction base the map $\K_2(\rA_4, S^{-1}B) \to \K_2(\Phi, S^{-1}B)$ is surjective.
    Since functor $\K_2$ commutes with filtered colimits (cf. \cite[Lemma~3.3]{LSV2}) there exists $f \in S$ such that $\lambda^*_f(\alpha_1)$ lies in the image of $\K_2(\rA_4, B_f) \to \K_2(\Phi, B_f)$.
    By the construction of $S$ we may assume that $f$ is unital in $X_1$.
    The required assertion now follows from~\cref{lem:horrocks-b}.
\end{proof}


\subsection{Horrocks ingredient}\label{subsec:horrocks-ingredient}


Now let $A$ be a local ring with maximal ideal $M$.
Set $B \coloneqq A[X\inv] + M[X],\ R \coloneqq A[X, X\inv]$.
Consider the following diagram:
%! suppress = EscapeAmpersand
\[ \xymatrix{ & \St(\Phi, A[X], M[X]) \ar[d]_{t_M} \ar[r] & \St(\Phi, A[X]) \ar[d]_{\lambda_X^*} \\
\St(\Phi, A[X\inv]) \ar[r]^{i_B} \ar@/_/[rr]_{\lambda_{X\inv}^*} & \St(\Phi, B) \ar[r]^{\lambda^*_{X\inv, B}} & \St(\Phi, R)
}\]
The arrow $t_M$ arises from the \("\)lifting property\("\) of Steinberg groups (cf. \cite[Lemma~3.3]{LS20} or~\cite[Theorem~3]{LS17}) as the composition of morphisms
$\lambda_{X}^{rel}\colon \St(\Phi, A[X], M[X]) \to \St(\Phi, R, M[X, X\inv])$ and $T \colon \St(\Phi, R, M[X, X\inv]) \to \St(\Phi, B)$.
%This is wrong: Its image $\Img(t_M) \leq \St(\Phi, B)$ is generated as a subgroup by elements $x_\alpha(f)$, $z_\alpha(f, a)$, $z_\alpha(X^2f, aX^{-1})$, for $\alpha\in \Phi$, $a \in A$, $f \in M[X]$, see the proof of~\cite[Lemma~5.41]{LS20}.

Set $G_M^{\geq 0} \coloneqq \Img(\St(\Phi, A[X], M[X]) \to \St(\Phi, R)), G_M^0 \coloneqq \overline{\St}(\Phi, A, M)$.% Show that G_M^0 is a subgroup of both first two factors
Denote by $\overline{V}$ the quotient-set of the product $V \coloneqq G_M^{\geq 0} \times \St(\Phi, A[X\inv]) \times (1 + M)^\times$
modulo the equivalence relation given by $(gh_0, h, u) \cong (g, h_0h, u)$ where $h_0 \in G_M^0, (g, h, u) \in V.$
Denote by $[g, h, u] \in \overline{V}$ the equivalence class corresponding to $(g, h, u)\in V$.

\begin{prop} \label{prop:horrocks-main} The group $\St(\Phi, B)$ acts on $\overline{V}$.
This action satisfies the following properties:
\begin{enumerate}
    \item For $(g, h, u) \in V$ and $g_0 \in \Img(t_M)$ one has
    \[g_0 \cdot [g, h, u] = [\lambda_{X\inv, B}^* (g_0) \cdot g, h, u]\]
    \item For $(1, h, u) \in V$ and $h_0 \in \St(\Phi, A[X\inv])$ one has
    \[ i_B(h_0) \cdot [1, h, u] = [1, h_0 \cdot h, u].\]
    \item If $g \in \Img(j\colon \St(\Psi, B) \to \St(\Phi, B)), h \in \St(\Phi, A[X\inv]), u \in 1 + M$ then
    \[ g \cdot [1, h, u] = [g', h', u']\] for some $g' \in \Img(j\colon G_{M, \Psi}^{\geq 0} \to G_{M, \Phi}^{\geq 0})$, $h' \in \St(\Phi, A[X\inv])$, $u'\in 1 + M$.
\end{enumerate}
\end{prop}
\begin{proof}
    The existence of the action of $\St(\Phi, B)$ and the second assertion are contained in~\cite[Proposition~5.39]{LS20}.
    The first assertion is what the proof of~\cite[Lemma~5.41]{LS20} actually shows without the assumption that $i_R$ is injective.
    The last assertion follows from the construction of the action. %TODO: Add more details!
\end{proof}

\begin{lemma} \label{lem:horrocks--ingredient}
Let $A$ be a local ring, $\alpha \in \K_2(\Phi, A[X], XA[X])$, $\beta \in \St(\Phi, A[X\inv]),$ $\gamma \in \St(\Psi, A[X, X\inv])$ be elements satisfying the equality (in $\St(\Phi, R)$)
\begin{equation} \label{eq:alpha-def}\lambda_X(\alpha) = j(\gamma) \cdot \lambda_{X\inv}(\beta) \end{equation}
Then $\alpha$ belongs to the image of $j\colon \St(\Psi, A[X]) \to \St(\Phi, A[X])$.
\end{lemma}
\begin{proof}
    Denote by $M$ the maximal ideal of $A$ and by $k$ its residue field.
    We denote by $\pi_A$ (resp. $\pi_{A[X]}$, resp. $\pi_{R}$) the canonical homomorphism $A \to k$ (resp. $A[X] \to k[X]$, resp. $R \to k[X, X\inv]$).

    Since $\K_2(\Psi, k[X])$ surjects onto $\K_2(\Phi, k[X])$ there exists $\alpha_0 \in \St(\Psi, A[X])$ such that $j(\alpha_0) \cdot \alpha \in \overline{\St}(\Phi, A[X], M[X])$ and
    hence $\pi_{A[X]}^*(j(\alpha_0) \cdot \alpha) = 1$.
    By definition of relative Steinberg groups, there exists $\widetilde{\alpha} \in \St(\Phi, A[X], M[X])$ such that
    $\iota \cdot \lambda_X^{rel}(\widetilde{\alpha}) = \lambda_X^* (j(\alpha_0) \cdot \alpha)$. %TODO: More details

    Now consider the element $\pi^*_{A[X\inv]}(\beta) \in \St(\Phi, k[X\inv])$.
    Its image in $\G(\Phi, k[X\inv])$ is contained in the subgroup $\G(\Psi, k[X\inv])$.
    Since $\K_1(\Psi, k[X\inv]) = 1$, there exist
    \[\beta_0 \in \St(\Psi, A[X\inv]),\ \beta_1 \in \St(\Phi, A[X\inv], M[X\inv])\] such that $\beta = j(\beta_0) \cdot \beta_1$.
    Set $\gamma_1 \coloneqq \lambda^*_X(\alpha_0) \cdot \gamma \cdot \lambda^*_{X\inv}(\beta_0) \in \St(\Psi, R)$.
    It follows from~\eqref{eq:alpha-def} that the element $\pi_R^*(j(\gamma_1)) \in \St(\Phi, k[X, X\inv])$ is trivial, hence $\pi^*_R(\gamma_1) \in \K_2(\Psi, k[X, X\inv])$.
    Recall from~\cite{Hur77} that $\K_2(\Psi, k[X, X\inv]) \to \K_2(\Phi, k[X, X\inv])$ is injective (see the assertion after Korollar~6).
    Thus, $\gamma_1 \in \overline{\St}(\Psi, R, M[X, X\inv]).$
    There exists $\gamma_2 \in \St(\Psi, R, M[X, X\inv])$ such that $\iota (\gamma_2) = \gamma_1$.
    Notice that $j(\gamma_2) \cdot \lambda_{X\inv}^{rel}(\beta_1) = \kappa \lambda_X^{rel}(\widetilde{\alpha})$ for $\kappa \in \Ker(\iota\colon \St(\Phi, R, M[X, X\inv]) \to \St(\Phi, R))$

    By the third assertion of~\cref{prop:horrocks-main} $j(\gamma_2) \cdot [1, \beta_1, 1] = [j(g), h, u]$ for some $g \in \G_{M, \Psi}^{\geq 0}$.
    Observe that
    \[ \lambda_{X\inv, B}^* (j(\gamma_2) \cdot i_B(\beta_1)) = j(\gamma_1) \lambda_{X\inv}^*(\beta_1) = \lambda_X^*(j(\alpha_0)) \cdot j(\gamma) \lambda_{X\inv}^*(\beta) = \lambda_{X\inv, B}^*(t_M(\widetilde{\alpha})), \]
    hence there exists $\kappa \in \Ker(\lambda^*_{X\inv, B})$ such that $j(\gamma_2) \cdot i_B(\beta_1) = \kappa t_M(\widetilde{\alpha})$.
    Consequently, from~\cref{prop:horrocks-main} we obtain that
    \[j(\gamma_2) \cdot [1, \beta_1, 1] = j(\gamma_2) \cdot i_B(\beta_1) [1, 1, 1] = \kappa [\lambda_X^*(j(\alpha_0) \cdot \alpha), 1, 1].\]
    By the definition of $\overline{V}$ one has $j(\alpha_0) \alpha = j(g) g_0$ for $g$ as above and some $g_0 \in \overline{\St}(\Phi, A, M)$.
    Since $\alpha(0) = 1$ we obtain that $g_0 = j(g)(0) \cdot j(\alpha_0)(0)$, which implies the required assertion.
\end{proof}