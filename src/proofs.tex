Throughout this section we also denote by $\lambda_a$ the homomorphism of principal localization at $a \in A$.
Also for a prime ideal $M \trianglelefteq A$ we denote by $\lambda_M \colon A \to A_M$ the homomorphism inverting the multiplicative subset $A \setminus M$.

For a commutative ring $A$, and a $A$-algebra $B$ and $b\in B$ we denote by $\ev{A}{B}{b}\colon A[t]\rightarrow B$ the morphism of $A$-algebras evaluating each polynomial
 $p(t)\in A[t]$ at $b$, i.e. $\ev{A}{B}{b} (p(t)) = p(b)$.
\subsection{Early stability theorem}
The aim of this subsection is to prepare necessary technical ingredients for the proof of early stability theorem.

For the rest of this subsection $\Psi$ is an arbitrary irreducible simply-laced root system of rank $\geq 3$ embedded into another irreducible simply-laced root system $\Phi$.
For a commutative ring $R$ we denote by $j_R$ the corresponding homomorphism of Steinberg groups $\St(\Psi, R) \to \St(\Phi, R)$ induced by this embedding.

The following result is the analogue of the so-called Dilation principle for subsystem embeddings.
\begin{lemma}\label{lem:dp-2}
Let $h\in\St(\Phi, A[X], X A[X])$ be such that $\lambda_a^*(h) \in \Img(j_{A_a[X]})$.
Then for sufficiently large $n$ one has
\[\ev{A}{A[X]}{a^n\cdot X}^*(h) \in \mathrm{Im}(j_{A[X]}).\]
\end{lemma}
\begin{proof}
 We denote by $A\ltimes XA_a[X]$ the semidirect product of $A$ and the ideal $XA_a[X]$, cf. e.\,g.~\cite[Definition~3.2]{S15}.
 Denote by $\theta$ the obvious map $A[X]\rightarrow A\ltimes XA_a[X]$ localizing at $a$ all coefficients of terms of degree $\geq 1$.

 Recall from~\cite[\S~2]{LS17} that there exists a homomorphism
 \[T_\Psi \colon \St(\Psi, A_a[X], XA_a[X]) \to \St(\Psi, A \ltimes XA_a[X])\]
 such that $\theta^* = T_\Psi \circ \lambda_a^*$.

 Let $(A_i, f_{ij})$ be the directed system of rings given by
 \[A_i\coloneqq A[X],\ f_{ij} \coloneqq \ev{A}{A[X]}{a^{j-i} \cdot X},\ 0 \leq i\leq j.\]
 It is easy to check that $\varinjlim A_i$ coincides with $A \ltimes XA_a[X]$.
 The canonical morphisms $A_j\rightarrow \varinjlim_i A_i \cong A \ltimes XA_a[X]$ can be easily computed as $\ev{A}{A\ltimes XA_a[X]}{a^{-j} \cdot X}$.

 By hypothesis $\lambda_a^*(h) = j_{A_a[X]}(h')$ for some $h' \in \St(\Psi, A_a[X], XA_a[X])$.
 Consequently, $\theta^*(h) = j_{A \ltimes XA_a[X]}(T_\Psi(h'))$
 and the assertion of the lemma follows from the fact that the Steinberg group functor commutes with
  colimits over directed systems (cf.~\cite[Lemma~2.2]{Tu83}):
 \[\St(\Psi, A\ltimes XA_a[X]) = \St(\Psi, \varinjlim_i A_i) \cong \varinjlim_i \St(\Psi,A_i). \qedhere\]
\end{proof}

\begin{lemma}\label{lem:L25-2}
Let $a$ and $b$ be a pair of coprime elements of $A$.
Let $g$ be an element of $\St(\Phi, A[X], XA[X])$ such that
$\lambda_a^*(g) \in \Img(j_{A_a[X]})$ and $\lambda_b^*(g) \in \Img(j_{A_b[X]})$.
Then $g$ lies in the image of $j_{A[X]} \colon \St(\Psi, A[X]) \to \St(\Phi, A[X])$.
\end{lemma}
\begin{proof}
    We reproduce the argument of~\cite[Lemma~2.5]{Tu83}, cf. also with~\cite[Lemma~16]{S15}.
    Set $S := A[X, Y]$.
    Consider the following element of $\St(\Phi, S[Z])$:
    \[h(X, Y, Z) := g(YX) \cdot  g^{-1}((Y+Z) X) = \ev{A}{S[Z]}{YX}^* \left(g\right) \cdot \ev{A}{S[Z]}{(Y + Z)X}^*\left(g^{-1}\right).\]
    It is easy to see that $h(X, Y, Z)$ belongs to
    \[\Ker\left(\eval{Z}{S}{S}{0}^*\colon\St(\Phi, S[Z]) \rightarrow \St(\Phi, S)\right)\]
    and hence by~\cite[Lemma~8]{S15} lies in $\St(\Phi, S[Z], Z S[Z])$.
    On the other hand, \begin{multline*}
                           \lambda_{a}^*(h(X, Y, Z)) = \left(\lambda_a\circ \ev{A}{S[Z]}{Y X}\right)^*(g) \cdot \left(\lambda_a\circ \ev{A}{S[Z]}{(Y + Z)X}\right)^*(g^{-1}) = \\
                           = \ev{A_a}{S_a[Z]}{YX}^*(\lambda_{a}^*(g)) \cdot \ev{A_a}{S_a[Z]}{(Y + Z)X}^*(\lambda_{a}^*(g^{-1})) \end{multline*}
    lies in the image of $j_{S_a[Z]}$.
    Similarly, $\lambda_{b}^*(h(X, Y, Z))$ lies in the image of $j_{S_b[Z]}$.

    We claim that there exists $n$ such that both $h(X, Y, a^n Z)$ and $h(X, Y, b^n Z)$
    lie in the image of $j(S[Z])$.

    By assumption, there exist $r, s\ \in A$ such that $r a^n + s b^n = 1$, consequently
    \begin{multline*}
        g(X) = g(X)\cdot g^{-1}(ra^n\cdot X) \cdot g(ra^n\cdot X) \cdot g^{-1}(0) = \\
         = h(X, 1, -sb^n) \cdot h(X, ra^n, -ra^n) \in \mathrm{Im}(j_{A[X]}). \qedhere
    \end{multline*} \end{proof}

The following result is the subsystem analogue of~\cite[Theorem~2]{LS17}, cf.\ also~\cite[Theorem~2.1]{Tu83}.
\begin{cor}[Local-global principle for subsystem embeddings]
    An element $g \in \St(\Phi, A[X], XA[X])$ lies in $\Img(j_{A[X]})$ if and only if
     $\lambda_M^*(g) \in \St(\Phi, A_M[X])$ lies in $\Img(j_{A_M[X]})$ for all maximal ideals $M \trianglelefteq A$.
\end{cor}
\begin{proof}
    It suffices to show ``if'' part of the statement.
    One defines the \textit{Quillen set} $Q(g)$ as the set consisting of those elements $a \in A$
    such that $\lambda_a^*(g) \in \Img(j_{A_a[X]})$.

    Repeating the same argument as in the proof of~\cite[Theorem~2]{S15} one shows using~\cref{lem:L25-2}
     that $Q(g)$ is an ideal of $A$, which can not be proper and therefore must coincide with $A$.
\end{proof}

Before we proceed further we would like to briefly recall the main construction from~\cite{LS20} upon which
 the proof of Theorem~1 ibid. is based.
Recall that one constructs an action of the group $\St(\Phi, A[X\inv] + M[X])$ on a certain set $\overline{V}$.
This set $\overline{V}$ is unrelated to the set $\overline{V}$ encountered in~\cref{sec:horrocks}.
Its construction proceeds as follows.

Set $G_{M, \Phi}^{\geq 0} \coloneqq \Img(\St(\Phi, A[X], M[X]) \to \St(\Phi, A[X, X\inv])), G_M^0 \coloneqq \overline{\St}(\Phi, A, M)$.
$G_M^0$ is easily seen to be a subgroup of both $\St(\Phi, A[X\inv])$ and $G_M^{\geq 0}$.
Denote by $\overline{V}$ the quotient-set of the product $V \coloneqq G_M^{\geq 0} \times \St(\Phi, A[X\inv]) \times (1 + M)^\times$
modulo the equivalence relation given by $(gh_0, h, u) \cong (g, h_0h, u)$ where $h_0 \in G_M^0, (g, h, u) \in V.$
Denote by $[g, h, u] \in \overline{V}$ the equivalence class corresponding to $(g, h, u)\in V$, cf.~\cite[\S~5.4]{LS20}.

Notice also that although the results in~\cite[\S~5.5]{LS20} are conditional,
 i.\,e. they are formulated under additional assumption that the canonical homomorphism
 $\St(\Phi, A[X\inv] + M[X]) \to \St(\Phi, A[X, X\inv])$ is injective,
this does not present a problem for us since this condition has been checked already during the proof of~\cref{lem:first-reduction}.

\begin{prop} \label{prop:horrocks-main} The group $\St(\Phi, A[X\inv] + M[X])$ acts simply transitively on $\overline{V}$.
This action satisfies the following additional property.
If
\[g \in \Img(j\colon \St(\Psi, A[X\inv] + M[X]) \to \St(\Phi, A[X\inv] + M[X])), \]
$h \in \St(\Phi, A[X\inv])$ and $u \in 1 + M$ then
\[ g \cdot [1, h, u] = [g', h', u']\] for some
 $g' \in \Img(j\colon G_{M, \Psi}^{\geq 0} \to G_{M, \Phi}^{\geq 0})$, $h' \in \St(\Phi, A[X\inv])$, $u'\in 1 + M$.
\end{prop}
\begin{proof}
    The existence of the action of $\St(\Phi, A[X\inv] + M[X])$ and its faithfullness are contained in
    Proposition~5.39 and Remark 5.42 of~\cite{LS20}.
    The stated property directly follows from the construction of the action in~\cite[\S~5.4]{LS20}.
\end{proof}

\begin{lemma} \label{lem:horrocks--ingredient}
Let $A$ be a local ring with maximal ideal $M$ and residue field $\kappa$.
Suppose that the image in $\St(\Phi, A[X, X\inv])$ of the element $x \in \K_2(\Phi, A[X], XA[X])$
 can be decomposed as $j_{A[X, X\inv]}(y) \cdot \lambda_{X^{-1}}^*(z)$ for
 some $y \in \St(\Psi, A[X, X\inv])$ and $z \in \St(\Phi, A[X\inv])$
Then $x$ belongs to $\Img(j_{A[X]})$.
\end{lemma}
\begin{proof}
    We denote by $\rho_{A}$ (resp. $\rho_{A[X]}$, $\rho_{A[X, X\inv]}$) the canonical homomorphism
     $A \to \kappa$ (resp. $A[X] \to \kappa[X]$, $A[X, X\inv] \to \kappa[X, X\inv]$).

    Recall from~\cite{Hur77} that $\K_2(\Phi, \kappa[X]) = \K_2(\Phi, \kappa)$ hence one has $\rho_{A[X]}(x) = 1$
     hence by our assumptions $\rho_{A[X, X\inv]}(j_{A[X, X\inv]}(y) \cdot z) = 1$.
    Consequently, we can find $z_0 \in \St(\Psi, A[X\inv])$ and $z_1 \in \overline{\St}(\Phi, A[X^{-1}], M[X^{-1}])$
     such that $z = j_{A[X\inv]}(z_0) \cdot z_1$.
    Set $y' = y \cdot \lambda_{X^{-1}}(z_0)$, from $\rho_{A[X, X\inv]}(y') = 1$ we obtain that
    $y' \in \overline{\St}(\Psi, A[X, X\inv], M[X, X\inv])$.

    By the third assertion of~\cref{prop:horrocks-main} $j(z_2) \cdot [1, y_1, 1] = [j(g), h, u]$ for some $g \in \G_{M, \Psi}^{\geq 0}$.
    Observe that
    \[ \lambda_{X\inv, B}^* (j(z_2) \cdot i_B(y_1)) = j(z_1) \lambda_{X\inv}^*(y_1) = \lambda_X^*(j(x_0)) \cdot j(z) \lambda_{X\inv}^*(y) = \lambda_{X\inv, B}^*(t_M(\widetilde{x})), \]
    hence there exists $\kappa \in \Ker(\lambda^*_{X\inv, B})$ such that $j(z_2) \cdot i_B(y_1) = \kappa t_M(\widetilde{x})$.
    Consequently, from~\cref{prop:horrocks-main} we obtain that
    \[j(z_2) \cdot [1, y_1, 1] = j(z_2) \cdot i_B(y_1) [1, 1, 1] = \kappa [\lambda_X^*(j(x_0) \cdot x), 1, 1].\]
    By the definition of $\overline{V}$ one has $j(x_0) x = j(g) g_0$ for $g$ as above and some $g_0 \in \overline{\St}(\Phi, A, M)$.
    Since $x(0) = 1$ we obtain that $g_0 = j(g)(0) \cdot j(x_0)(0)$, which implies the required assertion.
\end{proof}

\subsection{Main results}

We start by recalling the so-called Zariski excision property of Steinberg groups.
\begin{lemma} \label{lem:zariski-glueing}
Let $\Phi$ be any simply-laced root system of rank $\geq 3$.
Let $A$ be a commutative domain and $a, b \in A$ be a pair of coprime elements.
\begin{enumerate}
    \item Let $\delta$ be an element of $\St(\Phi, A_{ab})$.
    Then $\delta$ can be presented as $\lambda_b(x) \cdot \lambda_a(y)$ for some
    $x  \in \St(\Phi, A_a)$ and $y \in \St(\Phi, A_b)$.
    \item  Let $x \in \St(\Phi, A_a)$ and $y \in \St(\Phi, A_b)$ be such that the equality $\lambda_b(x) = \lambda_a(y)$ holds in $\St(\Phi, A_{ab})$.
    Then there exists $z \in \St(\Phi, A)$ such that $x = \lambda_a(z)$, $y = \lambda_b(z)$.
\end{enumerate}
\end{lemma}
\begin{proof}
    This is a special case of Nisnevich excision for domains, see~\cite[Proposition~4.5]{LSV2}
    (cf. also the proof of~\cite[Lemma~2.6]{LSV2}).
\end{proof}


\begin{lemma} \label{lem:horrocks-b}
 Let $A$ be a commutative domain and $f \in A[X]$ be a unitary polynomial.
 Let $\alpha$ be an element of $\K_2(\Phi, A[X], XA[X])$ such that $\lambda_f(\alpha)$ belongs to the image of the stabilization map
 $\St(\Psi, A[X]_f) \to \St(\Phi, A[X]_f)$.
 Then $\alpha$ lies in the image of $\St(\Psi, A[X]) \to \St(\Phi, A[X])$.
\end{lemma}
\begin{proof}
    Suppose that $f(X) = X^n + a_1 X^{n-1} \ldots + a_n$.
    Set $g(X\inv) = 1 + a_1 X\inv + \ldots + a_{n} X^{-n}$.
    It is clear that $A[X, X\inv]_f = A[X\inv]_{X\inv g}$ and, moreover, that $X\inv$ and $g$ are not zero divisors and together generate the unit ideal of $A$.

    Consider the following diagram:
    %! suppress = EscapeAmpersand
    \[ \xymatrix{\St(\Phi, A[X]) \ar[r]^{\lambda_X} \ar[d]_{\lambda_f} & \St(\Phi, A[X, X\inv]) \ar[d]_{\lambda_f}  & \St(\Phi, A[X\inv]) \ar[l]_{\lambda_{X\inv}} \ar[d]_{\lambda_g}  \\
                 \St(\Phi, A[X]_f) \ar[r] & \St(\Phi, A[X, X\inv]_f) & \St(\Phi, A[X\inv]_g) \ar[l] \\
                 \St(\Psi, A[X]_f) \ar[r] \ar[u]_j & \St(\Psi, A[X, X\inv]_f) \ar[u]_j & \St(\Psi, A[X\inv]_g) \ar[l] \ar[u]_j \\
                                   & \St(\Psi, A[X, X\inv]) \ar[u]_{\lambda_f^\Psi}   & \St(\Psi, A[X\inv]). \ar[l] \ar[u]_{\lambda_g^\Psi}}\]

    By assumption there exists $\widetilde{\alpha} \in \St(\Psi, A[X]_f)$ such that $j(\widetilde{\alpha}) = \lambda_f(\alpha)$.
    By the first part of~\cref{lem:zariski-glueing} one can write $\lambda_X(\widetilde{\alpha}) = \lambda_f^\Psi(\gamma) \cdot \lambda_{X\inv}(\beta)$
     for some $\beta \in \St(\Psi, A[X\inv]_g)$, $\gamma \in \St(\Psi, A[X, X\inv]).$
    Consequently, one has $\lambda_f(j(\gamma)\inv \cdot \lambda_X(\alpha)) = \lambda_{X\inv}(j(\beta))$.
    By the second part of~\cref{lem:zariski-glueing} there exists $\beta' \in \St(\Phi, A[X\inv])$ such that $\lambda_X(\alpha) = j(\gamma) \cdot \lambda_{X\inv}(\beta').$
    The assertion now follows from~\cref{lem:horrocks--ingredient} and~\cref{lem:horrocks--ingredient}.
\end{proof}

We now come to the main theorem of this section.
\begin{thm}\label{thm:early-stability}
Let $\Phi$ be a root system of type $\rA_{\geq 4}$, $\rD_{\geq 5}$ or $\rE_{6,7,8}$ and let $A$ be an arbitrary noetherian commutative ring of Krull dimension $\leq 1$.
Then for any $n \geq 0$ the obvious inclusion $\rA_4 \subseteq \Phi$ induces a surjection
\[\K_2(\rA_4, A[X_1,\ldots, X_n]) \to \K_2(\Phi, A[X_1, \ldots X_n]).\]
\end{thm}
\begin{proof}[Proof of Theorem 1]
    The proof is modeled after the proof of~\cite[Theorem~5.3]{Tu83}.
    We proceed by induction on $n$.
    Our assumption on the dimension of $A$ guarantees that it satisfies the condition $\mathrm{SR}_3$ in the sense of~\cite{St78}.
    Thus, from Corollary~3.2 and Theorem~4.1 of~\cite{St78} we conclude that the composite arrow in the following diagram is a surjection:
    \[\K_2(\rA_2, A) \to \K_2(\rA_4, A) \to \K_2(\Phi, A).\]
    Consequently, we obtain that the right arrow is a surjection, which yields the induction base.

    Now let us verify the induction step.
    Set $C = A[X_2, \ldots , X_n]$ and $B = C[X_1]$.
    We need to show that $\K_2(\rA_4, B) \to \K_2(\Phi, B)$ is surjective.
    Every element $\alpha \in \K_2(\Phi, B)$ can be decomposed as a product $\alpha = \alpha_0 \cdot \alpha_1$,
      where $\alpha_0 \in \K_2(\Phi, C)$ and $\alpha_1 \in \K_2(\Phi, B, X_1 B)$.
    By inductive assumption $\K_2(\rA_4, C)$ surjects onto $\K_2(\Phi, C)$, so it remains to show that $\alpha_1$ lies in the image of $\K_2(\rA_4, B)$.

    Denote by $S$ the multiplicative system $S \subseteq B$ consisting of polynomials $f$ such that for sufficiently large $m$
    the polynomial $f$ becomes unitary in $Y_1$ after substitutions $X_1 \coloneqq Y_1,$ $X_2 \coloneqq Y_2 + Y_1^m, \ldots X_n \coloneqq Y_n + Y_1^{m^n}$.
    Recall from~\cite[\S~6]{Su77} that $\dim(S^{-1}B) \leq \dim(A) = 1$.
    By induction base the map $\K_2(\rA_4, S^{-1}B) \to \K_2(\Phi, S^{-1}B)$ is surjective.
    Since functor $\K_2$ commutes with filtered colimits (cf. \cite[Lemma~3.3]{LSV2}) there exists $f \in S$ such that $\lambda^*_f(\alpha_1)$ lies in the image of $\K_2(\rA_4, B_f) \to \K_2(\Phi, B_f)$.
    By the construction of $S$ we may assume that $f$ is unital in $X_1$.
    The required assertion now follows from~\cref{lem:horrocks-b}.
\end{proof}

\begin{proof}[Proof of~\cref{cor:dedekind}]
    Consider the following diagram:
    \[ \K_2(\rA_3, A) \to \K_2(\rA_4, A) \to \K_2(\rA_4, A[X_1, \ldots, X_n]) \rightarrow \K_2(A[X_1, \ldots, X_n]) \to \K_2(A).\]
    The two right arrows on the above diagram are isomorphisms by the $\mathbb{A}^1$-invariance of the stable $\K_2$
    (see e.\,g. \cite[Theorem~V.6.3]{Kbook}) combined with the main result of~\cite{Tu83}.
    On the other hand, by~\cite[Corollary~3.2]{ST76} the left arrow and the composite arrow are isomorphisms.
    Our assertion now follows from~\cref{thm:early-stability}.
\end{proof}

\begin{comment}
\subsection{Horrocks ingredient}\label{subsec:horrocks-ingredient}


Now let $A$ be a local ring with maximal ideal $M$.
Set $B \coloneqq A[X\inv] + M[X],\ R \coloneqq A[X, X\inv]$.
Consider the following diagram:
%! suppress = EscapeAmpersand
\[ \xymatrix{ & \St(\Phi, A[X], M[X]) \ar[d]_{t_M} \ar[r] & \St(\Phi, A[X]) \ar[d]_{\lambda_X^*} \\
\St(\Phi, A[X\inv]) \ar[r]^{i_B} \ar@/_/[rr]_{\lambda_{X\inv}^*} & \St(\Phi, B) \ar[r]^{\lambda^*_{X\inv, B}} & \St(\Phi, R)
}\]
The arrow $t_M$ arises from the \("\)lifting property\("\) of Steinberg groups (cf. \cite[Lemma~3.3]{LS20} or~\cite[Theorem~3]{LS17}) as the composition of morphisms
$\lambda_{X}^{rel}\colon \St(\Phi, A[X], M[X]) \to \St(\Phi, R, M[X, X\inv])$ and $T \colon \St(\Phi, R, M[X, X\inv]) \to \St(\Phi, B)$.
%This is wrong: Its image $\Img(t_M) \leq \St(\Phi, B)$ is generated as a subgroup by elements $x_\alpha(f)$, $z_\alpha(f, a)$, $z_\alpha(X^2f, aX^{-1})$, for $\alpha\in \Phi$, $a \in A$, $f \in M[X]$, see the proof of~\cite[Lemma~5.41]{LS20}.
\end{comment}