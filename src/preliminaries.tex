\usepackage{amssymb}In this paper all rings are assumed to be commutative and all commutators are left-normed i.\,e. $[a, b] = a b a^{-1} b^{-1}$.

\subsection{Steinberg groups, $\K_2$-groups and symbols}\label{subsec:steinberg-preliminaries}
Let $\Phi$ be a finite root system of rank $\ell > 1$.
We assume that $\Phi$ is embedded into $\mathbb{R}^\ell$ whose scalar product we denote by $(\text{-}, \text{-})$.
We also fix some system of simple roots $\Pi = \{\alpha_1, \ldots, \alpha_\ell\} \subset \Phi$.
For a root $\alpha\in\Phi$ we denote by $m_i(\alpha)$ the $i$-th coefficient in the expansion of $\alpha$ in $\Pi$,
i.\,e. $\alpha = \sum_{i=1}^n m_i(\alpha) \alpha_i$.
We denote by $\Phi^+$ (resp. $\Phi^-$) the system of positive (resp. negative) roots with respect to the basis $\Pi$.

We denote by $\Phi^\vee$ the corresponding dual root system, which, by definition, consists of all coroots $\alpha^\vee = \frac{2}{(\alpha, \alpha)} \alpha$, where $\alpha \in \Phi$.
We denote by $P(\Phi^\vee)$ the integral lattice spanned by the \emph{fundamental coweights $\varpi_i^\vee$}.
Recall that the fundamental coweights $\varpi_i^\vee$ are uniquely determined by the property $(\varpi_i^\vee, \alpha_j) = \delta_{ij}$.
For $\alpha,\beta \in \Phi$ we denote by $\langle \alpha, \beta \rangle$ the integer $(\alpha, \beta^\vee) = \frac{2(\alpha, \beta)}{(\beta, \beta)}$.

Now let $R$ be an arbitrary commutative ring with $1$ and suppose that $\Phi$ is an irreducible root system of rank $\geq 2$.
Recall that to the pair $(\Phi, R)$ can associate an abstract group $\St(\Phi, R)$, called the \textit{Steinberg group} of type $\Phi$ over $R$.
By definition, $\St(\Phi, R)$ is the group presented by generators $x_\alpha(a)$, $a \in R$, $\alpha \in \Phi$ and an explicit list of relations (see e.\,g in.~\cite{Ma69, St71}).
In this paper we may restrict ourselves to the case when the root system $\Phi$ in question is \textit{simply-laced} (i.\,e. has type $\mathsf{ADE}$),
 in which case the defining relations of $\St(\Phi, R)$ reduce to the following shorter list:
\begin{align}
x_{\alpha}(a)\cdot x_{\alpha}(b)&=x_{\alpha}(a+b), \tag{R1} \label{x-additivity}\\
[x_{\alpha}(a),\,x_{\beta}(b)]  &=x_{\alpha+\beta}(N_{\alpha,\beta} \cdot ab),\text{ for }\alpha+\beta\in\Phi, \tag{R2} \label{R2} \\
[x_{\alpha}(a),\,x_{\beta}(b)]  &=1,\text{ for }\alpha+\beta\not\in\Phi\cup0. \tag{R3} \label{R3}
\end{align}
The coefficients $N_{\alpha,\beta}$ in the above formula are integers equal to $\pm 1$, they coincide with the structure constants of the complex Lie algebra of type $\Phi$.

Throughout this paper we denote by $\Gsc(\Phi, R)$ the group of points of the simply-connected Chevalley--Demazure group of type $\Phi$ over $R$.
We denote by $\Esc(\Phi, R)$ the \textit{elementary subgroup} of $\Gsc(\Phi, R)$, i.\,e. the subgroup generated by elementary root unipotents of $\Gsc(\Phi, R)$.
Notice that in~\cite{VP, Vav09} the notation $x_\alpha(a)$ is used to denote the elementary root unipotents.
To prevent confusion we will use different notation $t_\alpha(a)$ for them and reserve the notation $x_\alpha(a)$ solely for generators of Steinberg groups.

Recall that the map sending $x_\alpha(a)$ to $t_\alpha(a)$ gives rise to a well-defined homomorphism $\pi \colon \St(\Phi, R) \to \G_\mathrm{sc}(\Phi, R)$, see~\cite[\S~1A]{St78}.
The cokernel and the kernel of $\pi$ are called \textit{the unstable $\K_1$- and $\K_2$-functors modeled on the root system $\Phi$}:
\begin{equation} \label{eq:K1-K2-sequence}
  \xymatrix{ 1 \ar[r] & \K_2(\Phi, R) \ar[r] & \St(\Phi, R) \ar[r]^{\pi} & \Gsc(\Phi, R) \ar[r] & \K_1(\Phi, R) \ar[r] & 1}
\end{equation}

In the special case $\Phi =\rA_\ell$ the Steinberg $\St(\rA_\ell, R)$ coincides with the linear Steinberg group $\St(\ell + 1, R)$, while
  $\Gsc(\rA_\ell, R) = \mathrm{SL}(\ell + 1, R)$, $\Esc(\rA_\ell, R) = \E(\ell + 1, R)$.

For a \textit{special} subset of $\Phi$ (i.\,e. a subset $S \subseteq \Phi$ such that $S \cap -S = \varnothing$)
we denote by $\UU(S, R)$ the subgroup of $\St(\Phi, R)$ generated by $x_\alpha(a)$, where $a \in R$, $\alpha \in S$.
Notice that the restriction of $\pi$ to $\UU(S, R)$ is a monomorphism, so $\UU(S, R)$ is isomorphic to a unipotent subgroup of $\Gsc(\Phi, R)$.

Denote by $\Sigma^+_i$ (resp. $\Sigma^-_i$) the subset of $\Phi$ consisting of $\alpha \in \Phi$ such that $m_i(\alpha) > 0$ (resp. $m_i(\alpha) < 0$).
Denote by $\Delta_i$ the set of roots $\alpha \in \Phi$ such that $m_i(\alpha) = 0$.
It is clear that $\Phi = \Sigma_i^- \sqcup \Delta_i \sqcup \Sigma_i^+$ and that $\Sigma_i^+$, $\Sigma_i^-$ are special subsets of $\Phi$.

Following~\cite{Ma69} for $\alpha\in\Phi$ and $u \in R^\times$ we define the following elements of $\St(\Phi, R)$:
\begin{align} w_\alpha(u) & =  x_\alpha(u) \cdot x_{-\alpha}(-u^{-1}) \cdot x_\alpha(u), \label{eq:w-definition} \\
               h_\alpha(u) & =  w_\alpha(u) \cdot w_\alpha(-1).  \label{eq:h-definition} \end{align}
The subgroup generated by $w_\alpha(u)$ (resp. $h_\alpha(u)$) for all $\alpha\in \Phi$, $u \in R^\times$ is denoted by $\StW(\Phi, R)$ (resp. $\StH(\Phi, R)$).
By~\cite[Lemme~5.2]{Ma69} $\StH(\Phi, R)$ is a normal subgroup of $\StW(\Phi, R)$.

Denote by $W(\Phi, R)$ the image of $\StW(\Phi, R)$ in the Chevalley group $\Gsc(\Phi, R)$.
It is clear that $W(\Phi, R)$ is contained in the group-theoretic normalizer $N(\Phi, R)$ of the split maximal torus $H(\Phi, R) \leq \Gsc(\Phi, R)$.

In the sequel we will need the following explicit elements of the group $\K_2(\Phi, R)$.
Recall that for arbitrary $u, v \in R^\times$ one defines the \textit{Steinberg symbol} via the formula
\begin{equation} \label{eq:steinberg} \{ u, v \}_\alpha = h_\alpha(uv) \cdot h_\alpha^{-1}(u) \cdot h_\alpha^{-1}(v). \end{equation}
Recall also from~\cite[Lemme~5.4]{Ma69} that
\begin{equation} \label{eq:steinberg-2} [h_\alpha(u), h_\beta(v)] = \{u, v^{\langle \alpha, \beta \rangle}\}_\alpha. \end{equation}
Steinberg symbols depend only on the length of the root $\alpha$.
In particular, in the case when $\Phi$ is of simply-laced type, they do not depend on the choice of $\alpha$, which allows us to omit it from notation.

Steinberg symbols are central elements of $\St(\Phi, R)$.
Our assumptions on $\Phi$ guarantee that Steinberg symbols are antisymmetric and bimultiplicative, i.\,e. they satisfy the following identities:
\begin{equation} \label{eq:symbol-properties} \{ u, st \} = \{ u, s\} \{ u, t \}, \ \{ u, v \} = \{ v, u\}^{-1}. \end{equation}

Now let $A$ be a local ring with maximal ideal $M$ and residue field $k = A/M$.
For $w \in \StW(\Phi, A)$ we denote by $\overline{w}$ the image of $w$ in $W(\Phi)$ under the following composite homomorphism:
\[ \StW(\Phi, A) \to W(\Phi, A) \to W(\Phi, k) \to N(\Phi, k) \to N(\Phi, k) / H(\Phi, k) \cong W(\Phi). \]

\subsection{Relative Steinberg groups} \label{subsec:another-presentation}
In this paper we use the concept of a \textit{relative Steinberg group} introduced by F.~Keune and J.-L.~Loday in~\cite{Ke78, Lo78}.
We will only briefly mention the definition and basic properties of these groups and refer the reader to~\cite[\S~2.3]{LS20} for a more detailed exposition.

Let $R$ be a commutative ring, $I \trianglelefteq R$ be an ideal and let $p$ denote the canonical projection $R \to R/I$.
Denote by $D_{R, I}$ the pullback of two copies of $p$ i.\,e. the ring $R \times_{R/I} R$.
Elements of $D_{R, I}$ are pairs $(a; b)$ such that $a-b \in I$.
We also denote by $p_1$, $p_2$ the canonical projections $D_{R, I} \to R$ and by $p_1^*$, $p_2^*$ the corresponding homomorphisms of Steinberg groups induced by them.
Recall from~\cite[Definition~2.5]{LS20} that \textit{the relative Steinberg group} $\St(\Phi, R, I)$ is defined as the quotient
$\Ker(p_1^*) / C$, where $C = [\Ker(p_1^*), \Ker(p_2^*)]$.
Denote by $\mu$ the homomorphism $\St(\Phi, R, I) \to \St(\Phi, R)$ induced by $p_2^*$.
Also denote by $C(\Phi, R, I)$ the kernel of $\mu$.
Thus, we obtain an exact sequence
\begin{equation}
    \xymatrix{C(\Phi, R, I) \ar@{^{(}->}[r] & \St(\Phi, R, I) \ar[r]^\mu & \St(\Phi, R) \ar@{->>}[r]^-{p^*} & \St(\Phi, R/I). }\label{eq:relative-Steinberg}
\end{equation}
Alternatively, the group $\St(\Phi, R, I)$ can be defined via generators and relations as an $\St(\Phi, R)$-group, cf.~\cite[Proposition~6]{S15}
or even as an abstract group, see~\cite{V22}.

\begin{lemma} \label{lem:rel-Steinberg-crossed-module}
For any irreducible root system $\Phi$ of rank $\geq 3$ the homomorphism $\mu \colon \St(\Phi, R, I) \to \St(\Phi, R)$ is a crossed module.
\end{lemma}
\begin{proof}
    Denote by $\Delta$ the diagonal homomorphism $R \to D_{R, I}.$
    By~\cite[Proposition~6]{Lo78} it suffices to show that $[\Ker(p_1^*) \cap \Ker(p_2^*), \Delta^*(\St(\Phi, R))] = 1$.
    But the latter is clear since $\Ker(p_1^*) \cap \Ker(p_2^*)$ is contained in the subgroup $\K_2(\Phi, D_{R, I})$,
     which is a central subgroup of $\St(\Phi, D_{R, I})$ by~\cite{LSV1}.
\end{proof}

Recall that an ideal $I \trianglelefteq R$ is called \textit{splitting} if the canonical projection $R \twoheadrightarrow R/I$ splits.
If $I \trianglelefteq R$ is splitting then $C(\Phi, R, I)$ is trivial and $\St(\Phi, R)$ decomposes into the semidirect product
 $\St(\Phi, R, I) \rtimes \St(\Phi, R/I)$, see~\cite[\S~1]{LS17}.

We define the relative group $\K_2(\Phi, R, I)$ as the kernel of the homomorphism $\pi \mu$.
For a special subset of roots $S \subseteq \Phi$ we denote by $\UU(S, I)$ the subgroup of $\St(\Phi, R, I)$ generated by $x_\alpha((0;m))C$, $m \in I$, $\alpha \in S$.
It is clear that $\UU(S, I) \cap \K_2(\Phi, R, I) = 1$.

We will need a relative analogue of Steinberg symbol.
Let $A$ be a local unital ring with maximal ideal $M$ embedded as a subring into a larger unital ring $R$.
It is clear that under this assumption the subset $1+M \subseteq A$ forms a group under multiplication.
We denote this group by $(1+M)^\times$.
Clearly, it is isomorphic to the abelian group $(M, \circ)$ with the operation given by $m \circ m' = m + m' + mm'$.
Now for $a \in R^\times$ and $m \in M$ we denote by $\{a, 1+m\}_r$ the coset $\{(a; a), (1; 1+m)\}C \in \St(\Phi, R, RM)$.
It is clear that the map $1+m \mapsto \{a, 1+m\}_r$ specifies a group homomorphism
\begin{equation} \label{eq:relative-symbol} \{ a, -\}_r \colon (1+M)^\times \to \K_2(\Phi, R, RM). \end{equation}
It is also clear that $\mu(\{a, 1+m\}_r) = \{a, 1+m\}$.

\begin{lemma}\label{lem:symbols}
Assume that $A$ is a local domain with maximal ideal $M$.
Denote by $R$ the Laurent polynomial ring $A[X, X\inv]$.
Then the intersection of the image of the relative symbol map $\{X, -\}_r$ with $C(\Phi, R, M[X, X\inv])$ is trivial.
\end{lemma}
\begin{proof}
    Set $F = \mathrm{Frac}(A)$.
    Consider the following diagram:
    \[\begin{tikzcd}
    (1+M)^\times \ar[hookrightarrow, rr] \ar[d] &  & F^\times \ar[hookrightarrow, d] \\
    \K_2(\Phi ,R, M[X^{\pm 1}]) \ar[r] & \K_2(\Phi, R) \ar[r] & \K_2(\Phi, F[X^{\pm 1}]).
    \end{tikzcd}\]
    Since the right vertical arrow is injective by~\cite[Lemma~2.2]{LS20}, so is the left arrow.
\end{proof}

In the sequel we will need the following Steinberg-level analogue of Bruhat decomposition.
Denote by $\overline{\St}(\Phi, R, I)$ the kernel of $p^*$ (or what is the same, the image of $\mu$).
\begin{lemma}\label{lem:bruhat}
Let $A$ be a local ring with maximal ideal $M$ and residue field $k$.
Let $\Phi$ be any irreducible root system and $\Phi^+, \Phi^+' \leq \Phi$ be a pair of systems of positive roots of $\Phi$.
The Steinberg group $\St(\Phi, A)$ admits the following decomposition:
\begin{equation}\label{eq:bruhat}
\St(\Phi,A) =\UU(\Phi^+', A)\cdot \StW(\Phi,A)\cdot\UU(\Phi^+, A) \cdot \overline{\St}(\Phi, A, M).
\end{equation}
Moreover, if $uwvl=u'w'v'l'$ for some $u,u'\in \UU(\Phi^{+'}, A)$, $v, v' \in \UU(\Phi^+, A)$, $w,w'\in \StW(\Phi, A)$ and $l,l'\in \overline{\St}(\Phi, A, M)$, then
$\overline{w}=\overline{w'} \in W(\Phi)$ and $w^{-1}w' \in \StH(\Phi, A) \cap \overline{\St}(\Phi, A, M)$.
\end{lemma}
\begin{proof}
    Notice that $\E(\Phi, k) = \G_{sc}(\Phi, k)$ is a group with a BN-pair by~\cite[\S~4]{Ge17}.
    Since $\K_2(\Phi, k)$ is generated by Steinberg symbols, it is a subgroup of $\StW(\Phi, k)$.
    This allows us to consequently lift the BNB-decomposition from $\G_{sc}(\Phi, k)$ first to $\St(\Phi, k)$ and then to $\St(\Phi, A)$.
    This proves the first assertion of the lemma in the special case $\Phi^+' = \Phi^+$.
    To obtain the general case pick $w \in \StW(\Phi, A)$ such that ${}^w \UU(\Phi^+, A) = \UU(\Phi^+', A)$ and then multiply both sides of~\eqref{eq:bruhat} by $w$ on the left.

    Let us verify the second assertion.
    Projecting the equality to $\Gsc(\Phi, k)$ and using the fact that double B-cosets are disjoint we obtain that $w^{-1} w' \in \bigl(\StH(\Phi, A) \cdot \overline{\St}(\Phi, A, M)\bigr)\cap \StW(\Phi, A) \subseteq \StH(\Phi, A)$.
    Here we use the fact that $\K_2(\Phi, A)$ is generated by Steinberg symbols, see~\cite[Theorem~2.5]{Ste73}.
\end{proof}

\subsection{Weight automorphisms}\label{subsec:weight-automorphisms}
Recall that for every coweight $\omega \in P(\Phi^\vee)$ and $\beta \in \ZZ \Phi$ the scalar product $(\omega, \beta)$ is an integer.
Thus, a choice of $u \in R^\times$ and $\omega \in P(\Phi^\vee)$ specifies a permutation of the generating set for $\St(\Phi, R)$ via the following mapping:
\begin{equation*} x_\alpha(a) \mapsto x_\alpha(u^{(\omega, \alpha)} \cdot a),\ \alpha\in \Phi,\ a \in R. \end{equation*}
It is not hard to check that this permutation is compatible with relations~\eqref{x-additivity}--\eqref{R3} and therefore specifies a well-defined automorphism of $\St(\Phi, R)$, which we denote by $\chi_{\omega, u}$.

In the following lemma we check that an analogue of $\chi_{\omega, u}$ can also be defined for relative Steinberg groups.
\begin{lemma} \label{lem:relative-chi}
Let $R$ be a commutative ring, $I$ be its ideal and let $u \in R^\times$.
For every coweight $\omega \in P(\Phi^\vee)$ there exists a well-defined automorphism $\widetilde{\chi}_{\omega, u}$ of the relative Steinberg group $\St(\Phi, R, I)$ making the following diagram commute:
\[\begin{tikzcd} \St(\Phi, R, I) \ar[r, "\widetilde{\chi}_{\omega, u}"] \ar[d] & \St(\Phi, R, I) \ar[d] \\
\St(\Phi, R) \ar[r, "\chi_{\omega, X}"] & \St(\Phi, R]). \end{tikzcd}\]
\end{lemma}
\begin{proof}
    Observe that the automorphism $\chi_{\omega, (u; u)}$ of $\St(\Phi, D_{R, I})$ preserves subgroups
    $\Ker(p_i^*)$, $i=1, 2$ and hence their commutator subgroup $C$.
    The required automorphism $\widetilde{\chi}_{\omega, u}$ now can be obtained by restricting $\chi_{\omega, (u; u)}$ to $\Ker(p_1^*)$.
    The commutativity of the diagram is obvious.
\end{proof}

In the sequel we will use the following formulae describing the action of $\chi_{\omega, u}$ on the elements $w_\alpha(u)$, $h_\alpha(u)$:
\begin{align}
    \label{eq:chi-w} \chi_{\omega, u}\left(w_\alpha(v)\right) &= w_\alpha(u^{(\omega, \alpha)} \cdot v), \\
    \label{eq:chi-h} \chi_{\omega, u} (h_\alpha(v)) &= h_\alpha(u^{(\omega, \alpha)} \cdot v) \cdot h_\alpha(u^{(\omega, \alpha)})^{-1} = \{u^{(\omega, \alpha)}, v\} \cdot h_\alpha(v).
\end{align}

The following lemma is analogous to~\cite[Lemma~3.1(c)]{Tu83}.
\begin{lemma} \label{lem:winv-chiw}
For any $w \in \StW(\Phi, A[X^{\pm 1}])$ the element $w^{-1} \cdot \chi_{\omega, X}(w)$ belongs to $\StH(\Phi, A[X^{\pm 1}])$.
\end{lemma}
\begin{proof}
    Since $\StH(\Phi, A[X^{\pm 1}])$ is a normal subgroup of $\StW(\Phi, A[X^{\pm 1}])$, it suffices to verify the assertion for $w = w_\alpha(u)$.
    Set $h = w^{-1} \cdot \chi_{\omega, X}(w)$.
    Notice that
    \begin{multline*} w_\alpha(1) \cdot h\cdot  w_\alpha(-1) = w_\alpha(-1)^{-1} \cdot w_\alpha(u)^{-1} \cdot w_{\alpha}(X^{(\omega, \alpha)} u) \cdot w_\alpha(-1) = \\
    = h_\alpha(u)^{-1} \cdot h_\alpha(X^{(\omega, \alpha)}u) \in \StH(\Phi, A[X^{\pm 1}]).\end{multline*}
    Thus, we get from\cite[Lemme~5.2(b,g)]{Ma69} and~\eqref{eq:steinberg} that \[h = h_{\alpha}^{-1}(u^{-1}) \cdot h_{\alpha}(X^{-(\omega, \alpha)}u^{-1}) = h_{\alpha}^{-1}(X^{(\omega, \alpha)}) \cdot \{ X^{(\omega, \alpha)}, u^{-1} \}. \qedhere\]
\end{proof}

\begin{example} \label{exm:chi-linear}
Set $R = A[X^{\pm 1}]$.
Consider the following coweights:
\[\varepsilon_1 = \varpi_1^\vee,\ \varepsilon_2 = \varpi_2^\vee - \varpi_1^\vee,\ \ldots,\ \varepsilon_{\ell+1} = -\varpi^\vee_\ell.\]
For $1\leq k\leq \ell+1$ and $u \in R^\times$ denote by $d_k(u)$ the matrix from $\GL(\ell+1, R)$ which differs from the unit matrix only in that it has the element $u$ on the $k$-th position of its diagonal.
Recall from~\cite[Corollary~4]{Ka77} that for any $g \in \GL(\ell+1, R)$ there exists an automorphism $\beta_g$ of $\St(\ell+1, R)$ ''modeling`` the automorphism $\alpha_g \colon \GL(\ell+1, R) \to \GL(\ell+1, R)$ of inner conjugation by $g$, i.\,e. such that $\phi \beta_g = \alpha_g \phi$.


It is clear that in the linear case the map $\chi_{\varepsilon_k, u}$ coincides with $\beta_{d_k(u)}$,
while for other Chevalley groups the maps $\chi_{\omega, u}$ model automorphisms of inner conjugation by weight elements $h_\omega(u)$ in the sense of~\cite[\S~4]{Vav09}.
\end{example}


Let $\omega \in P(\Phi^\vee)$ be a coweight of $\Phi$.
Consider the subset $\mathcal{X}_\omega$ of the generating set of $\St(\Phi, A[X])$ consisting of those generators $x_{\alpha}(a) \in \St(\Phi, A[X])$ for which
$(\alpha, \omega) < 0$ implies that $a \in A[X]$ is divisible by $X^{-(\alpha, \omega)}$.
\begin{rem}
    Notice that according to this condition if $(\alpha, \omega) \geq 0$ then $\mathcal{X}$ contains $x_\alpha(a)$ for all $a \in A[X]$.
\end{rem}
Denote by $N_\omega$ the subgroup of $\St(\Phi, A[X])$ generated by $\mathcal{X}_\omega$.


\begin{dfn} \label{dfn:delta-pair}
Let $\omega \in P(\Phi^\vee)$ be a coweight.
By definition, an {\it $\omega$-pair} is a pair of mutually inverse group homomorphisms
$\xymatrix{ \sigma(\omega)\colon N_\omega \ar[r] & \ar@<-1.0ex>[l] N_{-\omega}\colon \sigma(-\omega) }$ satisfying the following identity:
\begin{equation} \label{eq:sigmadef}
\sigma(\pm \omega)(x_\alpha(\xi)) = x_\alpha(X^{(\pm \omega, \alpha)}\cdot \xi),
\text{ for all } x_\alpha(\xi) \in \mathcal{X}_{\pm\omega}.
\end{equation}\end{dfn}
It is clear that the maps $\sigma(\omega)$, $\sigma(-\omega)$ are uniquely determined by~\eqref{eq:sigmadef}, so at most one $\omega$-pair may exist for any given $\omega$.
Moreover, $\sigma(\omega), \sigma(-\omega)$ always make the following diagram commute:
\begin{equation} \label{eq:sigma-diagram}
\xymatrix{ N_\omega \ar[r]_{\sigma(\omega)}\ar@{^{(}->}[d] \ar@/^1.5pc/[rr]^{\mathrm{id}} & N_{-\omega} \ar@{^{(}->}[d] \ar[r]_{\sigma(-\omega)} & N_\omega \ar@{^{(}->}[d] \\
\St(\Phi, A[X]) \ar[d] & \St(\Phi, A[X]) \ar[d] & \St(\Phi, A[X]) \ar[d] \\
\St(\Phi, A[X^{\pm 1}]) \ar@<-0.0ex>[r]_{\chi_{\omega, X}} \ar@/_1.5pc/[rr]^{\mathrm{id}} & \St(\Phi, A[X^{\pm 1}]) \ar@<-0.0ex>[r]_{\chi_{-\omega, X}} & \St(\Phi, A[X^{\pm 1}]).} \end{equation}
The question of existence of an $\omega$-pair is rather complicated and will be addressed separately.
One of the necessary technical ingredients needed for this is the technique of ``another presentation'',
  which we recall in detail in the following subsection.

\subsection{Another presentation of the relative linear Steinberg group}
The aim of this subsection is twofold.
First of all, we recall the presentation of the relative linear Steinberg group $\St(\rA_{n-1}, R, I) = \St(n, R, I)$ from~\cite{LS17}.
This presentation is formulated in terms of rows and columns rather than root unipotents and is similar to
 M.~Tulenbaev's presentation of the relative linear Steinberg group from~\cite[Definition~1.5]{Tu83}.
The key advantage of this presentation over Tulenbaev's is that unlike the latter it applies to the Steinberg group $\St(4, R, I)$,
 which will be important in the sequel.

Our second main goal is to define certain elements $X^d(u, v)$, $Y^d(v, u)$ which will help us
 with the construction of morphisms $\sigma(\omega)$ from~\cref{dfn:delta-pair}.
These elements are inspired by Tulenbaev's elements $X_{v, Xw}^{\delta_k}$ defined on p.~145 of~\cite{Tu83} with
 the important difference that our definitions are also adapted to the situation $n=4$.

For a column $u \in R^n$ we denote by $u^t$ its transpose (which is a row of length $n$).
We denote by $\Um(n, R) \subseteq R^n$ the subset consisting of unimodular columns,
i.\,e. columns $u \in R^n$ such that $v^t u = 1$ for some $v \in R^n$.
We denote by $e_1, \ldots e_n$ the standard basis of $R^n$.
We also denote by $e$ the identity matrix of size $n$.
If $u, v \in R^n$ is a pair of orthogonal columns then $t(u, v) = e + uv^t$ is an invertible matrix of size $n.$
For an element $g \in \GL(n, R)$ we denote by $g^*$ the transpose-inverse of $g$, i.\,e. $g^* = {g^{t}}^{-1}.$

\begin{prop}
    \label{prop:rel-presentation}
    Assume that $I$ is a splitting ideal of a commutative ring $R$.
    Then for any $\ell\geq 3$ the group $\St(\rA_\ell,\,R,\,I)$ can be presented by means of two families of generators $F(u,\,v)$, $S(v,\,u)$
    (where $u\in \E(n,\,R)e_1,$ and $v\in I^n$ are such that $u^{t}v=0$) subject to the following relations:
    \begin{align}
        &F(u,\,v)F(u,\,w)=F(u,\,v+w), \label{add4}\\
        &S(u,\,v)S(w,\,v)=S(u+w,\,v), \label{add5}\\
        &F(u,\,v)F(u',\,v')F(u,\,v)^{-1}=F(t(u,\,v)u',\,t(v,\,u)^{-1} v'), \label{conj3} \\
        &F(me_1,\,m^{*}e_{2}a)=S(me_{1}a,\,m^{*}e_{2}),\ \text{for all $a\in I$,}\, m \in \E(n, R). \label{coef-move}
    \end{align}
\end{prop}
\begin{proof}
    This is precisely~\cite[Proposition 3.10]{LS17} combined with~\cite[Proposition~8]{S15}.
\end{proof}

In what follows we assume $n \geq 4$.
We denote by $D(u)$ the subset of $R^n$ consisting of all columns $v$ which are orthogonal to $u$
(i.\,e. $u^{t} v = 0$) and have at least two zero entries.
Recall from 3.2 of~\cite{Ka77} that for every $u, v, w \in R^n$ such that $u^t v = 0$ there
is a decomposition of $(w^t u) \cdot v$ into a sum of elements of $D(u)$, called \textit{canonical decomposition}:
\begin{equation}
    \label{eq:canonical} (w^tu) \cdot v=\sum_{i<j}u_{ij} c_{ij}(v, w),
\end{equation}
where $u_{ij} =e_i u_j-e_j u_i \in D(u)$ and $c_{ij}(v, w) =v_i w_j-v_j w_i \in R$.

\begin{lemma}
    \label{lem:xsmall-properties}
    Let $v, w \in R^n$ be such that $v^t w = 0$ and assume, moreover, that either $v$ or $w$ has at least one zero entry.
    Under these assumptions one can define certain element $x(v, w) \in \St(n, R)$ such that $\pi(x(v, w)) = t(v, w) = e + vw^t$.
    The elements $x(v, w)$ enjoy the following properties:
    \begin{lemlist}
        \item \label{itm:xsmall-scalar} If $v$ or $w$ has at least two zero entries, then $x(v, wa) = x(va, w)$ for $a\in R$.
        \item \label{itm:xsmall-additivity} If $w_1$ and $w_2$ have at least two zero entries of which at least one entry is common
        then $x(v, w_1) \cdot x(v, w_2) = x(v, w_1+w_2)$ and $x(w_1, v) \cdot x(w_2, v) = x(w_1 + w_2, v)$.
        \item \label{itm:xsmall-commute} If $v$, $v'$ are simultaneously orthogonal to $w$ and $w'$, and the elements $w, w'$ both have at least two zero entries then
        $[x(v, w),\ x(v', w')] = 1$.
        \item \label{itm:xsmall-conj} If $g = x_{ij}(\xi)$ is a generator of the Steinberg group and $v$ or $w$ has at least two zero entries then
        $g \cdot x(v, w) \cdot g^{-1} = x(gv, g^*w)$.
    \end{lemlist}
\end{lemma}
\begin{proof}
    See~\cite[Lemma~1.1]{Tu83}.
\end{proof}

Now assume that $I$ is an ideal of a ring $R$.
We also assume that $R$ itself is a subring of a ring $S$.
We now define two families of elements of $\St(n, R)$.
\begin{dfn} \label{dfn:xy-def}
    Let $u \in \Um(n, R)$ and $v \in I^n$ be a vector such that $u^{t}v = 0$.
    Let $v = \sum_r v_r$ be some decomposition of $v$ into a finite sum of elements $v_r \in D(u) \cap I^n$.
    Under our assumptions on $u$ and $v$ such decomposition always exists, see~\eqref{eq:canonical}.

    Let $d = \mathrm{diag}(d_1, \ldots, d_n)$ be an element of the subgroup $T(n, S)$ of diagonal matrices of $\GL(n, S)$ such that
    $d^{-1}u \in R^n,\ d \cdot I^n \subseteq R^n.$
    Under these assumptions we set
    \begin{equation*}
        X^d(u, v) \coloneqq \prod_i x(d^{-1}u, dv_i),
    \end{equation*}

    Not let $d'$ be an element of $T(n, S)$ such that $d' u\in R^n,\ {d'}^{-1} \cdot I^n \subseteq R^n$.
    Under these assumptions we set
    \begin{equation*}
        Y^{d'}(v, u) \coloneqq \prod_i x({d'}^{-1} v_i, {d'}u).
    \end{equation*}
\end{dfn}

\begin{lemma}
    \label{lem:xy-wd}
    The elements $X^d(u, v)$, $Y^{d'}(u, v)$ are well-defined, i.\,e. they do not depend on the choice of decomposition for $v$.
\end{lemma}
\begin{proof}
    Let $v = \sum_r v^r$ be a decomposition as above.
    Since $u$ is unimodular there exists $w$ such that $w^t u = 1$.
    Since each $v^r$ is orthogonal to $u$ we can write the canonical decomposition
    $v^r = (w^tu)v^r = \sum_{i<j} u_{ij} c_{ij}(v^r, w)$, moreover $\sum_{r} c_{ij}(v^r, w) = c_{ij}(v, w)$.
    Now using~\cref{lem:xsmall-properties} we obtain that
    \begin{multline*}
        \prod\limits_r x(d^{-1}u, dv^r) = \prod\limits_{r}\prod\limits_{i<j} x(d^{-1} u, du_{ij}c_{ij}(v^r, w)) =
        \prod\limits_{i<j} x(d^{-1} u, d u_{ij}c_{ij}(v, w)). \qedhere
    \end{multline*}
\end{proof}

\begin{lemma}
    \label{lem:xy-conj} Suppose that $g = x_{hk}(a)$ is a generator of $\St(n, R)$ such that $m = d\pi(g)d^{-1} \in \E(n, R)$, then
    \begin{equation*}
        g \cdot X^d(u, v) \cdot g^{-1} = X^d(mu, m^*v) \text{ and } g \cdot Y^d(v, u) \cdot g^{-1} = Y^d(mv, m^*u).
    \end{equation*}
\end{lemma}
\begin{proof}
 Choose $w\in R^n$ such that $w^t u = 1$ and write $v = \sum_{1\leq i<j\leq n} u_{ij} c_{ij}$, where $u_{ij}$ and $c_{ij} = c_{ij}(v, w)$ are as in~\eqref{eq:canonical}.
 By~\cref{lem:xsmall-properties} we can write $g' = g \cdot X^d(u, v) \cdot g^{-1}$ as follows:
 \begin{equation} \nonumber
   \prod\limits_{1\leq i<j\leq n} x(\pi(g) \cdot d^{-1} \cdot u, \pi(g)^* \cdot d \cdot u_{ij}c_{ij})
 = \prod\limits_{1\leq i<j\leq n} x( d^{-1} \cdot m \cdot u, d \cdot m^* \cdot u_{ij}c_{ij}).
 \end{equation}
 Now for every factor in the right-hand side we do the following:
 if $\{i, j\} = \{h, k\}$ or $\{i, j\} \cap \{h, k\} = \emptyset$ we leave the factor unchanged,
 otherwise, if, $|\{i, j\} \cap \{h, k\}| = 1$, say, $j = h$, $i\neq k$, we further decompose it as follows:
  \begin{equation} \nonumber
     x(d^{-1} \cdot u', d \cdot m^* \cdot u_{ij} c_{ij}) =
     x(d^{-1} \cdot u', d \cdot u'_{ij} c_{ij}) \cdot
     x(d^{-1} \cdot u', d \cdot u'_{ki} bc_{ij}).
  \end{equation}
  To check the last equality observe that $u' = m  u = t_{jk}(b) \cdot u$ for $b = d_j \cdot a \cdot d_k^{-1} \in R$.
  Notice that $m^* \cdot u_{ij} = t_{kj}(-b) \cdot u_{ij} = e_iu_j - e_ju_i + e_k b u_i =u'_{ij}+u'_{ki}b$.

  Thus, $g' = \prod_{s} x(d^{-1} \cdot u', d \cdot v'_s) $, where $v'_s \in D(u')$ and $\sum v'_s = m^*v$.
  It is clear now that $g' = X^d(m u, m^* v)$ by~\cref{dfn:xy-def}.

  The argument for the generator $Y^d(v, u)$ is similar.
\end{proof}
