In this paper all rings are assumed to be commutative and all commutators are left-normed i.\,e. $[a, b] = a b a^{-1} b^{-1}$.

\subsection{Formalism of triples}\label{subsec:triples}

Let $N$ be a group acting on itself by right conjugation.
Let $M$ be a group with a right action of $N$.
Recall that a group homomorphism $\varphi\colon M \to N$ is called a \textit{precrossed module} if $\varphi$ preserves the action of $N$, i.\,e.
\[\varphi(m^n) = \varphi(m)^n, \text{for all $m \in M$, $n\in N;$} \]
If, in addition, $\varphi$ satisfies the so-called \textit{Peiffer identity}, i.\,e.
\[{m}^{\varphi(m')} = {m'}^{-1} m m', \text{for all $m, m' \in M$,}\]
then $\varphi$ is called a \textit{crossed module}.

Let $\varphi\colon M \to N$ and $\varphi' \colon M' \to N'$ be a pair of precrossed modules.
A \textit{map of precrossed modules} $(f, g)\colon \varphi \to \varphi'$ is, by definition, any pair of homomorphisms
$f \colon M \to M'$, $g \colon N \to N'$ such that ${f(m)}^{g(n)} = f(m^n)$ for all $n \in N$, $m \in M$.

Now suppose that we are given the following cube-like commutative diagram of abstract groups:
\begin{equation} \label{eq:cube} \begin{split} \xymatrix{
    G_{123} \ar[rr]_{f_{23}} \ar[dd]_{f_{12}} \ar[rd]_{f_{13}} &                        & G_{23} \ar@{-->}[dd]^(.3){g_2^3} \ar[rd]^{g^2_3} &           \\
    & G_{13} \ar[rr]^(.3){g^1_3} \ar^(.3){g^3_1}[dd] &                   & G_3 \ar[dd]_{h_3} \\
    G_{12} \ar@{-->}[rr]^(.3){g_2^1} \ar[rd]_{g_1^2}          &                        & G_2 \ar@{-->}[rd]_{h_2}         &           \\
    & G_1 \ar[rr]_{h_1}              &                   & G.} \end{split} \end{equation}
Additionally, suppose that $g_1^3$ is a precrossed module, $h_3$ is a crossed module and, moreover, that $(g_3^1, h_1) \colon g_1^3 \to h_3$ is a map of precrossed modules.

Set $V = G_1 \times G_2 \times G_3$, $W = G_{12} \times G_{13} \times G_{23}$.
We define an operation $\star \colon W \times V \to V$ as follows. For $v = (x, y, z) \in V$ and $(a, b, c) \in W$ we set
\[(a, b, c) \star (x, y, z) = (x \cdot g_1^3(b) \cdot g_1^2(a),\ g_2^1(a)^{-1} \cdot y \cdot g_2^3(c),\ g_3^2(c)^{-1} \cdot g_3^1(b)^{-h_2(y)} \cdot z).\]

\begin{lemma} For every $v \in V$ one has
\begin{equation*}(a', b', c') \star \left( (a, b, c) \star v \right) = (a \cdot a', b \cdot {b'}^{g_1^2(a)^{-1}}, c \cdot c') \star v.\end{equation*}
\end{lemma}
\begin{proof}
    Set $v'=(x', y', z') = (a, b, c) \star v$ and $(x'', y'', z'') = (a', b', c') \star v'$.
    Since $g_1^3$ is a precrossed module we immediately obtain that
    \begin{align*}
        x'' =& x' \cdot g_1^3(b') \cdot g_1^2(a') = x \cdot g_1^3(b) \cdot g_1^2(a) \cdot g_1^3(b') \cdot g_1^2(a') = x \cdot g_1^3(b \cdot b'^{g_1^2(a)^{-1}}) g_1^2(a \cdot a'),\\
        y'' =& g_2^1(a')^{-1} \cdot g_2^1(a)^{-1} \cdot y \cdot g_2^3(c) \cdot g_2^3(c') = g_2^1(a\cdot a')^{-1} \cdot y \cdot g_2^{3}(c\cdot c'). \end{align*}
    Since $(g_3^1, h_1)$ is a map of precrossed modules, for every $a \in G_{12}$, $b \in G_{13}$ one has $g_3^1(b^{g_1^2(a)}) = g_3^1(b)^{h_1 g_1^2(a)} = g_3^1(b)^{h_2g_2^1(a)}$.
    Since $h_3$ satisfies Peiffer identity, for every $c \in G_{23}$ and $z \in G_3$ one has $z ^{h_2 g_2^3(c)} = z^{ h_3 g_3^2(c)} = g_3^2(c)^{-1} \cdot z \cdot g_3^2(c)$.
    Using these identities we obtain that
    \begin{multline*}
        z'' = g_3^2(c')^{-1} \cdot g_3^1(b')^{-h_2(y')} \cdot z' = \\
        = g_3^2(c')^{-1} \cdot g_3^1(b')^{- h_2 \left( g_2^1(a)^{-1} \cdot y \cdot g_2^3(c) \right)} g_3^2(c)^{-1} \cdot g_3^1(b)^{-h_2(y)} \cdot z = \\
        = g_3^2(c \cdot c')^{-1} \cdot g_3^1(b')^{- h_2 \left( g_2^1(a)^{-1} \cdot y \right)} \cdot g_3^1(b)^{-h_2(y)} \cdot z = \\
        = g_3^2(c \cdot c')^{-1} \cdot g_3^1(b \cdot {b'} ^ {g_1^2(a)^{-1}})^{- h_2 \left( y \right)} \cdot z. \qedhere
    \end{multline*}
\end{proof}

The above lemma allows us to define an equivalence relation on the set $V$ in the following fashion.
We declare two elements $v, v' \in V$ congruent (denoted $v \sim v'$) if $v' = w \star v$ for some triple $(a, b, c) \in W$.

\begin{lemma}\label{one-one-z} Assume that the back face $(f_{12}, f_{23}, g_2^1, g_2^3)$ of~\eqref{eq:cube} is a pullback square.
Assume additionally that the triple $(1, 1, 1)$ is congruent to $(1, 1, z)$ for some $z\in G_3$.
Then $z \in g_3^1(\Ker(g_1^3)).$ \end{lemma}
\begin{proof} By the definition of congruence relation there exists $(a, b, c)\in W$ such that
\[ (a, b, c) \star (1, 1, 1) = ( g_1^3(b) \cdot g_1^2(a),\ g_2^1(a)^{-1} \cdot g_2^3(c),\ g_3^2(c)^{-1} \cdot g_3^1(b)^{-1}) = (1,1,z). \]
By the Lemma's assumption there exists $e \in G_{123}$ such that $f_{12}(e) = a$, $f_{23}(e) = c$, hence
\[ 1 = g_1^3(b) \cdot g_1^2(f_{12}(e)) = g_1^3(b \cdot f_{13}(e)),\ z = g_3^2(f_{23}(e))^{-1} \cdot g_3^1(b)^{-1} = g_3^1(b \cdot f_{13}(e))^{-1}. \qedhere\] \end{proof}

\subsection{Steinberg groups, $\K_2$-groups and symbols}\label{subsec:steinberg-preliminaries}
Let $\Phi$ be a root system of rank $\ell \geq 1$.
We assume that $\Phi$ is embedded into $\mathbb{R}^\ell$ whose scalar product we denote by $(\text{-}, \text{-})$.
We also fix some system of simple roots $\Pi = \{\alpha_1, \ldots, \alpha_\ell\} \subset \Phi$.
For a root $\alpha\in\Phi$ we denote by $m_i(\alpha)$ the $i$-th coefficient in the expansion of $\alpha$ in $\Pi$,
i.\,e. $\alpha = \sum_{i=1}^n m_i(\alpha) \alpha_i$.

We denote by $\Phi^\vee$ the corresponding dual root system, which, by definition, consists of all coroots $\alpha^\vee = \frac{2}{(\alpha, \alpha)} \alpha$, where $\alpha \in \Phi$.
We denote by $P(\Phi^\vee)$ the integral lattice spanned by the \emph{fundamental coweights $\varpi_i^\vee$}.
Recall that the fundamental coweights $\varpi_i^\vee$ are uniquely determined by the property $(\varpi_i^\vee, \alpha_j) = \delta_{ij}$.
For $\alpha,\beta \in \Phi$ we denote by $\langle \alpha, \beta \rangle$ the integer $(\alpha, \beta^\vee) = \frac{2(\alpha, \beta)}{(\beta, \beta)}$.

Now let $R$ be an arbitrary commutative ring with $1$ and suppose that $\Phi$ is an irreducible root system of rank $\geq 2$.
Recall that to the pair $(\Phi, R)$ can associate an abstract group $\St(\Phi, R)$, called the \textit{Steinberg group} of type $\Phi$ over $R$.
By definition, $\St(\Phi, R)$ is the group presented by generators $x_\alpha(a)$, $a \in R$, $\alpha \in \Phi$ and an explicit list of relations (see e.\,g in.~\cite{Ma69, Re75, St71}).
In this paper we may restrict ourselves to the case when the root system $\Phi$ in question is \textit{simply-laced} (i.\,e. has type $\mathsf{ADE}$),
 in which case the defining relations of $\St(\Phi, R)$ reduce to the following shorter list:
\begin{align}
x_{\alpha}(a)\cdot x_{\alpha}(b)&=x_{\alpha}(a+b), \tag{R1} \label{x-additivity}\\
[x_{\alpha}(a),\,x_{\beta}(b)]  &=x_{\alpha+\beta}(N_{\alpha,\beta} \cdot ab),\text{ for }\alpha+\beta\in\Phi, \tag{R2} \label{R2} \\
[x_{\alpha}(a),\,x_{\beta}(b)]  &=1,\text{ for }\alpha+\beta\not\in\Phi\cup0. \tag{R3} \label{R3}
\end{align}
The coefficients $N_{\alpha,\beta}$ in the above formula are integers equal to $\pm 1$, they coincide with the structure constants of the complex Lie algebra of type $\Phi$.

Throughout this paper we denote by $\Gsc(\Phi, R)$ the group of points of the simply-connected Chevalley--Demazure group of type $\Phi$ over $R$.
We denote by $\Esc(\Phi, R)$ the \textit{elementary subgroup} of $\Gsc(\Phi, R)$, i.\,e. the subgroup generated by elementary root unipotents of $\Gsc(\Phi, R)$.
Notice that in~\cite{VP, Vav09} the notation $x_\alpha(a)$ is used to denote the elementary root unipotents.
To prevent confusion we will use different notation $t_\alpha(a)$ for them and reserve the notation $x_\alpha(a)$ solely for generators of Steinberg groups.

Recall that the map sending $x_\alpha(a)$ to $t_\alpha(a)$ gives rise to a well-defined homomorphism $\pi \colon \St(\Phi, R) \to \G_\mathrm{sc}(\Phi, R)$, see~\cite[\S~1A]{St78}.
The cokernel and the kernel of $\pi$ are called \textit{the unstable $\K_1$- and $\K_2$-functors modeled on the root system $\Phi$}:
\begin{equation} \label{eq:K1-K2-sequence}
  \xymatrix{ 1 \ar[r] & \K_2(\Phi, R) \ar[r] & \St(\Phi, R) \ar[r]^{\pi} & \Gsc(\Phi, R) \ar[r] & \K_1(\Phi, R) \ar[r] & 1}
\end{equation}

Following~\cite{Ma69} for $\alpha\in\Phi$ and $u \in R^\times$ we define the following elements of $\St(\Phi, R)$:
\begin{align*} w_\alpha(u) & =  x_\alpha(u) \cdot x_{-\alpha}(-u^{-1}) \cdot x_\alpha(u), \\
               h_\alpha(u) & =  w_\alpha(u) \cdot w_\alpha(-1).  \end{align*}
The subgroup generated by $w_\alpha(u)$ (resp. $h_\alpha(u)$) for all $\alpha\in \Phi$, $u \in R^\times$ is denoted by $\StW(\Phi, R)$ (resp. $\StH(\Phi, R)$).
By~\cite[Lemme~5.2]{Ma69} $\StH(\Phi, R)$ is a normal subgroup of $\StW(\Phi, R)$.

In the sequel we will need the following explicit elements of the group $\K_2(\Phi, R)$.
Recall that for arbitrary $u, v \in R^\times$ one defines the \textit{Steinberg symbol} via the formula
\begin{equation} \label{eq:steinberg} \{ u, v \}_\alpha = h_\alpha(uv) \cdot h_\alpha^{-1}(u) \cdot h_\alpha^{-1}(v). \end{equation}
Recall also from~\cite[Lemme~5.4]{Ma69} that
\begin{equation} \label{eq:steinberg-2} [h_\alpha(u), h_\beta(v)] = \{u, v^{\langle \alpha, \beta \rangle}\}_\alpha. \end{equation}
Steinberg symbols depend only on the length of the root $\alpha$.
In particular, in the case when $\Phi$ is of simply-laced type, they do not depend on the choice of $\alpha$, which allows us to omit it from notation.

Steinberg symbols are central elements of $\St(\Phi, R)$.
Our assumptions on $\Phi$ guarantee that Steinberg symbols are antisymmetric and bimultiplicative, i.\,e. they satisfy the following identities:
\begin{equation} \label{eq:symbol-properties} \{ u, st \} = \{ u, s\} \{ u, t \}, \ \{ u, v \} = \{ v, u\}^{-1}. \end{equation}

In this paper we will use also use the concept of a \textit{relative Steinberg group} introduced by F.~Keune and J.-L.~Loday in~\cite{Ke78, Lo78}.
We will only briefly mention the definition and basic properties of these groups and refer the reader to~\cite[\S~2.3]{LS20} for a more detailed exposition.

Let $R$ be a commutative ring, $I \trianglelefteq R$ be an ideal and let $p$ denote the canonical projection $R \to R/I$.
Denote by $D_{R, I}$ the pullback of two copies of $p$ i.\,e. the ring $R \times_{R/I} R$.
Elements of $D_{R, I}$ are pairs $(a; b)$ such that $a-b \in I$.
We also denote by $p_1$, $p_2$ the canonical projections $D_{R, I} \to R$ and by $p_1^*$, $p_2^*$ the corresponding homomorphisms of Steinberg groups induced by them.
Recall from~\cite[Definition~2.5]{LS20} that the relative Steinberg group $\St(\Phi, R, I)$ is defined as the quotient
 $\Ker(p_1^*) / C$, where $C = [\Ker(p_1^*), \Ker(p_2^*)]$.
If we denote by $\mu$ the homomorphism $\St(\Phi, R, I) \to \St(\Phi, R)$ induced by $p_2^*$, we obtain an exact sequence
\begin{equation}
    \xymatrix{1 \ar[r] & C(\Phi, R, I) \ar[r] & \St(\Phi, R, I) \ar[r]^\mu & \St(\Phi, R) \ar[r]^-{p^*} & \St(\Phi, R/I) \ar[r] & 1. }\label{eq:relative-Steinberg}
\end{equation}
Alternatively, the group $\St(\Phi, R, I)$ can be defined via generators and relations as an $\St(\Phi, R)$-group, cf.~\cite[Proposition~6]{S15}
 or even as an abstract group, see~\cite{V22}.
The relative group $\K_2(\Phi, R, I)$ is defined as the kernel of the homomorphism $\pi \mu$.

We will need a relative analogue of Steinberg symbol.
Let $A$ be a local unital ring with maximal ideal $M$ embedded as a subring into a larger unital ring $R$.
Under this assumption the subset $1+M \subseteq A$ forms a group under multiplication.
It is clear that $(1+M)^\times$ is isomorphic to the abelian group $(M, \circ)$ with the operation given by $m \circ m' = m + m' + mm'$.
Now for $a \in R^\times$ and $m \in M$ we denote by $\{a, 1+m\}_r$ the coset $\{(a; a), (1; 1+m)\}C \in \St(\Phi, R, RM)$.
It is clear that the map $1+m \mapsto \{a, 1+m\}_r$ specifies a group homomorphism
\[\{ a, -\}_r \colon (1+M)^\times \to \K_2(\Phi, R, RM). \]
It is also clear that $\mu(\{a, 1+m\}_r) = \{a, 1+m\}$.

\begin{lemma}\label{lem:symbols}
Assume that $A$ is a local domain with maximal ideal $M$.
Denote by $R$ the ring $A[X, X\inv]$.
Then the intersection of the image of the relative symbol map $\{X, -\}_r$ with $C(\Phi, R, M[X, X\inv])$ is trivial.
\end{lemma}
\begin{proof}
Set $F = \mathrm{Frac}(A)$.
Consider the following diagram:
\[\begin{tikzcd}
 (1+M)^\times \ar[hookrightarrow, rr] \ar[d] &  & F^\times \ar[hookrightarrow, d] \\
  \K_2(\Phi ,R, M[X^{\pm 1}]) \ar[r] & \K_2(\Phi, R) \ar[r] & \K_2(\Phi, F[X^{\pm 1}]).
\end{tikzcd}\]
Since the right vertical arrow is injective by~\cite[Lemma~2.2]{LS20}, so is the left arrow.
\end{proof}

\subsection{Weight elements}\label{subsec:weight-elements}
Recall that for every coweight $\omega \in P(\Phi^\vee)$ and $\beta \in \ZZ \Phi$ the scalar product $(\omega, \beta)$ is an integer.
Thus, for $u \in R^\times$ and $\omega \in P(\Phi^\vee)$ we can define a map of abelian groups $\chi \in \Hom(\ZZ \Phi, R^\times)$
 via the formula $\chi(\beta) = u ^ {(\omega, \beta)}$.
Such a map specifies an action on the set of generators of the Steinberg group $\St(\Phi, R)$ via the identity
\begin{equation*} \chi \cdot x_\alpha(\xi) = x_\alpha(\chi(\alpha) \cdot \xi),\ \alpha\in \Phi,\ \xi \in R. \end{equation*}
It is not hard to check that this action is compatible with Steinberg relations and
 hence gives a well-defined automorphism of $\St(\Phi, R)$ which we denote by the same symbol $\chi$.

The following lemma is the analogue of~\cite[Lemma~3.1(c)]{Tu83}.
\begin{lemma} \label{lem:winv-chiw}
For any root system $\Phi$ if $w \in \StW(\Phi, A[X^{\pm 1}])$ then $w^{-1} \cdot \chi_{\omega, X}(w)$ belongs to $\StH(\Phi, A[X^{\pm 1}])$.
\end{lemma}
\begin{proof}
Since $\StH(\Phi, A[X^{\pm 1}])$ is a normal subgroup of $\StW(\Phi, A[X^{\pm 1}])$, it suffices to verify the assertion for $w = w_\alpha(u)$.
Set $h = w^{-1} \cdot \chi_{\omega, X}(w)$.
It is not hard to check that
\begin{multline} w_\alpha(1) \cdot h\cdot  w_\alpha(-1) = w_\alpha(-1)^{-1} \cdot w_\alpha(u)^{-1} \cdot w_{\alpha}(X^{(\omega, \alpha)} u) \cdot w_\alpha(-1) = \\
= h_\alpha(u)^{-1} \cdot h_\alpha(X^{(\omega, \alpha)}u) \in \StH(\Phi, A[X^{\pm 1}]).\end{multline}
Thus, we get that \[h = h_{\alpha}(u^{-1})^{-1} \cdot h_{\alpha}(X^{-(\omega, \alpha)}u^{-1}) = h_{\alpha}(X^{-(\omega, \alpha)}) \cdot c,\ \text{for some $c \in \K_2(\Phi, A)$.} \qedhere\]
\end{proof}


In the sequel we will also need the following identity:
\begin{equation} \label{eq:chi-h} \chi_{\omega, X} (h_\alpha(u)) = h_\alpha(X^{(\omega, \alpha)} u) \cdot h_\alpha(X^{(\omega, \alpha)})^{-1} = \{X^{(\omega, \alpha)}, u\} \cdot h_\alpha(u) \end{equation}


\begin{lemma} \label{lem:relative-chi}
 For every coweight $\omega$ there is a well-defined automorphism $\widetilde{\chi}(\omega, X)$ of the relative group $\St(\Phi, A[X^{\pm 1}], M[X^{\pm 1}])$ making the following diagram commute:
 \[\begin{tikzcd} \St(\Phi, A[X^{\pm 1}], M[X^{\pm 1}]) \ar[r, "\widetilde{\chi}{(\omega, X)}"] \ar[d] & \St(\Phi, A[X^{\pm 1}], M[X^{\pm 1}]) \ar[d] \\ \St(\Phi, A[X^{\pm 1}]) \ar[r, "\chi{(\omega, X)}"] & \St(\Phi, A[X^{\pm 1}]). \end{tikzcd}\]
\end{lemma}
\begin{proof}
 Recall that $\St(\Phi, R, I)$ is defined as $\Ker(p_1^*)/[\Ker(p_1^*), \Ker(p_2^*)]$ where
 \[\begin{tikzcd} p_1, p_2 \colon R \times_{R/I} R \arrow[r, shift right=2] \arrow[r] & R \colon \Delta. \arrow[l, shift right=2] \end{tikzcd}\]
 Substitute $R = A[X^{\pm 1}]$, $I = M[X^{\pm 1}]$ and apply $\St(\Phi, -)$ to this diagram.
 Homomorphisms $\chi(\omega, (X, X))$, $\chi_{\omega, X}$ define its automorphism.
\end{proof}

\begin{example} \label{exm:chi-linear}
Set $R = A[X^{\pm 1}]$. Consider the following coweights: \[\varepsilon_1 = \varpi_1^\vee,\ \varepsilon_2 = \varpi_2^\vee - \varpi_1^\vee,\ \ldots,\ \varepsilon_{\ell+1} = -\varpi^\vee_\ell.\] For $1\leq k\leq \ell+1$ and $u \in R^\times$ denote by $d_k(u)$ the matrix from $\GL(\ell+1, R)$ which differs from the unit matrix only in that it has the element $u$ on the $k$-th place of its diagonal.
Recall from~\cite[Corollary~4]{Ka77} that for any $g \in \GL(\ell+1, R)$ there exists an automorphism $\beta_g$ of $\St(\ell+1, R)$ ''modeling`` the automorphism $\alpha_g \colon \GL(\ell+1, R) \to \GL(\ell+1, R)$ of inner conjugation by $g$, i.\,e. such that $\phi \beta_g = \alpha_g \phi$.


It is clear that in the linear case the map $\chi(\varepsilon_k, u)$ coincides with $\beta_{d_k(u)}$,
 while for other Chevalley groups the maps $\chi(\omega, u)$ model automorphisms of inner conjugation by weight elements $h_\omega(u)$ in the sense of~\cite[\S~4]{Vav09}.
\end{example}


Let $\omega \in P(\Phi^\vee)$ be a coweight of $\Phi$ as above.
Denote by $\XX(\omega)$ the subset of $\XX_{\Phi, A[X]}$ consisting of those generators $x_{\alpha}(\xi)$ of $\St(\Phi, A[X])$ for which
$(\alpha, \omega) < 0$ implies that $\xi \in A[X]$ is divisible by $X^{-(\alpha, \omega)}$.


Denote by $N(\omega)$ the subgroup of $\St(\Phi, A[X])$ generated by $\XX(\omega)$.


\begin{dfn} \label{dfn:delta-pair}
By definition, a {\it $\delta$-pair for $\omega$} is a pair of mutually inverse group homomorphisms
$\xymatrix{ \delta(\omega)\colon N(\omega) \ar[r] & \ar@<-1.0ex>[l] N(-\omega)\colon \delta(-\omega) }$ satisfying
\begin{equation} \label{eq:sigmadef}
\delta(\pm \omega)(x_\alpha(\xi)) = x_\alpha(X^{(\pm \omega, \alpha)}\cdot \xi),
 \text{ for all } x_\alpha(\xi) \in \XX(\pm\omega).
\end{equation}\end{dfn}
It is clear that the maps $\delta(\omega)$, $\delta(-\omega)$ are uniquely determined by~\eqref{eq:sigmadef}, so at most one $\delta$-pair may exist for any given $\omega$.
Moreover, it is automatically true that the homomorphisms $\delta(\omega), \delta(-\omega)$ make the following diagram commute:
\begin{equation} \label{eq:sigma-diagram}
\xymatrix{ N(\omega) \ar[r]^{\delta(\omega)}\ar@{^{(}->}[d] \ar@/^1.5pc/[rr]_{\mathrm{id}} & N(-\omega) \ar@{^{(}->}[d] \ar[r]^{\delta(-\omega)} & N(\omega) \ar@{^{(}->}[d] \\
          \St(\Phi, A[X]) \ar[d] & \St(\Phi, A[X]) \ar[d] & \St(\Phi, A[X]) \ar[d] \\
          \St(\Phi, A[X^{\pm 1}]) \ar@<-0.0ex>[r]_{\chi_{\omega, X}} \ar@/_1.5pc/[rr]^{\mathrm{id}} & \St(\Phi, A[X^{\pm 1}]) \ar@<-0.0ex>[r]_{\chi(-\omega, X)} & \St(\Phi, A[X^{\pm 1}]).} \end{equation}



\subsection{Horrocks ingredient}\label{subsec:horrocks-ingredient}
\begin{lemma} \label{lem:zariski-glueing}
Let $\Phi$ be any simply-laced root system of rank $\geq 3$.
Let $A$ be a commutative domain and $a, b \in A$ be a pair of coprime elements.
\begin{enumerate}
    \item Let $\delta$ be an element of $\St(\Phi, A_{ab})$.
    Then $\delta$ can be presented as $\lambda_b(\alpha) \cdot \lambda_a(\beta)$ for some
    $\alpha \in \St(\Phi, A_a)$ and \beta \in $\St(\Phi, A_b)$.
    \item  Let $\alpha \in \St(\Phi, A_a)$ and $\beta \in \St(\Phi, A_b)$ be such that the equality $\lambda_b(\alpha) = \lambda_a(\beta)$ holds in $\St(\Phi, A_{ab})$.
    Then there exists $\gamma \in \St(\Phi, A)$ such that $\alpha = \lambda_a(\gamma)$, $\beta = \lambda_b(\gamma)$.
\end{enumerate}

\end{lemma}
\begin{proof}
    This is a special case of Nisnevich excision for domains, see~\cite[Proposition~4.5]{LSV2}
    (cf. also the proof of~\cite[Lemma~2.6]{LSV2}).
\end{proof}

Now let $A$ be a local ring with maximal ideal $M$.
Set $B \coloneqq A[X\inv] + M[X],\ R \coloneqq A[X, X\inv]$.
Consider the following diagram:
%! suppress = EscapeAmpersand
\[ \xymatrix{ & \St(\Phi, A[X], M[X]) \ar[d]_{t_M} \ar[r] & \St(\Phi, A[X]) \ar[d]_{\lambda_X^*} \\
   \St(\Phi, A[X\inv]) \ar[r]^{i_B} \ar@/_/[rr]_{\lambda_{X\inv}^*} & \St(\Phi, B) \ar[r]^{\lambda^*_{X\inv, B}} & \St(\Phi, R)
}\]
The arrow $t_M$ arises from the \("\)lifting property\("\) of Steinberg groups (cf. \cite[Lemma~3.3]{LS20} or~\cite[Theorem~3]{LS17}) as the composition of morphisms
$\lambda_{X}^{rel}\colon \St(\Phi, A[X], M[X]) \to \St(\Phi, R, M[X, X\inv])$ and $T \colon \St(\Phi, R, M[X, X\inv]) \to \St(\Phi, B)$.
%This is wrong: Its image $\Img(t_M) \leq \St(\Phi, B)$ is generated as a subgroup by elements $x_\alpha(f)$, $z_\alpha(f, a)$, $z_\alpha(X^2f, aX^{-1})$, for $\alpha\in \Phi$, $a \in A$, $f \in M[X]$, see the proof of~\cite[Lemma~5.41]{LS20}.

Set $G_M^{\geq 0} \coloneqq \Img(\St(\Phi, A[X], M[X]) \to \St(\Phi, R)), G_M^0 \coloneqq \overline{\St}(\Phi, A, M)$.% Show that G_M^0 is a subgroup of both first two factors
Denote by $\overline{V}$ the quotient-set of the product $V \coloneqq G_M^{\geq 0} \times \St(\Phi, A[X\inv]) \times (1 + M)^\times$
 modulo the equivalence relation given by $(gh_0, h, u) \cong (g, h_0h, u)$ where $h_0 \in G_M^0, (g, h, u) \in V.$
Denote by $[g, h, u] \in \overline{V}$ the equivalence class corresponding to $(g, h, u)\in V$.

\begin{prop} \label{prop:horrocks-main} The group $\St(\Phi, B)$ acts on $\overline{V}$.
 This action satisfies the following properties:
    \begin{enumerate}
        \item For $(g, h, u) \in V$ and $g_0 \in \Img(t_M)$ one has
         \[g_0 \cdot [g, h, u] = [\lambda_{X\inv, B}^* (g_0) \cdot g, h, u]\]
        \item For $(1, h, u) \in V$ and $h_0 \in \St(\Phi, A[X\inv])$ one has
         \[ i_B(h_0) \cdot [1, h, u] = [1, h_0 \cdot h, u].\]
        \item If $g \in \Img(j\colon \St(\Psi, B) \to \St(\Phi, B)), h \in \St(\Phi, A[X\inv]), u \in 1 + M$ then
         \[ g \cdot [1, h, u] = [g', h', u']\] for some $g' \in \Img(j\colon G_{M, \Psi}^{\geq 0} \to G_{M, \Phi}^{\geq 0})$, $h' \in \St(\Phi, A[X\inv])$, $u'\in 1 + M$.
    \end{enumerate}
\end{prop}
\begin{proof}
    The existence of the action of $\St(\Phi, B)$ and the second assertion are contained in~\cite[Proposition~5.39]{LS20}.
    The first assertion is what the proof of~\cite[Lemma~5.41]{LS20} actually shows without the assumption that $i_R$ is injective.
    The last assertion follows from the construction of the action. %TODO: Add more details!
\end{proof}

\begin{lemma} \label{lem:horrocks--ingredient}
Let $A$ be a local ring, $\alpha \in \K_2(\Phi, A[X], XA[X])$, $\beta \in \St(\Phi, A[X\inv]),$ $\gamma \in \St(\Psi, A[X, X\inv])$ be elements satisfying the equality (in $\St(\Phi, R)$)
\begin{equation} \label{eq:alpha-def}\lambda_X(\alpha) = j(\gamma) \cdot \lambda_{X\inv}(\beta) \end{equation}
Then $\alpha$ belongs to the image of $j\colon \St(\Psi, A[X]) \to \St(\Phi, A[X])$.
\end{lemma}
\begin{proof}
    Denote by $M$ the maximal ideal of $A$ and by $k$ its residue field.
    We denote by $\pi_A$ (resp. $\pi_{A[X]}$, resp. $\pi_{R}$) the canonical homomorphism $A \to k$ (resp. $A[X] \to k[X]$, resp. $R \to k[X, X\inv]$).

    Since $\K_2(\Psi, k[X])$ surjects onto $\K_2(\Phi, k[X])$ there exists $\alpha_0 \in \St(\Psi, A[X])$ such that $j(\alpha_0) \cdot \alpha \in \overline{\St}(\Phi, A[X], M[X])$ and
     hence $\pi_{A[X]}^*(j(\alpha_0) \cdot \alpha) = 1$.
    By definition of relative Steinberg groups, there exists $\widetilde{\alpha} \in \St(\Phi, A[X], M[X])$ such that
     $\iota \cdot \lambda_X^{rel}(\widetilde{\alpha}) = \lambda_X^* (j(\alpha_0) \cdot \alpha)$. %TODO: More details

    Now consider the element $\pi^*_{A[X\inv]}(\beta) \in \St(\Phi, k[X\inv])$.
    Its image in $\G(\Phi, k[X\inv])$ is contained in the subgroup $\G(\Psi, k[X\inv])$.
    Since $\K_1(\Psi, k[X\inv]) = 1$, there exist
    \[\beta_0 \in \St(\Psi, A[X\inv]),\ \beta_1 \in \St(\Phi, A[X\inv], M[X\inv])\] such that $\beta = j(\beta_0) \cdot \beta_1$.
    Set $\gamma_1 \coloneqq \lambda^*_X(\alpha_0) \cdot \gamma \cdot \lambda^*_{X\inv}(\beta_0) \in \St(\Psi, R)$.
    It follows from~\eqref{eq:alpha-def} that the element $\pi_R^*(j(\gamma_1)) \in \St(\Phi, k[X, X\inv])$ is trivial, hence $\pi^*_R(\gamma_1) \in \K_2(\Psi, k[X, X\inv])$.
    Recall from~\cite{Hur77} that $\K_2(\Psi, k[X, X\inv]) \to \K_2(\Phi, k[X, X\inv])$ is injective (see the assertion after Korollar~6).
    Thus, $\gamma_1 \in \overline{\St}(\Psi, R, M[X, X\inv]).$
    There exists $\gamma_2 \in \St(\Psi, R, M[X, X\inv])$ such that $\iota (\gamma_2) = \gamma_1$.
    Notice that $j(\gamma_2) \cdot \lambda_{X\inv}^{rel}(\beta_1) = \kappa \lambda_X^{rel}(\widetilde{\alpha})$ for $\kappa \in \Ker(\iota\colon \St(\Phi, R, M[X, X\inv]) \to \St(\Phi, R))$

    By the third assertion of~\cref{prop:horrocks-main} $j(\gamma_2) \cdot [1, \beta_1, 1] = [j(g), h, u]$ for some $g \in \G_{M, \Psi}^{\geq 0}$.
    Observe that
    \[ \lambda_{X\inv, B}^* (j(\gamma_2) \cdot i_B(\beta_1)) = j(\gamma_1) \lambda_{X\inv}^*(\beta_1) = \lambda_X^*(j(\alpha_0)) \cdot j(\gamma) \lambda_{X\inv}^*(\beta) = \lambda_{X\inv, B}^*(t_M(\widetilde{\alpha})), \]
    hence there exists $\kappa \in \Ker(\lambda^*_{X\inv, B})$ such that $j(\gamma_2) \cdot i_B(\beta_1) = \kappa t_M(\widetilde{\alpha})$.
    Consequently, from~\cref{prop:horrocks-main} we obtain that
    \[j(\gamma_2) \cdot [1, \beta_1, 1] = j(\gamma_2) \cdot i_B(\beta_1) [1, 1, 1] = \kappa [\lambda_X^*(j(\alpha_0) \cdot \alpha), 1, 1].\]
    By the definition of $\overline{V}$ one has $j(\alpha_0) \alpha = j(g) g_0$ for $g$ as above and some $g_0 \in \overline{\St}(\Phi, A, M)$.
    Since $\alpha(0) = 1$ we obtain that $g_0 = j(g)(0) \cdot j(\alpha_0)(0)$, which implies the required assertion.
\end{proof}