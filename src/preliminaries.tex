In this paper all rings are assumed to be commutative and all commutators are left-normed i.\,e. $[a, b] = a b a^{-1} b^{-1}$.

\subsection{Formalism of triples}\label{subsec:triples}

Let $N$ be a group acting on itself by right conjugation.
Let $M$ be a group with a right action of $N$.
Recall that a group homomorphism $\mu\colon M \to N$ is called a \textit{precrossed module} if $\mu$ preserves the action of $N$, i.\,e.
\[\mu(m^n) = \mu(m)^n, \text{for all $m \in M$, $n\in N;$} \]
If, in addition, $\mu$ satisfies the so-called \textit{Peiffer identity}, i.\,e.
\[{m}^{\mu(m')} = {m'}^{-1} m m', \text{for all $m, m' \in M$,}\]
then $\mu$ is called a \textit{crossed module}.

If $\mu\colon M \to N$ and $\mu' \colon M' \to N'$ is a pair of precrossed modules.
A \textit{map of precrossed modules} $(f, g)\colon \mu \to \mu'$ is a pair of group homomorphisms $f\colon M \to M'$, $g\colon N \to N'$ such that
$\mu'f = g \mu$ and that the action of $N$ is preserved, i.\,e. ${f(m)}^{g(n)} = f(m^n)$ for all $n \in N$, $m \in M$.

Now suppose that we are given the following cube-like commutative diagram of abstract groups:
\begin{equation} \label{eq:cube} \begin{split} \xymatrix{
    G_{123} \ar[rr]_{f_{23}} \ar[dd]_{f_{12}} \ar[rd]_{f_{13}} &                        & G_{23} \ar@{-->}[dd]^(.3){g_2^3} \ar[rd]^{g^2_3} &           \\
    & G_{13} \ar[rr]^(.3){g^1_3} \ar^(.3){g^3_1}[dd] &                   & G_3 \ar[dd]_{h_3} \\
    G_{12} \ar@{-->}[rr]^(.3){g_2^1} \ar[rd]_{g_1^2}          &                        & G_2 \ar@{-->}[rd]_{h_2}         &           \\
    & G_1 \ar[rr]_{h_1}              &                   & G.} \end{split} \end{equation}
Additionally, we make the following assumptions:
\begin{itemize}
    \item $g_1^3$ is a precrossed module;
    \item $h_3$ is a crossed module;
    \item $(g_3^1, h_1) \colon g_1^3 \to h_3$ is a map of precrossed modules.
\end{itemize}
Set $V = G_1 \times G_2 \times G_3$, $W = G_{12} \times G_{13} \times G_{23}$.
We define an operation $\star \colon W \times V \to V$ as follows.
For $v = (x, y, z) \in V$ and $w = (a, b, c) \in W$ we set
\[(a, b, c) \star (x, y, z) = (x \cdot g_1^3(b) \cdot g_1^2(a),\ g_2^1(a)^{-1} \cdot y \cdot g_2^3(c),\ g_3^2(c)^{-1} \cdot g_3^1(b)^{-h_2(y)} \cdot z).\]
Consider the set-theoretic map $h \colon V \to G$ given by $(x, y, z) \mapsto h_1(x) \cdot h_2(y) \cdot h_3(z)$.
It is clear from the definition of $\star$-operation that for $w \in W$ one has $h(w \star v) = h(v).$
Now let us define the relation associated with $\star$-operation.
We declare two elements $v, v' \in V$ congruent (denoted $v \sim_W v'$) if $v' = w \star v$ for some triple $w=(a, b, c) \in W$.
As the following lemma shows, this relation is an equivalence relation.
\begin{lemma} For every $v \in V$ one has
\begin{equation*}(a', b', c') \star \left( (a, b, c) \star v \right) = (a \cdot a', b \cdot {b'}^{g_1^2(a)^{-1}}, c \cdot c') \star v.\end{equation*}
\end{lemma}
\begin{proof}
    Set $v'=(x', y', z') = (a, b, c) \star v$ and $(x'', y'', z'') = (a', b', c') \star v'$.
    Since $g_1^3$ is a precrossed module we immediately obtain that
    \begin{align*}
        x'' =& x' \cdot g_1^3(b') \cdot g_1^2(a') = x \cdot g_1^3(b) \cdot g_1^2(a) \cdot g_1^3(b') \cdot g_1^2(a') = x \cdot g_1^3(b \cdot b'^{g_1^2(a)^{-1}}) g_1^2(a \cdot a'),\\
        y'' =& g_2^1(a')^{-1} \cdot g_2^1(a)^{-1} \cdot y \cdot g_2^3(c) \cdot g_2^3(c') = g_2^1(a\cdot a')^{-1} \cdot y \cdot g_2^{3}(c\cdot c'). \end{align*}
    Since $(g_3^1, h_1)$ is a map of precrossed modules, for every $a \in G_{12}$, $b \in G_{13}$ one has $g_3^1(b^{g_1^2(a)}) = g_3^1(b)^{h_1 g_1^2(a)} = g_3^1(b)^{h_2g_2^1(a)}$.
    Since $h_3$ satisfies Peiffer identity, for every $c \in G_{23}$ and $z \in G_3$ one has $z ^{h_2 g_2^3(c)} = z^{ h_3 g_3^2(c)} = g_3^2(c)^{-1} \cdot z \cdot g_3^2(c)$.
    Using these identities we obtain that
    \begin{multline*}
        z'' = g_3^2(c')^{-1} \cdot g_3^1(b')^{-h_2(y')} \cdot z' = \\
        = g_3^2(c')^{-1} \cdot g_3^1(b')^{- h_2 \left( g_2^1(a)^{-1} \cdot y \cdot g_2^3(c) \right)} g_3^2(c)^{-1} \cdot g_3^1(b)^{-h_2(y)} \cdot z = \\
        = g_3^2(c \cdot c')^{-1} \cdot g_3^1(b')^{- h_2 \left( g_2^1(a)^{-1} \cdot y \right)} \cdot g_3^1(b)^{-h_2(y)} \cdot z = \\
        = g_3^2(c \cdot c')^{-1} \cdot g_3^1(b \cdot {b'} ^ {g_1^2(a)^{-1}})^{- h_2 \left( y \right)} \cdot z. \qedhere
    \end{multline*}
\end{proof}
Since $h$ is constant on the orbits of $\star$-action, $h$ gives rise to a well-defined map $\overline{h} \colon V/\sim_W \to G$.
We use the notation $[a, b, c]$ to denote the equivalence class of the triple $(a, b, c) \in V$.

\begin{lemma}\label{lem:one-one-z} Assume that the back face $(f_{12}, f_{23}, g_2^1, g_2^3)$ of~\eqref{eq:cube} is a pullback square.
Assume additionally that $[1, 1, 1] = [1, 1, z]$ for some $z\in G_3$.
Then $z \in g_3^1(\Ker(g_1^3)).$ \end{lemma}
\begin{proof} By the definition of congruence relation there exists $(a, b, c)\in W$ such that
\[ (a, b, c) \star (1, 1, 1) = ( g_1^3(b) \cdot g_1^2(a),\ g_2^1(a)^{-1} \cdot g_2^3(c),\ g_3^2(c)^{-1} \cdot g_3^1(b)^{-1}) = (1,1,z). \]
By the Lemma's assumption there exists $e \in G_{123}$ such that $f_{12}(e) = a$, $f_{23}(e) = c$, hence
\[ 1 = g_1^3(b) \cdot g_1^2(f_{12}(e)) = g_1^3(b \cdot f_{13}(e)),\ z = g_3^2(f_{23}(e))^{-1} \cdot g_3^1(b)^{-1} = g_3^1(b \cdot f_{13}(e))^{-1}. \qedhere\] \end{proof}

\subsection{Steinberg groups, $\K_2$-groups and symbols}\label{subsec:steinberg-preliminaries}
Let $\Phi$ be a root system of rank $\ell \geq 1$.
We assume that $\Phi$ is embedded into $\mathbb{R}^\ell$ whose scalar product we denote by $(\text{-}, \text{-})$.
We also fix some system of simple roots $\Pi = \{\alpha_1, \ldots, \alpha_\ell\} \subset \Phi$.
For a root $\alpha\in\Phi$ we denote by $m_i(\alpha)$ the $i$-th coefficient in the expansion of $\alpha$ in $\Pi$,
i.\,e. $\alpha = \sum_{i=1}^n m_i(\alpha) \alpha_i$.
We denote by $\Phi^+$ (resp. $\Phi^-$) the system of positive (resp. negative) roots with respect to the basis $\Pi$.

We denote by $\Phi^\vee$ the corresponding dual root system, which, by definition, consists of all coroots $\alpha^\vee = \frac{2}{(\alpha, \alpha)} \alpha$, where $\alpha \in \Phi$.
We denote by $P(\Phi^\vee)$ the integral lattice spanned by the \emph{fundamental coweights $\varpi_i^\vee$}.
Recall that the fundamental coweights $\varpi_i^\vee$ are uniquely determined by the property $(\varpi_i^\vee, \alpha_j) = \delta_{ij}$.
For $\alpha,\beta \in \Phi$ we denote by $\langle \alpha, \beta \rangle$ the integer $(\alpha, \beta^\vee) = \frac{2(\alpha, \beta)}{(\beta, \beta)}$.

Now let $R$ be an arbitrary commutative ring with $1$ and suppose that $\Phi$ is an irreducible root system of rank $\geq 2$.
Recall that to the pair $(\Phi, R)$ can associate an abstract group $\St(\Phi, R)$, called the \textit{Steinberg group} of type $\Phi$ over $R$.
By definition, $\St(\Phi, R)$ is the group presented by generators $x_\alpha(a)$, $a \in R$, $\alpha \in \Phi$ and an explicit list of relations (see e.\,g in.~\cite{Ma69, Re75, St71}).
In this paper we may restrict ourselves to the case when the root system $\Phi$ in question is \textit{simply-laced} (i.\,e. has type $\mathsf{ADE}$),
 in which case the defining relations of $\St(\Phi, R)$ reduce to the following shorter list:
\begin{align}
x_{\alpha}(a)\cdot x_{\alpha}(b)&=x_{\alpha}(a+b), \tag{R1} \label{x-additivity}\\
[x_{\alpha}(a),\,x_{\beta}(b)]  &=x_{\alpha+\beta}(N_{\alpha,\beta} \cdot ab),\text{ for }\alpha+\beta\in\Phi, \tag{R2} \label{R2} \\
[x_{\alpha}(a),\,x_{\beta}(b)]  &=1,\text{ for }\alpha+\beta\not\in\Phi\cup0. \tag{R3} \label{R3}
\end{align}
The coefficients $N_{\alpha,\beta}$ in the above formula are integers equal to $\pm 1$, they coincide with the structure constants of the complex Lie algebra of type $\Phi$.

Throughout this paper we denote by $\Gsc(\Phi, R)$ the group of points of the simply-connected Chevalley--Demazure group of type $\Phi$ over $R$.
We denote by $\Esc(\Phi, R)$ the \textit{elementary subgroup} of $\Gsc(\Phi, R)$, i.\,e. the subgroup generated by elementary root unipotents of $\Gsc(\Phi, R)$.
Notice that in~\cite{VP, Vav09} the notation $x_\alpha(a)$ is used to denote the elementary root unipotents.
To prevent confusion we will use different notation $t_\alpha(a)$ for them and reserve the notation $x_\alpha(a)$ solely for generators of Steinberg groups.

Recall that the map sending $x_\alpha(a)$ to $t_\alpha(a)$ gives rise to a well-defined homomorphism $\pi \colon \St(\Phi, R) \to \G_\mathrm{sc}(\Phi, R)$, see~\cite[\S~1A]{St78}.
The cokernel and the kernel of $\pi$ are called \textit{the unstable $\K_1$- and $\K_2$-functors modeled on the root system $\Phi$}:
\begin{equation} \label{eq:K1-K2-sequence}
  \xymatrix{ 1 \ar[r] & \K_2(\Phi, R) \ar[r] & \St(\Phi, R) \ar[r]^{\pi} & \Gsc(\Phi, R) \ar[r] & \K_1(\Phi, R) \ar[r] & 1}
\end{equation}

Following~\cite{Ma69} for $\alpha\in\Phi$ and $u \in R^\times$ we define the following elements of $\St(\Phi, R)$:
\begin{align*} w_\alpha(u) & =  x_\alpha(u) \cdot x_{-\alpha}(-u^{-1}) \cdot x_\alpha(u), \\
               h_\alpha(u) & =  w_\alpha(u) \cdot w_\alpha(-1).  \end{align*}
The subgroup generated by $w_\alpha(u)$ (resp. $h_\alpha(u)$) for all $\alpha\in \Phi$, $u \in R^\times$ is denoted by $\StW(\Phi, R)$ (resp. $\StH(\Phi, R)$).
By~\cite[Lemme~5.2]{Ma69} $\StH(\Phi, R)$ is a normal subgroup of $\StW(\Phi, R)$.

In the sequel we will need the following explicit elements of the group $\K_2(\Phi, R)$.
Recall that for arbitrary $u, v \in R^\times$ one defines the \textit{Steinberg symbol} via the formula
\begin{equation} \label{eq:steinberg} \{ u, v \}_\alpha = h_\alpha(uv) \cdot h_\alpha^{-1}(u) \cdot h_\alpha^{-1}(v). \end{equation}
Recall also from~\cite[Lemme~5.4]{Ma69} that
\begin{equation} \label{eq:steinberg-2} [h_\alpha(u), h_\beta(v)] = \{u, v^{\langle \alpha, \beta \rangle}\}_\alpha. \end{equation}
Steinberg symbols depend only on the length of the root $\alpha$.
In particular, in the case when $\Phi$ is of simply-laced type, they do not depend on the choice of $\alpha$, which allows us to omit it from notation.

Steinberg symbols are central elements of $\St(\Phi, R)$.
Our assumptions on $\Phi$ guarantee that Steinberg symbols are antisymmetric and bimultiplicative, i.\,e. they satisfy the following identities:
\begin{equation} \label{eq:symbol-properties} \{ u, st \} = \{ u, s\} \{ u, t \}, \ \{ u, v \} = \{ v, u\}^{-1}. \end{equation}

\subsection{Curtis--Tits type presentation of Steinberg groups} \label{subsec:curtis-tits}
In this subsection we briefly recall generalization of Steinberg groups to the Kac--Moody setting and their presentations.
For more details we refer the reader to \textbf{???}.
Recall from~\textbf{???} that to any generalized Cartan matrix $A$ and a commutative ring $R$ one can associate two groups:
 the so-called \textit{Steinberg group} $\St(A, R)$ and $\mathrm{PSt}(A, R)$, called the \textit{pre-Steinberg group} (or Steinberg group a la Tits) of type $A$ over $R$.
Steinberg and pre-Steinberg group differ from each other in the list of defining relations:
 the former definition imposes commutator relations over all prenilpotent pairs of real roots while the latter involves only classically prenilpotent pairs, cf.~\textbf{???}.

In the case when $A(\Phi)$ is the Cartan matrix of a spherical root system $\Phi$
 both these definitions agree with the previous one, i.\,e. $\St(\Phi, R) \cong \mathrm{PSt}(\Phi, R) \cong \St(A(\Phi), R)$.

Now let us briefly recall what happens in the affine case.
Denote by $\widetilde{A}(\Phi)$ the affine Cartan matrix corresponding to $\Phi$.
\begin{lemma} $\mathrm{PSt}(\widetilde{A}(\Phi), R) \cong \St(\widetilde{A}(\Phi), R) \cong \St(\Phi, R[X, X\inv])$.
\end{lemma}


\subsection{Relative Steinberg groups and their presentations} \label{subsec:another-presentation}
In this paper we will use also use the concept of a \textit{relative Steinberg group} introduced by F.~Keune and J.-L.~Loday in~\cite{Ke78, Lo78}.
We will only briefly mention the definition and basic properties of these groups and refer the reader to~\cite[\S~2.3]{LS20} for a more detailed exposition.

Let $R$ be a commutative ring, $I \trianglelefteq R$ be an ideal and let $p$ denote the canonical projection $R \to R/I$.
Denote by $D_{R, I}$ the pullback of two copies of $p$ i.\,e. the ring $R \times_{R/I} R$.
Elements of $D_{R, I}$ are pairs $(a; b)$ such that $a-b \in I$.
We also denote by $p_1$, $p_2$ the canonical projections $D_{R, I} \to R$ and by $p_1^*$, $p_2^*$ the corresponding homomorphisms of Steinberg groups induced by them.
Recall from~\cite[Definition~2.5]{LS20} that the relative Steinberg group $\St(\Phi, R, I)$ is defined as the quotient
$\Ker(p_1^*) / C$, where $C = [\Ker(p_1^*), \Ker(p_2^*)]$.
If we denote by $\mu$ the homomorphism $\St(\Phi, R, I) \to \St(\Phi, R)$ induced by $p_2^*$, we obtain an exact sequence
\begin{equation}
    \xymatrix{1 \ar[r] & C(\Phi, R, I) \ar[r] & \St(\Phi, R, I) \ar[r]^\mu & \St(\Phi, R) \ar[r]^-{p^*} & \St(\Phi, R/I) \ar[r] & 1. }\label{eq:relative-Steinberg}
\end{equation}
Alternatively, the group $\St(\Phi, R, I)$ can be defined via generators and relations as an $\St(\Phi, R)$-group, cf.~\cite[Proposition~6]{S15}
or even as an abstract group, see~\cite{V22}.
The relative group $\K_2(\Phi, R, I)$ is defined as the kernel of the homomorphism $\pi \mu$.

\begin{lemma} \label{lem:rel-Steinberg-crossed-module}
For any irreducible root system $\Phi$ of rank $\geq 3$ the homomorphism $\mu \colon \St(\Phi, R, I) \to \St(\Phi, R)$ is a crossed module.
\end{lemma}
\begin{proof}
    Denote by $\Delta$ the diagonal homomorphism $R \to D_{R, I}.$
    By~\cite[Proposition~6]{Lo78} it suffices to show that $[\Ker(p_1^*) \cap \Ker(p_2^*), \Delta^*(\St(\Phi, R))] = 1$.
    But the latter is clear since $\Ker(p_1^*) \cap \Ker(p_2^*) \subseteq \K_2(\Phi, D_{R, I})$ while the latter subgroup is central by~\cite{LSV1}.
\end{proof}

We will need a relative analogue of Steinberg symbol.
Let $A$ be a local unital ring with maximal ideal $M$ embedded as a subring into a larger unital ring $R$.
Under this assumption the subset $1+M \subseteq A$ forms a group under multiplication.
It is clear that $(1+M)^\times$ is isomorphic to the abelian group $(M, \circ)$ with the operation given by $m \circ m' = m + m' + mm'$.
Now for $a \in R^\times$ and $m \in M$ we denote by $\{a, 1+m\}_r$ the coset $\{(a; a), (1; 1+m)\}C \in \St(\Phi, R, RM)$.
It is clear that the map $1+m \mapsto \{a, 1+m\}_r$ specifies a group homomorphism
\begin{equation} \label{eq:relative-symbol} \{ a, -\}_r \colon (1+M)^\times \to \K_2(\Phi, R, RM). \end{equation}
It is also clear that $\mu(\{a, 1+m\}_r) = \{a, 1+m\}$.

\begin{lemma}\label{lem:symbols}
Assume that $A$ is a local domain with maximal ideal $M$.
Denote by $R$ the Laurent polynomial ring $A[X, X\inv]$.
Then the intersection of the image of the relative symbol map $\{X, -\}_r$ with $C(\Phi, R, M[X, X\inv])$ is trivial.
\end{lemma}
\begin{proof}
    Set $F = \mathrm{Frac}(A)$.
    Consider the following diagram:
    \[\begin{tikzcd}
    (1+M)^\times \ar[hookrightarrow, rr] \ar[d] &  & F^\times \ar[hookrightarrow, d] \\
    \K_2(\Phi ,R, M[X^{\pm 1}]) \ar[r] & \K_2(\Phi, R) \ar[r] & \K_2(\Phi, F[X^{\pm 1}]).
    \end{tikzcd}\]
    Since the right vertical arrow is injective by~\cite[Lemma~2.2]{LS20}, so is the left arrow.
\end{proof}

Our next goal is to recall an explicit presentation of the relative linear Steinberg group from~\cite{LS17}.
This presentation is formulated in terms of rows and columns rather than root unipotents and is similar to
 Tulenbaev's presentation of the relative linear Steinberg group from~\cite[Definition~1.5]{Tu83}.
Unlike Tulenbaev's presentation, however, our presentation applies to the group $\St(\rA_\ell, R, I)$ with $\ell \geq 3$ rather than $4$,
 which will be important in the sequel.

Let $n \geq 4$.
For $u \in R^n$ we denote by $D(u)$ the subset of $R^n$ consisting of all vectors $v$ which are orthogonal to $u$
(i.\,e. $u^{t} v = 0$) and have at least two zero entries.
Recall from 3.2 of~\cite{Ka77} that for every $u, v, w \in R^n$ such that $u^t v = 0$ there
is a decomposition of $(w^t u) \cdot v$ into a sum of elements of $D(u)$, called \textit{canonical decomposition}:
\begin{equation}
    \label{eq:canonical} (w^tu) \cdot v=\sum_{i<j}u_{ij} c_{ij}(v, w),
\end{equation}
where $u_{ij} =e_i u_j-e_j u_i \in D(u)$ and $c_{ij}(v, w) =v_i w_j-v_j w_i \in R$.

\begin{lemma}
    \label{lem:xsmall-properties}
    Let $v, w \in R^n$ be such that $v^t w = 0$ and assume, moreover, that either $v$ or $w$ has at least one zero entry.
    Under these assumptions one can define certain element $x(v, w) \in \St(n, R)$ such that $\phi(x(v, w)) = t(v, w) = 1 + vw^t$.
    The elements $x(v, w)$ enjoy the following properties:
    \begin{lemlist}
        \item \label{itm:xsmall-scalar} If $v$ or $w$ has at least two zero entries, then $x(v, wa) = x(va, w)$ for $a\in R$.
        \item \label{itm:xsmall-additivity} If $w_1$ and $w_2$ have at least two zero entries of which at least one entry is common
        then $x(v, w_1) \cdot x(v, w_2) = x(v, w_1+w_2)$ and $x(w_1, v) \cdot x(w_2, v) = x(w_1 + w_2, v)$.
        \item \label{itm:xsmall-commute} If $v$, $v'$ are simultaneously orthogonal to $w$ and $w'$, and the elements $w, w'$ both have at least two zero entries then
        $[x(v, w),\ x(v', w')] = 1$.
        \item \label{itm:xsmall-conj} If $g = x_{ij}(\xi)$ is a Steinberg generator and $v$ or $w$ has at least two zero entries then
        $g \cdot x(v, w) \cdot g^{-1} = x(gv, g^*w)$.
    \end{lemlist}
\end{lemma}
\begin{proof}
    See~\cite[Lemma~1.1]{Tu83}.
\end{proof}

Now assume that $I$ is an ideal of a ring $R$ which itself is a subring of a ring $S$.
We now define two families of elements of $\St(n, R)$.
\begin{dfn}
    Let $u \in \Um(n, R)$ and $v \in I^n$ be a vector such that $u^{t}v = 0$.
    Let $v = \sum_r v_r$ be some decomposition for $v$ into sum of elements $v_r \in D(u) \cap I^n$
    (under the assumptions on $u$ and $v$ such decomposition always exists, see~\eqref{eq:canonical}).

    Let $d$ be an element of the subgroup $T(n, S)$ of diagonal matrices of $\GL(n, S)$ such that
    $d^{-1}u \in R^n,\ d \cdot I^n \subseteq R^n.$
    Under these assumptions we set
    \begin{equation*}
        X^d(u, v) \coloneqq \prod_i x(d^{-1}u, dv_i),
    \end{equation*}

    Not let $d'$ be an element of $T(n, S)$ such that $d' u\in R^n,\ {d'}^{-1} \cdot I^n \subseteq R^n$.
    Under these assumptions we set
    \begin{equation*}
        Y^{d'}(v, u) \coloneqq \prod_i x({d'}^{-1} v_i, {d'}u).
    \end{equation*}
\end{dfn}

\begin{lemma}
    \label{lem:xy-wd}
    The elements $X^d(u, v)$, $Y^{d'}(u, v)$ are well-defined, i.\,e. they do not depend on the choice of decomposition for $v$.
\end{lemma}
\begin{proof}
    Let $v = \sum_r v^r$ be a decomposition as above.
    Since each $v^r$ is orthogonal to $u$ we can write the canonical decomposition
    $v^r = \sum_{i<j} u_{ij} c_{ij}(v^r, w)$, moreover $\sum_{r} c_{ij}(v^r, w) = c_{ij}(v, w)$.
    Now using~\cref{lem:xsmall-properties} we obtain:
    \begin{multline*}
        \prod\limits_r x(d^{-1}u, dv^r) = \prod\limits_{r}\prod\limits_{i<j} x(d^{-1} u, du_{ij}c_{ij}(v^r, w)) =
        \prod\limits_{i<j} x(d^{-1} u, d u_{ij}c_{ij}(v, w)). \qedhere
    \end{multline*}
\end{proof}

\begin{lemma}
    \label{lem:xy-conj} Suppose that $g = x_{hk}(\xi)$ is a generator of $\St(n, R)$ such that $m = d\phi(g)d^{-1} \in \E(n, R)$, then
    \begin{equation*}
        g \cdot X_d(u, v) \cdot g^{-1} = X_d(mu, m^*v) \text{ and } g \cdot Y_d(v, u) \cdot g^{-1} = Y_d(mv, m^*u).
    \end{equation*}
\end{lemma}
\begin{proof}
    Direct computation using~\cref{lem:xsmall-properties} (cf. with~\cite[3.14]{Ka77} or~\cite[Lemma~4.4d]{LS17}).
\end{proof}

The following result is the key result of~\cite{LS17}.
\begin{prop}[\text{\cite[Proposition 3.10]{LS17}}]
    \label{prop:rel-presentation}
    Assume that $I$ is a splitting ideal of a commutative ring $R$.
    Then for any $\ell\geq 3$ the group $\St(\rA_\ell,\,R,\,I)$ can be presented by means of two families of generators $F(u,\,v)$, $S(v,\,u)$
    (where $u\in \E(n,\,R)e_1,$ and $v\in I^n$ are such that $u^{t}v=0$) subject to the following relations:
    \begin{align}
        &F(u,\,v)F(u,\,w)=F(u,\,v+w), \label{add4}\\
        &S(u,\,v)S(w,\,v)=S(u+w,\,v), \label{add5}\\
        &F(u,\,v)F(u',\,v')F(u,\,v)^{-1}=F(t(u,\,v)u',\,t(v,\,u)^{-1} v'), \label{conj3} \\
        &F(me_1,\,m^{*}e_{2}a)=S(me_{1}a,\,m^{*}e_{2}),\ \text{for all $a\in I$,}\, m \in \E(n, R). \label{coef-move}
    \end{align}
\end{prop}

\subsection{Weight automorphisms}\label{subsec:weight-automorphisms}
Recall that for every coweight $\omega \in P(\Phi^\vee)$ and $\beta \in \ZZ \Phi$ the scalar product $(\omega, \beta)$ is an integer.
Thus, a choice of $u \in R^\times$ and $\omega \in P(\Phi^\vee)$ specifies a permutation of the generating set for $\St(\Phi, R)$ via the following mapping:
\begin{equation*} x_\alpha(a) \mapsto x_\alpha(u^{(\omega, \alpha)} \cdot a),\ \alpha\in \Phi,\ a \in R. \end{equation*}
It is not hard to check that this action is compatible with relations~\eqref{x-additivity}--\eqref{R3} and hence specifies a well-defined automorphism of $\St(\Phi, R)$, which we denote by $\chi_{\omega, u}$.

In the following lemma we check that an analogue of $\chi_{\omega, u}$ can also be defined for relative Steinberg groups.
\begin{lemma} \label{lem:relative-chi}
Let $R$ be a commutative ring, $I$ be its ideal and let $u \in R^\times$.
For every coweight $\omega \in P(\Phi^\vee)$ there exists a well-defined automorphism $\widetilde{\chi}_{\omega, u}$ of the relative Steinberg group $\St(\Phi, R, I)$ making the following diagram commute:
\[\begin{tikzcd} \St(\Phi, R, I) \ar[r, "\widetilde{\chi}_{\omega, u}"] \ar[d] & \St(\Phi, R, I) \ar[d] \\
\St(\Phi, R) \ar[r, "\chi_{\omega, X}"] & \St(\Phi, R]). \end{tikzcd}\]
\end{lemma}
\begin{proof}
    Observe that the automorphism $\chi_{\omega, (u; u)}$ of $\St(\Phi, D_{R, I})$ preserves subgroups
    $\Ker(p_i^*)$, $i=1, 2$ and hence their commutator subgroup $C$.
    The required automorphism $\widetilde{\chi}_{\omega, u}$ now can be obtained by restricting $\chi_{\omega, (u; u)}$ to $\Ker(p_1^*)$.
    The commutativity of the diagram is obvious.
\end{proof}

In the sequel we will use the following formulae describing the action of $\chi_{\omega, u}$ on the elements $w_\alpha(u)$, $h_\alpha(u)$:
\begin{align}
    \label{eq:chi-w} \chi_{\omega, u}\left(w_\alpha(v)\right) &= w_\alpha(u^{(\omega, \alpha)} \cdot v), \\
    \label{eq:chi-h} \chi_{\omega, u} (h_\alpha(v)) &= h_\alpha(u^{(\omega, \alpha)} \cdot v) \cdot h_\alpha(u^{(\omega, \alpha)})^{-1} = \{u^{(\omega, \alpha)}, v\} \cdot h_\alpha(v).
\end{align}

The following lemma is analogous to~\cite[Lemma~3.1(c)]{Tu83}.
\begin{lemma} \label{lem:winv-chiw}
For any $w \in \StW(\Phi, A[X^{\pm 1}])$ the element $w^{-1} \cdot \chi_{\omega, X}(w)$ belongs to $\StH(\Phi, A[X^{\pm 1}])$.
\end{lemma}
\begin{proof}
    Since $\StH(\Phi, A[X^{\pm 1}])$ is a normal subgroup of $\StW(\Phi, A[X^{\pm 1}])$, it suffices to verify the assertion for $w = w_\alpha(u)$.
    Set $h = w^{-1} \cdot \chi_{\omega, X}(w)$.
    Notice that
    \begin{multline*} w_\alpha(1) \cdot h\cdot  w_\alpha(-1) = w_\alpha(-1)^{-1} \cdot w_\alpha(u)^{-1} \cdot w_{\alpha}(X^{(\omega, \alpha)} u) \cdot w_\alpha(-1) = \\
    = h_\alpha(u)^{-1} \cdot h_\alpha(X^{(\omega, \alpha)}u) \in \StH(\Phi, A[X^{\pm 1}]).\end{multline*}
    Thus, we get from\cite[Lemme~5.2(b,g)]{Ma69} and~\eqref{eq:steinberg} that \[h = h_{\alpha}^{-1}(u^{-1}) \cdot h_{\alpha}(X^{-(\omega, \alpha)}u^{-1}) = h_{\alpha}^{-1}(X^{(\omega, \alpha)}) \cdot \{ X^{(\omega, \alpha)}, u^{-1} \}. \qedhere\]
\end{proof}

\begin{example} \label{exm:chi-linear}
Set $R = A[X^{\pm 1}]$.
Consider the following coweights:
\[\varepsilon_1 = \varpi_1^\vee,\ \varepsilon_2 = \varpi_2^\vee - \varpi_1^\vee,\ \ldots,\ \varepsilon_{\ell+1} = -\varpi^\vee_\ell.\]
For $1\leq k\leq \ell+1$ and $u \in R^\times$ denote by $d_k(u)$ the matrix from $\GL(\ell+1, R)$ which differs from the unit matrix only in that it has the element $u$ on the $k$-th place of its diagonal.
Recall from~\cite[Corollary~4]{Ka77} that for any $g \in \GL(\ell+1, R)$ there exists an automorphism $\beta_g$ of $\St(\ell+1, R)$ ''modeling`` the automorphism $\alpha_g \colon \GL(\ell+1, R) \to \GL(\ell+1, R)$ of inner conjugation by $g$, i.\,e. such that $\phi \beta_g = \alpha_g \phi$.


It is clear that in the linear case the map $\chi_{\varepsilon_k, u}$ coincides with $\beta_{d_k(u)}$,
while for other Chevalley groups the maps $\chi_{\omega, u}$ model automorphisms of inner conjugation by weight elements $h_\omega(u)$ in the sense of~\cite[\S~4]{Vav09}.
\end{example}


Let $\omega \in P(\Phi^\vee)$ be a coweight of $\Phi$.
Consider the subset $\mathcal{X}_\omega$ of the generating set of $\St(\Phi, A[X])$ consisting of those generators $x_{\alpha}(a) \in \St(\Phi, A[X])$ for which
$(\alpha, \omega) < 0$ implies that $a \in A[X]$ is divisible by $X^{-(\alpha, \omega)}$.
\begin{rem}
    Notice that according to this condition if $(\alpha, \omega) \geq 0$ then $\mathcal{X}$ contains $x_\alpha(a)$ for all $a \in A[X]$.
\end{rem}
Denote by $N_\omega$ the subgroup of $\St(\Phi, A[X])$ generated by $\mathcal{X}_\omega$.


\begin{dfn} \label{dfn:delta-pair}
Let $\omega \in P(\Phi^\vee)$ be a coweight.
By definition, an {\it $\omega$-pair} is a pair of mutually inverse group homomorphisms
$\xymatrix{ \sigma(\omega)\colon N_\omega \ar[r] & \ar@<-1.0ex>[l] N_{-\omega}\colon \sigma(-\omega) }$ satisfying the following identity:
\begin{equation} \label{eq:sigmadef}
\sigma(\pm \omega)(x_\alpha(\xi)) = x_\alpha(X^{(\pm \omega, \alpha)}\cdot \xi),
\text{ for all } x_\alpha(\xi) \in \mathcal{X}_{\pm\omega}.
\end{equation}\end{dfn}
It is clear that the maps $\delta(\omega)$, $\delta(-\omega)$ are uniquely determined by~\eqref{eq:sigmadef}, so at most one $\omega$-pair may exist for any given $\omega$.
Moreover, $\delta(\omega), \delta(-\omega)$ always make the following diagram commute:
\begin{equation} \label{eq:sigma-diagram}
\xymatrix{ N_\omega \ar[r]_{\sigma(\omega)}\ar@{^{(}->}[d] \ar@/^1.5pc/[rr]^{\mathrm{id}} & N_{-\omega} \ar@{^{(}->}[d] \ar[r]_{\sigma(-\omega)} & N_\omega \ar@{^{(}->}[d] \\
\St(\Phi, A[X]) \ar[d] & \St(\Phi, A[X]) \ar[d] & \St(\Phi, A[X]) \ar[d] \\
\St(\Phi, A[X^{\pm 1}]) \ar@<-0.0ex>[r]_{\chi_{\omega, X}} \ar@/_1.5pc/[rr]^{\mathrm{id}} & \St(\Phi, A[X^{\pm 1}]) \ar@<-0.0ex>[r]_{\chi_{-\omega, X}} & \St(\Phi, A[X^{\pm 1}]).} \end{equation}
The question of existence of an $\omega$-pair is rather complicated and will be addressed in the sequel.
One of the necessary technical ingredients needed for this is the technique of another presentation, which we recall in detail in the following subsection.

