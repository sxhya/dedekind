Throughout this section $\Phi$ denotes a simply laced root system of rank $\geq 3$ and $k$ is the index of
 a simple root $\alpha_k$ such that the corresponding weight $\varpi_k$ is a microweight.
The aim of this section is to prove the existence of a $\varpi_k$-pair for $\Phi$ in the sense of~\cref{dfn:delta-pair}
 under the additional assumption that $A$ is a local ring.

First of all, recall that the ideal $XA[X]$ is a splitting ideal of $A[X]$, therefore
$\St(\Phi, A[X])$ decomposes as $\St(\Phi, A[X], XA[X]) \rtimes \St(\Phi, A)$.
Notice also that under our assumptions on $\Phi$ and $k$ the subgroup $N_{\varpi_k}$ contains $\St(\Phi, A[X], XA[X])$.
In fact, $N_{\varpi_k}$ decomposes into the semidirect product of $N_0 := \St(\Phi, A[X], XA[X])$ and the parabolic subgroup $P_k \leq \St(\Phi, A)$
generated by $x_\alpha(a)$, $a \in A$ for all $\alpha \in \Delta \sqcup \Sigma^+_k$.

Recall that the universal property of semidirect products gives for any group $H$ acting on a group $N$
and any group homomorphisms $f_N\colon N \to G$, $f_H\colon H \to G$ satisfying
\begin{equation}
    \label{eq:coherence-condition} f_N({}^hn) = {}^{f_H(h)} f_N(n),\ n\in N,\ h\in H
\end{equation}
a unique map $f\colon N \rtimes H \to G$ extending $f_N$ and $f_H$.

Applying this universal property to our situation reduces the problem of construction of $\sigma(\varpi_k) \colon N_{\varpi_k} \to N_{-\varpi_k}$
to the construction problems for $\sigma(\varpi_k)_{P_k}$ and $\sigma(\varpi_k)_{N_0}$.
Since $P_k = L_k \ltimes U_k^+$, where $L_k = \mathrm{Im}(\St(\Delta_k, A) \to \St(\Phi, A))$
and $U_k^\pm = \UU(\Sigma_k^\pm, A)$ we can apply the universal property once again and further reduce the construction problem for $\sigma(\varpi_k)_{P_k}$
to the construction problems for $\sigma(\varpi_k)_{L_k}$ and $\sigma(\varpi_k)_{U_k}$.
The latter two homomorphisms now can be defined directly as follows:
\[\sigma(\varpi_k)_{L_k} \coloneqq \mathrm{id}_{L_k},\ \sigma(\varpi_k)\left(\prod\limits_{\alpha \in \Sigma_k^+} x_\alpha(a_\alpha)\right) \coloneqq \prod\limits_{\alpha \in \Sigma_k^+} x_\alpha(Xa_\alpha).\]

The first goal of this section is prove the existence of $\sigma(\varpi_1)$-pair for the root system $\Phi$ of type $\rA_\ell$.
The most important case of this construction for us will be $\ell = 3$, because it will be used in the construction $\varpi_k$-pair for other types $\Phi$ later in this subsection.
From now on we assume that the ground ring $A$ is local.

\begin{prop} \label{prop:sigma-construction}
    For a local ring $A$ and $\Phi = \rA_\ell$, $\ell + 1 = n \geq 4$ there exists a $\varpi_1$-pair.
\end{prop}
\begin{proof}
    By the above argument with semidirect products it remains to construct $\sigma(\varpi_1)_{N_0}$ where $N_0 = \St(n, A[X], XA[X])$.
    Denote by $d_1$ the matrix $d_1(X) = \mathrm{diag}(X, 1, \ldots, 1) \in \GL(n, A[X, X\inv])$.
    Now for $u \in \E(n, A[X])$ and $v \in (XA[X])^n$ we can define the map $\sigma(\varpi_1)_{N_0}$ on the generators of $N_{0}$
    using the elements defined in~\cref{subsec:another-presentation}:
    \begin{equation*}
    \sigma(\varpi_1)_{N_0} (F(u, v)) \coloneqq X^{d_1 \cdot X^{-1}}(u, v),\ \sigma(\varpi_1)_{N_0} (S(v, u)) \coloneqq Y^{d_1}(v, u).
    \end{equation*}
    It is clear that the above elements belong to $N_{-\varpi_1}$.
    We need to show that $(\sigma_1)_{N_0}$ respects the relations relations~\eqref{add4}--\eqref{coef-move}.

    For relations~\eqref{add4}--\eqref{add5} this is an immediate corollary of~\cref{itm:xsmall-additivity} and \cref{lem:xy-wd}.
    By~\cref{lem:xy-conj} for $g \in N_{-\varpi_1}$ one has
    \begin{equation}
        \label{eq:xy-conj-n1}
        g \cdot X^{d_1 X^{-1}}(u', v') \cdot g^{-1} = X^{d_1 X^{-1}}(mu', m^*v'), \text{ where } m = d_1 \cdot \pi(g) \cdot d_1^{-1}.
    \end{equation}
    To obtain that $\sigma(\varpi_1)_{N_0}$ respects~\eqref{conj3} it remains to specialize the above equality by setting $g = X^{d_1 X^{-1}}(u, v)$.

    Let us verify that $\sigma(\varpi_1)$ respects~\eqref{coef-move}, i.\,e. that for $a\in XA[X]$ and $m \in \E(n, A[X])$ one has
    $X^{d_1 X^{-1}}(me_1, m^*e_2 a) = Y^{d_1}(me_1 a, m^* e_2)$.
    First, let us verify this in the special case when $m \in \E(n, A[X])$ belongs to the subset
    \[G_0 = H_{12}(A) \cdot U^-_1(A) \cdot U^+_1(A).\]
    Here $H_{12}(A)$ denotes the subgroup of $T(n, A)$ generated by semisimple root elements $h_{12}(u)$, $u \in A^\times$.
    It is easy to check that in this case the only nonzero components of $u = m^* e_2$ are $u_1$ and $u_2$.
    Decompose $v = m e_1$ into a sum $v' + v''$, where $v' = (v_1, v_2, 0, \ldots 0)^t,$ $v'' = (0, 0, v_3, \ldots v_n)^t$.
    Since $v^t u = 0$ we obtain that $v', v'' \in D(u)$.
    Now from Lemmas~\ref{lem:xy-wd}--\ref{lem:xy-conj} we conclude that:
    \begin{multline}
        \label{eq:special-case}
        Y^{d_1}(me_{1}a, m^* e_2) = x(d_1^{-1} \cdot v'a, d_1\cdot  u) \cdot x(d_1^{-1}\cdot v''a, d_1 \cdot u) = \\
        = x(d_1^{-1} va, d_1 \cdot u) = x(d_1^{-1}X \cdot v, d_{1}X^{-1} \cdot u a) = X^{d_1 X^{-1}}(me_1, m^*e_2 a).
    \end{multline}

    To obtain the assertion in the general case notice that for local $A$ the group $\E(n, A)$ admits the following decomposition:
    \[\E(n, A) = \mathrm{EP}_1(A) \cdot H_{12}(A) \cdot U^-_1(A) \cdot U^+_1(A), \]
    where $\mathrm{EP}_1(A)$ is the image of $P_1$ in $\E(n, A)$ under $\pi$.
    This decomposition follows from the Gauss decomposition for Chevalley groups over local rings (see e.\,g. \cite[Theorem~1.1]{Sm12}).

    Thus, we obtain that $\E(n, A[X]) = \pi(N_{\varpi_1}) \cdot G_0$.
    Now factor $m \in \E(n, A[X])$ as $\pi(n) \cdot h$ for some $n\in N_{\varpi_1}$ and $h \in G_0$.
    Since $\pi(n) = d_1 \cdot \pi(g) \cdot d_1^{-1}$ for some $g \in N_{-\varpi_1}$ it remains to apply~\eqref{eq:xy-conj-n1} and~\eqref{eq:special-case}:
    \begin{multline}
        \nonumber X^{d_1 X^{-1}}(me_1, m^*e_{2}a) = {}^{g}(X^{d_1 X^{-1}}(he_1, h^*e_{2}a)) = \\
        = {}^{g}(Y^{d_1}(he_{1}a, h^*e_2)) = Y^{d_1}(me_{1}a, m^{*} e_{2}).
    \end{multline}
\end{proof}

\begin{cor} \label{cor:a3-microweight}
    For a local ring $A$ and $\Phi = \rA_3$ an $\omega$-pair exists for any microweight $\omega \in P(\Phi)$.
\end{cor}
\begin{proof}
    Recall from~\cref{lem:delta-weyl} that it suffices to prove the existence of $\omega$-pair for a representative $\omega$
     of every orbit of the action of $W(\Phi)$ on the set of microweights of $\Phi$.

    It is easy to see that up to the action of the Weyl group every nontrivial microweight in $\rA_3$ coincides with either
    $\pm\varpi_1$, $\pm\varpi_2$ or $\pm\varpi_3$.

    In \cref{prop:sigma-construction} we have proved the existence of $\sigma(\omega)$ for $\varepsilon_1 = \varpi_1$ (and hence also for $\varepsilon_2$ and $\varepsilon_3 = -\varpi_3$ from~\cref{exm:chi-linear}).
    Let us prove the existence of $\sigma(\varpi_2)$.
    The restriction of the homomorphism $\sigma(\varpi_2)$ on $N_0 = \St(4, A[X], XA[X])$ can be defined as the composition of
     $\sigma(\varepsilon_2)_{N_0}$ and $\sigma(\varepsilon_1)$.
    It easy to check that these homomorphisms can be composed correctly and that the image is contained in $N_{-\varpi_2, \Psi}$.
    To define $\sigma(\varpi_2)$ on the whole group $N_{\varepsilon_2, \Psi}$ it remains to use the above argument with semidirect products.
\end{proof}

Now we can prove the main result of this subsection.
\begin{thm} \label{thm:pairconstr}
    Let $\Phi$ be a root system of type $\rD_{\geq 4}$, $\rE_6$ or $\rE_7$ and $A$ be a local ring.
    Denote by $k$ the index of the simple root $\alpha_k$, whose weight $\varpi_k$ is a microweight
     ($k$ is the index of the vertex marked green on the Dynkin diagram from Figure 1 in~\cref{subsec:curtis-tits}).
    Then there exists a $\varpi_k$-pair.
\end{thm}
\begin{proof}
    Applying the semidirect product argument once again we reduce the construction problem for $\sigma(\varpi_k)$ of to the one for $\sigma({\varpi_k})_{N_0}$.
    In our situation \cref{thm:relPres} asserts that $N_0 = \St(\Phi, A[X], XA[X])$ can be decomposed into the free product of subgroups $\St(\Psi, A[X], XA[X])$,
    where $\Psi$ ranges over the set $A_3(\Phi)$ of root subsystems of type $\rA_3$ in $\Phi$, amalgamated over embeddings of generators $z_\alpha(Xf, g)$ into different groups $\St(\Psi, A[X], XA[X])$, whose root system $\Psi \in A_3(\Phi)$ contains $\alpha$.
    Thus, by the universal property of coproducts it remains to construct $\sigma(\varpi_k)_\Psi \colon \St(\Psi, A[X], XA[X]) \to N_{-\varpi_k}$ and then
    verify that these homomorphisms agree on generators $z_\alpha(Xf, g)$ of different $\St(\Psi, A[X], XA[X])$.

    In the case $\Psi \subseteq \Delta_k$ the homomorphism $\sigma(\varpi_k)_\Psi$ should have identical action, so we define it as the canonical map $\St(\Psi, A[X], XA[X]) \to N_{-\varpi_k}$
    induced by the embedding $\Psi \subseteq \Phi$.

    Now consider the nontrivial case $\Psi \not\subseteq \Delta_k$.
    It is clear that there is a unique nonzero weight $\omega \in P(\Psi)$ determined by $(\omega, \alpha) = (\varpi_k, \alpha)$, $\alpha \in \Psi$.
    By definition, the weight $\omega$ is a microweight for $\Psi$.
    By~\cref{cor:a3-microweight} there exists an $\omega$-pair $\xymatrix{ \sigma(\omega)\colon N_{\omega, \Psi} \ar[r] & \ar@<-1.0ex>[l] N_{-\omega, \Psi}\colon \sigma(-\omega) }$.
    The required homomorphisms $\sigma(\varpi_k)_\Psi$ now can be defined as the postcomposition of $\sigma(\omega)$ with the map $N_{-\omega, \Psi} \to N_{-\varpi_k, \Phi}$ induced by the embedding $\Psi \subseteq \Phi$.
    The fact that $\sigma(\varpi_k)_\Psi(z_\alpha(Xf, g))$ does not depend on $\Psi$ is obvious from the construction of $\sigma(\varpi_k)_\Psi$.
\end{proof}