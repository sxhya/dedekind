Throughout this section $\Phi$ denotes a simply laced root system of rank $\ell \geq 3$ and not of type $\rE_8$ and $k$ is the index of
 a simple root $\alpha_k$ such that the corresponding weight $\varpi_k$ is a microweight.
The aim of this section is to prove the existence of a $\varpi_k$-pair for $\Phi$ in the sense of~\cref{dfn:delta-pair}
 under the additional assumption that $A$ is a local ring.

First of all, recall that the ideal $XA[X]$ is a splitting ideal of $A[X]$, therefore
$\St(\Phi, A[X])$ decomposes as $\St(\Phi, A[X], XA[X]) \rtimes \St(\Phi, A)$.
Since $\varpi_k$ is a microweight subgroups $N_{\pm \varpi_k}$ both contain $\St(\Phi, A[X], XA[X])$.
In fact, $N_{\varpi_k}$ (resp. $N_{-\varpi_k}$) decomposes into the semidirect product of $N_0 := \St(\Phi, A[X], XA[X])$ and the parabolic subgroup $P_k^+ \leq \St(\Phi, A)$ (resp. $P_k^-\leq \St(\Phi, A)$).
Recall that $P_k^\pm$ is generated by $x_\alpha(a)$, $a \in A$ for all $\alpha \in \Delta \sqcup \Sigma^\pm_k$.

\begin{lemma} \label{lem:first-reduction}
 Let $\omega = \pm \varpi_k$ be a microweight.
 Suppose that there exists a restriction of $\sigma(\omega)$ to $N_0$ i.\,e. a homomorphism
 $\sigma(\omega)_{N_0} \colon N_0 \to N_{-\omega}$
 then there
 For microweight $\varpi_k$ to construct $\sigma(\varpi_k) \colon N_{\varpi_k} \to N_{-\varpi_k}$
 (resp. $\sigma(-\varpi_k)\colon N_{-\varpi_k} \to N_{\varpi_k}$) it is enough to construct the restriction of $\sigma(\varpi_k)$ to $N_0$.
\end{lemma}
\begin{proof}

    The universal property of semidirect products gives for any group $H$ acting on a group $N$
    and any group homomorphisms $f_N\colon N \to G$, $f_H\colon H \to G$ satisfying
    \begin{equation}
        \label{eq:coherence-condition} f_N({}^hn) = {}^{f_H(h)} f_N(n),\ n\in N,\ h\in H
    \end{equation}
    a unique map $f\colon N \rtimes H \to G$ extending $f_N$ and $f_H$.

    By leveraging the universal property, the problem of constructing \( \sigma(\varpi_k) \colon N_{\varpi_k} \to N_{-\varpi_k} \)
    (resp. $\sigma(-\varpi_k) \colon N_{-\varpi_k} \to N_{\varpi_k}$)
    reduces to solving the construction problems for \( \sigma(\varpi_k)_{P_k^+} \) (resp. $\sigma(-\varpi_k)_{P_k^-}$) and \( \sigma(\pm\varpi_k)_{N_0} \).

    First, let us solve the former easier problem.
    From the Levi decomposition~\eqref{eq:levi-decomp}, we know that \( P_k^\pm = L_k \ltimes U_k^\pm \), where \( L_k = \mathrm{Im}(\St(\Delta_k, A) \to \St(\Phi, A)) \) and \( U_k^\pm = \UU(\Sigma_k^\pm, A) \).
    This allows us to apply the universal property once more, further reducing the construction of \( \sigma(\varpi_k)_{P_k^+} \) (resp. $\sigma(-\varpi_k)_{P_k^-}$)
    to constructing \( \sigma(\varpi_k)_{L_k} \) and \( \sigma(\varpi_k)_{U_k^+} \) (resp. \( \sigma(-\varpi_k)_{L_k} \) and \( \sigma(-\varpi_k)_{U_k^-} \)).
    The latter homomorphisms can now be explicitly defined as follows:
    \begin{equation} \label{eq:sigma-Pk} \sigma(\varpi_k)_{L_k} \coloneqq \mathrm{id}_{L_k}, \quad \sigma(\varpi_k)_{U_k} \left(\prod_{\alpha \in \Sigma_k^+} x_\alpha(a_\alpha)\right) \coloneqq \prod_{\alpha \in \Sigma_k^+} x_\alpha(Xa_\alpha).\end{equation}
    These homomorphisms obviously satisfy~\eqref{eq:coherence-condition}, so the construction of \( \sigma(\varpi_k)_{P_k^+} \) is complete.
    The construction of \( \sigma(-\varpi_k)_{P_k^-} \) is analogous.
\end{proof}

Our next goal is to prove the existence of an $\varpi_1$-pair for the root system $\Phi$ of type $\rA_\ell$.
The following proposition is based on the construction described at the beginning of~\cite[\S~3]{Tu83}.
Its key novelty is that it also addresses the case \( \ell = 3 \) omitted in~\cite{Tu83}, which will be crucial for the subsequent construction of \( \varpi_k \)-pairs for other types of \( \Phi \) later in this subsection.
For the remainder of this subsection we assume that the ground ring $A$ is local.

\begin{prop} \label{prop:sigma-construction}
    For a local ring $A$ and $\Phi = \rA_\ell$, $\ell + 1 = n \geq 4$ there exists an $\varpi_1$-pair.
\end{prop}

By the preceding argument with semidirect products, it remains to construct $\sigma(\pm\varpi_1)_{N_0}$ where $N_0 = \St(n, A[X], XA[X])$.
Let \( d_1 \) denote the matrix $d_1(X) = \mathrm{diag}(X, 1, \ldots, 1) \in \GL(n, A[X, X\inv])$.

Define $R = A[X],$ $S = A[X, X\inv]$, $I = XA[X].$
For $u \in \E(n, R) \cdot e_1$ and $v \in I^n$ we can define homomorphisms $\sigma(\pm\varpi_1)_{N_0}$ on the generators of $N_{0}$ using the elements defined in~\cref{dfn:xy-def}:
\begin{align}
    \sigma(\varpi_1)_{N_0} \left(F(u, v)\right) \coloneqq X^{d_1^{-1}}(u, v), & \quad \sigma(\varpi_1)_{N_0} \left(S(v, u)\right) \coloneqq Y^{d_1^{-1} \cdot X}(v, u), \label{eq:def-sigma-1} \\
    \sigma(-\varpi_1)_{N_0} \left(F(u, v)\right) \coloneqq X^{d_1 \cdot X^{-1}}(u, v),& \quad \sigma(-\varpi_1)_{N_0} \left(S(v, u)\right) \coloneqq Y^{d_1}(v, u). \label{eq:def-sigma-2}
\end{align}
It is easy to see that the above definitions are correct, i.\,e. all conditions on $u$, $v$ and $d_1$ formulated in~\cref{dfn:xy-def} are satisfied.
To verify that $\sigma(\varpi_1)_{N_0}$ lands where it should we need the following lemma.
\begin{lemma}\label{lem:sigma-N0-image}
    The image of $\sigma(\varpi_1)_{N_0}$ is contained in $U_1^- \ltimes N_0 \leq N_{-\varpi_1}$.
\end{lemma}
\begin{proof}
Indeed, it suffices to show that $\mathrm{ev}_{X=0}^*(X^{d_1^{-1}}(u, v))$ and $\mathrm{ev}_{X=0}^*(Y^{d^{-1}_1 \cdot X}(v, u))$ belong to $U_1^-$.
It is clear that after expanding definitions, every factor $x(v, w)$ appearing in the right-hand sides of either~\eqref{eq:X-def} or~\eqref{eq:Y-def}
satisfies the following property: both the coordinate $v_1$ and the coordinates $w_2, \ldots, w_n$ are all divisible by $X$.
Consequently, $\mathrm{ev}_{X=0}^* \left(x(v, w)\right) \in U_1^-$, proving the assertion.
\end{proof}

\begin{proof}[Proof of~\cref{prop:sigma-construction}]
    We need to show that $\sigma(\varpi_1)_{N_0}$ respects the relations~\eqref{add4}--\eqref{coef-move}.
    For relations~\eqref{add4}--\eqref{add5} this is an immediate corollary of~\cref{itm:xsmall-additivity} and \cref{lem:xy-wd}.
    By~\cref{lem:xy-conj} for $g \in N_{-\varpi_1}$ one has
    \begin{equation}
        \label{eq:xy-conj-n1}
        g \cdot X^{d_1^{-1}}(u', v') \cdot g^{-1} = X^{d_1^{-1}}(mu', m^*v'), \text{ where } m = d_1^{-1} \cdot \pi(g) \cdot d_1.
    \end{equation}
    To obtain that $\sigma(\varpi_1)_{N_0}$ respects~\eqref{conj3} it remains to specialize the above equality setting $g = X^{d_1^{-1}}(u, v)$.

    Let us verify that $\sigma(\varpi_1)_{N_0}$ respects~\eqref{coef-move}, i.\,e. that for $a\in XA[X]$ and $m \in \E(n, A[X])$ one has
    $X^{d_1^{-1}}(me_1, m^*e_2 a) = Y^{d_1^{-1} \cdot X}(me_1 a, m^* e_2)$.
    First, we verify this in the special case when $m \in \E(n, A[X])$ belongs to the subset
    \[G_0 = H_{12} \cdot \mathrm{EU}^-_1 \cdot \mathrm{EU}^+_1,\]
    where $H_{12}$ denotes the subgroup of diagonal matrices $T(n, A)$ generated by semisimple elements $\pi(h_{12}(u))$, $u \in A^\times$
     and $\mathrm{EU}^\pm_1$ are (isomorphic) images of $U^\pm_1$ under homomorphism $\pi$ from~\eqref{eq:K1-K2-sequence}.

    It is easy to check that in this case the only nonzero components of $u = m^* e_2$ are $u_1$ and $u_2$.
    Decompose $v = m e_1$ into a sum $v' + v''$, where $v' = (v_1, v_2, 0, \ldots 0)^t,$ $v'' = (0, 0, v_3, \ldots v_n)^t$.
    Since $v^t u = 0$ we obtain that $v', v'' \in D(u)$.
    Now from Lemmas~\ref{lem:xy-wd}--\ref{lem:xy-conj} we obtain for any $a \in XA[X]$ that
    \begin{multline}
        \label{eq:special-case}
        X^{d_1^{-1}}(me_1, m^*e_2 a) =
        x(d_1 \cdot v, d_{1}^{-1} \cdot u a) =
        x(d_1 \cdot X^{-1} va, d_1^{-1} \cdot X u) = \\
        = x(d_1 X^{-1} \cdot v'a, d_1^{-1}\cdot X u) \cdot x(d_1 X^{-1}\cdot v''a, d_1^{-1} \cdot X u) =
        Y^{d_1^{-1} \cdot X}(me_{1}a, m^* e_2)
    \end{multline}

    To obtain the assertion in the general case notice that for local $A$ the group $\E(n, A)$ admits the following decomposition:
    \[\E(n, A) = \mathrm{EP}_1^+ \cdot H_{12} \cdot \mathrm{EU}^-_1 \cdot \mathrm{EU}^+_1, \]
    where $\mathrm{EP}_1^+$ is the image of $P_1^+$ in $\E(n, A)$ under $\pi$.
    This decomposition follows from the Gauss decomposition for Chevalley groups over local rings (see e.\,g. \cite[Theorem~1.1]{Sm12}).

    Thus, we obtain that $\E(n, A[X]) = \pi(N_{\varpi_1}) \cdot G_0$.
    Now factor $m \in \E(n, A[X])$ as $\pi(n) \cdot h$ for some $n\in N_{\varpi_1}$ and $h \in G_0$.
    Since $\pi(n) = d_1^{-1} \cdot \pi(g) \cdot d_1$ for some $g \in N_{-\varpi_1}$ it remains to apply~\eqref{eq:xy-conj-n1}, \eqref{eq:special-case} and~\cref{lem:xy-conj}:
    \begin{multline}
        \nonumber X^{d_1^{-1}}(me_1, m^*e_{2}a) = {}^{g}(X^{d_1^{-1}}(he_1, h^*e_{2}a)) = \\
        = {}^{g}(Y^{d_1^{-1} \cdot X}(he_{1}a, h^*e_2)) = Y^{d_1^{-1} \cdot X}(me_{1}a, m^{*} e_{2}).
    \end{multline}
    Homomorphisms $\sigma(\varpi_1)_{N_0}$ and $\sigma(\varpi_1)_{P_1^+}$ satisfy~\eqref{eq:coherence-condition} by~\cref{lem:xy-conj}.
    The construction of $\sigma(\varpi_1)$ is now complete.
    The argument for $\sigma(-\varpi_1)$ is analogous.
\end{proof}

\begin{cor} \label{cor:a3-microweight}
    For a local ring $A$ and $\Phi = \rA_3$ an $\omega$-pair exists for any microweight $\omega \in P(\Phi)$.
\end{cor}
\begin{proof}
    Recall from~\cref{lem:delta-weyl} that it suffices to prove the existence of an $\omega$-pair for a representative $\omega$
    of each orbit under the action of $W(\Phi) \cong S_4$ on the set of microweights of $\Phi$.

    It is straightforward to observe that, up to the action of the Weyl group $S_4$, every nontrivial microweight in $\rA_3$ coincides with
    $\varpi_1$, $\varpi_2$, or $\varpi_3$.

    In \cref{prop:sigma-construction}, we established the existence of an $\omega$-pair for $\omega = \varpi_1$
    (and by symmetry, also for $\omega = -\varpi_3$, cf.~\cref{exm:chi-linear}). Thus, it remains to demonstrate the existence of $\sigma(\varpi_2)$.

    We will adopt the same strategy as in the proof of~\cref{prop:sigma-construction}, beginning by defining the restriction of $\sigma(\varpi_2)$ to $N_0 = \St(4, A[X], XA[X])$.
    Specifically, we define $\sigma(\varpi_2)_{N_0}$ as the composition
    $\sigma(\varpi_1) \cdot \sigma((12) \cdot \varpi_1)_{N_0}$.
    By~\cref{lem:sigma-N0-image} and~\eqref{eq:sigma-gen}, the image of $\sigma((12) \cdot \varpi_1)_{N_0}$ is contained in
    \[{}^{w_{12}(1)}U_1^- \ltimes N_0 \leq N_{\varpi_1},\] so this definition is correct.
    Using~\eqref{eq:sigmadef} and the identity
    $\varpi_2 = (12) \cdot \varpi_1 + \varpi_1$ it is easy to show that the image of $\sigma(\varpi_2)_{N_0}$ is contained in $N_{-\varpi_2}$.

    To extend the definition of $\sigma(\varpi_2)$ to the entire group $N_{\varpi_2}$, it remains to apply the universal property of semidirect products to
    $\sigma(\varpi_2)_{N_0}$ and the homomorphism $\sigma(\varpi_2)_{P_2}^+$ constructed in~\eqref{eq:sigma-Pk}.
\end{proof}

Now we can prove the main result of this subsection.
\begin{thm} \label{thm:pairconstr}
    Let $\Phi$ be a root system of type $\rD_{\geq 4}$, $\rE_6$ or $\rE_7$ and $A$ be a local ring.
    Denote by $k$ the index of the simple root $\alpha_k$, whose weight $\varpi_k$ is a microweight
     ($k$ is the index of the vertex marked green on the Dynkin diagram from Figure 1 in~\cref{subsec:curtis-tits}).
    Then there exists a $\varpi_k$-pair.
\end{thm}
\begin{proof}
    Applying the semidirect product argument once again we reduce the construction problem for $\sigma(\varpi_k)$ of to the one for $\sigma({\varpi_k})_{N_0}$.
    In our situation \cref{thm:relPres} asserts that $N_0 = \St(\Phi, A[X], XA[X])$ can be decomposed into the free product of subgroups $\St(\Psi, A[X], XA[X])$,
    where $\Psi$ ranges over the set $A_3(\Phi)$ of root subsystems of type $\rA_3$ in $\Phi$, amalgamated over embeddings of generators $z_\alpha(Xf, g)$ into different groups $\St(\Psi, A[X], XA[X])$, whose root system $\Psi \in A_3(\Phi)$ contains $\alpha$.
    Thus, by the universal property of coproducts it remains to construct $\sigma(\varpi_k)_\Psi \colon \St(\Psi, A[X], XA[X]) \to N_{-\varpi_k}$ and then
    verify that these homomorphisms agree on generators $z_\alpha(Xf, g)$ of different $\St(\Psi, A[X], XA[X])$.

    In the case $\Psi \subseteq \Delta_k$ the homomorphism $\sigma(\varpi_k)_\Psi$ should have identical action, so we define it as the canonical map $\St(\Psi, A[X], XA[X]) \to N_{-\varpi_k}$
    induced by the embedding $\Psi \subseteq \Phi$.

    Now consider the nontrivial case $\Psi \not\subseteq \Delta_k$.
    It is clear that there is a unique nonzero weight $\omega \in P(\Psi)$ determined by $(\omega, \alpha) = (\varpi_k, \alpha)$, $\alpha \in \Psi$.
    By definition, the weight $\omega$ is a microweight for $\Psi$.
    By~\cref{cor:a3-microweight} there exists an $\omega$-pair $\xymatrix{ \sigma(\omega)\colon N_{\omega, \Psi} \ar[r] & \ar@<-1.0ex>[l] N_{-\omega, \Psi}\colon \sigma(-\omega) }$.
    The required homomorphisms $\sigma(\varpi_k)_\Psi$ now can be defined as the postcomposition of $\sigma(\omega)$ with the map $N_{-\omega, \Psi} \to N_{-\varpi_k, \Phi}$ induced by the embedding $\Psi \subseteq \Phi$.
    The fact that $\sigma(\varpi_k)_\Psi(z_\alpha(Xf, g))$ does not depend on $\Psi$ is obvious from the construction of $\sigma(\varpi_k)_\Psi$.
\end{proof}