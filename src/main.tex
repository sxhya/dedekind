\documentclass[oneside, 10pt]{amsart}
\usepackage{amscd}
\usepackage{amsfonts}
\usepackage{amsmath}
\usepackage{amsthm, amstext, amssymb, comment, enumitem, verbatim, mathtools, xfrac, microtype, nameref, thmtools, tikz, tikz-cd}
\usepackage[breaklinks=true]{hyperref}
\usepackage[capitalize]{cleveref}
\usepackage[matrix,arrow,curve]{xy}
\usepackage[notref,notcite]{showkeys}
\usepackage{longtable}
\usetikzlibrary{shapes, arrows, calc, arrows.meta, fit, positioning, decorations.markings}
\usepackage[natbib=true, backend=biber, firstinits=true, style=numeric, sortlocale=en_US, url=false, doi=false, eprint=true, maxbibnames=4]{biblatex}
\usepackage{mathtools}
\usepackage{tikz}
\addbibresource{main.bib}
\bibliographystyle{authordate1}
\renewbibmacro*{volume+number+eid}{\ifentrytype{article}{\- \iffieldundef{volume}{}{Vol.~\printfield{volume},}\iffieldundef{number}{}{ No.~\printfield{number},}}}
\renewbibmacro{in:}{\ifentrytype{article}{}{\printtext{\bibstring{in}\intitlepunct}}}
\newbibmacro{string+doi}[1]{\iffieldundef{doi}{\iffieldundef{url}{#1}{\href{\thefield{url}}{#1}}}{\href{https://dx.doi.org/\thefield{doi}}{#1}}}
\DeclareFieldFormat[article, inproceedings, inbook, book, online]{title}{\usebibmacro{string+doi}{\mkbibquote{#1}}}
\renewcommand*{\bibfont}{\footnotesize}

\DeclareMathOperator{\St}{St}
\DeclareMathOperator{\GL}{GL}
\DeclareMathOperator{\E}{E}
\DeclareMathOperator{\G}{G}
\DeclareMathOperator{\UU}{U}
\newcommand{\Gsc}{\G_\mathrm{sc}}
\newcommand{\Esc}{\E_\mathrm{sc}}
\DeclareMathOperator{\Hom}{Hom}
\DeclareMathOperator{\Um}{Um}
\DeclareMathOperator{\colim}{colim}
\DeclareMathOperator{\Aut}{Aut}
\DeclareMathOperator{\K}{K}
\DeclareMathOperator{\KO}{KO}
\DeclareMathOperator{\rk}{rk}
\DeclareMathOperator{\sr}{sr}
\DeclareMathOperator{\Ker}{Ker}
\DeclareMathOperator{\Stab}{Stab}
\DeclareMathOperator{\Img}{Im}
\newcommand{\catname}[1]{{\normalfont\textbf{#1}}}

\newcommand{\rA}{\mathsf{A}}
\newcommand{\rB}{\mathsf{B}}
\newcommand{\rC}{\mathsf{C}}
\newcommand{\rD}{\mathsf{D}}
\newcommand{\rE}{\mathsf{E}}


\newcommand{\ZZ}{\mathbb{Z}}
\newcommand{\StW}{\widehat{W}}
\newcommand{\StH}{\widehat{H}}
\newcommand{\inv}{^{-1}}
\newcommand{\RNum}[1]{\uppercase\expandafter{\romannumeral #1\relax}}

\newcommand{\card}[1]{
    \begin{tikzpicture}
    \draw (0,0) rectangle (40mm, 40mm);
    \node (title) at (20mm, 35mm){ #1 };
    \end{tikzpicture}
}

\newtheorem{thm}{Theorem}
\Crefname{thm}{Theorem}{Theorems}
\numberwithin{equation}{section}

\newtheorem{lemma}{Lemma}
\numberwithin{lemma}{section}
\Crefname{lemma}{Lemma}{Lemmas}
\newlist{lemlist}{enumerate}{1} \setlist[lemlist]{label=\textrm{(\arabic{lemlisti})}, ref=\thelemma.(\arabic{lemlisti}),noitemsep} \Crefname{lemlisti}{Lemma}{Lemma}

\newtheorem{cor}[lemma]{Corollary}
\Crefname{cor}{Corollary}{Corollaries}

\newtheorem{prop}[lemma]{Proposition}
\Crefname{prop}{Proposition}{Propositions}

\newtheorem*{thm*}{Theorem}
\newtheorem*{lemma*}{Lemma}

\theoremstyle{definition}

\newtheorem{dfn}[lemma]{Definition}
\Crefname{dfn}{Definition}{Definitions}
\newtheorem{example}[lemma]{Example}
\newtheorem{problem}[lemma]{Problem}
\Crefname{example}{Example}{Examples}

\theoremstyle{remark}

\newtheorem{rem}[lemma]{Remark}
\Crefname{rem}{Remark}{Remarks}

\title{On the $\mathbb{A}^1$-invariance of $K_2$ \RNum{2}: exceptional groups and Dedekind domains}
\keywords {Steinberg group, $K_2$-functor, Serre problem {\em Mathematical Subject Classification (2010):} 19C20}
\author {Sergei Sinchuk}
\address{Ludwigs Maximilian Universität München}
\email {sinchukss at gmail.com}
\date {\today}

\begin{document}
    \maketitle

    \section{Introduction}\label{sec:introduction}

    As a corollary we obtain the following result analogous to~\cite[Theorem~1.1]{St-Ded}:
    \begin{cor}
        \label{cor:dedekind}
        Let $\Phi$ and $A$ be as above.
        Assume additionally that $A$ is a regular domain.
        Then $\K_2(A)$ surjects onto $\K_2(\Phi, A[X_1,\ldots, X_n])$.
    \end{cor}
    \begin{proof}
        Consider the following diagram:
        \[ \K_2(\rA_3, A) \to \K_2(\rA_4, A) \to \K_2(\rA_4, A[X_1, \ldots, X_n]) \rightarrow \K_2(A[X_1, \ldots, X_n]) \to \K_2(A).\]
        The two right arrows on the above diagram are isomorphisms by the $\mathbb{A}^1$-invariance of the stable $\K_2$
        (see e.\,g. \cite[Theorem~V.6.3]{Kbook}) combined with the main result of~\cite{Tu83}.
        On the other hand, by~\cite[Corollary~3.2]{ST76} the left arrow and the composite arrow are isomorphisms.
        Our assertion now follows from~\cref{thm:early-stability}.
    \end{proof}


    \section{Preliminaries}\label{sec:preliminaries}
    In this paper all rings are assumed to be commutative and all commutators are left-normed i.\,e. $[a, b] = a b a^{-1} b^{-1}$.

\subsection{Formalism of triples}\label{subsec:triples}

Let $N$ be a group acting on itself by right conjugation.
Let $M$ be a group with a right action of $N$.
Recall that a group homomorphism $\varphi\colon M \to N$ is called a \textit{precrossed module} if $\varphi$ preserves the action of $N$, i.\,e.
\[\varphi(m^n) = \varphi(m)^n, \text{for all $m \in M$, $n\in N;$} \]
If, in addition, $\varphi$ satisfies the so-called \textit{Peiffer identity}, i.\,e.
\[{m}^{\varphi(m')} = {m'}^{-1} m m', \text{for all $m, m' \in M$,}\]
then $\varphi$ is called a \textit{crossed module}.

Let $\varphi\colon M \to N$ and $\varphi' \colon M' \to N'$ be a pair of precrossed modules.
A \textit{map of precrossed modules} $(f, g)\colon \varphi \to \varphi'$ is, by definition, any pair of homomorphisms
$f \colon M \to M'$, $g \colon N \to N'$ such that ${f(m)}^{g(n)} = f(m^n)$ for all $n \in N$, $m \in M$.

Now suppose that we are given the following cube-like commutative diagram of abstract groups:
\begin{equation} \label{eq:cube} \begin{split} \xymatrix{
    G_{123} \ar[rr]_{f_{23}} \ar[dd]_{f_{12}} \ar[rd]_{f_{13}} &                        & G_{23} \ar@{-->}[dd]^(.3){g_2^3} \ar[rd]^{g^2_3} &           \\
    & G_{13} \ar[rr]^(.3){g^1_3} \ar^(.3){g^3_1}[dd] &                   & G_3 \ar[dd]_{h_3} \\
    G_{12} \ar@{-->}[rr]^(.3){g_2^1} \ar[rd]_{g_1^2}          &                        & G_2 \ar@{-->}[rd]_{h_2}         &           \\
    & G_1 \ar[rr]_{h_1}              &                   & G.} \end{split} \end{equation}
Additionally, suppose that $g_1^3$ is a precrossed module, $h_3$ is a crossed module and, moreover, that $(g_3^1, h_1) \colon g_1^3 \to h_3$ is a map of precrossed modules.

Set $V = G_1 \times G_2 \times G_3$, $W = G_{12} \times G_{13} \times G_{23}$.
We define an operation $\star \colon W \times V \to V$ as follows. For $v = (x, y, z) \in V$ and $(a, b, c) \in W$ we set
\[(a, b, c) \star (x, y, z) = (x \cdot g_1^3(b) \cdot g_1^2(a),\ g_2^1(a)^{-1} \cdot y \cdot g_2^3(c),\ g_3^2(c)^{-1} \cdot g_3^1(b)^{-h_2(y)} \cdot z).\]

\begin{lemma} For every $v \in V$ one has
\begin{equation*}(a', b', c') \star \left( (a, b, c) \star v \right) = (a \cdot a', b \cdot {b'}^{g_1^2(a)^{-1}}, c \cdot c') \star v.\end{equation*}
\end{lemma}
\begin{proof}
    Set $v'=(x', y', z') = (a, b, c) \star v$ and $(x'', y'', z'') = (a', b', c') \star v'$.
    Since $g_1^3$ is a precrossed module we immediately obtain that
    \begin{align*}
        x'' =& x' \cdot g_1^3(b') \cdot g_1^2(a') = x \cdot g_1^3(b) \cdot g_1^2(a) \cdot g_1^3(b') \cdot g_1^2(a') = x \cdot g_1^3(b \cdot b'^{g_1^2(a)^{-1}}) g_1^2(a \cdot a'),\\
        y'' =& g_2^1(a')^{-1} \cdot g_2^1(a)^{-1} \cdot y \cdot g_2^3(c) \cdot g_2^3(c') = g_2^1(a\cdot a')^{-1} \cdot y \cdot g_2^{3}(c\cdot c'). \end{align*}
    Since $(g_3^1, h_1)$ is a map of precrossed modules, for every $a \in G_{12}$, $b \in G_{13}$ one has $g_3^1(b^{g_1^2(a)}) = g_3^1(b)^{h_1 g_1^2(a)} = g_3^1(b)^{h_2g_2^1(a)}$.
    Since $h_3$ satisfies Peiffer identity, for every $c \in G_{23}$ and $z \in G_3$ one has $z ^{h_2 g_2^3(c)} = z^{ h_3 g_3^2(c)} = g_3^2(c)^{-1} \cdot z \cdot g_3^2(c)$.
    Using these identities we obtain that
    \begin{multline*}
        z'' = g_3^2(c')^{-1} \cdot g_3^1(b')^{-h_2(y')} \cdot z' = \\
        = g_3^2(c')^{-1} \cdot g_3^1(b')^{- h_2 \left( g_2^1(a)^{-1} \cdot y \cdot g_2^3(c) \right)} g_3^2(c)^{-1} \cdot g_3^1(b)^{-h_2(y)} \cdot z = \\
        = g_3^2(c \cdot c')^{-1} \cdot g_3^1(b')^{- h_2 \left( g_2^1(a)^{-1} \cdot y \right)} \cdot g_3^1(b)^{-h_2(y)} \cdot z = \\
        = g_3^2(c \cdot c')^{-1} \cdot g_3^1(b \cdot {b'} ^ {g_1^2(a)^{-1}})^{- h_2 \left( y \right)} \cdot z. \qedhere
    \end{multline*}
\end{proof}

The above lemma allows us to define an equivalence relation on the set $V$ in the following fashion.
We declare two elements $v, v' \in V$ congruent (denoted $v \sim v'$) if $v' = w \star v$ for some triple $(a, b, c) \in W$.

\begin{lemma}\label{one-one-z} Assume that the back face $(f_{12}, f_{23}, g_2^1, g_2^3)$ of~\eqref{eq:cube} is a pullback square.
Assume additionally that the triple $(1, 1, 1)$ is congruent to $(1, 1, z)$ for some $z\in G_3$.
Then $z \in g_3^1(\Ker(g_1^3)).$ \end{lemma}
\begin{proof} By the definition of congruence relation there exists $(a, b, c)\in W$ such that
\[ (a, b, c) \star (1, 1, 1) = ( g_1^3(b) \cdot g_1^2(a),\ g_2^1(a)^{-1} \cdot g_2^3(c),\ g_3^2(c)^{-1} \cdot g_3^1(b)^{-1}) = (1,1,z). \]
By the Lemma's assumption there exists $e \in G_{123}$ such that $f_{12}(e) = a$, $f_{23}(e) = c$, hence
\[ 1 = g_1^3(b) \cdot g_1^2(f_{12}(e)) = g_1^3(b \cdot f_{13}(e)),\ z = g_3^2(f_{23}(e))^{-1} \cdot g_3^1(b)^{-1} = g_3^1(b \cdot f_{13}(e))^{-1}. \qedhere\] \end{proof}

\subsection{Steinberg groups, $\K_2$-groups and symbols}\label{subsec:steinberg-preliminaries}
Let $\Phi$ be a root system of rank $\ell \geq 1$.
We assume that $\Phi$ is embedded into $\mathbb{R}^\ell$ whose scalar product we denote by $(\text{-}, \text{-})$.
We also fix some system of simple roots $\Pi = \{\alpha_1, \ldots, \alpha_\ell\} \subset \Phi$.
For a root $\alpha\in\Phi$ we denote by $m_i(\alpha)$ the $i$-th coefficient in the expansion of $\alpha$ in $\Pi$,
i.\,e. $\alpha = \sum_{i=1}^n m_i(\alpha) \alpha_i$.

We denote by $\Phi^\vee$ the corresponding dual root system, which, by definition, consists of all coroots $\alpha^\vee = \frac{2}{(\alpha, \alpha)} \alpha$, where $\alpha \in \Phi$.
We denote by $P(\Phi^\vee)$ the integral lattice spanned by the \emph{fundamental coweights $\varpi_i^\vee$}.
Recall that the fundamental coweights $\varpi_i^\vee$ are uniquely determined by the property $(\varpi_i^\vee, \alpha_j) = \delta_{ij}$.
For $\alpha,\beta \in \Phi$ we denote by $\langle \alpha, \beta \rangle$ the integer $(\alpha, \beta^\vee) = \frac{2(\alpha, \beta)}{(\beta, \beta)}$.

Now let $R$ be an arbitrary commutative ring with $1$ and suppose that $\Phi$ is an irreducible root system of rank $\geq 2$.
Recall that to the pair $(\Phi, R)$ can associate an abstract group $\St(\Phi, R)$, called the \textit{Steinberg group} of type $\Phi$ over $R$.
By definition, $\St(\Phi, R)$ is the group presented by generators $x_\alpha(a)$, $a \in R$, $\alpha \in \Phi$ and an explicit list of relations (see e.\,g in.~\cite{Ma69, Re75, St71}).
In this paper we may restrict ourselves to the case when the root system $\Phi$ in question is \textit{simply-laced} (i.\,e. has type $\mathsf{ADE}$),
 in which case the defining relations of $\St(\Phi, R)$ reduce to the following shorter list:
\begin{align}
x_{\alpha}(a)\cdot x_{\alpha}(b)&=x_{\alpha}(a+b), \tag{R1} \label{x-additivity}\\
[x_{\alpha}(a),\,x_{\beta}(b)]  &=x_{\alpha+\beta}(N_{\alpha,\beta} \cdot ab),\text{ for }\alpha+\beta\in\Phi, \tag{R2} \label{R2} \\
[x_{\alpha}(a),\,x_{\beta}(b)]  &=1,\text{ for }\alpha+\beta\not\in\Phi\cup0. \tag{R3} \label{R3}
\end{align}
The coefficients $N_{\alpha,\beta}$ in the above formula are integers equal to $\pm 1$, they coincide with the structure constants of the complex Lie algebra of type $\Phi$.

Throughout this paper we denote by $\Gsc(\Phi, R)$ the group of points of the simply-connected Chevalley--Demazure group of type $\Phi$ over $R$.
We denote by $\Esc(\Phi, R)$ the \textit{elementary subgroup} of $\Gsc(\Phi, R)$, i.\,e. the subgroup generated by elementary root unipotents of $\Gsc(\Phi, R)$.
Notice that in~\cite{VP, Vav09} the notation $x_\alpha(a)$ is used to denote the elementary root unipotents.
To prevent confusion we will use different notation $t_\alpha(a)$ for them and reserve the notation $x_\alpha(a)$ solely for generators of Steinberg groups.

Recall that the map sending $x_\alpha(a)$ to $t_\alpha(a)$ gives rise to a well-defined homomorphism $\pi \colon \St(\Phi, R) \to \G_\mathrm{sc}(\Phi, R)$, see~\cite[\S~1A]{St78}.
The cokernel and the kernel of $\pi$ are called \textit{the unstable $\K_1$- and $\K_2$-functors modeled on the root system $\Phi$}:
\begin{equation} \label{eq:K1-K2-sequence}
  \xymatrix{ 1 \ar[r] & \K_2(\Phi, R) \ar[r] & \St(\Phi, R) \ar[r]^{\pi} & \Gsc(\Phi, R) \ar[r] & \K_1(\Phi, R) \ar[r] & 1}
\end{equation}

Following~\cite{Ma69} for $\alpha\in\Phi$ and $u \in R^\times$ we define the following elements of $\St(\Phi, R)$:
\begin{align*} w_\alpha(u) & =  x_\alpha(u) \cdot x_{-\alpha}(-u^{-1}) \cdot x_\alpha(u), \\
               h_\alpha(u) & =  w_\alpha(u) \cdot w_\alpha(-1).  \end{align*}
The subgroup generated by $w_\alpha(u)$ (resp. $h_\alpha(u)$) for all $\alpha\in \Phi$, $u \in R^\times$ is denoted by $\StW(\Phi, R)$ (resp. $\StH(\Phi, R)$).
By~\cite[Lemme~5.2]{Ma69} $\StH(\Phi, R)$ is a normal subgroup of $\StW(\Phi, R)$.

In the sequel we will need the following explicit elements of the group $\K_2(\Phi, R)$.
Recall that for arbitrary $u, v \in R^\times$ one defines the \textit{Steinberg symbol} via the formula
\begin{equation} \label{eq:steinberg} \{ u, v \}_\alpha = h_\alpha(uv) \cdot h_\alpha^{-1}(u) \cdot h_\alpha^{-1}(v). \end{equation}
Recall also from~\cite[Lemme~5.4]{Ma69} that
\begin{equation} \label{eq:steinberg-2} [h_\alpha(u), h_\beta(v)] = \{u, v^{\langle \alpha, \beta \rangle}\}_\alpha. \end{equation}
Steinberg symbols depend only on the length of the root $\alpha$.
In particular, in the case when $\Phi$ is of simply-laced type, they do not depend on the choice of $\alpha$, which allows us to omit it from notation.

Steinberg symbols are central elements of $\St(\Phi, R)$.
Our assumptions on $\Phi$ guarantee that Steinberg symbols are antisymmetric and bimultiplicative, i.\,e. they satisfy the following identities:
\begin{equation} \label{eq:symbol-properties} \{ u, st \} = \{ u, s\} \{ u, t \}, \ \{ u, v \} = \{ v, u\}^{-1}. \end{equation}

In this paper we will use also use the concept of a \textit{relative Steinberg group} introduced by F.~Keune and J.-L.~Loday in~\cite{Ke78, Lo78}.
We will only briefly mention the definition and basic properties of these groups and refer the reader to~\cite[\S~2.3]{LS20} for a more detailed exposition.

Let $R$ be a commutative ring, $I \trianglelefteq R$ be an ideal and let $p$ denote the canonical projection $R \to R/I$.
Denote by $D_{R, I}$ the pullback of two copies of $p$ i.\,e. the ring $R \times_{R/I} R$.
Elements of $D_{R, I}$ are pairs $(a; b)$ such that $a-b \in I$.
We also denote by $p_1$, $p_2$ the canonical projections $D_{R, I} \to R$ and by $p_1^*$, $p_2^*$ the corresponding homomorphisms of Steinberg groups induced by them.
Recall from~\cite[Definition~2.5]{LS20} that the relative Steinberg group $\St(\Phi, R, I)$ is defined as the quotient
 $\Ker(p_1^*) / C$, where $C = [\Ker(p_1^*), \Ker(p_2^*)]$.
If we denote by $\mu$ the homomorphism $\St(\Phi, R, I) \to \St(\Phi, R)$ induced by $p_2^*$, we obtain an exact sequence
\begin{equation}
    \xymatrix{1 \ar[r] & C(\Phi, R, I) \ar[r] & \St(\Phi, R, I) \ar[r]^\mu & \St(\Phi, R) \ar[r]^-{p^*} & \St(\Phi, R/I) \ar[r] & 1. }\label{eq:relative-Steinberg}
\end{equation}
Alternatively, the group $\St(\Phi, R, I)$ can be defined via generators and relations as an $\St(\Phi, R)$-group, cf.~\cite[Proposition~6]{S15}
 or even as an abstract group, see~\cite{V22}.
The relative group $\K_2(\Phi, R, I)$ is defined as the kernel of the homomorphism $\pi \mu$.

We will need a relative analogue of Steinberg symbol.
Let $A$ be a local unital ring with maximal ideal $M$ embedded as a subring into a larger unital ring $R$.
Under this assumption the subset $1+M \subseteq A$ forms a group under multiplication.
It is clear that $(1+M)^\times$ is isomorphic to the abelian group $(M, \circ)$ with the operation given by $m \circ m' = m + m' + mm'$.
Now for $a \in R^\times$ and $m \in M$ we denote by $\{a, 1+m\}_r$ the coset $\{(a; a), (1; 1+m)\}C \in \St(\Phi, R, RM)$.
It is clear that the map $1+m \mapsto \{a, 1+m\}_r$ specifies a group homomorphism
\[\{ a, -\}_r \colon (1+M)^\times \to \K_2(\Phi, R, RM). \]
It is also clear that $\mu(\{a, 1+m\}_r) = \{a, 1+m\}$.

\begin{lemma}\label{lem:symbols}
Assume that $A$ is a local domain with maximal ideal $M$.
Denote by $R$ the ring $A[X, X\inv]$.
Then the intersection of the image of the relative symbol map $\{X, -\}_r$ with $C(\Phi, R, M[X, X\inv])$ is trivial.
\end{lemma}
\begin{proof}
Set $F = \mathrm{Frac}(A)$.
Consider the following diagram:
\[\begin{tikzcd}
 (1+M)^\times \ar[hookrightarrow, rr] \ar[d] &  & F^\times \ar[hookrightarrow, d] \\
  \K_2(\Phi ,R, M[X^{\pm 1}]) \ar[r] & \K_2(\Phi, R) \ar[r] & \K_2(\Phi, F[X^{\pm 1}]).
\end{tikzcd}\]
Since the right vertical arrow is injective by~\cite[Lemma~2.2]{LS20}, so is the left arrow.
\end{proof}

\subsection{Weight elements}\label{subsec:weight-elements}
Recall that for every coweight $\omega \in P(\Phi^\vee)$ and $\beta \in \ZZ \Phi$ the scalar product $(\omega, \beta)$ is an integer.
%Thus, for $u \in R^\times$ and $\omega \in P(\Phi^\vee)$ we can define a map of abelian groups $\ZZ \Phi \to R^\times$ via the formula $\beta \mapsto u ^ {(\omega, \beta)}$.
Thus, a choice of $u \in R^\times$ and $\omega \in P(\Phi^\vee)$ specifies a permutation of the generating set for $\St(\Phi, R)$ via the following mapping:
\begin{equation*} x_\alpha(a) \mapsto x_\alpha(u^{(\omega, \alpha)} \cdot a),\ \alpha\in \Phi,\ a \in R. \end{equation*}
It is not hard to check that this action is compatible with relations~\eqref{x-additivity}--\eqref{R3} and hence specifies a well-defined automorphism of $\St(\Phi, R)$, which we denote by $\chi_{\omega, u}$.

In the following lemma we check that an analogue of $\chi_{\omega, u}$ can also be defined for relative Steinberg groups.
\begin{lemma} \label{lem:relative-chi}
Let $R$ be a commutative ring, $I$ be its ideal and let $u \in R^\times$.
For every coweight $\omega \in P(\Phi^\vee)$ there exists a well-defined automorphism $\widetilde{\chi}_{\omega, u}$ of the relative Steinberg group $\St(\Phi, R, I)$ making the following diagram commute:
\[\begin{tikzcd} \St(\Phi, R, I) \ar[r, "\widetilde{\chi}_{\omega, u}"] \ar[d] & \St(\Phi, R, I) \ar[d] \\
                 \St(\Phi, R) \ar[r, "\chi_{\omega, X}"] & \St(\Phi, R]). \end{tikzcd}\]
\end{lemma}
\begin{proof}
    Observe that the automorphism $\chi_{\omega, (u; u)}$ of $\St(\Phi, D_{R, I})$ preserves subgroups
     $\Ker(p_i^*)$, $i=1, 2$ and hence their commutator subgroup $C$.
    The required automorphism $\widetilde{\chi}_{\omega, u}$ now can be obtained by restricting $\chi_{\omega, (u; u)}$ to $\Ker(p_1^*)$.
    The commutativity of the diagram is obvious.
\end{proof}

In the sequel we will use the following formulae describing the action of $\chi_{\omega, u}$ on the elements $w_\alpha(u)$, $h_\alpha(u)$:
\begin{align}
    \label{eq:chi-w} \chi_{\omega, u}\left(w_\alpha(v)\right) &= w_\alpha(v \cdot u^{(\omega, \alpha)}), \\
    \label{eq:chi-h} \chi_{\omega, u} (h_\alpha(v)) &= h_\alpha(u^{(\omega, \alpha)} v) \cdot h_\alpha(u^{(\omega, \alpha)})^{-1} = \{u^{(\omega, \alpha)}, v\} \cdot h_\alpha(v).
\end{align}

The following lemma is the analogue of~\cite[Lemma~3.1(c)]{Tu83}.
\begin{lemma} \label{lem:winv-chiw}
For any $w \in \StW(\Phi, A[X^{\pm 1}])$ the element $w^{-1} \cdot \chi_{\omega, X}(w)$ belongs to $\StH(\Phi, A[X^{\pm 1}])$.
\end{lemma}
\begin{proof}
Since $\StH(\Phi, A[X^{\pm 1}])$ is a normal subgroup of $\StW(\Phi, A[X^{\pm 1}])$, it suffices to verify the assertion for $w = w_\alpha(u)$.
Set $h = w^{-1} \cdot \chi_{\omega, X}(w)$.
It is not hard to check that
\begin{multline} w_\alpha(1) \cdot h\cdot  w_\alpha(-1) = w_\alpha(-1)^{-1} \cdot w_\alpha(u)^{-1} \cdot w_{\alpha}(X^{(\omega, \alpha)} u) \cdot w_\alpha(-1) = \\
= h_\alpha(u)^{-1} \cdot h_\alpha(X^{(\omega, \alpha)}u) \in \StH(\Phi, A[X^{\pm 1}]).\end{multline}
Thus, we get that \[h = h_{\alpha}(u^{-1})^{-1} \cdot h_{\alpha}(X^{-(\omega, \alpha)}u^{-1}) = h_{\alpha}(X^{-(\omega, \alpha)}) \cdot c,\ \text{for some $c \in \K_2(\Phi, A)$.} \qedhere\]
\end{proof}

\begin{example} \label{exm:chi-linear}
Set $R = A[X^{\pm 1}]$. Consider the following coweights: \[\varepsilon_1 = \varpi_1^\vee,\ \varepsilon_2 = \varpi_2^\vee - \varpi_1^\vee,\ \ldots,\ \varepsilon_{\ell+1} = -\varpi^\vee_\ell.\] For $1\leq k\leq \ell+1$ and $u \in R^\times$ denote by $d_k(u)$ the matrix from $\GL(\ell+1, R)$ which differs from the unit matrix only in that it has the element $u$ on the $k$-th place of its diagonal.
Recall from~\cite[Corollary~4]{Ka77} that for any $g \in \GL(\ell+1, R)$ there exists an automorphism $\beta_g$ of $\St(\ell+1, R)$ ''modeling`` the automorphism $\alpha_g \colon \GL(\ell+1, R) \to \GL(\ell+1, R)$ of inner conjugation by $g$, i.\,e. such that $\phi \beta_g = \alpha_g \phi$.


It is clear that in the linear case the map $\chi(\varepsilon_k, u)$ coincides with $\beta_{d_k(u)}$,
 while for other Chevalley groups the maps $\chi(\omega, u)$ model automorphisms of inner conjugation by weight elements $h_\omega(u)$ in the sense of~\cite[\S~4]{Vav09}.
\end{example}


Let $\omega \in P(\Phi^\vee)$ be a coweight of $\Phi$ as above.
Denote by $\XX(\omega)$ the subset of $\XX_{\Phi, A[X]}$ consisting of those generators $x_{\alpha}(\xi)$ of $\St(\Phi, A[X])$ for which
$(\alpha, \omega) < 0$ implies that $\xi \in A[X]$ is divisible by $X^{-(\alpha, \omega)}$.


Denote by $N(\omega)$ the subgroup of $\St(\Phi, A[X])$ generated by $\XX(\omega)$.


\begin{dfn} \label{dfn:delta-pair}
By definition, a {\it $\delta$-pair for $\omega$} is a pair of mutually inverse group homomorphisms
$\xymatrix{ \delta(\omega)\colon N(\omega) \ar[r] & \ar@<-1.0ex>[l] N(-\omega)\colon \delta(-\omega) }$ satisfying
\begin{equation} \label{eq:sigmadef}
\delta(\pm \omega)(x_\alpha(\xi)) = x_\alpha(X^{(\pm \omega, \alpha)}\cdot \xi),
 \text{ for all } x_\alpha(\xi) \in \XX(\pm\omega).
\end{equation}\end{dfn}
It is clear that the maps $\delta(\omega)$, $\delta(-\omega)$ are uniquely determined by~\eqref{eq:sigmadef}, so at most one $\delta$-pair may exist for any given $\omega$.
Moreover, it is automatically true that the homomorphisms $\delta(\omega), \delta(-\omega)$ make the following diagram commute:
\begin{equation} \label{eq:sigma-diagram}
\xymatrix{ N(\omega) \ar[r]^{\delta(\omega)}\ar@{^{(}->}[d] \ar@/^1.5pc/[rr]_{\mathrm{id}} & N(-\omega) \ar@{^{(}->}[d] \ar[r]^{\delta(-\omega)} & N(\omega) \ar@{^{(}->}[d] \\
          \St(\Phi, A[X]) \ar[d] & \St(\Phi, A[X]) \ar[d] & \St(\Phi, A[X]) \ar[d] \\
          \St(\Phi, A[X^{\pm 1}]) \ar@<-0.0ex>[r]_{\chi_{\omega, X}} \ar@/_1.5pc/[rr]^{\mathrm{id}} & \St(\Phi, A[X^{\pm 1}]) \ar@<-0.0ex>[r]_{\chi(-\omega, X)} & \St(\Phi, A[X^{\pm 1}]).} \end{equation}



\subsection{Horrocks ingredient}\label{subsec:horrocks-ingredient}
\begin{lemma} \label{lem:zariski-glueing}
Let $\Phi$ be any simply-laced root system of rank $\geq 3$.
Let $A$ be a commutative domain and $a, b \in A$ be a pair of coprime elements.
\begin{enumerate}
    \item Let $\delta$ be an element of $\St(\Phi, A_{ab})$.
    Then $\delta$ can be presented as $\lambda_b(\alpha) \cdot \lambda_a(\beta)$ for some
    $\alpha \in \St(\Phi, A_a)$ and \beta \in $\St(\Phi, A_b)$.
    \item  Let $\alpha \in \St(\Phi, A_a)$ and $\beta \in \St(\Phi, A_b)$ be such that the equality $\lambda_b(\alpha) = \lambda_a(\beta)$ holds in $\St(\Phi, A_{ab})$.
    Then there exists $\gamma \in \St(\Phi, A)$ such that $\alpha = \lambda_a(\gamma)$, $\beta = \lambda_b(\gamma)$.
\end{enumerate}

\end{lemma}
\begin{proof}
    This is a special case of Nisnevich excision for domains, see~\cite[Proposition~4.5]{LSV2}
    (cf. also the proof of~\cite[Lemma~2.6]{LSV2}).
\end{proof}

Now let $A$ be a local ring with maximal ideal $M$.
Set $B \coloneqq A[X\inv] + M[X],\ R \coloneqq A[X, X\inv]$.
Consider the following diagram:
%! suppress = EscapeAmpersand
\[ \xymatrix{ & \St(\Phi, A[X], M[X]) \ar[d]_{t_M} \ar[r] & \St(\Phi, A[X]) \ar[d]_{\lambda_X^*} \\
   \St(\Phi, A[X\inv]) \ar[r]^{i_B} \ar@/_/[rr]_{\lambda_{X\inv}^*} & \St(\Phi, B) \ar[r]^{\lambda^*_{X\inv, B}} & \St(\Phi, R)
}\]
The arrow $t_M$ arises from the \("\)lifting property\("\) of Steinberg groups (cf. \cite[Lemma~3.3]{LS20} or~\cite[Theorem~3]{LS17}) as the composition of morphisms
$\lambda_{X}^{rel}\colon \St(\Phi, A[X], M[X]) \to \St(\Phi, R, M[X, X\inv])$ and $T \colon \St(\Phi, R, M[X, X\inv]) \to \St(\Phi, B)$.
%This is wrong: Its image $\Img(t_M) \leq \St(\Phi, B)$ is generated as a subgroup by elements $x_\alpha(f)$, $z_\alpha(f, a)$, $z_\alpha(X^2f, aX^{-1})$, for $\alpha\in \Phi$, $a \in A$, $f \in M[X]$, see the proof of~\cite[Lemma~5.41]{LS20}.

Set $G_M^{\geq 0} \coloneqq \Img(\St(\Phi, A[X], M[X]) \to \St(\Phi, R)), G_M^0 \coloneqq \overline{\St}(\Phi, A, M)$.% Show that G_M^0 is a subgroup of both first two factors
Denote by $\overline{V}$ the quotient-set of the product $V \coloneqq G_M^{\geq 0} \times \St(\Phi, A[X\inv]) \times (1 + M)^\times$
 modulo the equivalence relation given by $(gh_0, h, u) \cong (g, h_0h, u)$ where $h_0 \in G_M^0, (g, h, u) \in V.$
Denote by $[g, h, u] \in \overline{V}$ the equivalence class corresponding to $(g, h, u)\in V$.

\begin{prop} \label{prop:horrocks-main} The group $\St(\Phi, B)$ acts on $\overline{V}$.
 This action satisfies the following properties:
    \begin{enumerate}
        \item For $(g, h, u) \in V$ and $g_0 \in \Img(t_M)$ one has
         \[g_0 \cdot [g, h, u] = [\lambda_{X\inv, B}^* (g_0) \cdot g, h, u]\]
        \item For $(1, h, u) \in V$ and $h_0 \in \St(\Phi, A[X\inv])$ one has
         \[ i_B(h_0) \cdot [1, h, u] = [1, h_0 \cdot h, u].\]
        \item If $g \in \Img(j\colon \St(\Psi, B) \to \St(\Phi, B)), h \in \St(\Phi, A[X\inv]), u \in 1 + M$ then
         \[ g \cdot [1, h, u] = [g', h', u']\] for some $g' \in \Img(j\colon G_{M, \Psi}^{\geq 0} \to G_{M, \Phi}^{\geq 0})$, $h' \in \St(\Phi, A[X\inv])$, $u'\in 1 + M$.
    \end{enumerate}
\end{prop}
\begin{proof}
    The existence of the action of $\St(\Phi, B)$ and the second assertion are contained in~\cite[Proposition~5.39]{LS20}.
    The first assertion is what the proof of~\cite[Lemma~5.41]{LS20} actually shows without the assumption that $i_R$ is injective.
    The last assertion follows from the construction of the action. %TODO: Add more details!
\end{proof}

\begin{lemma} \label{lem:horrocks--ingredient}
Let $A$ be a local ring, $\alpha \in \K_2(\Phi, A[X], XA[X])$, $\beta \in \St(\Phi, A[X\inv]),$ $\gamma \in \St(\Psi, A[X, X\inv])$ be elements satisfying the equality (in $\St(\Phi, R)$)
\begin{equation} \label{eq:alpha-def}\lambda_X(\alpha) = j(\gamma) \cdot \lambda_{X\inv}(\beta) \end{equation}
Then $\alpha$ belongs to the image of $j\colon \St(\Psi, A[X]) \to \St(\Phi, A[X])$.
\end{lemma}
\begin{proof}
    Denote by $M$ the maximal ideal of $A$ and by $k$ its residue field.
    We denote by $\pi_A$ (resp. $\pi_{A[X]}$, resp. $\pi_{R}$) the canonical homomorphism $A \to k$ (resp. $A[X] \to k[X]$, resp. $R \to k[X, X\inv]$).

    Since $\K_2(\Psi, k[X])$ surjects onto $\K_2(\Phi, k[X])$ there exists $\alpha_0 \in \St(\Psi, A[X])$ such that $j(\alpha_0) \cdot \alpha \in \overline{\St}(\Phi, A[X], M[X])$ and
     hence $\pi_{A[X]}^*(j(\alpha_0) \cdot \alpha) = 1$.
    By definition of relative Steinberg groups, there exists $\widetilde{\alpha} \in \St(\Phi, A[X], M[X])$ such that
     $\iota \cdot \lambda_X^{rel}(\widetilde{\alpha}) = \lambda_X^* (j(\alpha_0) \cdot \alpha)$. %TODO: More details

    Now consider the element $\pi^*_{A[X\inv]}(\beta) \in \St(\Phi, k[X\inv])$.
    Its image in $\G(\Phi, k[X\inv])$ is contained in the subgroup $\G(\Psi, k[X\inv])$.
    Since $\K_1(\Psi, k[X\inv]) = 1$, there exist
    \[\beta_0 \in \St(\Psi, A[X\inv]),\ \beta_1 \in \St(\Phi, A[X\inv], M[X\inv])\] such that $\beta = j(\beta_0) \cdot \beta_1$.
    Set $\gamma_1 \coloneqq \lambda^*_X(\alpha_0) \cdot \gamma \cdot \lambda^*_{X\inv}(\beta_0) \in \St(\Psi, R)$.
    It follows from~\eqref{eq:alpha-def} that the element $\pi_R^*(j(\gamma_1)) \in \St(\Phi, k[X, X\inv])$ is trivial, hence $\pi^*_R(\gamma_1) \in \K_2(\Psi, k[X, X\inv])$.
    Recall from~\cite{Hur77} that $\K_2(\Psi, k[X, X\inv]) \to \K_2(\Phi, k[X, X\inv])$ is injective (see the assertion after Korollar~6).
    Thus, $\gamma_1 \in \overline{\St}(\Psi, R, M[X, X\inv]).$
    There exists $\gamma_2 \in \St(\Psi, R, M[X, X\inv])$ such that $\iota (\gamma_2) = \gamma_1$.
    Notice that $j(\gamma_2) \cdot \lambda_{X\inv}^{rel}(\beta_1) = \kappa \lambda_X^{rel}(\widetilde{\alpha})$ for $\kappa \in \Ker(\iota\colon \St(\Phi, R, M[X, X\inv]) \to \St(\Phi, R))$

    By the third assertion of~\cref{prop:horrocks-main} $j(\gamma_2) \cdot [1, \beta_1, 1] = [j(g), h, u]$ for some $g \in \G_{M, \Psi}^{\geq 0}$.
    Observe that
    \[ \lambda_{X\inv, B}^* (j(\gamma_2) \cdot i_B(\beta_1)) = j(\gamma_1) \lambda_{X\inv}^*(\beta_1) = \lambda_X^*(j(\alpha_0)) \cdot j(\gamma) \lambda_{X\inv}^*(\beta) = \lambda_{X\inv, B}^*(t_M(\widetilde{\alpha})), \]
    hence there exists $\kappa \in \Ker(\lambda^*_{X\inv, B})$ such that $j(\gamma_2) \cdot i_B(\beta_1) = \kappa t_M(\widetilde{\alpha})$.
    Consequently, from~\cref{prop:horrocks-main} we obtain that
    \[j(\gamma_2) \cdot [1, \beta_1, 1] = j(\gamma_2) \cdot i_B(\beta_1) [1, 1, 1] = \kappa [\lambda_X^*(j(\alpha_0) \cdot \alpha), 1, 1].\]
    By the definition of $\overline{V}$ one has $j(\alpha_0) \alpha = j(g) g_0$ for $g$ as above and some $g_0 \in \overline{\St}(\Phi, A, M)$.
    Since $\alpha(0) = 1$ we obtain that $g_0 = j(g)(0) \cdot j(\alpha_0)(0)$, which implies the required assertion.
\end{proof}

    \section{Curtis--Tits type presentations} \label{sec:affine}
    Throughout this section $A$ denotes an arbitrary commutative ring.

\subsection{Curtis--Tits type presentation of affine Steinberg groups} \label{subsec:curtis-tits}
In this subsection we briefly recall the theory of Steinberg groups in the Kac--Moody setting.
Only in this subsection we allow root systems to be infinite.

Recall that to any generalized Cartan matrix one can associate (possibly infinite) root system $\Phi$ and the Steinberg group functor $\St(\Phi, -)$.
Two definitions of this functor have been proposed: the definition of J.~Tits~\cite[\S~3.6]{Ti87} and the definition of J.~Morita and U.~Rehmann~\cite[\S~2]{MR90}.
These definitions agree if the Dynkin diagram of the GCM does not contain connected components of type $\rA_1$, see~\cite[\S~6]{A13}.

For our purposes it will be sufficient to restrict attention to the case of a GCM of \textit{affine} type, or, in other words, GCM whose Dynkin diagram is the \textit{extended} Dynkin diagram of a finite irreducible simply-laced root system $\Phi$ of rank $>1$.
Such digrams for $\Phi$ of type $\rD_\ell$, $\rE_6$ and $\rE_7$ are depicted on Figure 1 (the added root is marked with yellow color and index $0$).

\tikzset{
    root/.style={circle, draw, minimum size=0.2cm, inner sep=0},
    zeroroot/.style={circle, draw, minimum size=0.2cm, inner sep=0, fill=yellow},
    highlighted/.style={circle, draw, minimum size=0.2cm, inner sep=0, fill=green},
    levi/.style={draw, dashed, rounded corners},
    dottededge/.style={dotted},
    labeled/.style={below}
}
\begin{figure}[hb]\label{fig:dynkin-diagrams}
\begin{longtable}{ c c c }
    \scalebox{0.69}{\begin{tikzpicture}
                        \node[root, highlighted, label=$1$] (d1) at (0,0.5) {};
                        \node[root, label=$2$] (d2) at (1,0) {};
                        \node[root, label=$3$] (d3) at (2,0) {};
                        \node at (3,0) {\ldots};
                        \node[root, label={[xshift=-0.5cm, yshift=-0.6cm]$\ell-2$}] (dl2) at (4,0) {};
                        \node[root, label={[xshift=-0.6cm, yshift=-0.35cm]$\ell-1$}] (dl1) at (5,0.5) {};
                        \node[root, label={[xshift=-0.5cm, yshift=-0.3cm]$\ell$}] (dl) at (5,-0.5) {};
                        \node[root, zeroroot, label=below:{$0$}] (d0) at (0,-0.5) {};

                        \draw (d1) -- (d2) -- (d3) -- (2.5,0);
                        \draw (3.5,0) -- (dl2) -- (dl1);
                        \draw (dl2) -- (dl);
                        \draw[dottededge] (d2) -- (d0);

                        \begin{scope}
                            \node[levi, fit=(d2) (d3) (dl) (dl1) (dl2), label=above:{$\Delta$}] {};
                        \end{scope}
    \end{tikzpicture}}
    &
    \scalebox{0.69}{\begin{tikzpicture}
                        \begin{scope}
                            \node[root, highlighted, label=$1$] (e1) at (0,0) {};
                            \node[root, label=$3$] (e3) at (1,0) {};
                            \node[root, label={[xshift=-0.3cm]$4$}] (e4) at (2,0) {};
                            \node[root, label=$5$] (e5) at (3,0) {};
                            \node[root, label=$6$] (e6) at (4,0) {};
                            \node[root, label={[xshift=-0.3cm,yshift=-0.3cm]$2$}] (e2) at (2,1) {};
                            \node[root, zeroroot, label=left:$0$] (e0) at (2,2) {};
                            \draw (e1) -- (e3) -- (e4) -- (e5) -- (e6);
                            \draw (e2) -- (e4);
                            \draw[dottededge] (e2) -- (e0);

                            \begin{scope}
                                \node[levi, fit= (e2) (e3) (e4) (e5) (e6), label=above:{$\Delta$}] {};
                            \end{scope}
                        \end{scope}
    \end{tikzpicture}}
    &
    \scalebox{0.69}{\begin{tikzpicture}
                        \begin{scope}
                            \node[root, zeroroot, label={0}] (e0) at (0,0) {};
                            \node[root, label={$1$}] (e1) at (1,0) {};
                            \node[root, label={$3$}] (e3) at (2,0) {};
                            \node[root, label={[xshift=-0.3cm]$4$}] (e4) at (3,0) {};
                            \node[root, label={$5$}] (e5) at (4,0) {};
                            \node[root, label={$6$}] (e6) at (5,0) {};
                            \node[root, label={[xshift=-0.3cm,yshift=-0.3cm]$2$}] (e2) at (3,1) {};
                            \node[root, highlighted, label={$7$}] (e7) at (6,0) {};

                            \draw (e1) -- (e3) -- (e4) -- (e5) -- (e6) -- (e7);
                            \draw (e2) -- (e4);
                            \draw[dottededge] (e0) -- (e1);

                            \begin{scope}
                                \node[levi, fit=(e1) (e2) (e3) (e4) (e5) (e6), label=above:{$\Delta$}] {};
                            \end{scope}
                        \end{scope}
    \end{tikzpicture}} \\
    \text{$\rD_\ell$} &
    \text{$\rE_6$} &
    \text{$\rE_7$}
\end{longtable}
\caption{Root markings on extended Dynkin diagrams}
\end{figure}

Denote by $\widetilde{\Phi}$ the affine root system corresponding to $\Phi$.
Recall from~\cite[\S~4]{A16} that the set of real roots of $\widetilde{\Phi}$ is isomorphic to $\Phi \times \ZZ$.
\begin{lemma} \label{lem:affine-vs-loop} $\St(\widetilde{\Phi}, A) \cong \St(\Phi, A[X, X\inv])$.
\end{lemma}
\begin{proof}
    By definition, the group $\mathrm{St}(\widetilde{\Phi}, A)$ is presented by generators $x_{(\alpha, m)}(a)$, $a \in A$, $\alpha \in \Phi$ and
    the following relations:
    \begin{align}
        x_{(\alpha, m)}(a)\cdot x_{(\alpha, m)}(b)&=x_{(\alpha, m)}(a+b),  \label{AR1}\\
        [x_{(\alpha, m)}(a),\,x_{(\beta, n)}(b)]  &=x_{(\alpha+\beta, n+m)}(N_{\alpha,\beta} \cdot ab),\text{ for }\alpha+\beta\in\Phi, \label{AR2} \\
        [x_{(\alpha, m)}(a),\,x_{(\beta, n)}(b)]  &=1,\text{ for }\alpha+\beta\not\in\Phi\cup0. \label{AR3}
    \end{align}
    It is not hard to check that the homomorphism given by $x_{(\alpha, m)}(a) \mapsto x_\alpha(aX^m)$ is an isomorphism with the inverse given by
    $x_\alpha(a_{n}X^n + \ldots + a_m X^m) \mapsto \Pi\limits_{i=n}^m x_{(\alpha, i)}(a_i)$, $a_i \in A$, $n \leq m$, $n, m\in \ZZ$.
\end{proof}

Our next goal is to formulate the so-called \textit{Curtis--Tits presentation} of affine Steinberg groups discovered by D.~Allcock in~\cite{A16, A13}.
This presentation has the advantage of being formulated in terms of the Dynkin diagram of $\Phi$ rather than the (possibly infinite) set of real roots of $\Phi$.
Its other advantage is that it requires no choice of structure constants in its statement.
Allcock's result is a generalization of Curtis--Tits presentation of the Steinberg group of a finite root system.
Results of such type have been known since 1960's, see e.\,g.~\cite[Theorem~B]{DS74}.

Recall that $\{ \alpha_1, \ldots \alpha_\ell \}$ is the set of simple roots of $\Phi$.
We denote by $\alpha_0$ the opposite root to the maximal root of $\Phi$ (i.\,e. $\alpha_0 := -\alpha_\mathrm{max}$).
This root corresponds to the added $0$ node of the extended Dynkin diagram depicted on Figure 1.
We denote by $X_0(A)$ the root group $\{ X_0(a) \mid a \in A\}$ (as a group it is isomorphic to the additive group of $A$).
We also denote by $j$ the root adjacent to $0$ on the extended Dynkin diagram of $\Phi$.
\begin{prop} \label{prop:Allcock-affine} The group $\St(\Phi, A[X, X\inv])$ is isomorphic to the free product of $\St(\Phi, A)$, the group $X_0(A)$ and the free cyclic group $\langle S_0 \rangle$
     amalgamated over the subgroup generated by the following list of relations:
    {\allowdisplaybreaks\begin{align}
        [S_0^2, X_0(a)] & = 1 & \text{ for $a \in A$; } \label{eq:Allcock-2} \\
        X_0(1) \cdot {}^{S_0} X_0(1) \cdot X_0(1) & = S_0; \label{eq:Allcock-3} \\
        [S_0, w_{\alpha_i}(1)] & = 1; &  \label{eq:Allcock-4} \\
        [S_0, x_{\alpha_i}(a)] & = 1, &  \label{eq:Allcock-5-1}\\
        [w_{\alpha_i}(1), X_0(a)] & = 1 & \text{$i$ unjoined with $0$, $a \in A;$} \label{eq:Allcock-5-2} \\
        [X_0(a), x_{\alpha_i}(b)] & = 1 & \text{$i$ unjoined with $0$, $a, b \in A;$} \label{eq:Allcock-6} \\
        S_0 \cdot w_{\alpha_j}(1) \cdot S_0 & = w_{\alpha_j}(1) \cdot S_0 \cdot w_{\alpha_j}(1); \label{eq:Allcock-7} \\
        {}^{S_0^2} w_{\alpha_j}(1) & = w_{\alpha_j}(-1); \label{eq:Allcock-8-1} \\
        {}^{w_{\alpha_j}^2(1)} S_0 & = S_0^{-1}; \label{eq:Allcock-8-2} \\
        x_{\alpha_j}(a) \cdot S_0 \cdot w_{\alpha_j}(1) & = S_0 \cdot w_{\alpha_j}(1) \cdot X_0(a), & \label{eq:Allcock-9-1} \\
        X_0(a) \cdot w_{\alpha_j}(1) \cdot S_0 & = w_{\alpha_j}(1) \cdot S_0 \cdot x_{\alpha_j}(a), & \label{eq:Allcock-9-2} \\
        {}^{S_0^2} x_{\alpha_j}(a) & = x_{\alpha_j}(-a), & \label{eq:Allcock-10-1} \\
        {}^{w_{\alpha_j}^2(1)} X_0(a) & = X_0(-a) & \text{for $a \in A;$} \label{eq:Allcock-10-2} \\
        [X_0(a), {}^{S_0} x_{\alpha_j}(b)] &= 1, & \label{eq:Allcock-11-1} \\
        [x_{\alpha_j}(a), {}^{w_{\alpha_j}(1)}X_0(b)] &= 1, & \label{eq:Allcock-11-2} \\
        [X_0(a), x_{\alpha_j}(b)] &= {}^{S_0} x_{\alpha_j}(ab) & \text{for $a, b \in A.$} \label{eq:Allcock-12}
    \end{align}}
\end{prop}
\begin{proof}
    This is a direct consequence of our~\cref{lem:affine-vs-loop} and the presentation of~\cite[Theorem~1]{A16} (or, alternatively, \cite[Theorem~1.1]{A13} combined with~\cite[Theorem~1.3]{A13}).
    The list of relations in the statement is obtained from \cite[Table~1]{A16} by substituting concrete values of $i, j$, identifying Allcock's $X_{i}(a)$ with
     $x_{\alpha_i}(a)$ and $S_i$ with $w_{\alpha_i}(a)$ for all $i\neq 0$ and then omitting all the relations are already satisfied in $\St(\Phi, A)$ (i.\,e. the relations not involving $X_0(a)$ or $S_0$).
    The only further simplification is that we have omitted the relation $[X_{\alpha_j}(b), X_0(a)] = {}^{w_{\alpha_j}(1)} X_0(ab)$ (the relation symmetric to~\eqref{eq:Allcock-12})
    because it is a consequence of~\eqref{eq:Allcock-9-2} and~\eqref{eq:Allcock-12}.
\end{proof}

Our next goal is to slightly simplify the presentation of~\cref{prop:Allcock-affine} by using the larger group $\St(\Phi, A[X])$ instead of $\St(\Phi, A)$ in the statement.
\begin{cor} \label{cor:Allcock-simpler}
    For a finite simply-laced irreducible root system $\Phi$ and an arbitrary commutative ring $A$ the Steinberg group $\St(\Phi, A[X, X\inv])$ is isomorphic
    to the free product of $\St(\Phi, A[X])$ and the cyclic group $\langle S \rangle$ amalgamated over the subgroup generated by the following list of relations:
    {\allowdisplaybreaks\begin{align}
    [S^2, x_{\alpha_0}(aX)] & = 1 & \text{ for $a \in A$; } \label{eq:simpler-2} \\
    x_{\alpha_0}(X) \cdot {}^{S} x_{\alpha_0}(X) \cdot x_{\alpha_0}(X) & = S; \label{eq:simpler-3} \\
    [S, w_{\alpha_i}(1)] & = 1 & \text{ for $i$ unjoined with $0$;} \label{eq:simpler-4} \\
    [S, x_{\alpha_i}(a)] & = 1 &  \text{ $i$ unj. with $0$, $a \in A$; } \label{eq:simpler-5-1}\\
    S \cdot w_{\alpha_j}(1) \cdot S & = w_{\alpha_j}(1) \cdot S \cdot w_{\alpha_j}(1); \label{eq:simpler-7} \\
    {}^{S^2} w_{\alpha_j}(1) & = w_{\alpha_j}(-1); \label{eq:simpler-8-1} \\
    {}^{w_{\alpha_j}^2(1)} S & = S^{-1}; \label{eq:simpler-8-2} \\
    x_{\alpha_j}(a) \cdot S \cdot w_{\alpha_j}(1) & = S \cdot w_{\alpha_j}(1) \cdot x_{\alpha_0}(aX), & \label{eq:simpler-9-1} \\
    x_{\alpha_0}(aX) \cdot w_{\alpha_j}(1) \cdot S & = w_{\alpha_j}(1) \cdot S \cdot x_{\alpha_j}(a), & \label{eq:simpler-9-2} \\
    {}^{S^2} x_{\alpha_j}(a) & = x_{\alpha_j}(-a) & \text{ for $a \in A$; } \label{eq:simpler-10-1} \\
    [x_{\alpha_0}(aX), {}^{S} x_{\alpha_j}(b)] &= 1, & \label{eq:simpler-11-1} \\
    [x_{\alpha_0}(aX), x_{\alpha_j}(b)] &= {}^{S} x_{\alpha_j}(ab) & \text{for $a, b \in A.$} \label{eq:simpler-12}
    \end{align}}
\end{cor}
\begin{proof}
    The list of relations is obtained from~\eqref{eq:Allcock-2}--\eqref{eq:Allcock-12} by identifying $X_0(a)$ with $x_{\alpha_0}(aX)$, $S_0$ with $S$ and omitting those relations
     which are satisfied in the group $\St(\Phi, A[X])$ by virtue of Lemmas 5.1--5.2 of ~\cite{Ma69}.
\end{proof}

\subsection{Relationship between $\St(\Phi, A[X, X\inv])$ and $\St(\Phi, A[X])$.} \label{subsec:short-presentation}
Now let $\Phi$ be a root system of type $\rD_\ell$, $\rE_6$ or $\rE_7$.
Denote by $k$ the vertex of the extended Dynkin diagram marked green on Figure~1.
Recall that in each case $m_k(\alpha_\mathrm{\max}) = 1$, therefore $m_i(\alpha) = 1$ for all $\alpha \in \Sigma_k^+$ and the subgroup $\UU(\Sigma^+_k, R)$ is abelian.
Also the weight $\varpi_k$ is a \textit{microweight} of $\Phi$, cf. e.\,g.~\cite[\S~2]{Ge17}.

Denote by $G$ the amalgamated product of $\St(\Phi, A[X])$ and the cyclic group $\langle \sigma \rangle$ amalgamated over the subgroup generated by the following relations:
\begin{align}
    {}^\sigma x_{\alpha}(f) = & x_{\alpha} (Xf), & \alpha \in \Sigma^+_k, f \in A[X], \label{eq:sigma-sigma-plus} \\
    x_{\beta}(f)^ \sigma     =& x_{\beta} (Xf), & \beta \in \Sigma^-_k, f \in A[X], \label{eq:sigma-sigma-minus} \\
    [\sigma,\, x_\gamma(f)]   =& 1, & \gamma \in \Delta, f \in A[X]. \label{eq:sigma-delta}
\end{align}
It is clear that the action of the generator $\sigma$ is chosen to mimick the action of the weight element $\chi_{\omega_k, X}$ from~\cref{subsec:weight-automorphisms}.

We denote by $i_+$ the canonical homomorphism $\St(\Phi, A[X]) \to G$ and by $h_+$ the canonical embedding $A[X] \to A[X, X\inv]$.

The main result of this subsection is the following
\begin{prop} \label{prop:rel-poly-Laurent}
    For $\Phi$ as above there exists an arrow $\varphi$ making the diagram below commute:
    \[\begin{tikzcd} & \St(\Phi, A[X, X\inv]) \arrow[rd, dashrightarrow, "\varphi"] & \\
    \St(\Phi, A[X]) \arrow{ru}{h_+^*} \arrow{rr}{i_+} & & G.
    \end{tikzcd}\]
\end{prop}
\begin{proof}
 We will use the presentation of~\cref{cor:Allcock-simpler} for $\St(\Phi, A[X, X\inv])$.
 We only need to specify the value of $\varphi$ on the generator $S$.
 Observe that $-\alpha_0 = \alpha_{\max} \in \Sigma_k^+$, therefore the obvious candidate for the role of $\varphi(x_{-\alpha_0}(aX\inv))$ would be $x_{-\alpha_0}(a)^\sigma$, cf.~\eqref{eq:sigma-sigma-plus},
 which motivates the following definition for $\varphi(S)$:
 \[\varphi(S) := x_{\alpha_0}(X) \cdot x_{-\alpha_0}(-1)^\sigma \cdot x_{\alpha_0}(X).\]
 From~\eqref{eq:w-definition},\eqref{eq:sigma-sigma-minus} it follows that $\varphi(S) = w_{\alpha_0}(1)^\sigma$.

 We need to check that our definition is correct, namely that $\varphi$ respects the relations listed in~\cref{cor:Allcock-simpler}.
 We claim that this follows from the fact that the substitution of $w_{\alpha_0}(1)^\sigma$ into $S$ turns every formula from \cref{cor:Allcock-simpler}
     into a $\sigma$-conjugate of a valid relation in $\St(\Phi, A[X])$.

 Let us illustrate this claim with a few examples.
 For example, to see that $\varphi$ preserves~\eqref{eq:simpler-3} use~\eqref{eq:sigma-sigma-minus} combined with~\cite[Lemma~5.1b]{Ma69}:
 \begin{multline*}
     \varphi(x_{\alpha_0}(X) \cdot {}^{S} x_{\alpha_0}(X) \cdot x_{\alpha_0}(X)) = x_{\alpha_0}(1)^\sigma \cdot {}^{w_{\alpha_0}(1)^\sigma} x_{\alpha_0}(1)^\sigma \cdot x_{\alpha_0}(1)^\sigma = \\
     = \left( x_{\alpha_0}(1) \cdot {}^{w_{\alpha_0}(1)} x_{\alpha_0}(1) \cdot x_{\alpha_0}(1)\right)^\sigma = \left(x_{\alpha_0}(1) \cdot x_{-\alpha_0}(-1) \cdot x_{\alpha_0}(1)\right)^\sigma = w_{\alpha_0}(1)^\sigma = \varphi(S).
 \end{multline*}

 Let us show that $\varphi$ preserves~\eqref{eq:simpler-5-1}.
 Set $g = x_{\alpha_i}(a)$ if $i \neq k$ and $g = x_{\alpha_k}(aX)$ if $i = k$.
 From~\eqref{eq:sigma-sigma-plus},\eqref{eq:sigma-delta} and the commutator formulae we obtain that
 \begin{equation*}
     \varphi([S, x_{\alpha_i}(a)]) = [w_{\alpha_0}(1)^\sigma, x_{\alpha_i}(a)] = [w_{\alpha_0}(1)^\sigma, g^\sigma] = [w_{\alpha_0}(1), g]^\sigma = 1^\sigma = 1.
 \end{equation*}

 Let us show that $\varphi$ preserves~\eqref{eq:simpler-9-1}.
 We claim that in $\St(\Phi, A)$ one has
 \begin{equation} \label{eq:simpler-relation} {}^{w_{\alpha_0}(1) w_{\alpha_j}(1)} x_{\alpha_0}(a) = x_{\alpha_j}(a). \end{equation}
 This follows either from direct calculation involving~\cite[Lemma~5.1]{Ma69} and the standard identities for structure constants~\cite[\S~14]{VP},
 or, alternatively, from our~\cref{lem:affine-vs-loop}, D.~Allcock's presentation of the affine Steinberg group discussed above and
 the fact that the homomorphism $ev_{X=1}^*\colon \St(\Phi, A[X, X\inv]) \to \St(\Phi, A)$ is well-defined.
 Now from~\eqref{eq:sigma-sigma-minus},\eqref{eq:sigma-delta} and~\eqref{eq:simpler-relation} we obtain that
 \begin{multline*}
     \varphi(x_{\alpha_j}(a) \cdot S \cdot w_{\alpha_j}(1)) = x_{\alpha_j}(a) \cdot w_{\alpha_0}(1)^\sigma \cdot w_{\alpha_j}(1) = \left(x_{\alpha_j}(a) \cdot w_{\alpha_0}(1) \cdot w_{\alpha_j}(1)\right)^\sigma = \\
     = \left(w_{\alpha_0}(1) \cdot w_{\alpha_j}(1) \cdot x_{\alpha_0}(a)\right)^\sigma = w_{\alpha_0}(1)^\sigma w_{\alpha_j}(1) \cdot x_{\alpha_0}(aX) = \varphi(S \cdot w_{\alpha_j}(1) \cdot x_{\alpha_0}(aX)).
 \end{multline*}
\end{proof}

\subsection{Relative Curtis--Tits decompositions}
We aim of this subsection is to recall the amalgamation theorem for relative Steinberg groups from~\cite{S15}.

Notice that Allcock's presentation from~\cite{A16, A13} uses the identity element of $R$ in its statement,
 so it is not possible to directly generalize it to the relative case by replacing $R$ with an ideal $I$.
A less naive idea to amalgamate rank $2$ relative Steinberg groups over the edges of the Dynkin diagram of $\Phi$ also would not work
 because the resulting group would not even contain root subgroups $X_\alpha(I)$ for all $\alpha \in \Phi$.
Nevertheless, it turns out to be possible to obtain a weak analogue of Curtis--Tits decomposition for relative Steinberg groups of simply-laced type
 by decomposing $\St(\Phi, R, I)$ into an amalgamated product of multiple copies of Steinberg groups $\St(\rA_3, R, I) \cong \St(4, R, I)$.

Let $\Phi$ be an arbitrary simply-laced irreducible root system of rank $\geq 3$.
We denote by $A_3(\Phi)$ the set consisting of all root subsystems of type $\rA_3$ of $\Phi$.
Under our assumption on $\Phi$ the relative Steinberg group $\St(\Phi, R, I)$ is generated by elements
$z_\alpha(m, a) = x_\alpha((m; 0))^{x_{-\alpha}((a; a))}C$, $\alpha \in \Phi$, $a \in R$, $m \in I$, see~\cite[\S~3.1]{S15}.

Denote by $\widetilde{G}$ the free product of relative Steinberg groups $\St(\Psi, R, I)$, where $\Psi \in A_3(\Phi)$.
For $\Psi \in A_3(\Phi)$ also denote by $i_\Psi$ the canonical embedding $\St(\Psi, R, I) \to \widetilde{G}$.
We denote by $G$ the quotient of $\widetilde{G}$ modulo all relations of the form $i_{\Psi_1}(z_\alpha(m, a)) = i_{\Psi_2}(z_\alpha(m, a))$,
where $\Psi_1, \Psi_2 \in A_3(\Phi)$ both contain a common root $\alpha \in \Phi$ and $a\in R$, $m \in I$.

\begin{thm}\label{thm:relPres} The group $\St(\Phi, R, I)$ is isomorphic to $G$ as an abstract group. \end{thm}

\begin{rem}
    Recently E.~Voronetsky has found an explicit presentation of the relative Steinberg group $\St(\Phi, R, I)$
     by means of generators and relations, see~\cite{V22}.
    This result can be thought of as a further strengthening of~\cref{thm:relPres}
\end{rem}


    \section{Horrocks theorem for $\K_2$} \label{sec:horrocks}
    \begin{dfn}
    Let $K$ be a functor from commutative rings to groups.
    Recall from~\cite{LSV2} that $K$ is called \textit{locally acyclic (resp., locally acylic for domains)} if for every commutative local ring (resp., domain) $A$ the following diagram whose arrows are induced by natural embeddings is a pullback square:
    \begin{equation}\label{eq:P1-square} \begin{tikzcd} K(A) \ar[r] \ar[d] & K(A[X]) \arrow{d} \\ K(A[X\inv]) \ar{r} & K(A[X, X\inv]). \end{tikzcd} \end{equation}
\end{dfn}

Now we can formulate the main result of this section.
\begin{thm}[Horrocks theorem for $\K_2$]\label{thm:horrocks-k2}
Let $\Phi$ be a root system of type $\rA_{\geq 4}$, $\rD_{\geq 5}$, $\rE_6$ or $\rE_7$.
Then the functor $\K_{2}(\Phi, -)$ is locally acyclic for domains.
\end{thm}
Horrocks theorem for $\K_2$ has been previously known in the linear and even orthogonal case $\Phi=\rA_{\geq 4},\rD_{\geq 7}$, see~\cite[Theorem~1]{LS20},\cite[Proposition~4.3]{Tu83}.
Notice also that the orthogonal version of Horrocks theorem for $\K_2$ was proved under the additional assumption that $2 \in A^\times$.
Our theorem generalizes both these results and also improves them in the following respects:
\begin{itemize}
    \item It applies to exceptional root systems ($\Phi = \rE_6, \rE_7$);
    \item In the orthogonal case it makes no assumptions on the invertibility of $2$ and also requires a weaker assumption on the rank of $\Phi$.
\end{itemize}

\subsection{Initial reductions} \label{subsec:structure-theorem-overview}
Significant progress towards proving~\cref{thm:horrocks-k2} has already been achieved in~\cite{LS20}.
In this subsection we reduce~\cref{thm:horrocks-k2} to a specific technical statement which will be addressed in the following sections.

The following lemma provides the initial key reduction in the proof of~\cref{thm:horrocks-k2}.
\begin{lemma} \label{lem:first-reduction}
Let $A$ be a local ring with maximal ideal $M$ and residue field $k$.
Let $\Phi$ be as in the statement of~\cref{thm:horrocks-k2}.
Suppose that the canonical homomorphism
\begin{equation} \label{eq:c-surj} C(\Phi, A[X], M[X]) \to C(\Phi, A[X, X\inv], M[X, X\inv]) \end{equation}
is surjective.
Then the square~\eqref{eq:P1-square} is pullback.
\end{lemma}
\begin{proof}
    Denote by $R$ the Laurent polynomial ring $A[X, X\inv]$ and by $B$ the subring $A[X\inv] + M[X] \subseteq R$.
    We also set $I \subseteq M[X, X\inv]$, which is clearly an ideal of both $B$ and $R$.
    Consider the following diagram with rows obtained from~\eqref{eq:relative-Steinberg}:
    \[\begin{tikzcd}
          C(\Phi, B, I) \arrow{r} \arrow[d, swap, twoheadrightarrow] & \St(\Phi, B, I) \arrow{r}{\mu_B} \arrow{d} & \St(\Phi, B) \arrow{r} \arrow{d} & \St(\Phi, k[X^{-1}]) \arrow[hookrightarrow]{d} \\
          C(\Phi, R, I) \arrow{r} & \St(\Phi, R, I) \arrow{r}{\mu_R} \arrow[ur, "t", dashrightarrow] & \St(\Phi, R) \arrow{r} & \St(\Phi, k[X, X^{-1}]).
    \end{tikzcd}\]
    The lifting $t$ is obtained from~\cite[Lemma~3.3]{LS20}.
    The right-hand side vertical arrow is injective by~\cite[Lemma~2.2]{LS20}.
    The left-hand side vertical arrow is surjective since the composite arrow
    $\xymatrix{ C(\Phi, A[X], M[X]) \ar[r] & C(\Phi, B, I) \ar[r] & C(\Phi, R, I) }$
    is surjective by lemma's assumption.
    It remains to repeat the diagram chasing argument of~\cite[Theorem~1]{LS20} to conclude that the homomorphism $\St(\Phi, B) \to \St(\Phi, A[X, X\inv])$ is injective.
    The assertion of the lemma will then follow from~\cite[Theorem~3]{LS20}.
\end{proof}
The proof of Horrocks theorem is, thus, reduced to showing that~\eqref{eq:c-surj} is surjective for every local domain $(A, M)$.
In order to formulate the second reduction we need to recollect some notation pertaining to subgroups of $\St(\Phi, R)$.

For a \textit{special} subset of $\Phi$ (i.\,e. a subset $S \subseteq \Phi$ such that $S \cap -S = \varnothing$)
we denote by $\UU(S, R)$ the subgroup of $\St(\Phi, R)$ generated by $x_\alpha(a)$, where $a \in R$, $\alpha \in S$.
Similarly, for an ideal $I \trianglelefteq R$ we denote by $\UU(S, I)$ the subgroup of $\St(\Phi, R, I)$ generated by $x_\alpha((0;m))C$, $m \in I$, $\alpha \in S$.
Notice that the restriction of $\pi$ to $\UU(S, R)$ is a monomorphism, so $\UU(S, R)$ is isomorphic to a unipotent subgroup of $\Gsc(\Phi, R)$.

Not let us introduce shorter notation for Steinberg groups and their subgroups that will be relevant for the proof of Horrocks theorem.
\begin{align*}
    G     =& \St(\Phi, A[X, X^{-1}]),\\
    G^+   =& \St(\Phi, A[X]),\\
    B     =& \UU(\Phi^+, A[X, X\inv]) \rtimes \StH(\Phi, A[X, X\inv]) \leq G,\\
    G_M   =& \St(\Phi, A[X, X^{-1}], M[X, X^{-1}]),\\
    U^+   =& \UU(\Phi^+, A[X]),\\
    G^+_M =& \St(\Phi, A[X], M[X]),\\
    B_M   =& \UU(\Phi^+, M[X, X^{-1}]) \times \{X, 1+M\} \leq G_M,\\
    U^+_M =& \UU(\Phi^+, M[X]).
\end{align*}
In the definition of $B_M$ we denote by $\{X, 1+M\}$ the image of the homomorphism $\{X, -\}_{r}$ from~\eqref{eq:relative-symbol}.

The above groups can be organized into the following commutative diagram:
\begin{equation} \label{eq:cube-Steinberg} \xymatrix{
    U^+_M \ar@{^{(}->}[rr] \ar@{^{(}->}[dd] \ar@{^{(}->}[rd] &                        & B_M \ar[dd]^(.3){g_B^M} \ar@{^{(}->}[rd]^{g^B_M} &           \\
    & G^+_M \ar[rr]^(.3){g^+_M} \ar^(.3){g^M_+}[dd] &                   & G_M \ar[dd]_{h_M} \\
    U^+ \ar@{^{(}->}[rr]^(.3){g_B^+} \ar@{^{(}->}[rd]_{g_+^B}          &                        & B \ar@{^{(}->}[rd]_{h_B}       &           \\
    & G^+ \ar[rr]_{h_+}              &                   & G.}\end{equation}
Here $g^M_+$ and $h_M$ are just renamed homomorphisms $\mu$ from~\eqref{eq:relative-Steinberg}.
Homomorphism $g^M_B$ is also induced by $\mu$.
Maps $g_M^+$ and $h^+$ are induced by the ring embedding $A[X] \to A[X, X\inv]$.
The other homomorphisms on the diagram are all obvious subgroup embeddings.

Set $V = G^+\times B \times G_M$, $W = U^+\times G^+_M \times B_M$.
Both homomorphisms $g_+^M$ and $h_M$ are crossed modules by~\cref{lem:rel-Steinberg-crossed-module}.
The fact that $(g^+_M, h_+)$ is a map of precrossed modules is obvious.
Thus, we find ourselves in the situation of~\cref{subsec:triples}.

The following lemma provides the second key reduction in the proof of Horrocks theorem.
\begin{lemma}
    Suppose that $A$ is a domain.
    Suppose that $V/\sim_W$ admits an action of $G$ with the only property that
      for $g \in G_M$ one has \[ h_M(g) \cdot [1, 1, 1] = [1, 1, g]. \]
    Then the canonical homomorphism~\eqref{eq:c-surj} is surjective.
\end{lemma}
\begin{proof}
  Since $A$ is a domain, $g^M_B$ is injective by~\cref{lem:symbols}.
    Thus, the back face of~\eqref{eq:cube-Steinberg} is pull-back.
  Notice that for any $g \in C(\Phi, A[X, X\inv], M[X, X\inv]) \subseteq G_M$ one has
    \[ [1, 1, 1] = h_M(g) \cdot [1, 1, 1] = [1, 1, g].\]
  It is clear now that $g$ lies in the image of $C(\Phi, A[X], M[X])$ under $g^+_M$ by~\cref{lem:one-one-z}.
\end{proof}

\begin{rem}
    In the linear case ($\Phi = \rA_\ell$) the assertion of the lemma remains true without the assumption that $A$ is a domain.
\end{rem}

\begin{comment}
\subsection{Overview of Main structure theorem}

Let $A$ be a local ring with maximal ideal $M$ and residue field $k$.
Recall from~\cite[\S~4]{Su77} or~\cite[\S~VI.6]{Lam10} that for $r \geq 3$ the elementary linear group $\E_r(A[X, X\inv])$ admits the following decomposition:
\begin{equation}\label{eq:triple-decomposition}
\E_r(A[X^{\pm 1}]) = \E_r(A[X]) \cdot B(A[X^{\pm 1}]) \cdot \E_r(A[X^{\pm 1}], M[X^{\pm 1}]).
\end{equation}
Here $\E_r(-)$ denotes the same object as $\Esc(\rA_{r - 1}, -)$ before.
Recall that the relative elementary subgroup $\E_r(R, I)$ is defined as the kernel of canonical reduction homomorphism $\E_r(R) \to \E_r(R/I).$
Recall also that $B(R)$ denotes the Borel subgroup (i.\,e. the semidirect product of the unipotent radical $\UU(\Phi^+, R)$ and the group of diagonal matrices).

Decomposition~\eqref{eq:triple-decomposition} is a key tool in the study of the {$\K_1$-analogue of Serre's problem}.
We want to devise some analogue of it which would be suitable for study of Steinberg groups.
\end{comment}

\subsection{Hermitian presentation of affine Steinberg groups}

\tikzset{
    root/.style={circle, draw, minimum size=0.2cm, inner sep=0},
    zeroroot/.style={circle, draw, minimum size=0.2cm, inner sep=0, fill=yellow},
    highlighted/.style={circle, draw, minimum size=0.2cm, inner sep=0, fill=green},
    levi/.style={draw, dashed, rounded corners},
    dottededge/.style={dotted},
    labeled/.style={below}
}

\begin{longtable}{ c c c }
    \scalebox{0.69}{\begin{tikzpicture}
          \node[root, highlighted, label=$1$] (d1) at (0,0.5) {};
          \node[root, label=$2$] (d2) at (1,0) {};
          \node[root, label=$3$] (d3) at (2,0) {};
          \node at (3,0) {\ldots};
          \node[root, label={[xshift=-0.5cm, yshift=-0.6cm]$\ell-2$}] (dl2) at (4,0) {};
          \node[root, label={[xshift=-0.6cm, yshift=-0.35cm]$\ell-1$}] (dl1) at (5,0.5) {};
          \node[root, label={[xshift=-0.5cm, yshift=-0.3cm]$\ell$}] (dl) at (5,-0.5) {};
          \node[root, zeroroot, label=below:{$0$}] (d0) at (0,-0.5) {};

          \draw (d1) -- (d2) -- (d3) -- (2.5,0);
          \draw (3.5,0) -- (dl2) -- (dl1);
          \draw (dl2) -- (dl);
          \draw[dottededge] (d2) -- (d0);

          \begin{scope}
              \node[levi, fit=(d2) (d3) (dl) (dl1) (dl2), label=above:{$\Delta$}] {};
          \end{scope}
    \end{tikzpicture}}
    &
    \scalebox{0.69}{\begin{tikzpicture}
        \begin{scope}
            \node[root, highlighted, label=$1$] (e1) at (0,0) {};
            \node[root, label=$3$] (e3) at (1,0) {};
            \node[root, label={[xshift=-0.3cm]$4$}] (e4) at (2,0) {};
            \node[root, label=$5$] (e5) at (3,0) {};
            \node[root, label=$6$] (e6) at (4,0) {};
            \node[root, label={[xshift=-0.3cm,yshift=-0.3cm]$2$}] (e2) at (2,1) {};
            \node[root, zeroroot, label=left:$0$] (e0) at (2,2) {};
            \draw (e1) -- (e3) -- (e4) -- (e5) -- (e6);
            \draw (e2) -- (e4);
            \draw[dottededge] (e2) -- (e0);

            \begin{scope}
                \node[levi, fit= (e2) (e3) (e4) (e5) (e6), label=above:{$\Delta$}] {};
            \end{scope}
        \end{scope}
    \end{tikzpicture}}
    &
    \scalebox{0.69}{\begin{tikzpicture}
        \begin{scope}
            \node[root, zeroroot, label={0}] (e0) at (0,0) {};
            \node[root, label={$1$}] (e1) at (1,0) {};
            \node[root, label={$3$}] (e3) at (2,0) {};
            \node[root, label={[xshift=-0.3cm]$4$}] (e4) at (3,0) {};
            \node[root, label={$5$}] (e5) at (4,0) {};
            \node[root, label={$6$}] (e6) at (5,0) {};
            \node[root, label={[xshift=-0.3cm,yshift=-0.3cm]$2$}] (e2) at (3,1) {};
            \node[root, highlighted, label={$7$}] (e7) at (6,0) {};

            \draw (e1) -- (e3) -- (e4) -- (e5) -- (e6) -- (e7);
            \draw (e2) -- (e4);
            \draw[dottededge] (e0) -- (e1);

            \begin{scope}
                \node[levi, fit=(e1) (e2) (e3) (e4) (e5) (e6), label=above:{$\Delta$}] {};
            \end{scope}
        \end{scope}
    \end{tikzpicture}} \\
    \text{$\rD_\ell$} &
    \text{$\rE_6$} &
    \text{$\rE_7$}
\end{longtable}






$\Delta$, $\Sigma^+$, $\Sigma^-$.

By our assumption $\UU(\Sigma^+, R)$ is an abelian group.

Denote by $G$ the amalgamated product of $\St(\Phi, A[X])$ and the cyclic group $\mathbb{Z}$, whose generator we denote by $\sigma$ amalgamated over the following relations:
\begin{align}
    \sigma \cdot x_{\alpha}(Xf) \sigma^{-1} =& x_{\alpha} (f), & \alpha \in \Sigma^+, f \in A[X], \\
    \sigma \cdot x_{\beta}(f) \sigma^{-1}   =& x_{\beta} (Xf), & \beta \in \Sigma^-, f \in A[X], \\
    \sigma \cdot x_\gamma(f) \sigma^{-1}    =& x_\gamma(f), & \gamma \in \Delta, f \in A[X].
\end{align}
We denote by $i_+$ the canonical homomorphism $\St(\Phi, A[X]) \to G$.

The main result of this subsection is the following
\begin{prop}
    There exists an arrow $j$ making the diagram below commute:
    \[\begin{tikzcd}           & \St(\Phi, A[X, X\inv]) \arrow[rd, dashrightarrow, "j"] & \\
           \St(\Phi, A[X]) \arrow{ru}{h_+} \arrow{rr}{i_+} &                                & G.
    \end{tikzcd}\]
\end{prop}

To construct the arrow $j$ we will use an alternative presentation of $\St(\Phi, A[X, X\inv])$ with fewer generators from~\cite{LS20}.
To formulate it we need to introduce additional notation.

Recall that the Laurent polynomial ring $R = A[X, X\inv]$ can be reinterpreted as a $\ZZ$-graded ring in which $X^{-1}$ has degree $1$.
We denote by $R_d$ the degree $d$ component of $R$.
It is clear that $R_d$ is isomorphic to $A$ as additive group.

We call a generator $x_\alpha(r)$ of the group $\St(\Phi, R)$ \textit{homogeneous of degree $d$} if $r = aX^d \in R_d$ for some $a\in A$.
We denote by $\St^{\leq n}(\Phi, R)$ the group presented by all homogeneous generators of degree $\leq d$ and the following 4 families of homogeneous relations (in which $d, e \leq n$):
\begin{align}
    x_{\alpha}(r) x_{\alpha}(s)    &= x_{\alpha}(a + b), & \text{ for } r, s \in R_d, \tag{R$1_d$} \\
    [x_{\alpha}(r),\ x_{\beta}(s)] &= x_{\alpha + \beta}(N_{\alpha, \beta} \cdot rs), &\text{ for } \alpha + \beta \in \Phi,\ r \in R_d, s \in R_e, \tag{R$2_{d, e}$} \\
    [x_{\alpha}(r),\ x_{\beta}(s)] &= 1, &\text{ for }\alpha - \beta \in \Phi, r \in R_d, s \in R_e, \tag{R$3_{d, e}^\angle$} \\
    [x_{\alpha}(r),\ x_{\beta}(s)] &= 1, &\text{ for }\alpha \perp \beta, r \in R_d, s \in R_e. \tag{R$3_{d,e}^\perp$}
\end{align}
By a \textit{degree of a relation} we mean the maximum of degrees of generators appearing in the relation.
For example, the degree of R$3_{d, e}$ is $\max(d, e)$, while the degree of R$2_{d, e}$ is $\max(d, e, d+e).$

\begin{lemma} \label{lem:homog-presentation}
    Let $\Phi$ be a simply-laced root system of rank $\geq 3$. %TODO: Check if we may have such generality.
    The group $\St(\Phi, R)$ admits presentation by homogeneous generators and homogeneous relations of degree $\leq 1$.
    Moreover, we may omit from this presentation all relations of type $R3^\perp_{1, 1}$.
\end{lemma}
\begin{proof}
    This is precisely the assertion of~\cite[Proposition~5.3]{LS20} combined with~\cite[Lemma~5.2]{LS20} and~\cite[Remark~5.5]{LS20}.
\end{proof}

We now proceed with the construction of homomorphism $j$.
First of all, we set
\begin{align}
    j(x_\alpha(a))       & = x_\alpha(a) & \text{ for } r \in R_d,\, d \geq 0, \\
    j(x_\alpha(a X\inv)) & = \sigma \cdot x_\alpha(a) \sigma\inv & \text{ for } a \in A,\, \alpha \in \Sigma^+, \\
    j(x_\beta(bX \inv))  & = \sigma^{-1} \cdot x_\beta(b) \sigma & \text{ for } b \in A,\, \alpha \in \Sigma^-.
\end{align}

It remains to define the value of $j$ on the generator $x_\gamma(cX\inv)$ for $\gamma \in \Delta$.
To achieve this we decompose $\gamma$ as a sum $\alpha + \beta$ for some $\alpha \in \Sigma^+$ and $\Sigma^-$
 and present $c$ as a product $N_{\alpha, \beta} \cdot a b$ for some $a, b \in A$. % TODO: (why is this possible?)
Now we set $j(x_\gamma(c X\inv)) = [j(x_\alpha(a X\inv)), x_\beta(b)]$.
We need to check that this definition is unambiguous.
It is \textbf{NOT} clear if $j(x_\gamma(c X\inv)) = [x_\alpha(a), j(x_\beta(bX\inv))]$.


    \section{Proof of main results} \label{sec:main}
    Throughout this section we also denote by $\lambda_a$ the homomorphism of principal localization at $a \in A$.
Also for a prime ideal $M \trianglelefteq A$ we denote by $\lambda_M \colon A \to A_M$ the homomorphism inverting the multiplicative subset $A \setminus M$.

For a commutative ring $A$, and a $A$-algebra $B$ and $b\in B$ we denote by $\ev{A}{B}{b}\colon A[t]\rightarrow B$ the morphism of $A$-algebras evaluating each polynomial
 $p(t)\in A[t]$ at $b$, i.e. $\ev{A}{B}{b} (p(t)) = p(b)$.
\subsection{Early stability theorem}
The aim of this subsection is to prepare necessary technical ingredients for the proof of early stability theorem~\cref{cor:dedekind}.

For the rest of this subsection $\Psi$ is an arbitrary irreducible simply-laced root system of rank $\geq 3$ embedded into another irreducible simply-laced root system $\Phi$.
For a commutative ring $R$ we denote by $j_R$ the corresponding homomorphism of Steinberg groups $\St(\Psi, R) \to \St(\Phi, R)$ induced by this embedding.

The following result is the analogue of the so-called dilation principle for subsystem embeddings.
\begin{lemma}\label{lem:dp-2}
Let $h\in\St(\Phi, A[X], X A[X])$ be such that $\lambda_a^*(h) \in \Img(j_{A_a[X]})$.
Then for sufficiently large $n$ one has
\[\ev{A}{A[X]}{a^n\cdot X}^*(h) \in \mathrm{Im}(j_{A[X]}).\]
\end{lemma}
\begin{proof}
 We denote by $A\ltimes XA_a[X]$ the semidirect product of $A$ and the ideal $XA_a[X]$, cf. e.\,g.~\cite[Definition~3.2]{S15}.
 Denote by $\theta$ the obvious map $A[X]\rightarrow A\ltimes XA_a[X]$ localizing at $a$ all coefficients of terms of degree $\geq 1$.

 Recall from~\cite[\S~2]{LS17} that there exists a homomorphism
 \[T_\Psi \colon \St(\Psi, A_a[X], XA_a[X]) \to \St(\Psi, A \ltimes XA_a[X])\]
 such that $\theta^* = T_\Psi \circ \lambda_a^*$.

 Let $(A_i, f_{ij})$ be the directed system of rings given by
 \[A_i\coloneqq A[X],\ f_{ij} \coloneqq \ev{A}{A[X]}{a^{j-i} \cdot X},\ 0 \leq i\leq j.\]
 It is easy to check that $\varinjlim A_i$ coincides with $A \ltimes XA_a[X]$.
 The canonical morphisms $A_j\rightarrow \varinjlim_i A_i \cong A \ltimes XA_a[X]$ can be easily computed as $\ev{A}{A\ltimes XA_a[X]}{a^{-j} \cdot X}$.

 By hypothesis $\lambda_a^*(h) = j_{A_a[X]}(h')$ for some $h' \in \St(\Psi, A_a[X], XA_a[X])$.
 Consequently, $\theta^*(h) = j_{A \ltimes XA_a[X]}(T_\Psi(h'))$
 and the assertion of the lemma follows from the fact that the Steinberg group functor commutes with
  colimits over directed systems (cf.~\cite[Lemma~2.2]{Tu83}):
 \[\St(\Psi, A\ltimes XA_a[X]) = \St(\Psi, \varinjlim_i A_i) \cong \varinjlim_i \St(\Psi,A_i). \qedhere\]
\end{proof}

\begin{lemma}\label{lem:L25-2}
Let $a$ and $b$ be a pair of coprime elements of $A$.
Let $g$ be an element of $\St(\Phi, A[X], XA[X])$ such that
$\lambda_a^*(g) \in \Img(j_{A_a[X]})$ and $\lambda_b^*(g) \in \Img(j_{A_b[X]})$.
Then $g$ lies in the image of $j_{A[X]} \colon \St(\Psi, A[X]) \to \St(\Phi, A[X])$.
\end{lemma}
\begin{proof}
    We reproduce the argument of~\cite[Lemma~2.5]{Tu83}, cf. also with~\cite[Lemma~16]{S15}.
    Set $S := A[X, Y]$.
    Consider the following element of $\St(\Phi, S[Z])$:
    \[h(X, Y, Z) := g(YX) \cdot  g^{-1}((Y+Z) X) = \ev{A}{S[Z]}{YX}^* \left(g\right) \cdot \ev{A}{S[Z]}{(Y + Z)X}^*\left(g^{-1}\right).\]
    It is easy to see that $h(X, Y, Z)$ belongs to
    \[\Ker\left(\eval{Z}{S}{S}{0}^*\colon\St(\Phi, S[Z]) \rightarrow \St(\Phi, S)\right)\]
    and hence by~\cite[Lemma~8]{S15} lies in $\St(\Phi, S[Z], Z S[Z])$.
    On the other hand, \begin{multline*}
                           \lambda_{a}^*(h(X, Y, Z)) = \left(\lambda_a\circ \ev{A}{S[Z]}{Y X}\right)^*(g) \cdot \left(\lambda_a\circ \ev{A}{S[Z]}{(Y + Z)X}\right)^*(g^{-1}) = \\
                           = \ev{A_a}{S_a[Z]}{YX}^*(\lambda_{a}^*(g)) \cdot \ev{A_a}{S_a[Z]}{(Y + Z)X}^*(\lambda_{a}^*(g^{-1})) \end{multline*}
    lies in the image of $j_{S_a[Z]}$.
    Similarly, $\lambda_{b}^*(h(X, Y, Z))$ lies in the image of $j_{S_b[Z]}$.

    We claim that there exists $n$ such that both $h(X, Y, a^n Z)$ and $h(X, Y, b^n Z)$
    lie in the image of $j(S[Z])$.

    By assumption, there exist $r, s\ \in A$ such that $r a^n + s b^n = 1$, consequently
    \begin{multline*}
        g(X) = g(X)\cdot g^{-1}(ra^n\cdot X) \cdot g(ra^n\cdot X) \cdot g^{-1}(0) = \\
         = h(X, 1, -sb^n) \cdot h(X, ra^n, -ra^n) \in \mathrm{Im}(j_{A[X]}). \qedhere
    \end{multline*} \end{proof}

The following result is the subsystem analogue of~\cite[Theorem~2]{LS17}, cf.\ also~\cite[Theorem~2.1]{Tu83}.
\begin{cor}[Local-global principle for subsystem embeddings] \label{cor:QS-subsystem}
    An element $g \in \St(\Phi, A[X], XA[X])$ lies in $\Img(j_{A[X]})$ if and only if
     $\lambda_M^*(g) \in \St(\Phi, A_M[X])$ lies in $\Img(j_{A_M[X]})$ for all maximal ideals $M \trianglelefteq A$.
\end{cor}
\begin{proof}
    It suffices to show ``if'' part of the statement.
    One defines the \textit{Quillen set} $Q(g)$ as the set consisting of those elements $a \in A$
    such that $\lambda_a^*(g) \in \Img(j_{A_a[X]})$.

    Repeating the same argument as in the proof of~\cite[Theorem~2]{S15} one shows using~\cref{lem:L25-2}
     that $Q(g)$ is an ideal of $A$, which can not be proper and therefore must coincide with $A$.
\end{proof}

Before we proceed further we would like to briefly recall the main construction from~\cite{LS20} upon which
 the proof of Theorem~1 ibid. is based.
Recall that one constructs an action of the group $\St(\Phi, A[X\inv] + M[X])$ on a certain set $\overline{V}$.
This set $\overline{V}$ is unrelated to the set $\overline{V}$ encountered in~\cref{sec:horrocks}.
Its construction proceeds as follows.

Set $G_{M, \Phi}^{\geq 0} \coloneqq \Img(\St(\Phi, A[X], M[X]) \to \St(\Phi, A[X, X\inv])), G_M^0 \coloneqq \overline{\St}(\Phi, A, M)$.
$G_M^0$ is easily seen to be a subgroup of both $\St(\Phi, A[X\inv])$ and $G_M^{\geq 0}$.
Denote by $\overline{V}$ the quotient-set of the product $V \coloneqq G_M^{\geq 0} \times \St(\Phi, A[X\inv]) \times (1 + M)^\times$
modulo the equivalence relation given by $(gh_0, h, u) \cong (g, h_0h, u)$ where $h_0 \in G_M^0, (g, h, u) \in V.$
Denote by $[g, h, u] \in \overline{V}$ the equivalence class corresponding to $(g, h, u)\in V$, cf.~\cite[\S~5.4]{LS20}.

Notice also that although the results in~\cite[\S~5.5]{LS20} are conditional,
 i.\,e. they are formulated under additional assumption that the canonical homomorphism
 $\St(\Phi, A[X\inv] + M[X]) \to \St(\Phi, A[X, X\inv])$ is injective,
this does not present a problem for us since this condition has been checked already during the proof of~\cref{lem:first-reduction}.

\begin{prop} \label{prop:horrocks-main} The group $\St(\Phi, A[X\inv] + M[X])$ acts simply transitively on $\overline{V}$.
This action satisfies the following additional property.
Suppose that for some
\[g \in \Img(j\colon \St(\Psi, A[X\inv] + M[X]) \to \St(\Phi, A[X\inv] + M[X])), \]
$h, h' \in \St(\Phi, A[X\inv])$ and $u, h' \in 1 + M$ one has
\[ g \cdot [1, h, u] = [g', h', u'].\]
Then $g'$ belongs to $\Img(j\colon G_{M, \Psi}^{\geq 0} \to G_{M, \Phi}^{\geq 0})$.
\end{prop}
\begin{proof}
    The existence of the action of $\St(\Phi, A[X\inv] + M[X])$ and its faithfullness are contained in
    Proposition~5.39 and Remark 5.42 of~\cite{LS20}.
    The stated property follows from the construction of the action in~\cite[\S~5.4]{LS20}.
\end{proof}

The following lemma generalizes~\cite[Proposition~4.3(b)]{Tu83}.
\begin{lemma} \label{lem:horrocks-subsystem-local}
Let $A$ be a local domain with maximal ideal $M$ and residue field $\kappa$.
Suppose that the image in $\St(\Phi, A[X, X\inv])$ of the element $x \in \K_2(\Phi, A[X], XA[X])$
 can be decomposed as $j_{A[X, X\inv]}(y) \cdot \lambda_{X^{-1}}^*(z)$ for
 some $y \in \St(\Psi, A[X, X\inv])$ and $z \in \St(\Phi, A[X\inv])$
Then $x$ belongs to $\Img(j_{A[X]})$.
\end{lemma}
\begin{proof}
    We denote by $\rho_{A}$ (resp. $\rho_{A[X]}$, $\rho_{A[X, X\inv]}$) the canonical homomorphism
     $A \to \kappa$ (resp. $A[X] \to \kappa[X]$, $A[X, X\inv] \to \kappa[X, X\inv]$).

    Recall from~\cite{Hur77} that $\K_2(\Phi, \kappa[X]) = \K_2(\Phi, \kappa)$ hence one has $\rho_{A[X]}(x) = 1$
     hence by our assumptions $\rho_{A[X, X\inv]}(j_{A[X, X\inv]}(y) \cdot z) = 1$.
    Consequently, we can find $z_0 \in \St(\Psi, A[X\inv])$ and $z_1 \in \overline{\St}(\Phi, A[X^{-1}], M[X^{-1}])$
     such that $z = j_{A[X\inv]}(z_0) \cdot z_1$.
    It is clear that $\rho_{A[X, X\inv]}(y \cdot \lambda_{X^{-1}}(z_0)) = 1$ hence we can find
     $y' \in \St(\Psi, A[X, X\inv], M[X, X\inv])$ such that $\mu(y') = \cdot \lambda_{X^{-1}}(z_0)$.
    Notice also that $t(y') \in \St(\Phi, A[X\inv] + M[X])$, where $t$ denotes the homomorphism discussed in~\cref{lem:first-reduction}.
    From~\cref{prop:horrocks-main} we obtain that
    \[ [x, 1, 1 ] = t(y') [1, z_1, 1] = [g', h', u']. \]
    for some $g' \in \Img(\G_{M, \Psi}^{\geq 0} \to G_{M, \Phi}^{\geq 0})$.
    From the definition of $\overline{V}$ we obtain that $x = g' \cdpt h_0$ for some $h_0 \in \overline{\St}(\Phi, A, M)$.
    But since $x(0) = 1$ we conclude that $x = g' \cdot (g'(0))^{-1}$ from which the assertion of the lemma follows.
\end{proof}

\begin{cor} \label{cor:horrocks--ingredient}
    Let $A$ be an arbitrary commutative domain.
    Suppose that the image in $\St(\Phi, A[X, X\inv])$ of the element $x \in \K_2(\Phi, A[X], XA[X])$
    can be decomposed as $j_{A[X, X\inv]}(y) \cdot \lambda_{X^{-1}}^*(z)$ for
    some $y \in \St(\Psi, A[X, X\inv])$ and $z \in \St(\Phi, A[X\inv])$
    Then $x$ belongs to $\Img(j_{A[X]})$.
\end{cor}
\begin{proof}
    This is a consequence of~\cref{lem:horrocks-subsystem-local} and~\cref{cor:QS-subsystem}.
\end{proof}

\subsection{Main results}

We start by recalling the so-called Zariski excision property of Steinberg groups.
\begin{lemma} \label{lem:zariski-glueing}
Let $\Phi$ be any simply-laced root system of rank $\geq 3$.
Let $A$ be a commutative domain and $a, b \in A$ be a pair of coprime elements.
\begin{enumerate}
    \item Let $\delta$ be an element of $\St(\Phi, A_{ab})$.
    Then $\delta$ can be presented as $\lambda_b(x) \cdot \lambda_a(y)$ for some
    $x  \in \St(\Phi, A_a)$ and $y \in \St(\Phi, A_b)$.
    \item  Let $x \in \St(\Phi, A_a)$ and $y \in \St(\Phi, A_b)$ be such that the equality $\lambda_b(x) = \lambda_a(y)$ holds in $\St(\Phi, A_{ab})$.
    Then there exists $z \in \St(\Phi, A)$ such that $x = \lambda_a(z)$, $y = \lambda_b(z)$.
\end{enumerate}
\end{lemma}
\begin{proof}
    This is a special case of Nisnevich excision for domains, see~\cite[Proposition~4.5]{LSV2}
    (cf. also the proof of~\cite[Lemma~2.6]{LSV2}).
\end{proof}


\begin{lemma} \label{lem:horrocks-b}
 Let $A$ be a commutative domain and $f \in A[X]$ be a unitary polynomial.
 Let $h$ be an element of $\K_2(\Phi, A[X], XA[X])$ such that $\lambda_f(h)$ belongs to the image of the stabilization map
 $\St(\Psi, A[X]_f) \to \St(\Phi, A[X]_f)$.
 Then $h$ lies in the image of $\St(\Psi, A[X]) \to \St(\Phi, A[X])$.
\end{lemma}
\begin{proof}
    Suppose that $f(X) = X^n + a_1 X^{n-1} \ldots + a_n$.
    Set $g(X\inv) = 1 + a_1 X\inv + \ldots + a_{n} X^{-n}$.
    It is clear that $A[X, X\inv]_f = A[X\inv]_{X\inv g}$ and, moreover, that $X\inv$ and $g$ are not zero divisors and together generate the unit ideal of $A$.

    Consider the following diagram:
    %! suppress = EscapeAmpersand
    \[ \xymatrix{\St(\Phi, A[X]) \ar[r]^{\lambda_X} \ar[d]_{\lambda_f} & \St(\Phi, A[X, X\inv]) \ar[d]_{\lambda_f}  & \St(\Phi, A[X\inv]) \ar[l]_{\lambda_{X\inv}} \ar[d]_{\lambda_g}  \\
                 \St(\Phi, A[X]_f) \ar[r] & \St(\Phi, A[X, X\inv]_f) & \St(\Phi, A[X\inv]_g) \ar[l] \\
                 \St(\Psi, A[X]_f) \ar[r] \ar[u]_j & \St(\Psi, A[X, X\inv]_f) \ar[u]_j & \St(\Psi, A[X\inv]_g) \ar[l] \ar[u]_j \\
                                   & \St(\Psi, A[X, X\inv]) \ar[u]_{\lambda_f^\Psi}   & \St(\Psi, A[X\inv]). \ar[l] \ar[u]_{\lambda_g^\Psi}}\]

    By assumption there exists $\widetilde{h} \in \St(\Psi, A[X]_f)$ such that $j(\widetilde{h}) = \lambda_f(h)$.
    By the first part of~\cref{lem:zariski-glueing} one can write 
     $\lambda_X(\widetilde{h}) = \lambda_f^\Psi(z) \cdot \lambda_{X\inv}(y)$
     for some $y \in \St(\Psi, A[X\inv]_g)$, $z \in \St(\Psi, A[X, X\inv]).$
    Consequently, one has $\lambda_f(j(z)\inv \cdot \lambda_X(\alpha)) = \lambda_{X\inv}(j(y))$.
    By the second part of~\cref{lem:zariski-glueing} there exists $y' \in \St(\Phi, A[X\inv])$ such that $\lambda_X(\alpha) = j(z) \cdot \lambda_{X\inv}(y').$
    The assertion now follows from~\cref{cor:horrocks--ingredient}.
\end{proof}

\begin{thm}\label{thm:early-stability}
Let $\Phi$ be a root system of type $\rA_{\geq 4}$, $\rD_{\geq 5}$ or $\rE_{6,7,8}$ and let $A$ be an arbitrary noetherian commutative
 domain of Krull dimension $\leq 1$.
Then for any $n \geq 0$ the obvious inclusion $\rA_4 \subseteq \Phi$ induces a surjection
\[\K_2(\rA_4, A[X_1,\ldots, X_n]) \to \K_2(\Phi, A[X_1, \ldots X_n]).\]
\end{thm}
\begin{proof}[Proof of Theorem 1]
    The proof is modeled after the proof of~\cite[Theorem~5.3]{Tu83}.
    We proceed by induction on $n$.
    Our assumption on the dimension of $A$ guarantees that it satisfies the condition $\mathrm{SR}_3$ in the sense of~\cite{St78}.
    Thus, from Corollary~3.2 and Theorem~4.1 of~\cite{St78} we conclude that the composite arrow in the following diagram is a surjection:
    \[\K_2(\rA_2, A) \to \K_2(\rA_4, A) \to \K_2(\Phi, A).\]
    Consequently, we obtain that the right arrow is a surjection, which yields the induction base.

    Now let us verify the induction step.
    Set $C = A[X_2, \ldots , X_n]$ and $B = C[X_1]$.
    We need to show that $\K_2(\rA_4, B) \to \K_2(\Phi, B)$ is surjective.
    Every element $\alpha \in \K_2(\Phi, B)$ can be decomposed as a product $\alpha = \alpha_0 \cdot \alpha_1$,
      where $\alpha_0 \in \K_2(\Phi, C)$ and $\alpha_1 \in \K_2(\Phi, B, X_1 B)$.
    By inductive assumption $\K_2(\rA_4, C)$ surjects onto $\K_2(\Phi, C)$, so it remains to show that $\alpha_1$ lies in the image of $\K_2(\rA_4, B)$.

    Denote by $S$ the multiplicative system $S \subseteq B$ consisting of polynomials $f$ such that for sufficiently large $m$
    the polynomial $f$ becomes unitary in $Y_1$ after substitutions $X_1 \coloneqq Y_1,$ $X_2 \coloneqq Y_2 + Y_1^m, \ldots X_n \coloneqq Y_n + Y_1^{m^n}$.
    Recall from~\cite[\S~6]{Su77} that $\dim(S^{-1}B) \leq \dim(A) = 1$.
    By induction base the map $\K_2(\rA_4, S^{-1}B) \to \K_2(\Phi, S^{-1}B)$ is surjective.
    Since functor $\K_2$ commutes with filtered colimits (cf. \cite[Lemma~3.3]{LSV2}) there exists $f \in S$ such that $\lambda^*_f(\alpha_1)$ lies in the image of $\K_2(\rA_4, B_f) \to \K_2(\Phi, B_f)$.
    By the construction of $S$ we may assume that $f$ is unital in $X_1$.
    The required assertion now follows from~\cref{lem:horrocks-b}.
\end{proof}

\begin{proof}[Proof of~\cref{thm:LP-for-K2}]
    This is what the proof of~\cite[Theorem~1.1]{LSV2} actually shows if we use the more general~\cref{thm:horrocks-k2}.
\end{proof}

\begin{proof}[Proof of~\cref{cor:motivic-pi1}]
    Repeat the proof of~\cite[Corollary~1.2]{LSV2} verbatim.
\end{proof}

\begin{proof}[Proof of~\cref{cor:dedekind}]
    Consider the following diagram:
    \[ \K_2(\rA_3, A) \to \K_2(\rA_4, A) \to \K_2(\rA_4, A[X_1, \ldots, X_n]) \rightarrow \K_2(A[X_1, \ldots, X_n]) \to \K_2(A).\]
    The two right arrows on the above diagram are isomorphisms by the $\mathbb{A}^1$-invariance of the stable $\K_2$
    (see e.\,g. \cite[Theorem~V.6.3]{Kbook}) combined with the main result of~\cite{Tu83}.
    On the other hand, by~\cite[Corollary~3.2]{ST76} the left arrow and the composite arrow are isomorphisms.
    Our assertion now follows from~\cref{thm:early-stability}.
\end{proof}

    \printbibliography
\end{document}