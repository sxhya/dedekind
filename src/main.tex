\documentclass[oneside, 10pt]{amsart}
\usepackage{amscd}
\usepackage{amsfonts}
\usepackage{amsmath}
\usepackage{amsthm, amstext, amssymb, comment, enumitem, verbatim, mathtools, xfrac, microtype, nameref, thmtools, tikz, tikz-cd}
\usepackage[breaklinks=true]{hyperref}
\usepackage[capitalize]{cleveref}
\usepackage[matrix,arrow,curve]{xy}
\usepackage[notref,notcite]{showkeys}
\usepackage{longtable}
\usepackage{mathrsfs}
\usetikzlibrary{shapes, arrows, calc, arrows.meta, fit, positioning, decorations.markings}
\usepackage[natbib=true, backend=biber, firstinits=true, style=numeric, sortlocale=en_US, url=false, doi=false, eprint=true, maxbibnames=4]{biblatex}
\usepackage{mathtools}
\usepackage{tikz}
\addbibresource{main.bib}
\renewbibmacro*{volume+number+eid}{\ifentrytype{article}{\- \iffieldundef{volume}{}{Vol.~\printfield{volume},}\iffieldundef{number}{}{ No.~\printfield{number},}}}
\renewbibmacro{in:}{\ifentrytype{article}{}{\printtext{\bibstring{in}\intitlepunct}}}
\newbibmacro{string+doi}[1]{\iffieldundef{doi}{\iffieldundef{url}{#1}{\href{\thefield{url}}{#1}}}{\href{https://dx.doi.org/\thefield{doi}}{#1}}}
\DeclareFieldFormat[article, inproceedings, inbook, book, online]{title}{\usebibmacro{string+doi}{\mkbibquote{#1}}}
\renewcommand*{\bibfont}{\footnotesize}

\DeclareMathOperator{\St}{St}
\DeclareMathOperator{\GL}{GL}
\DeclareMathOperator{\E}{E}
\DeclareMathOperator{\G}{G}
\DeclareMathOperator{\UU}{U}
\newcommand{\Gsc}{\G_\mathrm{sc}}
\newcommand{\Esc}{\E_\mathrm{sc}}
\DeclareMathOperator{\Hom}{Hom}
\DeclareMathOperator{\Um}{Um}
\DeclareMathOperator{\colim}{colim}
\DeclareMathOperator{\Aut}{Aut}
\DeclareMathOperator{\K}{K}
\DeclareMathOperator{\KO}{KO}
\DeclareMathOperator{\rk}{rk}
\DeclareMathOperator{\sr}{sr}
\DeclareMathOperator{\Ker}{Ker}
\DeclareMathOperator{\Stab}{Stab}
\DeclareMathOperator{\Img}{Im}
\newcommand{\catname}[1]{{\normalfont\textbf{#1}}}

\newcommand{\rA}{\mathsf{A}}
\newcommand{\rB}{\mathsf{B}}
\newcommand{\rC}{\mathsf{C}}
\newcommand{\rD}{\mathsf{D}}
\newcommand{\rE}{\mathsf{E}}


\newcommand{\ZZ}{\mathbb{Z}}
\newcommand{\StW}{\widehat{W}}
\newcommand{\StH}{\widehat{H}}
\newcommand{\inv}{^{-1}}
\newcommand{\RNum}[1]{\uppercase\expandafter{\romannumeral #1\relax}}
\newcommand{\eval}[4]{\mathrm{ev} \textstyle \left[\frac{#2[#1] \rightarrow #3}{#1 \mapsto #4}\right]}
\newcommand{\ev}[3]{\eval{X}{#1}{#2}{#3}}

\newcommand{\card}[1]{
    \begin{tikzpicture}
    \draw (0,0) rectangle (40mm, 40mm);
    \node (title) at (20mm, 35mm){ #1 };
    \end{tikzpicture}
}

\newtheorem{thm}{Theorem}
\Crefname{thm}{Theorem}{Theorems}
\numberwithin{equation}{section}
\numberwithin{thm}{section}

\newtheorem{lemma}[thm]{Lemma}
\numberwithin{lemma}{section}
\Crefname{lemma}{Lemma}{Lemmas}
\newlist{lemlist}{enumerate}{1} \setlist[lemlist]{label=\textrm{(\arabic{lemlisti})}, ref=\thelemma.(\arabic{lemlisti}),noitemsep} \Crefname{lemlisti}{Lemma}{Lemma}

\newtheorem{cor}[thm]{Corollary}
\Crefname{cor}{Corollary}{Corollaries}

\newtheorem{prop}[lemma]{Proposition}
\Crefname{prop}{Proposition}{Propositions}

\newtheorem*{thm*}{Theorem}
\newtheorem*{lemma*}{Lemma}

\theoremstyle{definition}

\newtheorem{dfn}[lemma]{Definition}
\Crefname{dfn}{Definition}{Definitions}
\newtheorem{example}[lemma]{Example}
\newtheorem{problem}[lemma]{Problem}
\Crefname{example}{Example}{Examples}

\theoremstyle{remark}

\newtheorem{rem}[lemma]{Remark}
\Crefname{rem}{Remark}{Remarks}

\title{On the $\mathbb{A}^1$-invariance of $\K_2$ of Chevalley groups of simply-laced type}
\keywords {Steinberg group, $K_2$-functor, Serre problem {\em Mathematical Subject Classification (2010):} 19C20}
\author {Sergei Sinchuk}
\dedicatory{In the memory of my teacher Nikolai Vavilov}
\email {sinchukss at gmail.com}
\date {\today}
\begin{document}
    \maketitle


\begin{abstract}
In this paper we study the $\mathbb{A}^1$-invariance of the unstable functor $\mathrm{K}_2(\Phi, R)$
in the case when $\Phi$ is an irreducible root system of type $\mathsf{ADE}$ containing $\rA_4$ and not of type $\rE_8$.
We show that in the geometric case, i.\,e. when $R$ is a regular ring containing a field $k$
one has $\K_2(\Phi, R[t]) = \K_2(\Phi, R)$, which allows one to interpret the unstable $\K_2$ groups
as $\mathbb{A}^1$-fundamental groups of Chevalley--Demazure group schemes in the $\mathbb{A}^1$-homotopy category.
We also prove a variant of "early stability" theorem which allows one to find a generating set
of $\K_2(\Phi, A[X_1, \ldots X_n])$ in the case when $A$ is a Dedekind domain.
\end{abstract}

\section{Introduction}\label{sec:introduction}

Let $R$ be a commutative ring with a unit and $\Phi$ be a reduced irreducible root system of rank $\geq 2$.
Recall that a choice of a lattice $\Lambda$ lying between the root lattice $\mathbb{Z}\Phi$ and the weight lattice $P(\Phi)$
 specifies a Chevalley--Demazure group scheme $G=\G_\Lambda(\Phi, -)$ over $\ZZ$.
The Steinberg group $\St(\Phi, R)$ is a certain group defined by means of generators and relations, see~\cref{subsec:steinberg-preliminaries} for the definition.
By definition, the \textit{unstable $\K_2$-functor} $\K_2^G(R)$ is defined as the kernel of the canonical homomorphism $\St(\Phi, R) \to \G_\Lambda(\Phi, R)$.

This work continues the series of papers~\cite{LS20, LSV2} and has a twofold aim.
The first goal of this work is to generalize the main result of~\cite{LSV2} concerning the $\mathbb{A}^1$-invariance of the functor $\K_2^G(R)$.
Specifically, we prove the following invariance theorem, which generalizes~\cite[Theorem~1.1]{LSV2}:

\begin{thm}[The $\K_2$-analogue of Lindel--Popescu theorem] \label{thm:LP-for-K2}
 Let $k$ be an arbitrary field and $R$ be a regular ring containing $k$.
 Let $\Phi$ be an irreducible simply-laced root system containing $\rA_4$ but not of type $\rE_8$.
 Then for any lattice $\Lambda$ as above and $G = \mathrm{G}_\Lambda(\Phi, -)$ one has
 $\K_2^G(R[t])\cong\K_2^G(R).$
\end{thm}

The primary corollary of \cref{thm:LP-for-K2} is the following result, which interprets the unstable $\K_2$-functors as fundamental groups of group schemes
 $\mathrm{G}_\Lambda(\Phi, -)$ in the motivic category of F.~Morel and V.~Voevodsky extending the result of~\cite[Corollary~1.2]{LSV2}.

\begin{cor} \label{cor:motivic-pi1} Let $\Phi$, $\Lambda$ and $k$ be as in~\cref{thm:LP-for-K2}.
Then for any $k$-smooth algebra $R$ and $G = \mathrm{G}_{\Lambda}(\Phi, -)$ one has
\[ \pi_1^{\mathbb{A}^1}(G)(R) := \mathrm{Hom}_{\mathscr{H}_{\bullet}(k)}(S^1 \wedge \mathrm{Spec}(R)_+, G) = \mathrm{KV}_2^{G}(R) = \K_2^G(R).\]
\end{cor}
In the above statement $\mathscr{H}_\bullet(k)$ denotes the pointed unstable $\mathbb{A}^1$-homotopy category over $k$, see~\cite{MV99}.
$\mathrm{KV}_2^{G}(R)$ denotes the second Karoubi--Villamayor K-functor, i.\,e. the fundamental group of
 the simplicial group $G(R[\Delta^\bullet])$, see e.\,g.~\cite[\S~3]{Jar83} or~\cite[\S~3.2]{LSV2}.

The second primary objective of this paper is to establish a result that describes the structure of unstable $\K_2$-groups
 over multivariate polynomial rings over Dedekind domains.
Conceptually, this should be compared to the solution of the Bass--Quillen conjecture for one-dimensional rings, as discussed in~\cite[\S~V.3]{Lam10},
 and to analogous results for the unstable $\K_1$-functor, e.\,g.~\cite[Theorem~1.1]{St-Ded}.

\begin{thm} \label{cor:dedekind}
Let $\Phi$ be as in the statement of~\cref{thm:LP-for-K2} and let $A$ be a Dedekind domain.
Then for any $n \geq 0$ the group $\K_2(\rA_4, A)$ surjects onto $\K_2(\Phi, A[X_1,\ldots, X_n])$.
In particular, $\K_2(\Phi, A[X_1,\ldots, X_n]) = \K_2(\Phi, A)$.
Moreover, if $\K_2(A)$ is generated by Dennis--Stein symbols then for any $n \geq 0$ so is $\K_2(\Phi, A[X_1,\ldots, X_n])$.
\end{thm}

In turn, \cref{thm:LP-for-K2} can be viewed as an analogue, for the functor $\K_2^G$, of the theorem of H.~Lindel and D.~Popescu,
which resolved the classical Bass--Quillen conjecture in the geometric case — i.\,e., for commutative regular $k$-algebras of arbitrary dimension.
We highlight here that the main novelty of \cref{thm:LP-for-K2} is its applicability to root system types $\Phi = \rE_6, \rE_7$,
as well as its extension to orthogonal groups over the base ring, without requiring that $2$ be invertible in $R$.

In the special case when $R$ is a field, homotopy invariance type results for $\K_2$ similar to~\cref{thm:LP-for-K2} have been known since 1970's, cf. e.\,g.~\cite{Hur77, Re75, VW16}.
Notice also that the homotopy invariance for the unstable $\K_1$-functors and their interpretation as groups of connected components of the corresponding group schemes in the motivic homotopy category have been
 recently obtained in much greater generality by A.~Stavrova, see~\cite[Theorem~1.3]{St-poly}, \cite[Theorem~5.2]{St22}.
Notice also that the Nisnevich sheaf $\bm{\pi}_1^{\mathbb{A}^1}(G)$ associated to the presheaf of fundamental groups $\pi_1^{\mathbb{A}^1}(G)(R)$
 has been recently computed by F.~Morel and A.~Sawant for all split reductive $G$ without any assumptions on the rank of $G$, see~\cite[Theorem~1]{MS23}.

The main technical ingredient in the proof of all our results is the refined Horrocks theorem for $\K_2$,~\cref{thm:horrocks-k2},
 which generalizes~\cite[Theorem~1]{LS20} and~\cite[Proposition~4.3]{Tu83}.
As noted in~\cite{LSV2}, the restrictive nature of these two Horrocks-type theorems for $\K_2$ was a major obstacle to proving a more
 general result in~\cite{LSV2}, which this paper addresses.
Our proof of~\cref{thm:horrocks-k2} follows the same general approach as that of~\cite[Proposition~4.3]{Tu83},
 with the key difference being our use of more modern technical tools, such as the amalgamation theorem for affine Steinberg groups introduced by A.~Allcock in~\cite{A13}.
For a more detailed discussion of the ideas behind our proof of~\cref{thm:horrocks-k2} and its key differences from~\cite{LS20, Tu83}, we refer the reader to~\cref{sec:horrocks}.


\subsection{Acknowledgements}
This work was supported by the Russian Science Foundation grant №19-71-30002.

    \section{Preliminaries}\label{sec:preliminaries}
    In this paper all rings are assumed to be commutative and all commutators are left-normed i.\,e. $[a, b] = a b a^{-1} b^{-1}$.

\subsection{Formalism of triples}\label{subsec:triples}

Let $N$ be a group acting on itself by right conjugation.
Let $M$ be a group with a right action of $N$.
Recall that a group homomorphism $\varphi\colon M \to N$ is called a \textit{precrossed module} if $\varphi$ preserves the action of $N$, i.\,e.
\[\varphi(m^n) = \varphi(m)^n, \text{for all $m \in M$, $n\in N;$} \]
If, in addition, $\varphi$ satisfies the so-called \textit{Peiffer identity}, i.\,e.
\[{m}^{\varphi(m')} = {m'}^{-1} m m', \text{for all $m, m' \in M$,}\]
then $\varphi$ is called a \textit{crossed module}.

Let $\varphi\colon M \to N$ and $\varphi' \colon M' \to N'$ be a pair of precrossed modules.
A \textit{map of precrossed modules} $(f, g)\colon \varphi \to \varphi'$ is, by definition, any pair of homomorphisms
$f \colon M \to M'$, $g \colon N \to N'$ such that ${f(m)}^{g(n)} = f(m^n)$ for all $n \in N$, $m \in M$.

Now suppose that we are given the following cube-like commutative diagram of abstract groups:
\begin{equation} \label{eq:cube} \begin{split} \xymatrix{
    G_{123} \ar[rr]_{f_{23}} \ar[dd]_{f_{12}} \ar[rd]_{f_{13}} &                        & G_{23} \ar@{-->}[dd]^(.3){g_2^3} \ar[rd]^{g^2_3} &           \\
    & G_{13} \ar[rr]^(.3){g^1_3} \ar^(.3){g^3_1}[dd] &                   & G_3 \ar[dd]_{h_3} \\
    G_{12} \ar@{-->}[rr]^(.3){g_2^1} \ar[rd]_{g_1^2}          &                        & G_2 \ar@{-->}[rd]_{h_2}         &           \\
    & G_1 \ar[rr]_{h_1}              &                   & G.} \end{split} \end{equation}
Additionally, suppose that $g_1^3$ is a precrossed module, $h_3$ is a crossed module and, moreover, that $(g_3^1, h_1) \colon g_1^3 \to h_3$ is a map of precrossed modules.

Set $V = G_1 \times G_2 \times G_3$, $W = G_{12} \times G_{13} \times G_{23}$.
We define an operation $\star \colon W \times V \to V$ as follows. For $v = (x, y, z) \in V$ and $(a, b, c) \in W$ we set
\[(a, b, c) \star (x, y, z) = (x \cdot g_1^3(b) \cdot g_1^2(a),\ g_2^1(a)^{-1} \cdot y \cdot g_2^3(c),\ g_3^2(c)^{-1} \cdot g_3^1(b)^{-h_2(y)} \cdot z).\]

\begin{lemma} For every $v \in V$ one has
\begin{equation*}(a', b', c') \star \left( (a, b, c) \star v \right) = (a \cdot a', b \cdot {b'}^{g_1^2(a)^{-1}}, c \cdot c') \star v.\end{equation*}
\end{lemma}
\begin{proof}
    Set $v'=(x', y', z') = (a, b, c) \star v$ and $(x'', y'', z'') = (a', b', c') \star v'$.
    Since $g_1^3$ is a precrossed module we immediately obtain that
    \begin{align*}
        x'' =& x' \cdot g_1^3(b') \cdot g_1^2(a') = x \cdot g_1^3(b) \cdot g_1^2(a) \cdot g_1^3(b') \cdot g_1^2(a') = x \cdot g_1^3(b \cdot b'^{g_1^2(a)^{-1}}) g_1^2(a \cdot a'),\\
        y'' =& g_2^1(a')^{-1} \cdot g_2^1(a)^{-1} \cdot y \cdot g_2^3(c) \cdot g_2^3(c') = g_2^1(a\cdot a')^{-1} \cdot y \cdot g_2^{3}(c\cdot c'). \end{align*}
    Since $(g_3^1, h_1)$ is a map of precrossed modules, for every $a \in G_{12}$, $b \in G_{13}$ one has $g_3^1(b^{g_1^2(a)}) = g_3^1(b)^{h_1 g_1^2(a)} = g_3^1(b)^{h_2g_2^1(a)}$.
    Since $h_3$ satisfies Peiffer identity, for every $c \in G_{23}$ and $z \in G_3$ one has $z ^{h_2 g_2^3(c)} = z^{ h_3 g_3^2(c)} = g_3^2(c)^{-1} \cdot z \cdot g_3^2(c)$.
    Using these identities we obtain that
    \begin{multline*}
        z'' = g_3^2(c')^{-1} \cdot g_3^1(b')^{-h_2(y')} \cdot z' = \\
        = g_3^2(c')^{-1} \cdot g_3^1(b')^{- h_2 \left( g_2^1(a)^{-1} \cdot y \cdot g_2^3(c) \right)} g_3^2(c)^{-1} \cdot g_3^1(b)^{-h_2(y)} \cdot z = \\
        = g_3^2(c \cdot c')^{-1} \cdot g_3^1(b')^{- h_2 \left( g_2^1(a)^{-1} \cdot y \right)} \cdot g_3^1(b)^{-h_2(y)} \cdot z = \\
        = g_3^2(c \cdot c')^{-1} \cdot g_3^1(b \cdot {b'} ^ {g_1^2(a)^{-1}})^{- h_2 \left( y \right)} \cdot z. \qedhere
    \end{multline*}
\end{proof}

The above lemma allows us to define an equivalence relation on the set $V$ in the following fashion.
We declare two elements $v, v' \in V$ congruent (denoted $v \sim v'$) if $v' = w \star v$ for some triple $(a, b, c) \in W$.

\begin{lemma}\label{one-one-z} Assume that the back face $(f_{12}, f_{23}, g_2^1, g_2^3)$ of~\eqref{eq:cube} is a pullback square.
Assume additionally that the triple $(1, 1, 1)$ is congruent to $(1, 1, z)$ for some $z\in G_3$.
Then $z \in g_3^1(\Ker(g_1^3)).$ \end{lemma}
\begin{proof} By the definition of congruence relation there exists $(a, b, c)\in W$ such that
\[ (a, b, c) \star (1, 1, 1) = ( g_1^3(b) \cdot g_1^2(a),\ g_2^1(a)^{-1} \cdot g_2^3(c),\ g_3^2(c)^{-1} \cdot g_3^1(b)^{-1}) = (1,1,z). \]
By the Lemma's assumption there exists $e \in G_{123}$ such that $f_{12}(e) = a$, $f_{23}(e) = c$, hence
\[ 1 = g_1^3(b) \cdot g_1^2(f_{12}(e)) = g_1^3(b \cdot f_{13}(e)),\ z = g_3^2(f_{23}(e))^{-1} \cdot g_3^1(b)^{-1} = g_3^1(b \cdot f_{13}(e))^{-1}. \qedhere\] \end{proof}

Consider the set-theoretic map $h \colon V \to G$ given by $(x, y, z) \mapsto h_1(x) \cdot h_2(y) \cdot h_3(z)$.
It is clear that for any $w \in W$ one has $h(w \star (x, y, z)) = h(x, y, z)$, consequently we obtain a well-defined set-theoretic map $\overline{h} \colon V/W \to G$.

\subsection{Steinberg groups, $\K_2$-groups and symbols}\label{subsec:steinberg-preliminaries}
Let $\Phi$ be a root system of rank $\ell \geq 1$.
We assume that $\Phi$ is embedded into $\mathbb{R}^\ell$ whose scalar product we denote by $(\text{-}, \text{-})$.
We also fix some system of simple roots $\Pi = \{\alpha_1, \ldots, \alpha_\ell\} \subset \Phi$.
For a root $\alpha\in\Phi$ we denote by $m_i(\alpha)$ the $i$-th coefficient in the expansion of $\alpha$ in $\Pi$,
i.\,e. $\alpha = \sum_{i=1}^n m_i(\alpha) \alpha_i$.
We denote by $\Phi^+$ (resp. $\Phi^-$) the system of positive (resp. negative) roots with respect to the basis $\Pi$.

We denote by $\Phi^\vee$ the corresponding dual root system, which, by definition, consists of all coroots $\alpha^\vee = \frac{2}{(\alpha, \alpha)} \alpha$, where $\alpha \in \Phi$.
We denote by $P(\Phi^\vee)$ the integral lattice spanned by the \emph{fundamental coweights $\varpi_i^\vee$}.
Recall that the fundamental coweights $\varpi_i^\vee$ are uniquely determined by the property $(\varpi_i^\vee, \alpha_j) = \delta_{ij}$.
For $\alpha,\beta \in \Phi$ we denote by $\langle \alpha, \beta \rangle$ the integer $(\alpha, \beta^\vee) = \frac{2(\alpha, \beta)}{(\beta, \beta)}$.

Now let $R$ be an arbitrary commutative ring with $1$ and suppose that $\Phi$ is an irreducible root system of rank $\geq 2$.
Recall that to the pair $(\Phi, R)$ can associate an abstract group $\St(\Phi, R)$, called the \textit{Steinberg group} of type $\Phi$ over $R$.
By definition, $\St(\Phi, R)$ is the group presented by generators $x_\alpha(a)$, $a \in R$, $\alpha \in \Phi$ and an explicit list of relations (see e.\,g in.~\cite{Ma69, Re75, St71}).
In this paper we may restrict ourselves to the case when the root system $\Phi$ in question is \textit{simply-laced} (i.\,e. has type $\mathsf{ADE}$),
 in which case the defining relations of $\St(\Phi, R)$ reduce to the following shorter list:
\begin{align}
x_{\alpha}(a)\cdot x_{\alpha}(b)&=x_{\alpha}(a+b), \tag{R1} \label{x-additivity}\\
[x_{\alpha}(a),\,x_{\beta}(b)]  &=x_{\alpha+\beta}(N_{\alpha,\beta} \cdot ab),\text{ for }\alpha+\beta\in\Phi, \tag{R2} \label{R2} \\
[x_{\alpha}(a),\,x_{\beta}(b)]  &=1,\text{ for }\alpha+\beta\not\in\Phi\cup0. \tag{R3} \label{R3}
\end{align}
The coefficients $N_{\alpha,\beta}$ in the above formula are integers equal to $\pm 1$, they coincide with the structure constants of the complex Lie algebra of type $\Phi$.

Throughout this paper we denote by $\Gsc(\Phi, R)$ the group of points of the simply-connected Chevalley--Demazure group of type $\Phi$ over $R$.
We denote by $\Esc(\Phi, R)$ the \textit{elementary subgroup} of $\Gsc(\Phi, R)$, i.\,e. the subgroup generated by elementary root unipotents of $\Gsc(\Phi, R)$.
Notice that in~\cite{VP, Vav09} the notation $x_\alpha(a)$ is used to denote the elementary root unipotents.
To prevent confusion we will use different notation $t_\alpha(a)$ for them and reserve the notation $x_\alpha(a)$ solely for generators of Steinberg groups.

Recall that the map sending $x_\alpha(a)$ to $t_\alpha(a)$ gives rise to a well-defined homomorphism $\pi \colon \St(\Phi, R) \to \G_\mathrm{sc}(\Phi, R)$, see~\cite[\S~1A]{St78}.
The cokernel and the kernel of $\pi$ are called \textit{the unstable $\K_1$- and $\K_2$-functors modeled on the root system $\Phi$}:
\begin{equation} \label{eq:K1-K2-sequence}
  \xymatrix{ 1 \ar[r] & \K_2(\Phi, R) \ar[r] & \St(\Phi, R) \ar[r]^{\pi} & \Gsc(\Phi, R) \ar[r] & \K_1(\Phi, R) \ar[r] & 1}
\end{equation}

Following~\cite{Ma69} for $\alpha\in\Phi$ and $u \in R^\times$ we define the following elements of $\St(\Phi, R)$:
\begin{align*} w_\alpha(u) & =  x_\alpha(u) \cdot x_{-\alpha}(-u^{-1}) \cdot x_\alpha(u), \\
               h_\alpha(u) & =  w_\alpha(u) \cdot w_\alpha(-1).  \end{align*}
The subgroup generated by $w_\alpha(u)$ (resp. $h_\alpha(u)$) for all $\alpha\in \Phi$, $u \in R^\times$ is denoted by $\StW(\Phi, R)$ (resp. $\StH(\Phi, R)$).
By~\cite[Lemme~5.2]{Ma69} $\StH(\Phi, R)$ is a normal subgroup of $\StW(\Phi, R)$.

In the sequel we will need the following explicit elements of the group $\K_2(\Phi, R)$.
Recall that for arbitrary $u, v \in R^\times$ one defines the \textit{Steinberg symbol} via the formula
\begin{equation} \label{eq:steinberg} \{ u, v \}_\alpha = h_\alpha(uv) \cdot h_\alpha^{-1}(u) \cdot h_\alpha^{-1}(v). \end{equation}
Recall also from~\cite[Lemme~5.4]{Ma69} that
\begin{equation} \label{eq:steinberg-2} [h_\alpha(u), h_\beta(v)] = \{u, v^{\langle \alpha, \beta \rangle}\}_\alpha. \end{equation}
Steinberg symbols depend only on the length of the root $\alpha$.
In particular, in the case when $\Phi$ is of simply-laced type, they do not depend on the choice of $\alpha$, which allows us to omit it from notation.

Steinberg symbols are central elements of $\St(\Phi, R)$.
Our assumptions on $\Phi$ guarantee that Steinberg symbols are antisymmetric and bimultiplicative, i.\,e. they satisfy the following identities:
\begin{equation} \label{eq:symbol-properties} \{ u, st \} = \{ u, s\} \{ u, t \}, \ \{ u, v \} = \{ v, u\}^{-1}. \end{equation}

In this paper we will use also use the concept of a \textit{relative Steinberg group} introduced by F.~Keune and J.-L.~Loday in~\cite{Ke78, Lo78}.
We will only briefly mention the definition and basic properties of these groups and refer the reader to~\cite[\S~2.3]{LS20} for a more detailed exposition.

Let $R$ be a commutative ring, $I \trianglelefteq R$ be an ideal and let $p$ denote the canonical projection $R \to R/I$.
Denote by $D_{R, I}$ the pullback of two copies of $p$ i.\,e. the ring $R \times_{R/I} R$.
Elements of $D_{R, I}$ are pairs $(a; b)$ such that $a-b \in I$.
We also denote by $p_1$, $p_2$ the canonical projections $D_{R, I} \to R$ and by $p_1^*$, $p_2^*$ the corresponding homomorphisms of Steinberg groups induced by them.
Recall from~\cite[Definition~2.5]{LS20} that the relative Steinberg group $\St(\Phi, R, I)$ is defined as the quotient
 $\Ker(p_1^*) / C$, where $C = [\Ker(p_1^*), \Ker(p_2^*)]$.
If we denote by $\mu$ the homomorphism $\St(\Phi, R, I) \to \St(\Phi, R)$ induced by $p_2^*$, we obtain an exact sequence
\begin{equation}
    \xymatrix{1 \ar[r] & C(\Phi, R, I) \ar[r] & \St(\Phi, R, I) \ar[r]^\mu & \St(\Phi, R) \ar[r]^-{p^*} & \St(\Phi, R/I) \ar[r] & 1. }\label{eq:relative-Steinberg}
\end{equation}
Alternatively, the group $\St(\Phi, R, I)$ can be defined via generators and relations as an $\St(\Phi, R)$-group, cf.~\cite[Proposition~6]{S15}
 or even as an abstract group, see~\cite{V22}.
The relative group $\K_2(\Phi, R, I)$ is defined as the kernel of the homomorphism $\pi \mu$.

We will need a relative analogue of Steinberg symbol.
Let $A$ be a local unital ring with maximal ideal $M$ embedded as a subring into a larger unital ring $R$.
Under this assumption the subset $1+M \subseteq A$ forms a group under multiplication.
It is clear that $(1+M)^\times$ is isomorphic to the abelian group $(M, \circ)$ with the operation given by $m \circ m' = m + m' + mm'$.
Now for $a \in R^\times$ and $m \in M$ we denote by $\{a, 1+m\}_r$ the coset $\{(a; a), (1; 1+m)\}C \in \St(\Phi, R, RM)$.
It is clear that the map $1+m \mapsto \{a, 1+m\}_r$ specifies a group homomorphism
\begin{equation} \label{eq:relative-symbol} \{ a, -\}_r \colon (1+M)^\times \to \K_2(\Phi, R, RM). \end{equation}
It is also clear that $\mu(\{a, 1+m\}_r) = \{a, 1+m\}$.

\begin{lemma}\label{lem:symbols}
Assume that $A$ is a local domain with maximal ideal $M$.
Denote by $R$ the ring $A[X, X\inv]$.
Then the intersection of the image of the relative symbol map $\{X, -\}_r$ with $C(\Phi, R, M[X, X\inv])$ is trivial.
\end{lemma}
\begin{proof}
Set $F = \mathrm{Frac}(A)$.
Consider the following diagram:
\[\begin{tikzcd}
 (1+M)^\times \ar[hookrightarrow, rr] \ar[d] &  & F^\times \ar[hookrightarrow, d] \\
  \K_2(\Phi ,R, M[X^{\pm 1}]) \ar[r] & \K_2(\Phi, R) \ar[r] & \K_2(\Phi, F[X^{\pm 1}]).
\end{tikzcd}\]
Since the right vertical arrow is injective by~\cite[Lemma~2.2]{LS20}, so is the left arrow.
\end{proof}

\subsection{Weight automorphisms}\label{subsec:weight-automorphisms}
Recall that for every coweight $\omega \in P(\Phi^\vee)$ and $\beta \in \ZZ \Phi$ the scalar product $(\omega, \beta)$ is an integer.
Thus, a choice of $u \in R^\times$ and $\omega \in P(\Phi^\vee)$ specifies a permutation of the generating set for $\St(\Phi, R)$ via the following mapping:
\begin{equation*} x_\alpha(a) \mapsto x_\alpha(u^{(\omega, \alpha)} \cdot a),\ \alpha\in \Phi,\ a \in R. \end{equation*}
It is not hard to check that this action is compatible with relations~\eqref{x-additivity}--\eqref{R3} and hence specifies a well-defined automorphism of $\St(\Phi, R)$, which we denote by $\chi_{\omega, u}$.

In the following lemma we check that an analogue of $\chi_{\omega, u}$ can also be defined for relative Steinberg groups.
\begin{lemma} \label{lem:relative-chi}
Let $R$ be a commutative ring, $I$ be its ideal and let $u \in R^\times$.
For every coweight $\omega \in P(\Phi^\vee)$ there exists a well-defined automorphism $\widetilde{\chi}_{\omega, u}$ of the relative Steinberg group $\St(\Phi, R, I)$ making the following diagram commute:
\[\begin{tikzcd} \St(\Phi, R, I) \ar[r, "\widetilde{\chi}_{\omega, u}"] \ar[d] & \St(\Phi, R, I) \ar[d] \\
                 \St(\Phi, R) \ar[r, "\chi_{\omega, X}"] & \St(\Phi, R]). \end{tikzcd}\]
\end{lemma}
\begin{proof}
    Observe that the automorphism $\chi_{\omega, (u; u)}$ of $\St(\Phi, D_{R, I})$ preserves subgroups
     $\Ker(p_i^*)$, $i=1, 2$ and hence their commutator subgroup $C$.
    The required automorphism $\widetilde{\chi}_{\omega, u}$ now can be obtained by restricting $\chi_{\omega, (u; u)}$ to $\Ker(p_1^*)$.
    The commutativity of the diagram is obvious.
\end{proof}

In the sequel we will use the following formulae describing the action of $\chi_{\omega, u}$ on the elements $w_\alpha(u)$, $h_\alpha(u)$:
\begin{align}
    \label{eq:chi-w} \chi_{\omega, u}\left(w_\alpha(v)\right) &= w_\alpha(u^{(\omega, \alpha)} \cdot v), \\
    \label{eq:chi-h} \chi_{\omega, u} (h_\alpha(v)) &= h_\alpha(u^{(\omega, \alpha)} \cdot v) \cdot h_\alpha(u^{(\omega, \alpha)})^{-1} = \{u^{(\omega, \alpha)}, v\} \cdot h_\alpha(v).
\end{align}

The following lemma is analogous to~\cite[Lemma~3.1(c)]{Tu83}.
\begin{lemma} \label{lem:winv-chiw}
For any $w \in \StW(\Phi, A[X^{\pm 1}])$ the element $w^{-1} \cdot \chi_{\omega, X}(w)$ belongs to $\StH(\Phi, A[X^{\pm 1}])$.
\end{lemma}
\begin{proof}
Since $\StH(\Phi, A[X^{\pm 1}])$ is a normal subgroup of $\StW(\Phi, A[X^{\pm 1}])$, it suffices to verify the assertion for $w = w_\alpha(u)$.
Set $h = w^{-1} \cdot \chi_{\omega, X}(w)$.
Notice that
\begin{multline*} w_\alpha(1) \cdot h\cdot  w_\alpha(-1) = w_\alpha(-1)^{-1} \cdot w_\alpha(u)^{-1} \cdot w_{\alpha}(X^{(\omega, \alpha)} u) \cdot w_\alpha(-1) = \\
= h_\alpha(u)^{-1} \cdot h_\alpha(X^{(\omega, \alpha)}u) \in \StH(\Phi, A[X^{\pm 1}]).\end{multline*}
Thus, we get from\cite[Lemme~5.2(b,g)]{Ma69} and~\eqref{eq:steinberg} that \[h = h_{\alpha}^{-1}(u^{-1}) \cdot h_{\alpha}(X^{-(\omega, \alpha)}u^{-1}) = h_{\alpha}^{-1}(X^{(\omega, \alpha)}) \cdot \{ X^{(\omega, \alpha)}, u^{-1} \}. \qedhere\]
\end{proof}

\begin{example} \label{exm:chi-linear}
Set $R = A[X^{\pm 1}]$.
Consider the following coweights:
\[\varepsilon_1 = \varpi_1^\vee,\ \varepsilon_2 = \varpi_2^\vee - \varpi_1^\vee,\ \ldots,\ \varepsilon_{\ell+1} = -\varpi^\vee_\ell.\]
For $1\leq k\leq \ell+1$ and $u \in R^\times$ denote by $d_k(u)$ the matrix from $\GL(\ell+1, R)$ which differs from the unit matrix only in that it has the element $u$ on the $k$-th place of its diagonal.
Recall from~\cite[Corollary~4]{Ka77} that for any $g \in \GL(\ell+1, R)$ there exists an automorphism $\beta_g$ of $\St(\ell+1, R)$ ''modeling`` the automorphism $\alpha_g \colon \GL(\ell+1, R) \to \GL(\ell+1, R)$ of inner conjugation by $g$, i.\,e. such that $\phi \beta_g = \alpha_g \phi$.


It is clear that in the linear case the map $\chi_{\varepsilon_k, u}$ coincides with $\beta_{d_k(u)}$,
 while for other Chevalley groups the maps $\chi_{\omega, u}$ model automorphisms of inner conjugation by weight elements $h_\omega(u)$ in the sense of~\cite[\S~4]{Vav09}.
\end{example}


Let $\omega \in P(\Phi^\vee)$ be a coweight of $\Phi$.
Consider the subset $\mathcal{X}_\omega$ of the generating set of $\St(\Phi, A[X])$ consisting of those generators $x_{\alpha}(a) \in \St(\Phi, A[X])$ for which
$(\alpha, \omega) < 0$ implies that $a \in A[X]$ is divisible by $X^{-(\alpha, \omega)}$.
\begin{rem}
Notice that according to this condition if $(\alpha, \omega) \geq 0$ then $\mathcal{X}$ contains $x_\alpha(a)$ for all $a \in A[X]$.
\end{rem}
Denote by $N_\omega$ the subgroup of $\St(\Phi, A[X])$ generated by $\mathcal{X}_\omega$.


\begin{dfn} \label{dfn:delta-pair}
Let $\omega \in P(\Phi^\vee)$ be a coweight.
By definition, an {\it $\omega$-pair} is a pair of mutually inverse group homomorphisms
$\xymatrix{ \sigma(\omega)\colon N_\omega \ar[r] & \ar@<-1.0ex>[l] N_{-\omega}\colon \sigma(-\omega) }$ satisfying the following identity:
\begin{equation} \label{eq:sigmadef}
\sigma(\pm \omega)(x_\alpha(\xi)) = x_\alpha(X^{(\pm \omega, \alpha)}\cdot \xi),
 \text{ for all } x_\alpha(\xi) \in \mathcal{X}_{\pm\omega}.
\end{equation}\end{dfn}
It is clear that the maps $\delta(\omega)$, $\delta(-\omega)$ are uniquely determined by~\eqref{eq:sigmadef}, so at most one $\omega$-pair may exist for any given $\omega$.
Moreover, $\delta(\omega), \delta(-\omega)$ always make the following diagram commute:
\begin{equation} \label{eq:sigma-diagram}
\xymatrix{ N_\omega \ar[r]_{\sigma(\omega)}\ar@{^{(}->}[d] \ar@/^1.5pc/[rr]^{\mathrm{id}} & N_{-\omega} \ar@{^{(}->}[d] \ar[r]_{\sigma(-\omega)} & N_\omega \ar@{^{(}->}[d] \\
          \St(\Phi, A[X]) \ar[d] & \St(\Phi, A[X]) \ar[d] & \St(\Phi, A[X]) \ar[d] \\
          \St(\Phi, A[X^{\pm 1}]) \ar@<-0.0ex>[r]_{\chi_{\omega, X}} \ar@/_1.5pc/[rr]^{\mathrm{id}} & \St(\Phi, A[X^{\pm 1}]) \ar@<-0.0ex>[r]_{\chi_{-\omega, X}} & \St(\Phi, A[X^{\pm 1}]).} \end{equation}
The question of existence of an $\omega$-pair is rather complicated and will be addressed in the sequel.
One of the necessary technical ingredients needed for this is the technique of another presentation, which we recall in detail in the following subsection.
\subsection{Another presentation of $\St(\rA_3, R)$.} \label{subsec:another-presentation}
Throughout this subsection $R$ denotes a commutative ring and $n \geq 4$.

For $u \in R^n$ we denote by $D(u)$ the subset of $R^n$ consisting of all vectors $v$ which are orthogonal to $u$
(i.\,e. $u^{t} v = 0$) and have at least two zero entries.
Recall from 3.2 of~\cite{Ka77} that for every $u, v, w \in R^n$ such that $u^t v = 0$ there
is a decomposition of $(w^t u) \cdot v$ into a sum of elements of $D(u)$, called \textit{canonical decomposition}:
\begin{equation}
    \label{eq:canonical} (w^tu) \cdot v=\sum_{i<j}u_{ij} c_{ij}(v, w),
\end{equation}
where $u_{ij} =e_i u_j-e_j u_i \in D(u)$ and $c_{ij}(v, w) =v_i w_j-v_j w_i \in R$.

\begin{lemma}
    \label{lem:xsmall-properties}
    Let $v, w \in R^n$ be such that $v^t w = 0$ and assume, moreover, that either $v$ or $w$ has at least one zero entry.
    Under these assumptions one can define certain element $x(v, w) \in \St(n, R)$ such that $\phi(x(v, w)) = t(v, w) = 1 + vw^t$.
    The elements $x(v, w)$ enjoy the following properties:
    \begin{lemlist}
        \item \label{itm:xsmall-scalar} If $v$ or $w$ has at least two zero entries, then $x(v, wa) = x(va, w)$ for $a\in R$.
        \item \label{itm:xsmall-additivity} If $w_1$ and $w_2$ have at least two zero entries of which at least one entry is common
        then $x(v, w_1) \cdot x(v, w_2) = x(v, w_1+w_2)$ and $x(w_1, v) \cdot x(w_2, v) = x(w_1 + w_2, v)$.
        \item \label{itm:xsmall-commute} If $v$, $v'$ are simultaneously orthogonal to $w$ and $w'$, and the elements $w, w'$ both have at least two zero entries then
        $[x(v, w),\ x(v', w')] = 1$.
        \item \label{itm:xsmall-conj} If $g = x_{ij}(\xi)$ is a Steinberg generator and $v$ or $w$ has at least two zero entries then
        $g \cdot x(v, w) \cdot g^{-1} = x(gv, g^*w)$.
    \end{lemlist}
\end{lemma}
\begin{proof}
    See~\cite[Lemma~1.1]{Tu83}.
\end{proof}

Now assume that $I$ is an ideal of a ring $R$ which itself is a subring of a ring $S$.
We now define two families of elements of $\St(n, R)$.
\begin{dfn}
    Let $u \in \Um(n, R)$ and $v \in I^n$ be a vector such that $u^{t}v = 0$.
    Let $v = \sum_r v_r$ be some decomposition for $v$ into sum of elements $v_r \in D(u) \cap I^n$
    (under the assumptions on $u$ and $v$ such decomposition always exists, see~\eqref{eq:canonical}).

    Let $d$ be an element of the subgroup $T(n, S)$ of diagonal matrices of $\GL(n, S)$ such that
    $d^{-1}u \in R^n,\ d \cdot I^n \subseteq R^n.$
    Under these assumptions we set
    \begin{equation*}
        X^d(u, v) \coloneqq \prod_i x(d^{-1}u, dv_i),
    \end{equation*}

    Not let $d'$ be an element of $T(n, S)$ such that $d' u\in R^n,\ {d'}^{-1} \cdot I^n \subseteq R^n$.
    Under these assumptions we set
    \begin{equation*}
        Y^{d'}(v, u) \coloneqq \prod_i x({d'}^{-1} v_i, {d'}u).
    \end{equation*}
\end{dfn}

\begin{lemma}
    \label{lem:xy-wd}
    The elements $X^d(u, v)$, $Y^{d'}(u, v)$ are well-defined, i.\,e. they do not depend on the choice of decomposition for $v$.
\end{lemma}
\begin{proof}
    Let $v = \sum_r v^r$ be a decomposition as above.
    Since each $v^r$ is orthogonal to $u$ we can write the canonical decomposition
    $v^r = \sum_{i<j} u_{ij} c_{ij}(v^r, w)$, moreover $\sum_{r} c_{ij}(v^r, w) = c_{ij}(v, w)$.
    Now using~\cref{lem:xsmall-properties} we obtain:
    \begin{multline*}
        \prod\limits_r x(d^{-1}u, dv^r) = \prod\limits_{r}\prod\limits_{i<j} x(d^{-1} u, du_{ij}c_{ij}(v^r, w)) =
        \prod\limits_{i<j} x(d^{-1} u, d u_{ij}c_{ij}(v, w)). \qedhere
    \end{multline*}
\end{proof}

\begin{lemma}
    \label{lem:xy-conj} Suppose that $g = x_{hk}(\xi)$ is a generator of $\St(n, R)$ such that $m = d\phi(g)d^{-1} \in \E(n, R)$, then
    \begin{equation*}
        g \cdot X_d(u, v) \cdot g^{-1} = X_d(mu, m^*v) \text{ and } g \cdot Y_d(v, u) \cdot g^{-1} = Y_d(mv, m^*u).
    \end{equation*}
\end{lemma}
\begin{proof}
    Direct computation using~\cref{lem:xsmall-properties} (cf. with~\cite[3.14]{Ka77} or~\cite[Lemma~4.4d]{LS17}).
\end{proof}

We will need the following explicit presentation of the relative linear Steinberg group.
\begin{prop}[\text{\cite[Proposition 3.10]{LS17}}]
    \label{prop:rel-presentation}
    Assume that $I$ is a splitting ideal of a commutative ring $R$.
    Then for any $\ell\geq 3$ the group $\St(\rA_\ell,\,R,\,I)$ can be presented by means of two families of generators $F(u,\,v)$, $S(v,\,u)$
    (where $u\in \E(n,\,R)e_1,$ and $v\in I^n$ are such that $u^{t}v=0$) subject to the following relations:
    \begin{align}
        &F(u,\,v)F(u,\,w)=F(u,\,v+w), \label{add4}\\
        &S(u,\,v)S(w,\,v)=S(u+w,\,v), \label{add5}\\
        &F(u,\,v)F(u',\,v')F(u,\,v)^{-1}=F(t(u,\,v)u',\,t(v,\,u)^{-1} v'), \label{conj3} \\
        &F(me_1,\,m^{*}e_{2}a)=S(me_{1}a,\,m^{*}e_{2}),\ \text{for all $a\in I$}\, m \in \E(n, R). \label{coef-move}
    \end{align}
\end{prop}


    \section{Curtis--Tits type presentations} \label{sec:affine}
    Throughout this section $A$ denotes an arbitrary commutative ring.

\subsection{Curtis--Tits type presentation of affine Steinberg groups} \label{subsec:curtis-tits}
In this subsection we briefly recall generalization of Steinberg groups to the Kac--Moody setting.
Only in this subsection we allow root systems to be infinite.

Recall that to any generalized Cartan matrix one can associate (possibly infinite) root system $\Phi$ and the Steinberg group functor $\St(\Phi, -)$.
Two definitions of this functor have been proposed: the definition of J.~Tits~\cite[\S~3.6]{Ti87} and the definition of J.~Morita and U.~Rehmann~\cite[\S~2]{MR90}.
These definitions agree if the Dynkin diagram of the GCM does not contain connected components of type $\rA_1$, see~\cite[\S~6]{A13}.

For our purposes it will be sufficient to restrict attention to the case of a GCM of \textit{affine} type, or, in other words, GCM whose Dynkin diagram is the \textit{extended} Dynkin diagram of a finite irreducible simply-laced root system $\Phi$ of rank $>1$.
Such digrams for $\Phi$ of type $\rD_\ell$, $\rE_6$ and $\rE_7$ are depicted on Figure 1 (the added root is marked with yellow color and index $0$).

\tikzset{
    root/.style={circle, draw, minimum size=0.2cm, inner sep=0},
    zeroroot/.style={circle, draw, minimum size=0.2cm, inner sep=0, fill=yellow},
    highlighted/.style={circle, draw, minimum size=0.2cm, inner sep=0, fill=green},
    levi/.style={draw, dashed, rounded corners},
    dottededge/.style={dotted},
    labeled/.style={below}
}
\begin{figure}[hb]\label{fig:dynkin-diagrams}
\begin{longtable}{ c c c }
    \scalebox{0.69}{\begin{tikzpicture}
                        \node[root, highlighted, label=$1$] (d1) at (0,0.5) {};
                        \node[root, label=$2$] (d2) at (1,0) {};
                        \node[root, label=$3$] (d3) at (2,0) {};
                        \node at (3,0) {\ldots};
                        \node[root, label={[xshift=-0.5cm, yshift=-0.6cm]$\ell-2$}] (dl2) at (4,0) {};
                        \node[root, label={[xshift=-0.6cm, yshift=-0.35cm]$\ell-1$}] (dl1) at (5,0.5) {};
                        \node[root, label={[xshift=-0.5cm, yshift=-0.3cm]$\ell$}] (dl) at (5,-0.5) {};
                        \node[root, zeroroot, label=below:{$0$}] (d0) at (0,-0.5) {};

                        \draw (d1) -- (d2) -- (d3) -- (2.5,0);
                        \draw (3.5,0) -- (dl2) -- (dl1);
                        \draw (dl2) -- (dl);
                        \draw[dottededge] (d2) -- (d0);

                        \begin{scope}
                            \node[levi, fit=(d2) (d3) (dl) (dl1) (dl2), label=above:{$\Delta$}] {};
                        \end{scope}
    \end{tikzpicture}}
    &
    \scalebox{0.69}{\begin{tikzpicture}
                        \begin{scope}
                            \node[root, highlighted, label=$1$] (e1) at (0,0) {};
                            \node[root, label=$3$] (e3) at (1,0) {};
                            \node[root, label={[xshift=-0.3cm]$4$}] (e4) at (2,0) {};
                            \node[root, label=$5$] (e5) at (3,0) {};
                            \node[root, label=$6$] (e6) at (4,0) {};
                            \node[root, label={[xshift=-0.3cm,yshift=-0.3cm]$2$}] (e2) at (2,1) {};
                            \node[root, zeroroot, label=left:$0$] (e0) at (2,2) {};
                            \draw (e1) -- (e3) -- (e4) -- (e5) -- (e6);
                            \draw (e2) -- (e4);
                            \draw[dottededge] (e2) -- (e0);

                            \begin{scope}
                                \node[levi, fit= (e2) (e3) (e4) (e5) (e6), label=above:{$\Delta$}] {};
                            \end{scope}
                        \end{scope}
    \end{tikzpicture}}
    &
    \scalebox{0.69}{\begin{tikzpicture}
                        \begin{scope}
                            \node[root, zeroroot, label={0}] (e0) at (0,0) {};
                            \node[root, label={$1$}] (e1) at (1,0) {};
                            \node[root, label={$3$}] (e3) at (2,0) {};
                            \node[root, label={[xshift=-0.3cm]$4$}] (e4) at (3,0) {};
                            \node[root, label={$5$}] (e5) at (4,0) {};
                            \node[root, label={$6$}] (e6) at (5,0) {};
                            \node[root, label={[xshift=-0.3cm,yshift=-0.3cm]$2$}] (e2) at (3,1) {};
                            \node[root, highlighted, label={$7$}] (e7) at (6,0) {};

                            \draw (e1) -- (e3) -- (e4) -- (e5) -- (e6) -- (e7);
                            \draw (e2) -- (e4);
                            \draw[dottededge] (e0) -- (e1);

                            \begin{scope}
                                \node[levi, fit=(e1) (e2) (e3) (e4) (e5) (e6), label=above:{$\Delta$}] {};
                            \end{scope}
                        \end{scope}
    \end{tikzpicture}} \\
    \text{$\rD_\ell$} &
    \text{$\rE_6$} &
    \text{$\rE_7$}
\end{longtable}
\caption{Root markings on extended Dynkin diagrams}
\end{figure}

Denote by $\widetilde{\Phi}$ the affine root system corresponding to $\Phi$.
Recall from~\cite[\S~4]{A16} that the set of real roots of $\widetilde{\Phi}$ is isomorphic to $\Phi \times \ZZ$.
\begin{lemma} \label{lem:affine-vs-loop} $\St(\widetilde{\Phi}, A) \cong \St(\Phi, A[X, X\inv])$.
\end{lemma}
\begin{proof}
    By definition, the group $\mathrm{St}(\widetilde{\Phi}, A)$ is presented by generators $x_{(\alpha, m)}(a)$, $a \in A$, $\alpha \in \Phi$ and
    the following relations:
    \begin{align}
        x_{(\alpha, m)}(a)\cdot x_{(\alpha, m)}(b)&=x_{(\alpha, m)}(a+b),  \label{AR1}\\
        [x_{(\alpha, m)}(a),\,x_{(\beta, n)}(b)]  &=x_{(\alpha+\beta, n+m)}(N_{\alpha,\beta} \cdot ab),\text{ for }\alpha+\beta\in\Phi, \label{AR2} \\
        [x_{(\alpha, m)}(a),\,x_{(\beta, n)}(b)]  &=1,\text{ for }\alpha+\beta\not\in\Phi\cup0. \label{AR3}
    \end{align}
    It is not hard to check that the homomorphism given by $x_{(\alpha, m)}(a) \mapsto x_\alpha(aX^m)$ is an isomorphism with the inverse given by
    $x_\alpha(a_{n}X^n + \ldots + a_m X^m) \mapsto \Pi\limits_{i=n}^m x_{(\alpha, i)}(a_i)$, $a_i \in A$, $n \leq m$, $n, m\in \ZZ$.
\end{proof}

Our next goal is to formulate the so-called \textit{Curtis--Tits presentation} of affine Steinberg groups discovered by D.~Allcock in~\cite{A16, A13}.
This presentation has the advantage of being formulated in terms of the Dynkin diagram of $\Phi$ rather than the (possibly infinite) set of real roots of $\Phi$.
Its other advantage is that it requires no choice of structure constants in its statement.
Allcock's result is the generalization of Curtis--Tits presentations of finite root system, which has been known since 1960's, see e.\,g.~\cite[Theorem~B]{DS74}.

Recall that $\{ \alpha_1, \ldots \alpha_\ell \}$ is the set of simple roots of $\Phi$.
We denote by $\alpha_0$ the opposite root to the maximal root of $\Phi$ (i.\,e. $\alpha_0 := -\alpha_\mathrm{max}$).
This root corresponds to the added $0$ node of the extended Dynkin diagram depicted on Figure 1.
We denote by $X_0(A)$ the root group $\{ X_0(a) \mid a \in A\}$ (as a group it is isomorphic to the additive group of $A$).
We also denote by $j$ the root adjacent to $0$ on the extended Dynkin diagram of $\Phi$.
\begin{prop} \label{prop:Allcock-affine} The group $\St(\Phi, A[X, X\inv])$ is isomorphic to the free product of $\St(\Phi, A)$, the group $X_0(A)$ and the free cyclic group $\langle S_0 \rangle$
     amalgamated over the subgroup generated by the following list of relations:
    {\allowdisplaybreaks\begin{align}
        [S_0^2, X_0(a)] & = 1 & \text{ for $a \in A$; } \label{eq:Allcock-2} \\
        X_0(1) \cdot {}^{S_0} X_0(1) \cdot X_0(1) & = S_0; \label{eq:Allcock-3} \\
        [S_0, w_{\alpha_i}(1)] & = 1; &  \label{eq:Allcock-4} \\
        [S_0, x_{\alpha_i}(a)] & = 1, &  \label{eq:Allcock-5-1}\\
        [w_{\alpha_i}(1), X_0(a)] & = 1 & \text{$i$ unjoined with $0$, $a \in A;$} \label{eq:Allcock-5-2} \\
        [X_0(a), x_{\alpha_i}(b)] & = 1 & \text{$i$ unjoined with $0$, $a, b \in A;$} \label{eq:Allcock-6} \\
        S_0 \cdot w_{\alpha_j}(1) \cdot S_0 & = w_{\alpha_j}(1) \cdot S_0 \cdot w_{\alpha_j}(1); \label{eq:Allcock-7} \\
        {}^{S_0^2} w_{\alpha_j}(1) & = w_{\alpha_j}(-1); \label{eq:Allcock-8-1} \\
        {}^{w_{\alpha_j}^2(1)} S_0 & = S_0^{-1}; \label{eq:Allcock-8-2} \\
        x_{\alpha_j}(a) \cdot S_0 \cdot w_{\alpha_j}(1) & = S_0 \cdot w_{\alpha_j}(1) \cdot X_0(a), & \label{eq:Allcock-9-1} \\
        X_0(a) \cdot w_{\alpha_j}(1) \cdot S_0 & = w_{\alpha_j}(1) \cdot S_0 \cdot x_{\alpha_j}(a), & \label{eq:Allcock-9-2} \\
        {}^{S_0^2} x_{\alpha_j}(a) & = x_{\alpha_j}(-a), & \label{eq:Allcock-10-1} \\
        {}^{w_{\alpha_j}^2(1)} X_0(a) & = X_0(-a) & \text{for $a \in A;$} \label{eq:Allcock-10-2} \\
        [X_0(a), {}^{S_0} x_{\alpha_j}(b)] &= 1, & \label{eq:Allcock-11-1} \\
        [x_{\alpha_j}(a), {}^{w_{\alpha_j}(1)}X_0(b)] &= 1, & \label{eq:Allcock-11-2} \\
        [X_0(a), x_{\alpha_j}(b)] &= {}^{S_0} x_{\alpha_j}(ab) & \text{for $a, b \in A.$} \label{eq:Allcock-12}
    \end{align}}
\end{prop}
\begin{proof}
    This is a direct consequence of our~\cref{lem:affine-vs-loop} and the presentation of~\cite[Theorem~1]{A16} (or, alternatively, \cite[Theorem~1.1]{A13} combined with~\cite[Theorem~1.3]{A13}).
    The list of relations in the statement is obtained from \cite[Table~1]{A16} by substituting concrete values of $i, j$, identifying Allcock's $X_{i}(a)$ with
     $x_{\alpha_i}(a)$ and $S_i$ with $w_{\alpha_i}(a)$ for all $i\neq 0$ and then omitting all the relations are already satisfied in $\St(\Phi, A)$ (relations not involving $X_0(a)$ or $S_0$).
    The only further simplification is that we have omitted the relation $[X_{\alpha_j}(b), X_0(a)] = {}^{w_{\alpha_j}(1)} X_0(ab)$ (the relation symmetric to~\eqref{eq:Allcock-12})
    because it is a consequence of~\eqref{eq:Allcock-9-2} and~\eqref{eq:Allcock-12}.
\end{proof}

Our next goal is to slightly simplify the presentation of~\cref{prop:Allcock-affine} by using the larger group $\St(\Phi, A[X])$ instead of $\St(\Phi, A)$ in the statement.
\begin{cor} \label{cor:Allcock-simpler}
    For a finite simply-laced irreducible root system $\Phi$ and an arbitrary commutative ring $A$ the Steinberg group $\St(\Phi, A[X, X\inv])$ is isomorphic
    to the free product of $\St(\Phi, A[X])$ and the cyclic group $\langle S \rangle$ amalgamated over the subgroup generated by the following list of relations:
    \begin{align}
    [S^2, x_{\alpha_0}(aX)] & = 1 & \text{ for $a \in A$; } \label{eq:simpler-2} \\
    x_{\alpha_0}(X) \cdot {}^{S} x_{\alpha_0}(X) \cdot x_{\alpha_0}(X) & = S; \label{eq:simpler-3} \\
    [S, w_{\alpha_i}(1)] & = 1 & \text{ for $i$ unjoined with $0$;} \label{eq:simpler-4} \\
    [S, x_{\alpha_i}(a)] & = 1 &  \text{ $i$ unj. with $0$, $a \in A$; } \label{eq:simpler-5-1}\\
    S \cdot w_{\alpha_j}(1) \cdot S & = w_{\alpha_j}(1) \cdot S \cdot w_{\alpha_j}(1); \label{eq:simpler-7} \\
    {}^{S^2} w_{\alpha_j}(1) & = w_{\alpha_j}(-1); \label{eq:simpler-8-1} \\
    {}^{w_{\alpha_j}^2(1)} S & = S^{-1}; \label{eq:simpler-8-2} \\
    x_{\alpha_j}(a) \cdot S \cdot w_{\alpha_j}(1) & = S \cdot w_{\alpha_j}(1) \cdot x_{\alpha_0}(aX), & \label{eq:simpler-9-1} \\
    x_{\alpha_0}(aX) \cdot w_{\alpha_j}(1) \cdot S & = w_{\alpha_j}(1) \cdot S \cdot x_{\alpha_j}(a), & \label{eq:simpler-9-2} \\
    {}^{S^2} x_{\alpha_j}(a) & = x_{\alpha_j}(-a) & \text{ for $a \in A$; } \label{eq:simpler-10-1} \\
    [x_{\alpha_0}(aX), {}^{S} x_{\alpha_j}(b)] &= 1, & \label{eq:simpler-11-1} \\
    [x_{\alpha_0}(aX), x_{\alpha_j}(b)] &= {}^{S} x_{\alpha_j}(ab) & \text{for $a, b \in A.$} \label{eq:simpler-12}
    \end{align}
\end{cor}
\begin{proof}
    The list of relations is obtained from~\eqref{eq:Allcock-2}--\eqref{eq:Allcock-12} by identifying $X_0(a)$ with $x_{\alpha_0}(aX)$, $S_0$ with $S$ and omitting those relations
     which are satisfied in the group $\St(\Phi, A[X])$ by virtue of Lemmas 5.1--5.2 of ~\cite{Ma69}.
\end{proof}

\subsection{Relationship between $\St(\Phi, A[X, X\inv])$ and $\St(\Phi, A[X])$.} \label{subsec:short-presentation}
Now let $\Phi$ be a root system of type $\rD_\ell$, $\rE_6$ or $\rE_7$.
Denote by $k$ the vertex of the extended Dynkin diagram marked green on Figure~1.
Recall that in each case $m_k(\alpha_\mathrm{\max}) = 1$, therefore $m_i(\alpha) = 1$ for all $\alpha \in \Sigma_k^+$ and the subgroup $\UU(\Sigma^+_k, R)$ is abelian.

Denote by $G$ the amalgamated product of $\St(\Phi, A[X])$ and the cyclic group $\langle \sigma \rangle$ amalgamated over the subgroup generated by the following relations:
\begin{align}
    {}^\sigma x_{\alpha}(f) = & x_{\alpha} (Xf), & \alpha \in \Sigma^+_k, f \in A[X], \label{eq:sigma-sigma-plus} \\
    x_{\beta}(f)^ \sigma     =& x_{\beta} (Xf), & \beta \in \Sigma^-_k, f \in A[X], \label{eq:sigma-sigma-minus} \\
    [\sigma,\, x_\gamma(f)]   =& 1, & \gamma \in \Delta, f \in A[X]. \label{eq:sigma-delta}
\end{align}
It is clear that the action of the generator $\sigma$ is chosen to mimick the action of the weight element $\chi_{\omega_k, X}$ from~\cref{subsec:weight-automorphisms}.

We denote by $i_+$ the canonical homomorphism $\St(\Phi, A[X]) \to G$ and by $h_+$ the canonical embedding $A[X] \to A[X, X\inv]$.

The main result of this subsection is the following
\begin{prop}
    For $\Phi$ as above there exists an arrow $\varphi$ making the diagram below commute:
    \[\begin{tikzcd}           & \St(\Phi, A[X, X\inv]) \arrow[rd, dashrightarrow, "\varphi"]] & \\
    \St(\Phi, A[X]) \arrow{ru}{h_+^*} \arrow{rr}{i_+} &                                & G.
    \end{tikzcd}\]
\end{prop}
\begin{proof}
 We will use the presentation of~\cref{cor:Allcock-simpler} for $\St(\Phi, A[X, X\inv])$.
 We only need to specify the value of $\varphi$ on the generator $S$.
 Observe that $-\alpha_0 = \alpha_{\max} \in \Sigma_k^+$, therefore the obvious candidate for the role of $\varphi(x_{-\alpha_0}(aX\inv))$ would be $x_{-\alpha_0}(a)^\sigma$, cf.~\eqref{eq:sigma-sigma-plus},
 which motivates the following definition for $\varphi(S)$:
 Set \[\varphi(S) := x_{\alpha_0}(X) \cdot x_{-\alpha_0}(-1)^\sigma \cdot x_{\alpha_0}(X).\]
 From~\eqref{eq:w-definition},\eqref{eq:sigma-sigma-minus} it follows that $\varphi(S) = w_{\alpha_0}(1)^\sigma$.

 We need to check that our definition is correct, namely that $\varphi$ preserves all the relations listed in~\cref{cor:Allcock-simpler}.
 We claim that this follows from the fact that substitution of $w_{\alpha_0}(1)^\sigma$ into $S$ turns every formula from \cref{cor:Allcock-simpler}
     into a $\sigma$-conjugate of a valid relation in $\St(\Phi, A[X])$.

 Let us illustrate this idea with a few examples.
 For example, to see that $\varphi$ preserves~\eqref{eq:simpler-3} use~\eqref{eq:sigma-sigma-minus} combined with~\cite[Lemma~5.1b]{Ma69}:
 \begin{multline*}
     \varphi(x_{\alpha_0}(X) \cdot {}^{S} x_{\alpha_0}(X) \cdot x_{\alpha_0}(X)) = x_{\alpha_0}(1)^\sigma \cdot {}^{w_{\alpha_0}(1)^\sigma} x_{\alpha_0}(1)^\sigma \cdot x_{\alpha_0}(1)^\sigma = \\
     = \left( x_{\alpha_0}(1) \cdot {}^{w_{\alpha_0}(1)} x_{\alpha_0}(1) \cdot x_{\alpha_0}(1)\right)^\sigma = \left(x_{\alpha_0}(1) \cdot x_{-\alpha_0}(-1) \cdot x_{\alpha_0}(1)\right)^\sigma = w_{\alpha_0}(1)^\sigma = \varphi(S).
 \end{multline*}

 Let us show that $\varphi$ preserves~\eqref{eq:simpler-5-1}.
 Set $g = x_{\alpha_i}(a)$ if $i \neq k$ and $g = x_{\alpha_k}(aX)$ if $i = k$.
 From~\eqref{eq:sigma-sigma-plus},\eqref{eq:sigma-delta} and the commutator formula we obtain that
 \begin{equation*}
     \varphi([S, x_{\alpha_i}(a)]) = [w_{\alpha_0}(1)^\sigma, x_{\alpha_i}(a)] = [w_{\alpha_0}(1)^\sigma, g^\sigma] = [w_{\alpha_0}(1), g]^\sigma = 1^\sigma = 1.
 \end{equation*}

 Let us show that $\varphi$ preserves~\eqref{eq:simpler-9-1}.
 First of all, notice that ${}^{w_{\alpha_0}(1) w_{\alpha_j}(1)} x_{\alpha_0}(a) = x_{\alpha_j}(a)$.
 This can be showed directly
 \begin{multline*}
     \varphi(x_{\alpha_j}(a) \cdot S \cdot w_{\alpha_j}(1)) = x_{\alpha_j}(a) \cdot w_{\alpha_0}(1)^\sigma \cdot w_{\alpha_j}(1) = \ldots \\
     \ldots = w_{\alpha_0}(1)^\sigma w_{\alpha_j}(1) \cdot x_{\alpha_0}(aX) = \varphi(S \cdot w_{\alpha_j}(1) \cdot x_{\alpha_0}(aX)).
 \end{multline*}
\end{proof}

    \section{Horrocks theorem for $\K_2$} \label{sec:horrocks}
    \begin{dfn}
    Let $K$ be a functor from commutative rings to groups.
    Recall from~\cite{LSV2} that $K$ is called \textit{locally acyclic (resp., locally acyclic for domains)} if for every commutative local ring (resp., domain)
     $A$ the following diagram whose arrows are induced by natural embeddings is a pullback square:
    \begin{equation}\label{eq:P1-square} \begin{tikzcd} K(A) \ar[r] \ar[d] & K(A[X]) \arrow{d} \\ K(A[X\inv]) \ar{r} & K(A[X, X\inv]). \end{tikzcd} \end{equation}
\end{dfn}

Now we can formulate the main result of this section.
\begin{thm}[Horrocks theorem for $\K_2$]\label{thm:horrocks-k2}
Let $\Phi$ be a root system of type $\rA_{\geq 4}$, $\rD_{\geq 5}$, $\rE_6$ or $\rE_7$.
Then the functor $\K_{2}(\Phi, -)$ is locally acyclic for domains.
\end{thm}
Horrocks theorem for $\K_2$ has been previously known in the linear and even orthogonal case $\Phi=\rA_{\geq 4},\rD_{\geq 7}$, see~\cite[Proposition~4.3]{Tu83} and~\cite[Theorem~1]{LS20}, respectively.
Notice that the orthogonal Horrocks theorem for $\K_2$ \cite[Theorem~1]{LS20} was proved only under the additional assumption $2 \in A^\times$.
Thus, the novelty of~\cref{thm:horrocks-k2} consists not only in the fact that it applies to root systems of types $\rD_5$, $\rD_6$, $\rE_6$, $\rE_7$ but also that
 it is proved without the assumption $2 \in A^\times$.

Since our proof of \cref{thm:horrocks-k2} is rather long and technical we will first try to describe its plan.

Let $A$ be a local ring with maximal ideal $M$ and residue field $k$.
Recall that proofs of the Horrocks theorem for $\K_1$ are usually based on the decomposition of the elementary subgroup over Laurent polynomial ring
 called \textit{Suslin's lemma} or \textit{Suslin's structure theorem}, cf.~\cite{Abe83, Su77}, \cite[\S~VI.6]{Lam10}.
In the linear case this decomposition asserts that the elementary linear group $\E(n, A[X, X\inv]) = \Esc(\rA_{n-1}, A[X, X\inv])$ admits the following group factorization for $n \geq 3$:
\begin{equation}\label{eq:triple-decomposition}
\E(n, A[X^{\pm 1}]) = \E(n, A[X]) \cdot B(A[X^{\pm 1}]) \cdot \E(n, A[X^{\pm 1}], M[X^{\pm 1}]).
\end{equation}
In the above formula $\E(n, R, I)$ denotes the relative elementary subgroup, i.\,e. the kernel of the canonical reduction homomorphism $\E(n, R) \to \E(n, R/I),$
 while $B(R)$ denotes the Borel subgroup (i.\,e. the semidirect product of the unipotent radical $\UU(\Phi^+, R)$ and the group of diagonal matrices).

To prove the Suslin's structure theorem one constructs a collection of diagonal automorphisms $\delta_i$, $1\leq i\leq n$
 and shows that the product $V$ of the three subgroups in the right-hand side of~\eqref{eq:triple-decomposition} is stabilized by these automorphisms.
On the other hand, the product $V$ is clearly stabilized by left translations by elements of $\E(n, A[X])$.
Since the minimal subgroup of $\E(n, A[X, X\inv])$ that contains the image of $\E(n, A[X])$ and at the same time is invariant with respect to the conjugation by all $\delta_i$
  coincides with all of $\E(n, A[X, X\inv])$, the decomposition~\eqref{eq:triple-decomposition} follows.

Tulenbaev's proof of the Horrocks theorem for $\K_2$ is similar to Suslin's proof of the Horrocks theorem for $\K_1$.
The main difference, however, is that simply proving an factorization of the form~\eqref{eq:triple-decomposition} for the Steinberg group $\St(\Phi, A[X, X\inv])$
 would not be sufficient.
Instead, one has to prove a much more precise result, namely one has to construct a ``model set'' $V$ for the group $\St(\Phi, A[X, X\inv])$
 out of the same three ingredient groups: $\St(\Phi, A[X])$, Borel subgroup and the relative group $\St(\Phi, A[X^\pm], M[X^\pm])$.
By a ``model set'' we mean a set upon which $\St(\Phi, A[X, X\inv])$ would act simply-transitively.
This technique probably dates back to~\cite{ST76}, where it was used to prove the injective stability theorem for $\K_2$, it was also reused
 by M.~Tulenbaev in his proof of the Horrocks theorem for $\K_2$ in the linear case, cf.\ Propositions 4.1 and 4.3 in~\cite{Tu83}.
The same technique is also used in the proof of~\cite[Theorem~3]{LS20}, which is a direct generalization of~\cite[Proposition~4.3]{Tu83}.
We will use the same technique for the proof of~\cref{thm:horrocks-k2}, which, by analogy, should be thought of as a generalization of~\cite[Proposition~4.1]{Tu83}.

Now let us list the key steps of the proof:
\begin{itemize}
    \item Step 1. First of all, in~\cref{subsec:triples} we introduce abstract formalism of group decompositions which will allow us to construct the set $V$ modeling the Steinberg group $\St(\Phi, A[X, X\inv])$.
                  This should be thought of as preparatory work needed for the formulation of the $\K_2$-analogue of Suslin's structure theorem~\eqref{eq:triple-decomposition}.
    \item Step 2. In~\cref{subsec:structure-theorem-overview} we use this formalism to construct the "model set" $V$ and, thus, formulate the $\K_2$-analogue of Suslin's structure theorem precisely.
                  We also show how~\cref{thm:horrocks-k2} can be reduced to a concrete technical statement, namely,
                  to the existence of the action of $\St(\Phi, A[X, X\inv])$ on $V$.
\end{itemize}
Obviously, \cref{prop:rel-poly-Laurent} reduces the problem of construction of the action of $\St(\Phi, A[X, X\inv])$ on $V$ to the construction of
 the automorphism $\sigma_k$ of $V$ modeling the action of the weight automorphism $\chi_{\varpi_k, X}$ (cf.~\cref{subsec:weight-automorphisms}).
Here index $k$ has the same meaning as in Figure 1 of~\cref{sec:subsec:curtis-tits}, namely it is the index of the simple root marked green on the extended Dynkin diagram.
The construction of $\sigma_k$ proceeds as follows:
\begin{itemize}
    \item Step 3. First, we have to show the existence of $\varpi_k^\vee$-pair in the sense of~\cref{dfn:delta-pair}.
                  Firstly, we construct $\varpi_1^\vee$-pair for the root system of type $\rA_3$ using the presentation from~\cref{prop:rel-presentation} for this purpose.
                  Next, we use the amalgamation theorem~\cref{thm:relPres} to obtain the general result.
                  This is achieved in~\cref{subsec:construction-sigma}.
    \item Step 4. Finally, we complete the construction of the action of $\St(\Phi, A[X, X\inv])$ on $V$ and verify all the neccessary relations.
                  This is achieved in~\cref{sec:construction-delta}.
\end{itemize}

\subsection{Formalism of triples}\label{subsec:triples}
Let $N$ be a group acting on itself by right conjugation.
Let $M$ be a group with a right action of $N$.
Recall that a group homomorphism $\mu\colon M \to N$ is called a \textit{precrossed module} if $\mu$ preserves the action of $N$, i.\,e.
\[\mu(m^n) = \mu(m)^n, \text{for all $m \in M$, $n\in N;$} \]
If, in addition, $\mu$ satisfies the so-called \textit{Peiffer identity}, i.\,e.
\[{m}^{\mu(m')} = {m'}^{-1} m m', \text{for all $m, m' \in M$,}\]
then $\mu$ is called a \textit{crossed module}.

If $\mu\colon M \to N$ and $\mu' \colon M' \to N'$ is a pair of precrossed modules.
A \textit{map of precrossed modules} $(f, g)\colon \mu \to \mu'$ is a pair of group homomorphisms $f\colon M \to M'$, $g\colon N \to N'$ such that
$\mu'f = g \mu$ and that the action of $N$ is preserved, i.\,e. ${f(m)}^{g(n)} = f(m^n)$ for all $n \in N$, $m \in M$.

Now suppose that we are given the following cube-like commutative diagram of abstract groups:
\begin{equation} \label{eq:cube} \begin{split} \xymatrix{
    G_{123} \ar[rr]_{f_{23}} \ar[dd]_{f_{12}} \ar[rd]_{f_{13}} &                        & G_{23} \ar@{-->}[dd]^(.3){g_2^3} \ar[rd]^{g^2_3} &           \\
    & G_{13} \ar[rr]^(.3){g^1_3} \ar^(.3){g^3_1}[dd] &                   & G_3 \ar[dd]_{h_3} \\
    G_{12} \ar@{-->}[rr]^(.3){g_2^1} \ar[rd]_{g_1^2}          &                        & G_2 \ar@{-->}[rd]_{h_2}         &           \\
    & G_1 \ar[rr]_{h_1}              &                   & G.} \end{split} \end{equation}
Additionally, we make the following assumptions:
\begin{itemize}
    \item $g_1^3$ is a precrossed module;
    \item $h_3$ is a crossed module;
    \item $(g_3^1, h_1) \colon g_1^3 \to h_3$ is a map of precrossed modules.
\end{itemize}
Set $V = G_1 \times G_2 \times G_3$, $W = G_{12} \times G_{13} \times G_{23}$.
We define an operation $\star \colon W \times V \to V$ as follows.
For $v = (x, y, z) \in V$ and $w = (a, b, c) \in W$ we set
\[(a, b, c) \star (x, y, z) = (x \cdot g_1^3(b) \cdot g_1^2(a),\ g_2^1(a)^{-1} \cdot y \cdot g_2^3(c),\ g_3^2(c)^{-1} \cdot g_3^1(b)^{-h_2(y)} \cdot z).\]
Consider the set-theoretic map $h \colon V \to G$ given by $(x, y, z) \mapsto h_1(x) \cdot h_2(y) \cdot h_3(z)$.
It is clear from the definition of $\star$-operation that for $w \in W$ one has $h(w \star v) = h(v).$
Now let us define the relation associated with $\star$-operation.
We declare two elements $v, v' \in V$ congruent (denoted $v \sim_W v'$) if $v' = w \star v$ for some triple $w=(a, b, c) \in W$.
As the following lemma shows, this relation is an equivalence relation.
\begin{lemma} For every $v \in V$ one has
\begin{equation*}(a', b', c') \star \left( (a, b, c) \star v \right) = (a \cdot a', b \cdot {b'}^{g_1^2(a)^{-1}}, c \cdot c') \star v.\end{equation*}
\end{lemma}
\begin{proof}
    Set $v'=(x', y', z') = (a, b, c) \star v$ and $(x'', y'', z'') = (a', b', c') \star v'$.
    Since $g_1^3$ is a precrossed module we immediately obtain that
    \begin{align*}
        x'' =& x' \cdot g_1^3(b') \cdot g_1^2(a') = x \cdot g_1^3(b) \cdot g_1^2(a) \cdot g_1^3(b') \cdot g_1^2(a') = x \cdot g_1^3(b \cdot b'^{g_1^2(a)^{-1}}) g_1^2(a \cdot a'),\\
        y'' =& g_2^1(a')^{-1} \cdot g_2^1(a)^{-1} \cdot y \cdot g_2^3(c) \cdot g_2^3(c') = g_2^1(a\cdot a')^{-1} \cdot y \cdot g_2^{3}(c\cdot c'). \end{align*}
    Since $(g_3^1, h_1)$ is a map of precrossed modules, for every $a \in G_{12}$, $b \in G_{13}$ one has $g_3^1(b^{g_1^2(a)}) = g_3^1(b)^{h_1 g_1^2(a)} = g_3^1(b)^{h_2g_2^1(a)}$.
    Since $h_3$ satisfies Peiffer identity, for every $c \in G_{23}$ and $z \in G_3$ one has $z ^{h_2 g_2^3(c)} = z^{ h_3 g_3^2(c)} = g_3^2(c)^{-1} \cdot z \cdot g_3^2(c)$.
    Using these identities we obtain that
    \begin{multline*}
        z'' = g_3^2(c')^{-1} \cdot g_3^1(b')^{-h_2(y')} \cdot z' = \\
        = g_3^2(c')^{-1} \cdot g_3^1(b')^{- h_2 \left( g_2^1(a)^{-1} \cdot y \cdot g_2^3(c) \right)} g_3^2(c)^{-1} \cdot g_3^1(b)^{-h_2(y)} \cdot z = \\
        = g_3^2(c \cdot c')^{-1} \cdot g_3^1(b')^{- h_2 \left( g_2^1(a)^{-1} \cdot y \right)} \cdot g_3^1(b)^{-h_2(y)} \cdot z = \\
        = g_3^2(c \cdot c')^{-1} \cdot g_3^1(b \cdot {b'} ^ {g_1^2(a)^{-1}})^{- h_2 \left( y \right)} \cdot z. \qedhere
    \end{multline*}
\end{proof}
Since $h$ is constant on the orbits of $\star$-action, $h$ gives rise to a well-defined map $\overline{h} \colon V/\sim_W \to G$.
We use the notation $[a, b, c]$ to denote the equivalence class of the triple $(a, b, c) \in V$.

\begin{lemma}\label{lem:one-one-z} Assume that the back face $(f_{12}, f_{23}, g_2^1, g_2^3)$ of~\eqref{eq:cube} is a pullback square.
Assume additionally that $[1, 1, 1] = [1, 1, z]$ for some $z\in G_3$.
Then $z \in g_3^1(\Ker(g_1^3)).$ \end{lemma}
\begin{proof} By the definition of congruence relation there exists $(a, b, c)\in W$ such that
\[ (a, b, c) \star (1, 1, 1) = ( g_1^3(b) \cdot g_1^2(a),\ g_2^1(a)^{-1} \cdot g_2^3(c),\ g_3^2(c)^{-1} \cdot g_3^1(b)^{-1}) = (1,1,z). \]
By lemma's assumption there exists $e \in G_{123}$ such that $f_{12}(e) = a$, $f_{23}(e) = c$, hence
\[ 1 = g_1^3(b) \cdot g_1^2(f_{12}(e)) = g_1^3(b \cdot f_{13}(e)),\ z = g_3^2(f_{23}(e))^{-1} \cdot g_3^1(b)^{-1} = g_3^1(b \cdot f_{13}(e))^{-1}. \qedhere\] \end{proof}


\subsection{First reductions} \label{subsec:structure-theorem-overview}
%sSignificant progress towards proving~\cref{thm:horrocks-k2} has already been achieved in~\cite{LS20}.
%In this subsection we reduce~\cref{thm:horrocks-k2} to a specific technical statement which will be addressed in the following sections.

The following lemma provides the first key reduction in the proof of~\cref{thm:horrocks-k2}.
\begin{lemma} \label{lem:first-reduction}
Let $A$ be a local ring with maximal ideal $M$ and residue field $k$.
Let $\Phi$ be as in the statement of~\cref{thm:horrocks-k2}.
Suppose that the canonical homomorphism
\begin{equation} \label{eq:c-surj} C(\Phi, A[X], M[X]) \to C(\Phi, A[X, X\inv], M[X, X\inv]) \end{equation}
is surjective.
Then the square~\eqref{eq:P1-square} is pullback and Horrocks theorem for $\K_2$ holds.
\end{lemma}
\begin{proof}
    Denote by $R$ the Laurent polynomial ring $A[X, X\inv]$ and by $B$ the subring $A[X\inv] + M[X] \subseteq R$.
    We also set $I \subseteq M[X, X\inv]$, which is clearly an ideal of both $B$ and $R$.
    Consider the following diagram with rows obtained from~\eqref{eq:relative-Steinberg}:
    \[\begin{tikzcd}
          C(\Phi, B, I) \arrow{r} \arrow[d, swap, twoheadrightarrow] & \St(\Phi, B, I) \arrow{r}{\mu_B} \arrow{d} & \St(\Phi, B) \arrow{r} \arrow{d} & \St(\Phi, k[X^{-1}]) \arrow[hookrightarrow]{d} \\
          C(\Phi, R, I) \arrow{r} & \St(\Phi, R, I) \arrow{r}{\mu_R} \arrow[ur, "t", dashrightarrow] & \St(\Phi, R) \arrow{r} & \St(\Phi, k[X, X^{-1}]).
    \end{tikzcd}\]
    The lifting $t$ is obtained from~\cite[Lemma~3.3]{LS20}.
    The right-hand side vertical arrow is injective by~\cite[Lemma~2.2]{LS20}.
    The left-hand side vertical arrow is surjective since the composite arrow
    $\xymatrix{ C(\Phi, A[X], M[X]) \ar[r] & C(\Phi, B, I) \ar[r] & C(\Phi, R, I) }$
    is surjective by lemma's assumption.
    It remains to repeat the diagram chasing argument of~\cite[Theorem~1]{LS20} to conclude that the homomorphism $\St(\Phi, B) \to \St(\Phi, A[X, X\inv])$ is injective.
    The assertion of the lemma will then follow from~\cite[Theorem~3]{LS20}.
\end{proof}
The proof of Horrocks theorem is, thus, reduced to showing that~\eqref{eq:c-surj} is surjective for every local domain $(A, M)$.
We need to shorter notation for certain subgroups of $\St(\Phi, R)$.

%Not let us introduce shorter notation for Steinberg groups and their subgroups relevant for the proof of Horrocks theorem.
{\allowdisplaybreaks\begin{align*}
    G     =& \St(\Phi, A[X, X^{-1}]),\\
    G^+   =& \St(\Phi, A[X]),\\
    B     =& \UU(\Phi^+, A[X, X\inv]) \rtimes \StH(\Phi, A[X, X\inv]) \leq G,\\
    G_M   =& \St(\Phi, A[X, X^{-1}], M[X, X^{-1}]),\\
    U^+   =& \UU(\Phi^+, A[X]),\\
    G^+_M =& \St(\Phi, A[X], M[X]),\\
    B_M   =& \UU(\Phi^+, M[X, X^{-1}]) \times \{X, 1+M\} \leq G_M,\\
    U^+_M =& \UU(\Phi^+, M[X]).
\end{align*}}
In the definition of $B_M$ we denote by $\{X, 1+M\}$ the image of the homomorphism $\{X, -\}_{r}$ from~\eqref{eq:relative-symbol}.

The above groups can be organized into the following commutative diagram:
\begin{equation} \label{eq:cube-Steinberg} \xymatrix{
    U^+_M \ar@{^{(}->}[rr] \ar@{^{(}->}[dd] \ar@{^{(}->}[rd] &                        & B_M \ar[dd]^(.3){g_B^M} \ar@{^{(}->}[rd]^{g^B_M} &           \\
    & G^+_M \ar[rr]^(.3){g^+_M} \ar^(.3){g^M_+}[dd] &                   & G_M \ar[dd]_{h_M} \\
    U^+ \ar@{^{(}->}[rr]^(.3){g_B^+} \ar@{^{(}->}[rd]_{g_+^B}          &                        & B \ar@{^{(}->}[rd]_{h_B}       &           \\
    & G^+ \ar[rr]_{h_+}              &                   & G.}\end{equation}
Here $g^M_+$ and $h_M$ are just renamed homomorphisms $\mu$ from~\eqref{eq:relative-Steinberg}.
The homomorphism $g^M_B$ is also induced by $\mu$.
Maps $g_M^+$ and $h^+$ are induced by the ring embedding $A[X] \to A[X, X\inv]$.
The other homomorphisms on the diagram are all obvious subgroup embeddings.

Set $V = G^+\times B \times G_M$, $W = U^+\times G^+_M \times B_M$.
Both homomorphisms $g_+^M$ and $h_M$ are crossed modules by~\cref{lem:rel-Steinberg-crossed-module}.
The fact that $(g^+_M, h_+)$ is a map of precrossed modules is obvious.
Thus, we find ourselves in the situation of~\cref{subsec:triples}.

The following lemma provides the second key reduction in the proof of Horrocks theorem.
\begin{lemma}
    Suppose that $A$ is a domain.
    Suppose that $V/\sim_W$ admits an action of $G$ such that
      for $g \in G_M$ one has \[ h_M(g) \cdot [1, 1, 1] = [1, 1, g]. \]
    Then the canonical homomorphism~\eqref{eq:c-surj} is surjective.
\end{lemma}
\begin{proof}
  Since $A$ is a domain, $g^M_B$ is injective by~\cref{lem:symbols}.
    Thus, the back face of~\eqref{eq:cube-Steinberg} is pull-back.
  Notice that for any $g \in C(\Phi, A[X, X\inv], M[X, X\inv]) \subseteq G_M$ one has
    \[ [1, 1, 1] = h_M(g) \cdot [1, 1, 1] = [1, 1, g].\]
  It is clear now that $g$ lies in the image of $C(\Phi, A[X], M[X])$ under $g^+_M$ by~\cref{lem:one-one-z}.
\end{proof}

\begin{rem}
    Notice that in the linear case the symbol homomorphism $A^\times \to \K_2(A[X, X\inv])$, $a \mapsto \{a, X\}$ admits section
     for a general commutative ring $A$ by~\cite{Wa71}.
    Thus, in the linear case the assertion of the lemma remains true without the assumption that $A$ is a domain, cf.~\cite[Lemma~3.1g]{Tu83}.
\end{rem}

\subsection{Construction of the homomorphism $\sigma(\varpi_k^\vee$)} \label{subsec:construction-sigma}
Throughout this section $\Phi$ denotes a simply laced root system of rank $\geq 3$ and $k$ is the index of
 a simple root $\alpha_k$ such that the corresponding weight $\varpi_k$ is a microweight.
The aim of this section is to prove the existence of a $\varpi_k$-pair for $\Phi$ in the sense of~\cref{dfn:delta-pair}
 under the additional assumption that $A$ is a local ring.

First of all, recall that the ideal $XA[X]$ is a splitting ideal of $A[X]$, therefore
$\St(\Phi, A[X])$ decomposes as $\St(\Phi, A[X], XA[X]) \rtimes \St(\Phi, A)$.
Notice also that under our assumptions on $\Phi$ and $k$ the subgroup $N_{\varpi_k}$ contains $\St(\Phi, A[X], XA[X])$.
In fact, $N_{\varpi_k}$ decomposes into the semidirect product of $N_0 := \St(\Phi, A[X], XA[X])$ and the parabolic subgroup $P_k \leq \St(\Phi, A)$
generated by $x_\alpha(a)$, $a \in A$ for all $\alpha \in \Delta \sqcup \Sigma^+_k$.

Recall that the universal property of semidirect products gives for any group $H$ acting on a group $N$
and any group homomorphisms $f_N\colon N \to G$, $f_H\colon H \to G$ satisfying
\begin{equation}
    \label{eq:coherence-condition} f_N({}^hn) = {}^{f_H(h)} f_N(n),\ n\in N,\ h\in H
\end{equation}
a unique map $f\colon N \rtimes H \to G$ extending $f_N$ and $f_H$.

Applying this universal property to our situation reduces the problem of construction of $\sigma(\varpi_k) \colon N_{\varpi_k} \to N_{-\varpi_k}$
to the construction problems for $\sigma(\varpi_k)_{P_k}$ and $\sigma(\varpi_k)_{N_0}$.
Since $P_k = L_k \ltimes U_k^+$, where $L_k = \mathrm{Im}(\St(\Delta_k, A) \to \St(\Phi, A))$
and $U_k^\pm = \UU(\Sigma_k^\pm, A)$ we can apply the universal property once again and further reduce the construction problem for $\sigma(\varpi_k)_{P_k}$
to the construction problems for $\sigma(\varpi_k)_{L_k}$ and $\sigma(\varpi_k)_{U_k}$.
The latter two homomorphisms now can be defined directly as follows:
\[\sigma(\varpi_k)_{L_k} \coloneqq \mathrm{id}_{L_k},\ \sigma(\varpi_k)\left(\prod\limits_{\alpha \in \Sigma_k^+} x_\alpha(a_\alpha)\right) \coloneqq \prod\limits_{\alpha \in \Sigma_k^+} x_\alpha(Xa_\alpha).\]

The first goal of this section is prove the existence of $\sigma(\varpi_1)$-pair for the root system $\Phi$ of type $\rA_\ell$.
The most important case of this construction for us will be $\ell = 3$, because it will be used in the construction $\varpi_k$-pair for other types $\Phi$ later in this subsection.
From now on we assume that the ground ring $A$ is local.

\begin{prop} \label{prop:sigma-construction}
    For a local ring $A$ and $\Phi = \rA_\ell$, $\ell + 1 = n \geq 4$ there exists a $\varpi_1$-pair.
\end{prop}
\begin{proof}
    By the above argument with semidirect products it remains to construct $\sigma(\varpi_1)_{N_0}$ where $N_0 = \St(n, A[X], XA[X])$.
    Denote by $d_1$ the matrix $d_1(X) = \mathrm{diag}(X, 1, \ldots, 1) \in \GL(n, A[X, X\inv])$.
    Now for $u \in \E(n, A[X])$ and $v \in (XA[X])^n$ we can define the map $\sigma(\varpi_1)_{N_0}$ on the generators of $N_{0}$
    using the elements defined in~\cref{subsec:another-presentation}:
    \begin{equation*}
    \sigma(\varpi_1)_{N_0} (F(u, v)) \coloneqq X^{d_1 \cdot X^{-1}}(u, v),\ \sigma(\varpi_1)_{N_0} (S(v, u)) \coloneqq Y^{d_1}(v, u).
    \end{equation*}
    It is clear that the above elements belong to $N_{-\varpi_1}$.
    We need to show that $(\sigma_1)_{N_0}$ respects the relations relations~\eqref{add4}--\eqref{coef-move}.

    For relations~\eqref{add4}--\eqref{add5} this is an immediate corollary of~\cref{itm:xsmall-additivity} and \cref{lem:xy-wd}.
    By~\cref{lem:xy-conj} for $g \in N_{-\varpi_1}$ one has
    \begin{equation}
        \label{eq:xy-conj-n1}
        g \cdot X^{d_1 X^{-1}}(u', v') \cdot g^{-1} = X^{d_1 X^{-1}}(mu', m^*v'), \text{ where } m = d_1 \cdot \pi(g) \cdot d_1^{-1}.
    \end{equation}
    To obtain that $\sigma(\varpi_1)_{N_0}$ respects~\eqref{conj3} it remains to specialize the above equality by setting $g = X^{d_1 X^{-1}}(u, v)$.

    Let us verify that $\sigma(\varpi_1)$ respects~\eqref{coef-move}, i.\,e. that for $a\in XA[X]$ and $m \in \E(n, A[X])$ one has
    $X^{d_1 X^{-1}}(me_1, m^*e_2 a) = Y^{d_1}(me_1 a, m^* e_2)$.
    First, let us verify this in the special case when $m \in \E(n, A[X])$ belongs to the subset
    \[G_0 = H_{12}(A) \cdot U^-_1(A) \cdot U^+_1(A).\]
    Here $H_{12}(A)$ denotes the subgroup of $T(n, A)$ generated by semisimple root elements $h_{12}(u)$, $u \in A^\times$.
    It is easy to check that in this case the only nonzero components of $u = m^* e_2$ are $u_1$ and $u_2$.
    Decompose $v = m e_1$ into a sum $v' + v''$, where $v' = (v_1, v_2, 0, \ldots 0)^t,$ $v'' = (0, 0, v_3, \ldots v_n)^t$.
    Since $v^t u = 0$ we obtain that $v', v'' \in D(u)$.
    Now from Lemmas~\ref{lem:xy-wd}--\ref{lem:xy-conj} we conclude that:
    \begin{multline}
        \label{eq:special-case}
        Y^{d_1}(me_{1}a, m^* e_2) = x(d_1^{-1} \cdot v'a, d_1\cdot  u) \cdot x(d_1^{-1}\cdot v''a, d_1 \cdot u) = \\
        = x(d_1^{-1} va, d_1 \cdot u) = x(d_1^{-1}X \cdot v, d_{1}X^{-1} \cdot u a) = X^{d_1 X^{-1}}(me_1, m^*e_2 a).
    \end{multline}

    To obtain the assertion in the general case notice that for local $A$ the group $\E(n, A)$ admits the following decomposition:
    \[\E(n, A) = \mathrm{EP}_1(A) \cdot H_{12}(A) \cdot U^-_1(A) \cdot U^+_1(A), \]
    where $\mathrm{EP}_1(A)$ is the image of $P_1$ in $\E(n, A)$ under $\pi$.
    This decomposition follows from the Gauss decomposition for Chevalley groups over local rings (see e.\,g. \cite[Theorem~1.1]{Sm12}).

    Thus, we obtain that $\E(n, A[X]) = \pi(N_{\varpi_1}) \cdot G_0$.
    Now factor $m \in \E(n, A[X])$ as $\pi(n) \cdot h$ for some $n\in N_{\varpi_1}$ and $h \in G_0$.
    Since $\pi(n) = d_1 \cdot \pi(g) \cdot d_1^{-1}$ for some $g \in N_{-\varpi_1}$ it remains to apply~\eqref{eq:xy-conj-n1} and~\eqref{eq:special-case}:
    \begin{multline}
        \nonumber X^{d_1 X^{-1}}(me_1, m^*e_{2}a) = {}^{g}(X^{d_1 X^{-1}}(he_1, h^*e_{2}a)) = \\
        = {}^{g}(Y^{d_1}(he_{1}a, h^*e_2)) = Y^{d_1}(me_{1}a, m^{*} e_{2}).
    \end{multline}
\end{proof}

\begin{cor} \label{cor:a3-microweight}
    For a local ring $A$ and $\Phi = \rA_3$ an $\omega$-pair exists for any microweight $\omega \in P(\Phi)$.
\end{cor}
\begin{proof}
    Recall from~\cref{lem:delta-weyl} that it suffices to prove the existence of $\omega$-pair for a representative $\omega$
     of every orbit of the action of $W(\Phi)$ on the set of microweights of $\Phi$.

    It is easy to see that up to the action of the Weyl group every nontrivial microweight in $\rA_3$ coincides with either
    $\pm\varpi_1$, $\pm\varpi_2$ or $\pm\varpi_3$.

    In \cref{prop:sigma-construction} we have proved the existence of $\sigma(\omega)$ for $\varepsilon_1 = \varpi_1$ (and hence also for $\varepsilon_2$ and $\varepsilon_3 = -\varpi_3$ from~\cref{exm:chi-linear}).
    Let us prove the existence of $\sigma(\varpi_2)$.
    The restriction of the homomorphism $\sigma(\varpi_2)$ on $N_0 = \St(4, A[X], XA[X])$ can be defined as the composition of
     $\sigma(\varepsilon_2)_{N_0}$ and $\sigma(\varepsilon_1)$.
    It easy to check that these homomorphisms can be composed correctly and that the image is contained in $N_{-\varpi_2, \Psi}$.
    To define $\sigma(\varpi_2)$ on the whole group $N_{\varepsilon_2, \Psi}$ it remains to use the above argument with semidirect products.
\end{proof}

Now we can prove the main result of this subsection.
\begin{thm} \label{thm:pairconstr}
    Let $\Phi$ be a root system of type $\rD_{\geq 4}$, $\rE_6$ or $\rE_7$ and $A$ be a local ring.
    Denote by $k$ the index of the simple root $\alpha_k$, whose weight $\varpi_k$ is a microweight
     ($k$ is the index of the vertex marked green on the Dynkin diagram from Figure 1 in~\cref{subsec:curtis-tits}).
    Then there exists a $\varpi_k$-pair.
\end{thm}
\begin{proof}
    Applying the semidirect product argument once again we reduce the construction problem for $\sigma(\varpi_k)$ of to the one for $\sigma({\varpi_k})_{N_0}$.
    In our situation \cref{thm:relPres} asserts that $N_0 = \St(\Phi, A[X], XA[X])$ can be decomposed into the free product of subgroups $\St(\Psi, A[X], XA[X])$,
    where $\Psi$ ranges over the set $A_3(\Phi)$ of root subsystems of type $\rA_3$ in $\Phi$, amalgamated over embeddings of generators $z_\alpha(Xf, g)$ into different groups $\St(\Psi, A[X], XA[X])$, whose root system $\Psi \in A_3(\Phi)$ contains $\alpha$.
    Thus, by the universal property of coproducts it remains to construct $\sigma(\varpi_k)_\Psi \colon \St(\Psi, A[X], XA[X]) \to N_{-\varpi_k}$ and then
    verify that these homomorphisms agree on generators $z_\alpha(Xf, g)$ of different $\St(\Psi, A[X], XA[X])$.

    In the case $\Psi \subseteq \Delta_k$ the homomorphism $\sigma(\varpi_k)_\Psi$ should have identical action, so we define it as the canonical map $\St(\Psi, A[X], XA[X]) \to N_{-\varpi_k}$
    induced by the embedding $\Psi \subseteq \Phi$.

    Now consider the nontrivial case $\Psi \not\subseteq \Delta_k$.
    It is clear that there is a unique nonzero weight $\omega \in P(\Psi)$ determined by $(\omega, \alpha) = (\varpi_k, \alpha)$, $\alpha \in \Psi$.
    By definition, the weight $\omega$ is a microweight for $\Psi$.
    By~\cref{cor:a3-microweight} there exists an $\omega$-pair $\xymatrix{ \sigma(\omega)\colon N_{\omega, \Psi} \ar[r] & \ar@<-1.0ex>[l] N_{-\omega, \Psi}\colon \sigma(-\omega) }$.
    The required homomorphisms $\sigma(\varpi_k)_\Psi$ now can be defined as the postcomposition of $\sigma(\omega)$ with the map $N_{-\omega, \Psi} \to N_{-\varpi_k, \Phi}$ induced by the embedding $\Psi \subseteq \Phi$.
    The fact that $\sigma(\varpi_k)_\Psi(z_\alpha(Xf, g))$ does not depend on $\Psi$ is obvious from the construction of $\sigma(\varpi_k)_\Psi$.
\end{proof}

\subsection{Construction of the action of $\St(\Phi, A[X, X\inv])$ on $V$} \label{sec:construction-delta}
From now on we assume that $A$ is a local ring.
Recall from \cref{subsec:structure-theorem-overview} that \[V = G^+ \times B \times G_M,\ W = U^+ \times G_M^+ \times B_M\]
and that $W$ acts upon $V$ via $\star$-action defined in~\cref{subsec:triples}.

For a coweight $\omega \in P(\Phi^\vee)$ denote by $V_\omega$ the subset of $V$ consisting of those triples $v = (x, y, z)$ for which $x \in N_\omega\cdot \StW(\Phi, A)$.
Throughout this section we assume that $\Phi$ is a root system of type $\rD_{\geq 4}$, $\rE_6$ or $\rE_7$
 and $\omega = \pm \varpi_k^\vee$, where $k$ is the same as in the statement of \cref{thm:pairconstr}.

\begin{dfn} \label{sigma-def}
  We define the function $\delta_\omega \colon V_\omega \to V_{-\omega}$ via the following formula:
  \begin{equation} \label{eq:sigma-def} \delta_\omega(x_1 \cdot x_2, y, z) \mapsto (\sigma(\omega)(x_1)\cdot x_2, x_2^{-1} \cdot \chi_{\omega, X}(x_2 \cdot y), \widetilde{\chi}_{\omega, X}(z)), \end{equation}
  where $x_1 \in N(\omega)$, $x_2 \in \StW(\Phi, A)$,  $y \in B$, $z \in G_M$.
\end{dfn}

In the above definition $\chi_{\omega, X}$ and $\widetilde{\chi}_{\omega, X}$ denote homomorphisms defined in~\cref{subsec:weight-automorphisms},
 while $\sigma(\omega)$ is the homomorphism constructed in~\cref{thm:pairconstr}.

By~\cref{lem:winv-chiw} the factor $x_2^{-1} \cdot \chi_{\omega, X}(x_2)$ lies in $\StH(\Phi, A[X^{\pm 1}])$,
 so the second component of the above triple belongs to $B$.
At this point, it is not immediately clear why the above definition does not depend on the choice of decomposition $x = x_1 \cdot x_2$.

Denote by $\Lambda$ the set of microweights of $\Phi$.
Recall that $\Lambda$ coincides with the orbit set $W(\Phi) \cdot \varpi_k^\vee$.
Since $W(\Delta_k)$ stabilizes $\varpi_k^\vee$ we obtain that the coset set $W(\Phi)/W(\Delta_k)$ is in one-to-one correspondence with $\Lambda$.
%For each minuscule weight $\lambda \in \Lambda = W(\Phi)\cdot \varpi_k$ we choose a representative $w_\lambda \in W(\Phi)$ a representative for the coset $W(\Delta) w_\lambda$.

%Denote by $w_{\lambda, u}$ an element of the form $\prod w_{\alpha_i}(u_i) \in W(\Phi, A)$ such that $\prod u_i = u$, $\alpha_i \in \Sigma_k$ and $\varpi_k + \sum \alpha_i = \lambda$.
%{\bf In this sum all $\alpha_i$'s are different!}
%If $\lambda = \varpi_k$ set $w_{\lambda, \varpi_k}(u) = h_\alpha(u)$ for some $\alpha \in \Sigma_k$.
%There is a map $\iota \colon W(\Phi) \to W(\Phi, A)$ given by $s_{\alpha_i} \mapsto w_{\alpha_i}(\pm 1)$ (this does not depend on the choice of the set $S = \{s_{\alpha_i}\}_{\alpha_i \in \Pi}$ of simple reflections of $W(\Phi)$, however, it is {\it not} a homomorphism of groups).

We denote by $V$ the microweight representation of $\Gsc(\Phi, A)$ with the highest weight $\varpi_k$ (see e.\,g. \cite[\S~2]{Ge17} or \cite[\S~1.1]{V00}).
This representation has dimension $|\Lambda| $ and we denote by $v^\lambda$ its basis vector corresponding to $\lambda \in \Lambda$.
We also denote by $v^+$ the highest weight vector of this representation.

Recall from~\cite[Lemma~6]{V00} that for every $w \in W(\Phi, A)$ there exists a unique $\lambda \in \Lambda$ and $u \in A^\times $ such that
 \begin{equation}\label{eq:w-highestweight} w \cdot v^+ = u v^\lambda. \end{equation}
Conversely, for every pair $(u, \lambda) \in A^\times \times \Lambda$ there exists some element $w = w_{\lambda, u} \in W(\Phi, A)$ such that~\eqref{eq:w-highestweight} holds.
\begin{rem} Each $w_{\lambda, u}$ can be chosen to be a product of at most $2$ elements $w_\beta(v)$ for $\beta\in\Sigma_k$ in the cases $\Phi = \rD_\ell, \rE_6$ or $3$ such elements in the case $\Phi = \rE_7$. \end{rem}

\begin{lemma} \label{lem:can-repr}
  For $\omega = \pm \varpi_k^\vee$ one has \[N_{\omega} \cdot \StW(\Phi, A) = \bigsqcup\limits_{(u, \lambda) \in A^\times \times \Lambda} N_{\omega} \cdot w_{\lambda, u}. \]
\end{lemma}
\begin{proof}
  First of all, observe that the natural homomorphism $\K_2(\Delta_k, A) \to \K_2(\Phi, A)$ is surjective.
  This follows from the fact that both groups are generated by Steinberg symbols by~\cite[Theorem~2.5]{Ste73}.
  Consequently, the coset set $\StW(\Phi, A) / \StW(\Delta_k, A)$ is in bijective correspondence with the coset set $W(\Phi, A)/W(\Delta_k, A)$.

  Now we claim that the coset set $W(\Phi, A)/W(\Delta_k, A)$ is isomorphic to the set $A^\times \times \Lambda$
   via the mapping $w W(\Delta_k, A) \mapsto (u, \lambda)$, where $u$ is determined from~\eqref{eq:w-highestweight}.
  This map is obviously surjective, its injectivity follows from Chevalley--Matsumoto decomposition~\cite[Theorem~1.3]{St78}.

  It remains to notice that $N_{\omega} \supseteq \StW(\Delta_k, A)$ and that for each $g \in N_{\omega}$ the element $\pi(g(0))$ stabilizes the highest weight vector $v^+$.
\end{proof}

Notice that for $g \in \StW(\Delta, A)$ one has $\chi_{\omega, X}(g) = g$ therefore we obtain the following
\begin{cor}
 The right-hand side of~\eqref{eq:sigma-def} does not depend on the choice of decomposition $x = x_1 \cdot x_2$, so $\delta_\omega$ is defined unambigously.
\end{cor}

%We will also need to compute $x^{-1} \cdot \chi_{\omega, X}(x)$ for the representative $x = h_{\alpha_k}(u)\iota(w_\lambda)$. For simplicity, we may assume $w_\lambda = w_\beta(1)$, $\beta \in \Sigma_k$.
%\begin{multline} x^{-1} \cdot \chi_{\omega, X}(x) = \{ X, u\} \cdot \iota(w_\lambda)^{-1} \chi_{\omega, X}(w_\lambda) = \\ = \{X, u\} \cdot w_\beta(1)^{-1} w_\beta(X) \end{multline}

Our next goal is to show that the function $\delta_\omega$ gives rise to a well-defined function
 \[\overline{\delta}_\omega \colon V/\sim_W \to V/\sim_W.\]
The first step towards achieving this is the following
\begin{lemma}
 For every $v \in V$ there exists $(a, b, 1) \in W$ such that $(a, b, 1) \star v \in V_\omega$.
\end{lemma}
\begin{proof}
 Choose system of positive roots $\Phi^+'$ to be either $\Phi^+$ or $\Phi^-$ depending on whether $\omega $ is $\varpi_k^\vee$ or $-\varpi_k^\vee$.
 From~\cref{lem:bruhat} and the normality of the subgroup $\St(\Phi, A[X], X[A]) \leq G^+$ we obtain that
  \[G^+ = \UU(\Phi^{+}', A)\cdot \St(\Phi, A[X], XA[X]) \cdot \StW(\Phi,A)\cdot \overline{\St}(\Phi, A, M) \cdot\UU(\Phi^+, A).\]
 The first two factors in the above decomposition are contained in $N_\omega$, so the assertion follows from the definition of $\star$.
\end{proof}

\begin{prop}
 For every $w \in W$, $v \in V_\omega$ such that $v' = w \star v \in V_\omega$ there exists $w' \in W$ such that
  $\delta_\omega(w \star v) = w' \star \delta_\omega(v)$.
\end{prop}
\begin{proof}
 It suffices to consider the case $\omega = \varpi_k^\vee$.
 Suppose that $v' = (x_1' x_2', y', z') = (a, b, 1) \star v$, where $v = (x_1 x_2, y, z)$.
 We have that
 \begin{equation}\label{eq:x-xprime} x_1' x_2' = x_1 x_2 \cdot \mu(b) \cdot a  \text{ for some }x_1, x_1' \in N_\omega,\ x_2, x_2' \in \StW(\Phi, A) \end{equation}
 and some $a \in \UU(\Phi^+, A[X]),\ b \in \St(\Phi, A[X], M[X])$.
 Thanks to~\cref{lem:can-repr} we may assume, without loss of generality, that $x_2 = w_{\lambda, u}$, $x_2' = w_{\lambda', u'}$
 for some $u, u' \in A^\times$, $\lambda, \lambda' \in \Lambda$.

 Evaluating both sides of~\eqref{eq:x-xprime} at $X=0$ and projecting them to $\St(\Phi, k)$ we obtain the equality
  $\overline{x_1'}(0) \cdot \overline{x_2'} = \overline{x_1}(0) \cdot \overline{x_2}\cdot \overline{a}(0).$
 Since $x_1(0)$, $x_1'(0)$ and $a$ stabilize the highest weight vector $v^+$ we conclude that $\lambda = \lambda'$ and that $u - u' \in M$.
 %Again, without loss of generality we may assume $w_{\lambda, u} = w_{\alpha'}(u) \cdot [ w_{\alpha''}(1) \cdot w_{\alpha'''}(1)]$.
 Set $m = 1 - u/u' \in M$.
 Since $\K_2(\Phi, A) \subseteq \StW(\Delta_k, A)$ it follows from Chevalley--Matsumoto decomposition that
  $x_2 = c \cdot h_{\alpha}(1 + m) \cdot x_2'$ for some $c \in \StW(\Delta_k, A)$ and $\alpha \in \Sigma_k$.
 %Set $\epsilon = 1$ if $\lambda\neq \varpi_k$, otherwise set $\epsilon = -1$.
 By~\cref{lem:winv-chiw} $\chi_{\omega, X}$ acts identically on $\StW(\Delta_k, A)$, consequently we obtain from~\eqref{eq:chi-h} that
\begin{multline*}\label{eq:central-term}
 {x_2}^{-1} \cdot \chi_{\omega, X}(x_2) =
    {x_2'}^{-1} \cdot h_{\alpha}^{-1}(1 + m) \cdot c^{-1} \cdot \chi_{\omega, X}(c \cdot h_{\alpha}(1 + m) \cdot x_2') = \\
    = {x_2'}^{-1} \cdot h_{\alpha}^{-1}(1 + m) \cdot \chi_{\omega, X}(h_{\alpha}(1 + m) \cdot x_2') = \{X, 1+m\} \cdot {x_2'}^{-1} \chi_{\omega, X}(x_2').
\end{multline*}
 Since $\StW(\Delta_k, A) \subseteq N_\omega$ we obtain that from~\eqref{eq:x-xprime} that
 \begin{equation} h_{\alpha}(1 + m) \cdot x_2' \cdot \mu(b) \cdot a \cdot {x_2'}^{-1} \in N_\omega. \end{equation}
 Projecting this equality into $\St(\Phi, \kappa[x])$ we conclude that $\overline{x_2'} \overline{a} \overline{x_2'}^{-1}$ stabilizes the highest weight vector $v^+$.
 Consequently, there exists $q \in \UU(\Phi^+, M[X])$ such that $a' = x_2' \cdot q \cdot a \cdot {x_2'}^{-1}$ and $b' = h_\alpha(1 + m) \cdot x_2' \cdot \mu(b) \cdot q^{-1} x_2'^{-1}$ both stabilize the highest weight vector $v^+$ and hence both belong to $N_\omega$.

\begin{multline} \label{eq:computation1}
 \delta_\omega(x_1'x_2', y', z') =
 \bigl(\sigma(\omega)(x_1') \cdot x_2',\, x_2'^{-1}\cdot \chi_{\omega, X}(x_2' y'),\, \widetilde{\chi}_{\omega, X}(z')\bigr) = \\
 = \bigl(\sigma(\omega)(x_1) \cdot \sigma(\omega)(c b') \sigma(\omega)(a') \cdot x_2',\, \{1+m, X\} \cdot x_2^{-1}\cdot \chi_{\omega, X}(x_2 y'), \chi_{\omega, X}(z') \bigr) = \\
 = \bigl(\delta_\omega(x_1) \cdot x_2 \cdot \delta_{\omega'}(aq^{-1}) \cdot \delta_{\omega'}(\pi(b')) \cdot h^{-1}_{\alpha'}(1+m),\, \{X^\epsilon, 1+m\} \cdot x_2^{-1}\cdot \chi_{\omega, X}(x_2 y'), \widetilde{\chi}_{\omega, X}(z') \bigr) = \\
 = (1, 1, \widetilde{\chi}_{\omega', X}(b') \cdot h_{\alpha'}^{-1}(1+m)) \star \\ \star \bigl(\delta_\omega(x_1) \cdot x_2 \cdot \delta_{\omega'}(aq^{-1}),\, \{X^\epsilon, 1+m\} \cdot x_2^{-1}\cdot \chi_{\omega, X}(x_2 y'), \bigl(\widetilde{\chi}_{\omega', X}(b') h^{-1}_{\alpha'}(1+m)\bigr)^{x_2^{-1} \cdot \chi_{\omega, X}(x_2 y')} \widetilde{\chi}_{\omega, X}(z') \bigr).
\end{multline}


Now let us compute the third coordinate:
\begin{multline} \bigl(\widetilde{\chi}_{\omega', X}(b') \cdot h^{-1}_{\alpha'}(1+m)\bigr)^{x_2^{-1} \cdot \chi_{\omega, X}(x_2 y')} \widetilde{\chi}_{\omega, X}(z') = \\ =
\bigl(\widetilde{\chi}_{\omega', X}(q \cdot b^a \cdot h_{\alpha'}(1+m)) \cdot h^{-1}_{\alpha'}(1+m)\bigr)^{x_2^{-1} \cdot \chi_{\omega, X}(x_2 y')} \widetilde{\chi}_{\omega, X}(b^{-y}z) = \text{by~\cref{h-computation}}  \\ = \bigl(\widetilde{\chi}_{\omega, X}((q \cdot b^a \cdot h_{\alpha'}(1+m))^{x_2^{-1}}) \bigr)^{\chi_{\omega, X}(x_2 y')} \cdot h_{\alpha'}(1+m)^{-\chi_{\omega, X}(y')} \cdot \{1 + m, X^{\epsilon + 1} \} \cdot \widetilde{\chi}_{\omega, X}(b^{-y}z) = \\ = \widetilde{\chi}_{\omega, X}\bigl((q \cdot b^a \cdot h_{\alpha'}(1+m))^{y'}\bigr) \cdot \widetilde{\chi}_{\omega, X}\bigl(h_{\alpha'}(1+m)\bigr)^{-\chi_{\omega, X}(y')} \cdot \{1 + m, X^\epsilon \} \cdot \widetilde{\chi}_{\omega, X}(b^{-y}z) = \\ = \{ 1+m, X^\epsilon \} \cdot \widetilde{\chi}_{\omega, X}(q^{y'} \cdot z).
\end{multline}


Now recalling that $y' = a ^{-1} y, z' = b^{-y} z$ we can continue~\eqref{eq:computation1} as follows:
\begin{multline*}
 \bigl(\delta_\omega(x_1) \cdot x_2 \cdot \delta_{\omega'}(aq^{-1}),\, x_2^{-1}\cdot \chi_{\omega, X}(x_2 y'), \widetilde{\chi}_{\omega, X}(q^{y'} z) \bigr) = \\ = (\delta_{\omega'}(qa^{-1}), 1, \chi_{\omega, X}(q^{y'})) \star \bigl(\delta_\omega(x_1) \cdot x_2,\, \cdot \chi_{\omega', X}(aq^{-1}) \cdot x_2^{-1}\cdot \chi_{\omega, X}(x_2 q a^{-1} \cdot y ), \widetilde{\chi}_{\omega, X}(z) \bigr) = \\ = (\ldots) \star (\delta_\omega(x_1) \cdot x_2,\, x_2^{-1} \chi_{\omega, X}(x_2y), \widetilde{\chi}_{\omega, X}(z)) \end{multline*} \end{proof}


 For $w\in W(\Phi)$ denote by $\Stab_{w}$ the algebraic subgroup of $\G_{sc}(\Phi, -)$ consisting of elements stabilizing the vector $\iota(w) \cdot v^+$. It is clear that the resulting subgroup does not depend on the exact choice of $\iota$. Recall that the projection $\varphi \colon \St(\Phi, R) \to \G_{sc}(\Phi, R)$ is always injective on unipotent subgroups. Moreover, it is clear that $N(w \cdot \varpi_k)(0) = \varphi^{-1}(\Stab_w(A[X]))$.

\begin{lemma} \label{lem:q}
Let $R$ be a commutative ring, $I$ be an ideal of $R$.
Suppose that $a \in \UU(\Phi^+, R)$ is such that its image $\varphi(\overline{a})$ lies in $\Stab_{w}(R/I)$ for some $w \in W(\Phi)$.
Then there exists $q \in \UU(\Phi^+, I)$ such that $\varphi(aq)$ lies in $\Stab_{w}(R)$.
\end{lemma}
\begin{proof}
Set $b = a^{\iota(w)}$.
Notice that $(b \cdot v_+)_{\varpi_k} = 1$, therefore by the Chevalley--Matsumoto decomposition we can write $b = b_1 \cdot b_2$ for some $b_1$ such that $\varphi(b_1) \in \Stab_1(R)$ and $b_2 \in \UU(\Sigma_k^-, R)$.
It is also clear from our assumptions that $b_2$, in fact, belongs to $\UU(\Sigma_k^-, I)$.
Set $a_i = {}^{\iota(w)}b_i$.
It is clear that $\varphi(a_1) \in \Stab_w(R)$, $\varphi(a_2) \in \UU(\Phi^+ \cap w(\Sigma_k^-), I)$ and $a = a_1 \cdot a_2$ as claimed.
\end{proof}


    \section{Proof of main results} \label{sec:main}
    \subsection{Early stability theorem}
The aim of this subsection is to prepare necessary technical ingredients for the proof of early stability theorem~\cref{cor:dedekind}.

For the rest of this subsection $\Psi$ is an arbitrary irreducible simply-laced root system of rank $\geq 3$ embedded into another irreducible simply-laced root system $\Phi$.
For a commutative ring $R$ we denote by $j_R$ the corresponding homomorphism of Steinberg groups $\St(\Psi, R) \to \St(\Phi, R)$ induced by this embedding.

The following result is the analogue of the so-called dilation principle for subsystem embeddings.
\begin{lemma}\label{lem:dp-2}
Let $h\in\St(\Phi, A[X], X A[X])$ be such that $\lambda_a^*(h) \in \Img(j_{A_a[X]})$.
Then for sufficiently large $n$ one has
\[\ev{A}{A[X]}{a^n\cdot X}^*(h) \in \mathrm{Im}(j_{A[X]}).\]
\end{lemma}
\begin{proof}
 We denote by $A\ltimes XA_a[X]$ the semidirect product of $A$ and the ideal $XA_a[X]$, cf. e.\,g.~\cite[Definition~3.2]{S15}.
 Denote by $\theta$ the obvious map $A[X]\rightarrow A\ltimes XA_a[X]$ localizing at $a$ all coefficients of terms of degree $\geq 1$.

 Recall from~\cite[\S~2]{LS17} that there exists a homomorphism
 \[T_\Psi \colon \St(\Psi, A_a[X], XA_a[X]) \to \St(\Psi, A \ltimes XA_a[X])\]
 such that $\theta^* = T_\Psi \circ \lambda_a^*$.

 Let $(A_i, f_{ij})$ be the directed system of rings given by
 \[A_i\coloneqq A[X],\ f_{ij} \coloneqq \ev{A}{A[X]}{a^{j-i} \cdot X},\ 0 \leq i\leq j.\]
 It is easy to check that $\varinjlim A_i$ coincides with $A \ltimes XA_a[X]$.
 The canonical morphisms $A_j\rightarrow \varinjlim_i A_i \cong A \ltimes XA_a[X]$ can be easily computed as $\ev{A}{A\ltimes XA_a[X]}{a^{-j} \cdot X}$.

 By hypothesis $\lambda_a^*(h) = j_{A_a[X]}(h')$ for some $h' \in \St(\Psi, A_a[X], XA_a[X])$.
 Consequently, $\theta^*(h) = j_{A \ltimes XA_a[X]}(T_\Psi(h'))$
 and the assertion of the lemma follows from the fact that the Steinberg group functor commutes with
  colimits over directed systems (cf.~\cite[Lemma~2.2]{Tu83}):
 \[\St(\Psi, A\ltimes XA_a[X]) = \St(\Psi, \varinjlim_i A_i) \cong \varinjlim_i \St(\Psi,A_i). \qedhere\]
\end{proof}

\begin{lemma}\label{lem:L25-2}
Let $a$ and $b$ be a pair of coprime elements of $A$.
Let $g$ be an element of $\St(\Phi, A[X], XA[X])$ such that
$\lambda_a^*(g) \in \Img(j_{A_a[X]})$ and $\lambda_b^*(g) \in \Img(j_{A_b[X]})$.
Then $g$ lies in the image of $j_{A[X]} \colon \St(\Psi, A[X]) \to \St(\Phi, A[X])$.
\end{lemma}
\begin{proof}
    We reproduce the argument of~\cite[Lemma~2.5]{Tu83}, cf. also with~\cite[Lemma~16]{S15}.
    Set $S := A[X, Y]$.
    Consider the following element of $\St(\Phi, S[Z])$:
    \[h(X, Y, Z) := g(YX) \cdot  g^{-1}((Y+Z) X) = \ev{A}{S[Z]}{YX}^* \left(g\right) \cdot \ev{A}{S[Z]}{(Y + Z)X}^*\left(g^{-1}\right).\]
    It is easy to see that $h(X, Y, Z)$ belongs to
    \[\Ker\left(\eval{Z}{S}{S}{0}^*\colon\St(\Phi, S[Z]) \rightarrow \St(\Phi, S)\right)\]
    and hence by~\cite[Lemma~8]{S15} lies in $\St(\Phi, S[Z], Z S[Z])$.
    On the other hand, \begin{multline*}
                           \lambda_{a}^*(h(X, Y, Z)) = \left(\lambda_a\circ \ev{A}{S[Z]}{Y X}\right)^*(g) \cdot \left(\lambda_a\circ \ev{A}{S[Z]}{(Y + Z)X}\right)^*(g^{-1}) = \\
                           = \ev{A_a}{S_a[Z]}{YX}^*(\lambda_{a}^*(g)) \cdot \ev{A_a}{S_a[Z]}{(Y + Z)X}^*(\lambda_{a}^*(g^{-1})) \end{multline*}
    lies in the image of $j_{S_a[Z]}$.
    Similarly, $\lambda_{b}^*(h(X, Y, Z))$ lies in the image of $j_{S_b[Z]}$.

    We claim that there exists $n$ such that both $h(X, Y, a^n Z)$ and $h(X, Y, b^n Z)$
    lie in the image of $j(S[Z])$.

    By assumption, there exist $r, s\ \in A$ such that $r a^n + s b^n = 1$, consequently
    \begin{multline*}
        g(X) = g(X)\cdot g^{-1}(ra^n\cdot X) \cdot g(ra^n\cdot X) \cdot g^{-1}(0) = \\
         = h(X, 1, -sb^n) \cdot h(X, ra^n, -ra^n) \in \mathrm{Im}(j_{A[X]}). \qedhere
    \end{multline*} \end{proof}

The following result is the subsystem analogue of~\cite[Theorem~2]{LS17}, cf.\ also~\cite[Theorem~2.1]{Tu83}.
\begin{cor}[Local-global principle for subsystem embeddings] \label{cor:QS-subsystem}
    An element $g \in \St(\Phi, A[X], XA[X])$ lies in $\Img(j_{A[X]})$ if and only if
     $\lambda_M^*(g) \in \St(\Phi, A_M[X])$ lies in $\Img(j_{A_M[X]})$ for all maximal ideals $M \trianglelefteq A$.
\end{cor}
\begin{proof}
    It suffices to show ``if'' part of the statement.
    One defines the \textit{Quillen set} $Q(g)$ as the set consisting of those elements $a \in A$
    such that $\lambda_a^*(g) \in \Img(j_{A_a[X]})$.

    Repeating the same argument as in the proof of~\cite[Theorem~2]{S15} one shows using~\cref{lem:L25-2}
     that $Q(g)$ is an ideal of $A$, which can not be proper and therefore must coincide with $A$.
\end{proof}

Before we proceed further we would like to briefly recall the main construction from~\cite{LS20} upon which
 the proof of Theorem~1 ibid. is based.
Recall that one constructs an action of the group $\St(\Phi, A[X\inv] + M[X])$ on a certain set $\overline{V}$.
This set $\overline{V}$ is unrelated to the set $\overline{V}$ encountered in~\cref{sec:horrocks}.
Its construction proceeds as follows.

Set $G_{M, \Phi}^{\geq 0} \coloneqq \Img(\St(\Phi, A[X], M[X]) \to \St(\Phi, A[X, X\inv])), G_M^0 \coloneqq \overline{\St}(\Phi, A, M)$.
$G_M^0$ is easily seen to be a subgroup of both $\St(\Phi, A[X\inv])$ and $G_M^{\geq 0}$.
Denote by $\overline{V}$ the quotient-set of the product $V \coloneqq G_M^{\geq 0} \times \St(\Phi, A[X\inv]) \times (1 + M)^\times$
modulo the equivalence relation given by $(gh_0, h, u) \cong (g, h_0h, u)$ where $h_0 \in G_M^0, (g, h, u) \in V.$
Denote by $[g, h, u] \in \overline{V}$ the equivalence class corresponding to $(g, h, u)\in V$, cf.~\cite[\S~5.4]{LS20}.

Notice also that although the results in~\cite[\S~5.5]{LS20} are conditional,
 i.\,e. they are formulated under additional assumption that the canonical homomorphism
 $\St(\Phi, A[X\inv] + M[X]) \to \St(\Phi, A[X, X\inv])$ is injective,
this does not present a problem for us since this condition has been checked already during the proof of~\cref{lem:first-reduction}.

\begin{prop} \label{prop:horrocks-main} The group $\St(\Phi, A[X\inv] + M[X])$ acts simply transitively on $\overline{V}$.
This action satisfies the following additional property.
Suppose that for some
\[g \in \Img(j\colon \St(\Psi, A[X\inv] + M[X]) \to \St(\Phi, A[X\inv] + M[X])), \]
$h, h' \in \St(\Phi, A[X\inv])$ and $u, h' \in 1 + M$ one has
\[ g \cdot [1, h, u] = [g', h', u'].\]
Then $g'$ belongs to $\Img(j\colon G_{M, \Psi}^{\geq 0} \to G_{M, \Phi}^{\geq 0})$.
\end{prop}
\begin{proof}
    The existence of the action of $\St(\Phi, A[X\inv] + M[X])$ and its faithfullness are contained in
    Proposition~5.39 and Remark 5.42 of~\cite{LS20}.
    The stated property follows from the construction of the action in~\cite[\S~5.4]{LS20}.
\end{proof}

The following lemma generalizes~\cite[Proposition~4.3(b)]{Tu83}.
\begin{lemma} \label{lem:horrocks-subsystem-local}
Let $A$ be a local domain with maximal ideal $M$ and residue field $\kappa$.
Suppose that the image in $\St(\Phi, A[X, X\inv])$ of the element $x \in \K_2(\Phi, A[X], XA[X])$
 can be decomposed as $j_{A[X, X\inv]}(y) \cdot \lambda_{X^{-1}}^*(z)$ for
 some $y \in \St(\Psi, A[X, X\inv])$ and $z \in \St(\Phi, A[X\inv])$
Then $x$ belongs to $\Img(j_{A[X]})$.
\end{lemma}
\begin{proof}
    We denote by $\rho_{A}$ (resp. $\rho_{A[X]}$, $\rho_{A[X, X\inv]}$) the canonical homomorphism
     $A \to \kappa$ (resp. $A[X] \to \kappa[X]$, $A[X, X\inv] \to \kappa[X, X\inv]$).

    Recall from~\cite{Hur77} that $\K_2(\Phi, \kappa[X]) = \K_2(\Phi, \kappa)$ hence one has $\rho_{A[X]}(x) = 1$
     hence by our assumptions $\rho_{A[X, X\inv]}(j_{A[X, X\inv]}(y) \cdot z) = 1$.
    Consequently, we can find $z_0 \in \St(\Psi, A[X\inv])$ and $z_1 \in \overline{\St}(\Phi, A[X^{-1}], M[X^{-1}])$
     such that $z = j_{A[X\inv]}(z_0) \cdot z_1$.
    It is clear that $\rho_{A[X, X\inv]}(y \cdot \lambda_{X^{-1}}(z_0)) = 1$ hence we can find
     $y' \in \St(\Psi, A[X, X\inv], M[X, X\inv])$ such that $\mu(y') = \cdot \lambda_{X^{-1}}(z_0)$.
    Notice also that $t(y') \in \St(\Phi, A[X\inv] + M[X])$, where $t$ denotes the homomorphism discussed in~\cref{lem:first-reduction}.
    From~\cref{prop:horrocks-main} we obtain that
    \[ [x, 1, 1 ] = t(y') [1, z_1, 1] = [g', h', u']. \]
    for some $g' \in \Img(\G_{M, \Psi}^{\geq 0} \to G_{M, \Phi}^{\geq 0})$.
    From the definition of $\overline{V}$ we obtain that $x = g' \cdot h_0$ for some $h_0 \in \overline{\St}(\Phi, A, M)$.
    But since $x(0) = 1$ we conclude that $x = g' \cdot (g'(0))^{-1}$ from which the assertion of the lemma follows.
\end{proof}

\begin{cor} \label{cor:horrocks--ingredient}
    Let $A$ be an arbitrary commutative domain.
    Suppose that the image in $\St(\Phi, A[X, X\inv])$ of the element $x \in \K_2(\Phi, A[X], XA[X])$
    can be decomposed as $j_{A[X, X\inv]}(y) \cdot \lambda_{X^{-1}}^*(z)$ for
    some $y \in \St(\Psi, A[X, X\inv])$ and $z \in \St(\Phi, A[X\inv])$
    Then $x$ belongs to $\Img(j_{A[X]})$.
\end{cor}
\begin{proof}
    This is a consequence of~\cref{lem:horrocks-subsystem-local} and~\cref{cor:QS-subsystem}.
\end{proof}

\subsection{Main results}

We start by recalling the so-called Zariski excision property of Steinberg groups.
\begin{lemma} \label{lem:zariski-glueing}
Let $\Phi$ be any simply-laced root system of rank $\geq 3$.
Let $A$ be a commutative domain and $a, b \in A$ be a pair of coprime elements.
\begin{enumerate}
    \item Let $\delta$ be an element of $\St(\Phi, A_{ab})$.
    Then $\delta$ can be presented as $\lambda_b(x) \cdot \lambda_a(y)$ for some
    $x  \in \St(\Phi, A_a)$ and $y \in \St(\Phi, A_b)$.
    \item  Let $x \in \St(\Phi, A_a)$ and $y \in \St(\Phi, A_b)$ be such that the equality $\lambda_b(x) = \lambda_a(y)$ holds in $\St(\Phi, A_{ab})$.
    Then there exists $z \in \St(\Phi, A)$ such that $x = \lambda_a(z)$, $y = \lambda_b(z)$.
\end{enumerate}
\end{lemma}
\begin{proof}
    This is a special case of Nisnevich excision for domains, see~\cite[Proposition~4.5]{LSV2}
    (cf. also the proof of~\cite[Lemma~2.6]{LSV2}).
\end{proof}


\begin{lemma} \label{lem:horrocks-b}
 Let $A$ be a commutative domain and $f \in A[X]$ be a unitary polynomial.
 Let $h$ be an element of $\K_2(\Phi, A[X], XA[X])$ such that $\lambda_f(h)$ belongs to the image of the stabilization map
 $\St(\Psi, A[X]_f) \to \St(\Phi, A[X]_f)$.
 Then $h$ lies in the image of $\St(\Psi, A[X]) \to \St(\Phi, A[X])$.
\end{lemma}
\begin{proof}
    Suppose that $f(X) = X^n + a_1 X^{n-1} \ldots + a_n$.
    Set $g(X\inv) = 1 + a_1 X\inv + \ldots + a_{n} X^{-n}$.
    It is clear that $A[X, X\inv]_f = A[X\inv]_{X\inv g}$ and, moreover, that $X\inv$ and $g$ are not zero divisors and together generate the unit ideal of $A$.

    Consider the following diagram:
    %! suppress = EscapeAmpersand
    \[ \xymatrix{\St(\Phi, A[X]) \ar[r]^{\lambda_X} \ar[d]_{\lambda_f} & \St(\Phi, A[X, X\inv]) \ar[d]_{\lambda_f}  & \St(\Phi, A[X\inv]) \ar[l]_{\lambda_{X\inv}} \ar[d]_{\lambda_g}  \\
                 \St(\Phi, A[X]_f) \ar[r] & \St(\Phi, A[X, X\inv]_f) & \St(\Phi, A[X\inv]_g) \ar[l] \\
                 \St(\Psi, A[X]_f) \ar[r] \ar[u]_j & \St(\Psi, A[X, X\inv]_f) \ar[u]_j & \St(\Psi, A[X\inv]_g) \ar[l] \ar[u]_j \\
                                   & \St(\Psi, A[X, X\inv]) \ar[u]_{\lambda_f^\Psi}   & \St(\Psi, A[X\inv]). \ar[l] \ar[u]_{\lambda_g^\Psi}}\]

    By assumption there exists $\widetilde{h} \in \St(\Psi, A[X]_f)$ such that $j(\widetilde{h}) = \lambda_f(h)$.
    By the first part of~\cref{lem:zariski-glueing} one can write 
     $\lambda_X(\widetilde{h}) = \lambda_f^\Psi(z) \cdot \lambda_{X\inv}(y)$
     for some $y \in \St(\Psi, A[X\inv]_g)$, $z \in \St(\Psi, A[X, X\inv]).$
    Consequently, one has $\lambda_f(j(z)\inv \cdot \lambda_X(\alpha)) = \lambda_{X\inv}(j(y))$.
    By the second part of~\cref{lem:zariski-glueing} there exists $y' \in \St(\Phi, A[X\inv])$ such that $\lambda_X(\alpha) = j(z) \cdot \lambda_{X\inv}(y').$
    The assertion now follows from~\cref{cor:horrocks--ingredient}.
\end{proof}

\begin{thm}\label{thm:early-stability}
Let $\Phi$ be a root system of type $\rA_{\geq 4}$, $\rD_{\geq 5}$ or $\rE_{6,7,8}$ and let $A$ be an arbitrary noetherian commutative
 domain of Krull dimension $\leq 1$.
Then for any $n \geq 0$ the obvious inclusion $\rA_4 \subseteq \Phi$ induces a surjection
\[\K_2(\rA_4, A[X_1,\ldots, X_n]) \to \K_2(\Phi, A[X_1, \ldots X_n]).\]
\end{thm}
\begin{proof}[Proof of Theorem 1]
    The proof is modeled after the proof of~\cite[Theorem~5.3]{Tu83}.
    We proceed by induction on $n$.
    Our assumption on the dimension of $A$ guarantees that it satisfies the condition $\mathrm{SR}_3$ in the sense of~\cite{St78}.
    Thus, from Corollary~3.2 and Theorem~4.1 of~\cite{St78} we conclude that the composite arrow in the following diagram is a surjection:
    \[\K_2(\rA_2, A) \to \K_2(\rA_4, A) \to \K_2(\Phi, A).\]
    Consequently, we obtain that the right arrow is a surjection, which yields the induction base.

    Now let us verify the induction step.
    Set $C = A[X_2, \ldots , X_n]$ and $B = C[X_1]$.
    We need to show that $\K_2(\rA_4, B) \to \K_2(\Phi, B)$ is surjective.
    Every element $\alpha \in \K_2(\Phi, B)$ can be decomposed as a product $\alpha = \alpha_0 \cdot \alpha_1$,
      where $\alpha_0 \in \K_2(\Phi, C)$ and $\alpha_1 \in \K_2(\Phi, B, X_1 B)$.
    By inductive assumption $\K_2(\rA_4, C)$ surjects onto $\K_2(\Phi, C)$, so it remains to show that $\alpha_1$ lies in the image of $\K_2(\rA_4, B)$.

    Denote by $S$ the multiplicative system $S \subseteq B$ consisting of polynomials $f$ such that for sufficiently large $m$
    the polynomial $f$ becomes unitary in $Y_1$ after substitutions $X_1 \coloneqq Y_1,$ $X_2 \coloneqq Y_2 + Y_1^m, \ldots X_n \coloneqq Y_n + Y_1^{m^n}$.
    Recall from~\cite[\S~6]{Su77} that $\dim(S^{-1}B) \leq \dim(A) = 1$.
    By induction base the map $\K_2(\rA_4, S^{-1}B) \to \K_2(\Phi, S^{-1}B)$ is surjective.
    Since functor $\K_2$ commutes with filtered colimits (cf. \cite[Lemma~3.3]{LSV2}) there exists $f \in S$ such that $\lambda^*_f(\alpha_1)$ lies in the image of $\K_2(\rA_4, B_f) \to \K_2(\Phi, B_f)$.
    By the construction of $S$ we may assume that $f$ is unital in $X_1$.
    The required assertion now follows from~\cref{lem:horrocks-b}.
\end{proof}

\begin{proof}[Proof of~\cref{thm:LP-for-K2}]
    This is what the proof of~\cite[Theorem~1.1]{LSV2} actually shows if we use the more general~\cref{thm:horrocks-k2}.
\end{proof}

\begin{proof}[Proof of~\cref{cor:motivic-pi1}]
    Repeat the proof of~\cite[Corollary~1.2]{LSV2} verbatim.
\end{proof}

\begin{proof}[Proof of~\cref{cor:dedekind}]
    Consider the following diagram:
    \[ \K_2(\rA_3, A) \to \K_2(\rA_4, A) \to \K_2(\rA_4, A[X_1, \ldots, X_n]) \rightarrow \K_2(A[X_1, \ldots, X_n]) \to \K_2(A).\]
    The two right arrows on the above diagram are isomorphisms by the $\mathbb{A}^1$-invariance of the stable $\K_2$
    (see e.\,g. \cite[Theorem~V.6.3]{Kbook}) combined with the main result of~\cite{Tu83}.
    On the other hand, by~\cite[Corollary~3.2]{ST76} the left arrow and the composite arrow are isomorphisms.
    Our assertion now follows from~\cref{thm:early-stability}.
\end{proof}

    \printbibliography
\end{document}