
\subsection{Construction of the homomorphism $\sigma_1$}\label{subsec:construction-sigma}
To simplify notation, throughout this section we set $N_1 = N(\varpi_1^\vee),$ $N^1 = N(-\varpi_1^\vee)$.
We only construct one of the desired maps from the definition of $\sigma$-pair, namely $\sigma(\varpi_1^\vee) \colon N_1 \to N^1$,
for shortness we denote it by $\sigma_1$.

In order to define $\sigma_1$ we will use the description of the subgroup $N_1$ given in terms of ``another presentation''.
We start with the following simple observation.
\begin{lemma}
    \label{lem:n1-decomp} For $n\geq 4$ there is an isomorphism $N_1 \cong N_{0} \rtimes P_1^-(A)$,
    where $N_{0}$ denotes the subgroup $\St(n, A[t], tA[t])$ and $P_1^-(A)$ is the subgroup of $\St(n, A)$ generated by $x_{ij}(\xi)$ with $i\neq 1$.
\end{lemma}

Recall that the universal property of semidirect products gives for any group $H$ acting on a group $N$ via $\phi \colon H \to \Aut(N)$
and any group homomorphisms $f_N\colon N \to G$, $f_H\colon H \to G$ satisfying
\begin{equation}
    \label{eq:coherence-condition} f_N(\phi(h)(n)) = f_H(h) f_N(n) f_H(h)^{-1},\ (n\in N,\ h\in H)
\end{equation}
a unique map $f\colon N \rtimes H \to G$ extending $f_N$ and $f_H$.

Thus, by the above lemma, in order to construct the map $\sigma_1$ we need to construct two maps
\[ (\sigma_1)_{P_1^-(A)} \colon P_1^-(A) \to N^1, \ \ (\sigma_1)_{N_{0}} \colon N_{0} \to N^1\]
and then verify that they satisfy~\eqref{eq:coherence-condition}.

It is easy to define the first map $(\sigma_1)_{P_1^-(A)}$, indeed, using the decomposition $P_1^-(A) \cong U^-_1 \rtimes L_1$ where %TODO: Language???
\[U^-_1 = \langle x_{i1}(\xi) \mid i\neq 1,\ \xi\in A \rangle \text{ and } L_1 = \langle x_{ij}(\xi) \mid i,  j \neq 1,\ \xi\in A\rangle \]
we apply the universal property of semidirect products once again
and define $(\sigma_1)_{P_1^-(A)}$ by requiring that it acts identically on $L_1$ (notice that $L_1 \subseteq N^1$) %TODO: Language???
and acts on elements of $U^-_1$ via the formula $(\sigma_1)_{P_1^-(A)}(x_{i1}(\xi))= x_{i1}(t\xi)$.

%To define the map $(\sigma_1)_{N_{0}}$ we invoke the presentation of the relative Steinberg group $\St(n, A[t], tA[t])$ given in~\cref{prop:rel-presentation}.

Denote by $\delta_1$ the matrix $\mathrm{diag}(t, 1, \ldots, 1) \in \GL(A[t, t^{-1}])$.
Now for $u \in \E(n, A[t])$ and $v \in (tA[t])^n$ we can define the map $(\sigma_1)_{N_0}$ on the generators of $N_{0}=\St(n, A[t], tA[t])$
using the elements defined in~\cref{subsec:another-presentation}:
\begin{equation*}
(\sigma_1)
    _{N_0} (F(u, v)) = X_{\delta_1 \cdot t^{-1}}(u, v),\ (\sigma_1)_{N_0} (S(v, u)) = Y_{\delta_1}(v, u).
\end{equation*}
It is clear that the above elements belong to $N^1$.

\begin{prop}
    If $A$ is a local ring them the map $(\sigma_1)_{N_0}$ preserves relations~\eqref{add4}--\eqref{coef-move}.
    In particular, the map $(\sigma_1)_{N_0}$ is well-defined.
\end{prop}
\begin{proof}
    For~\eqref{add4}--\eqref{add5} this is an immediate corollary of~\cref{itm:xsmall-additivity} and~\cref{lem:xy-wd}.
    By~\cref{lem:xy-conj} for $g \in N^1$ one has
    \begin{equation}
        \label{eq:xy-conj-n1}
        g \cdot X_{\delta_1 t^{-1}}(u', v') \cdot g^{-1} = X_{\delta_1 t^{-1}}(mu', m^*v'), \text{ where } m = \delta_1 \cdot \phi(g) \cdot \delta_1^{-1}.
    \end{equation}
    To obtain that $(\sigma_1)_{N_0}$ preserves~\eqref{conj3} it remains to put $g = X_{\delta_1 t^{-1}}(u, v)$.

    Let us verify that $\sigma_1$ preserves~\eqref{coef-move}, i.\,e. that for $a\in tA[t]$ and $m \in \E(n, A[t])$ one has
    $X_{\delta_1 t^{-1}}(me_1, m^*e_2 a) = Y_{\delta_1}(me_1 a, m^* e_2)$.
    At first let us verify this in the special case when $m^* \in \E(n, A[t])$ belongs to the subset $G_0 = H_{12}(A) \cdot U^+_1(A) \cdot U^-_1(A)$
    (here $H_{12}(A)$ denotes the subgroup of $T(n, A)$ generated by semisimple root elements $h_{12}(\xi)$, $\xi \in A^\times$).
    It is easy to check that in this case the only nonzero components of $v = m^* e_2$ are $v_1$ and $v_2$.
    Write $u = m e_1$ as the sum $u' + u''$, where $u' = (u_1, u_2, 0, \ldots 0)^t,$ $u'' = (0, 0, u_3, \ldots u_n)^t$.
    Since $u^t v = 0$ we obtain that $u', u'' \in D(v)$.
    Now from Lemmas~\ref{lem:xy-wd}--\ref{lem:xy-conj} we conclude that:
    \begin{multline}
        \label{eq:special-case}
        Y_{\delta_1}(me_{1}a, m^* e_2) = x(\delta_1^{-1} \cdot u'a, \delta_1\cdot  v) \cdot x(\delta_1^{-1}\cdot u''a, \delta_1 \cdot v) = \\
        = x(\delta_1^{-1} ua, \delta_1 \cdot v) = x(\delta_1^{-1}t \cdot u, \delta_{1}t^{-1} \cdot v a) = X_{\delta_1 t^{-1}}(me_1, m^*e_2 a).
    \end{multline}

    To obtain the assertion in the general case notice that for local $A$ the group $\E(n, A)$ admits the following decomposition:
    \[\E(n, A) = \mathrm{EP}_1(A) \cdot H_{12}(A) \cdot U^-_1(A) \cdot U^+_1(A).\]
    This can be either obtained as a corollary of Gauss decomposition or can be proved by a direct calculation. %TODO: Give reference

    Applying transpose-inverse automorphism $(-)^*$ to the above decomposition and invoking~\cref{lem:n1-decomp} we obtain that $\E(n, A[t]) = \phi(N_1) \cdot G_0$.
    Thus, every $m \in \E(n, A[t])$ can be factored as $\phi(n) \cdot h$ for some $n\in N_1$ and $h \in G_0$.
    Since $\phi(n) = \delta_1 \phi(n') \delta_1^{-1}$ for some $n' \in N^1$ it remains to apply~\eqref{eq:xy-conj-n1} and~\eqref{eq:special-case}:
    \begin{multline}
        \nonumber X_{\delta_1 t^{-1}}(me_1, m^*e_{2}a) = {}^{n'}(X_{\delta_1 t^{-1}}(he_1, h^*e_{2}a)) = \\
        = {}^{n'}(Y_{\delta_1}(he_{1}a, h^*e_2)) = Y_{\delta_1}(me_{1}a, m^{*} e_{2}).
    \end{multline}
\end{proof}

Thus, we have completed the construction of the map $(\sigma_1)_{N_{1,0}}$.
It is not hard to verify that the condition~\eqref{eq:coherence-condition} is satisfied with these definitions,
therefore, we have also completed the construction of the map $\sigma_1$.


% Panin's theorem asserts $\KO_2(8, A) \twoheadrightarrow \KO_2(10, A) \cong \KO_2(10, A)$ for a Dedekind domain $A$.
% We also may need Horrocks theorem for SO_5 :(

\printbibliography