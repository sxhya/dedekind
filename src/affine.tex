\subsection{Curtis--Tits type presentation of affine Steinberg groups} \label{subsec:curtis-tits}
In this subsection we briefly recall generalization of Steinberg groups to the Kac--Moody setting.
Only in this section we allow root systems to be infinite.

Recall that to any generalized Cartan matrix $A$ one can associate (possibly infinite) root system $\Phi$ and the Steinberg group functor $\St(\Phi, -)$.
Two definitions of this functor have been proposed: the definition of Tits~\cite[\S~3.6]{Ti87} and the definition of Morita--Rehmann~\cite[\S~2]{MR90}.
These definitions agree if the Dynkin diagram of $A$ does not contain connected components of type $\rA_1$, see~\cite[\S~6]{A13}.

For our purposes it will be sufficient to restrict attention to the case when $A$ is the GCM corresponding to the extended Dynkin diagram of a finite irreducible simply-laced root system $\Phi$ of rank $>1$.
Such digrams for $\Phi$ of type $\rD_\ell$, $\rE_6$ and $\rE_7$ are depicted on Figure 1 (the added root is marked with index $0$).

\tikzset{
    root/.style={circle, draw, minimum size=0.2cm, inner sep=0},
    zeroroot/.style={circle, draw, minimum size=0.2cm, inner sep=0, fill=yellow},
    highlighted/.style={circle, draw, minimum size=0.2cm, inner sep=0, fill=green},
    levi/.style={draw, dashed, rounded corners},
    dottededge/.style={dotted},
    labeled/.style={below}
}
\begin{figure}[hb]\label{fig:dynkin-diagrams}
\begin{longtable}{ c c c }
    \scalebox{0.69}{\begin{tikzpicture}
                        \node[root, highlighted, label=$1$] (d1) at (0,0.5) {};
                        \node[root, label=$2$] (d2) at (1,0) {};
                        \node[root, label=$3$] (d3) at (2,0) {};
                        \node at (3,0) {\ldots};
                        \node[root, label={[xshift=-0.5cm, yshift=-0.6cm]$\ell-2$}] (dl2) at (4,0) {};
                        \node[root, label={[xshift=-0.6cm, yshift=-0.35cm]$\ell-1$}] (dl1) at (5,0.5) {};
                        \node[root, label={[xshift=-0.5cm, yshift=-0.3cm]$\ell$}] (dl) at (5,-0.5) {};
                        \node[root, zeroroot, label=below:{$0$}] (d0) at (0,-0.5) {};

                        \draw (d1) -- (d2) -- (d3) -- (2.5,0);
                        \draw (3.5,0) -- (dl2) -- (dl1);
                        \draw (dl2) -- (dl);
                        \draw[dottededge] (d2) -- (d0);

                        \begin{scope}
                            \node[levi, fit=(d2) (d3) (dl) (dl1) (dl2), label=above:{$\Delta$}] {};
                        \end{scope}
    \end{tikzpicture}}
    &
    \scalebox{0.69}{\begin{tikzpicture}
                        \begin{scope}
                            \node[root, highlighted, label=$1$] (e1) at (0,0) {};
                            \node[root, label=$3$] (e3) at (1,0) {};
                            \node[root, label={[xshift=-0.3cm]$4$}] (e4) at (2,0) {};
                            \node[root, label=$5$] (e5) at (3,0) {};
                            \node[root, label=$6$] (e6) at (4,0) {};
                            \node[root, label={[xshift=-0.3cm,yshift=-0.3cm]$2$}] (e2) at (2,1) {};
                            \node[root, zeroroot, label=left:$0$] (e0) at (2,2) {};
                            \draw (e1) -- (e3) -- (e4) -- (e5) -- (e6);
                            \draw (e2) -- (e4);
                            \draw[dottededge] (e2) -- (e0);

                            \begin{scope}
                                \node[levi, fit= (e2) (e3) (e4) (e5) (e6), label=above:{$\Delta$}] {};
                            \end{scope}
                        \end{scope}
    \end{tikzpicture}}
    &
    \scalebox{0.69}{\begin{tikzpicture}
                        \begin{scope}
                            \node[root, zeroroot, label={0}] (e0) at (0,0) {};
                            \node[root, label={$1$}] (e1) at (1,0) {};
                            \node[root, label={$3$}] (e3) at (2,0) {};
                            \node[root, label={[xshift=-0.3cm]$4$}] (e4) at (3,0) {};
                            \node[root, label={$5$}] (e5) at (4,0) {};
                            \node[root, label={$6$}] (e6) at (5,0) {};
                            \node[root, label={[xshift=-0.3cm,yshift=-0.3cm]$2$}] (e2) at (3,1) {};
                            \node[root, highlighted, label={$7$}] (e7) at (6,0) {};

                            \draw (e1) -- (e3) -- (e4) -- (e5) -- (e6) -- (e7);
                            \draw (e2) -- (e4);
                            \draw[dottededge] (e0) -- (e1);

                            \begin{scope}
                                \node[levi, fit=(e1) (e2) (e3) (e4) (e5) (e6), label=above:{$\Delta$}] {};
                            \end{scope}
                        \end{scope}
    \end{tikzpicture}} \\
    \text{$\rD_\ell$} &
    \text{$\rE_6$} &
    \text{$\rE_7$}
\end{longtable}
\caption{Root markings on extended Dynkin diagrams}
\end{figure}

Denote by $\widetilde{\Phi}$ the affine root system corresponding to $\Phi$.
Recall from~\cite[\S~4]{A16} that the set of real roots of $\widetilde{\Phi}$ is isomorphic to $\Phi \times \ZZ$.
\begin{lemma} \label{lem:affine-vs-loop} $\St(\widetilde{\Phi}, A) \cong \St(\Phi, A[X, X\inv])$.
\end{lemma}
\begin{proof}
    By definition, the group $\mathrm{St}(\widetilde{\Phi}, A)$ is presented by generators $x_{(\alpha, m)}(a)$, $a \in A$, $\alpha \in \Phi$ and
    the following relations:
    \begin{align}
        x_{(\alpha, m)}(a)\cdot x_{(\alpha, m)}(b)&=x_{(\alpha, m)}(a+b),  \label{AR1}\\
        [x_{(\alpha, m)}(a),\,x_{(\beta, n)}(b)]  &=x_{(\alpha+\beta, n+m)}(N_{\alpha,\beta} \cdot ab),\text{ for }\alpha+\beta\in\Phi, \label{AR2} \\
        [x_{(\alpha, m)}(a),\,x_{(\beta, n)}(b)]  &=1,\text{ for }\alpha+\beta\not\in\Phi\cup0. \label{AR3}
    \end{align}
    It is not hard to check that the homomorphism given by $x_{(\alpha, m)}(a) \mapsto x_\alpha(aX^m)$ is an isomorphism with the inverse given by
    $x_\alpha(a_{n}X^n + \ldots + a_m X^m) \mapsto \Pi\limits_{i=n}^m x_{(\alpha, i)}(a_i)$, $a_i \in A$, $n \leq m$, $n, m\in \ZZ$.
\end{proof}

Our next goal is to formulate the so-called \textit{Curtis--Tits presentation} of affine Steinberg groups discovered by D.~Allcock in~\cite{A16, A13}.
This presentation has the advantage of being formulated in terms of the Dynkin diagram of $\Phi$ rather than the (possibly infinite) set of real roots of $\Phi$.
Its other advantage is that it requires no choice of structure constants in its statement.
Allcock's result is the generalization of Curtis--Tits presentations of finite root system, which have been known since 1960's, see e.\,g.~\cite[Theorem~B]{DS74}.

Recall that $\{ \alpha_1, \ldots \alpha_\ell \}$ is the set of simple roots of $\Phi$.
We denote by $\alpha_0$ the opposite root to the maximal root of $\Phi$ (i.\,e. $\alpha_0 := -\alpha_\mathrm{max}$).
This roots corresponds to the added $0$ node of the extended Dynkin diagram depicted on Figure 1.
We denote by $X_0(A)$ the root group $\{ X_0(a) \mid a \in A\}$ (as a group it is isomorphic to the additive group of $A$).
We also denote by $j$ the root adjacent to $0$.
\begin{prop} \label{prop:Allcock-affine} The group $\St(\Phi, A[X, X\inv])$ is isomorphic to the free product of $\St(\Phi, A)$, the group $X_0(A)$ and the free cyclic group $\langle S_0 \rangle$
     amalgamated over the subgroup generated by the following relations:
    \begin{align}
        [S_0^2, X_0(a)] & = 1 & \text{ for $a \in A$; } \label{eq:Allcock-2} \\
        X_0(1) \cdot {}^{S_0} X_0(1) \cdot X_0(1) & = S_0; \label{eq:Allcock-3} \\
        [S_0, w_{\alpha_j}(1)] & = 1; &  \label{eq:Allcock-4} \\
        [S_0, x_{\alpha_i}(a)] & = 1, &  \label{eq:Allcock-5-1}\\
        [w_{\alpha_i}(1), X_0(a)] & = 1 & \text{$i$ unjoined with $0$, $a \in A;$} \label{eq:Allcock-5-2} \\
        [X_0(a), x_{\alpha_i}(b)] & = 1 & \text{$i$ unjoined with $0$, $a, b \in A;$} \label{eq:Allcock-6} \\
        S_0 \cdot w_{\alpha_j}(1) \cdot S_0 & = w_{\alpha_j}(1) \cdot S_0 \cdot w_{\alpha_j}(1); \label{eq:Allcock-7} \\
        {}^{S_0^2} w_{\alpha_j}(1) & = w_{\alpha_j}(-1); \label{eq:Allcock-8-1} \\
        {}^{w_{\alpha_j}^2(1)} S_0 & = S_0^{-1}; \label{eq:Allcock-8-2} \\
        x_{\alpha_j}(a) \cdot S_0 \cdot w_{\alpha_j}(1) & = S_0 \cdot w_{\alpha_j}(1) \cdot X_0(a), & \label{eq:Allcock-9-1} \\
        X_0(a) \cdot w_{\alpha_j}(1) \cdot S_0 & = w_{\alpha_j}(1) \cdot S_0 \cdot x_{\alpha_j}(a), & \label{eq:Allcock-9-2} \\
        {}^{S_0^2} X_{\alpha_j}(a) & = x_{\alpha_j}(-a), & \label{eq:Allcock-10-1} \\
        {}^{w_{\alpha_j}^2(1)} X_0(a) & = X_0(-a) & \text{for $a \in A;$} \label{eq:Allcock-10-2} \\
        [X_0(a), {}^{S_0} x_{\alpha_j}(b)] &= 1, & \label{eq:Allcock-11-1} \\
        [x_{\alpha_j}(a), {}^{w_{\alpha_j}(1)}X_0(b)] &= 1, & \label{eq:Allcock-11-2} \\
        [X_0(a), x_{\alpha_j}(b)] &= {}^{S_0} x_{\alpha_j}(ab) & \text{for $a, b \in A.$} \label{eq:Allcock-12}
    \end{align}
\end{prop}
\begin{proof}
    This is a direct consequence of our~\cref{lem:affine-vs-loop} and the presentation of~\cite[Theorem~1]{A16} (or, alternatively, \cite[Theorem~1.1]{A13} combined with~\cite[Theorem~1.3]{A13}).
    The list of relations in the statement is obtained from \cite[Table~1]{A16} by substituting concrete values of $i, j$, identifying Allcock's $X_{i}(a)$ with  $x_{\alpha_i}(a)$ and $S_i$ with $w_{\alpha_i}(a)$ for all $i\neq 0$ and then omitting all the relations which already hold true in $\St(\Phi, A)$.
    The only further simplification is that we have omitted the relation $[X_{\alpha_j}(b), X_0(a)] = {}^{w_{\alpha_j}(1)} X_0(ab)$ (the relation symmetric to~\eqref{eq:Allcock-12})
    because it is a consequence of~\eqref{eq:Allcock-9-2} and~\eqref{eq:Allcock-12}.
\end{proof}

Our next goal is to simplify the presentation of~\cref{prop:Allcock-affine} by using the larger group $\St(\Phi, A[X])$ instead of $\St(\Phi, A)$.
\begin{prop}
    For a finite simply-laced irreducible root system $\Phi$ and an arbitrary commutative ring $A$ the Steinberg group $\St(\Phi, A[X, X\inv])$ is isomorphic
    to the free product of $\St(\Phi, A[X])$ and the cyclic group $\langle S \rangle$ amalgamated over the following list relations:
    \begin{align}
        lol & lol \\
        foo & bar
    \end{align}
\end{prop}

\subsection{Weight automorphisms}\label{subsec:weight-automorphisms}
Recall that for every coweight $\omega \in P(\Phi^\vee)$ and $\beta \in \ZZ \Phi$ the scalar product $(\omega, \beta)$ is an integer.
Thus, a choice of $u \in R^\times$ and $\omega \in P(\Phi^\vee)$ specifies a permutation of the generating set for $\St(\Phi, R)$ via the following mapping:
\begin{equation*} x_\alpha(a) \mapsto x_\alpha(u^{(\omega, \alpha)} \cdot a),\ \alpha\in \Phi,\ a \in R. \end{equation*}
It is not hard to check that this action is compatible with relations~\eqref{x-additivity}--\eqref{R3} and hence specifies a well-defined automorphism of $\St(\Phi, R)$, which we denote by $\chi_{\omega, u}$.

In the following lemma we check that an analogue of $\chi_{\omega, u}$ can also be defined for relative Steinberg groups.
\begin{lemma} \label{lem:relative-chi}
Let $R$ be a commutative ring, $I$ be its ideal and let $u \in R^\times$.
For every coweight $\omega \in P(\Phi^\vee)$ there exists a well-defined automorphism $\widetilde{\chi}_{\omega, u}$ of the relative Steinberg group $\St(\Phi, R, I)$ making the following diagram commute:
\[\begin{tikzcd} \St(\Phi, R, I) \ar[r, "\widetilde{\chi}_{\omega, u}"] \ar[d] & \St(\Phi, R, I) \ar[d] \\
\St(\Phi, R) \ar[r, "\chi_{\omega, X}"] & \St(\Phi, R]). \end{tikzcd}\]
\end{lemma}
\begin{proof}
    Observe that the automorphism $\chi_{\omega, (u; u)}$ of $\St(\Phi, D_{R, I})$ preserves subgroups
    $\Ker(p_i^*)$, $i=1, 2$ and hence their commutator subgroup $C$.
    The required automorphism $\widetilde{\chi}_{\omega, u}$ now can be obtained by restricting $\chi_{\omega, (u; u)}$ to $\Ker(p_1^*)$.
    The commutativity of the diagram is obvious.
\end{proof}

In the sequel we will use the following formulae describing the action of $\chi_{\omega, u}$ on the elements $w_\alpha(u)$, $h_\alpha(u)$:
\begin{align}
    \label{eq:chi-w} \chi_{\omega, u}\left(w_\alpha(v)\right) &= w_\alpha(u^{(\omega, \alpha)} \cdot v), \\
    \label{eq:chi-h} \chi_{\omega, u} (h_\alpha(v)) &= h_\alpha(u^{(\omega, \alpha)} \cdot v) \cdot h_\alpha(u^{(\omega, \alpha)})^{-1} = \{u^{(\omega, \alpha)}, v\} \cdot h_\alpha(v).
\end{align}

The following lemma is analogous to~\cite[Lemma~3.1(c)]{Tu83}.
\begin{lemma} \label{lem:winv-chiw}
For any $w \in \StW(\Phi, A[X^{\pm 1}])$ the element $w^{-1} \cdot \chi_{\omega, X}(w)$ belongs to $\StH(\Phi, A[X^{\pm 1}])$.
\end{lemma}
\begin{proof}
    Since $\StH(\Phi, A[X^{\pm 1}])$ is a normal subgroup of $\StW(\Phi, A[X^{\pm 1}])$, it suffices to verify the assertion for $w = w_\alpha(u)$.
    Set $h = w^{-1} \cdot \chi_{\omega, X}(w)$.
    Notice that
    \begin{multline*} w_\alpha(1) \cdot h\cdot  w_\alpha(-1) = w_\alpha(-1)^{-1} \cdot w_\alpha(u)^{-1} \cdot w_{\alpha}(X^{(\omega, \alpha)} u) \cdot w_\alpha(-1) = \\
    = h_\alpha(u)^{-1} \cdot h_\alpha(X^{(\omega, \alpha)}u) \in \StH(\Phi, A[X^{\pm 1}]).\end{multline*}
    Thus, we get from\cite[Lemme~5.2(b,g)]{Ma69} and~\eqref{eq:steinberg} that \[h = h_{\alpha}^{-1}(u^{-1}) \cdot h_{\alpha}(X^{-(\omega, \alpha)}u^{-1}) = h_{\alpha}^{-1}(X^{(\omega, \alpha)}) \cdot \{ X^{(\omega, \alpha)}, u^{-1} \}. \qedhere\]
\end{proof}

\begin{example} \label{exm:chi-linear}
Set $R = A[X^{\pm 1}]$.
Consider the following coweights:
\[\varepsilon_1 = \varpi_1^\vee,\ \varepsilon_2 = \varpi_2^\vee - \varpi_1^\vee,\ \ldots,\ \varepsilon_{\ell+1} = -\varpi^\vee_\ell.\]
For $1\leq k\leq \ell+1$ and $u \in R^\times$ denote by $d_k(u)$ the matrix from $\GL(\ell+1, R)$ which differs from the unit matrix only in that it has the element $u$ on the $k$-th place of its diagonal.
Recall from~\cite[Corollary~4]{Ka77} that for any $g \in \GL(\ell+1, R)$ there exists an automorphism $\beta_g$ of $\St(\ell+1, R)$ ''modeling`` the automorphism $\alpha_g \colon \GL(\ell+1, R) \to \GL(\ell+1, R)$ of inner conjugation by $g$, i.\,e. such that $\phi \beta_g = \alpha_g \phi$.


It is clear that in the linear case the map $\chi_{\varepsilon_k, u}$ coincides with $\beta_{d_k(u)}$,
while for other Chevalley groups the maps $\chi_{\omega, u}$ model automorphisms of inner conjugation by weight elements $h_\omega(u)$ in the sense of~\cite[\S~4]{Vav09}.
\end{example}


Let $\omega \in P(\Phi^\vee)$ be a coweight of $\Phi$.
Consider the subset $\mathcal{X}_\omega$ of the generating set of $\St(\Phi, A[X])$ consisting of those generators $x_{\alpha}(a) \in \St(\Phi, A[X])$ for which
$(\alpha, \omega) < 0$ implies that $a \in A[X]$ is divisible by $X^{-(\alpha, \omega)}$.
\begin{rem}
    Notice that according to this condition if $(\alpha, \omega) \geq 0$ then $\mathcal{X}$ contains $x_\alpha(a)$ for all $a \in A[X]$.
\end{rem}
Denote by $N_\omega$ the subgroup of $\St(\Phi, A[X])$ generated by $\mathcal{X}_\omega$.


\begin{dfn} \label{dfn:delta-pair}
Let $\omega \in P(\Phi^\vee)$ be a coweight.
By definition, an {\it $\omega$-pair} is a pair of mutually inverse group homomorphisms
$\xymatrix{ \sigma(\omega)\colon N_\omega \ar[r] & \ar@<-1.0ex>[l] N_{-\omega}\colon \sigma(-\omega) }$ satisfying the following identity:
\begin{equation} \label{eq:sigmadef}
\sigma(\pm \omega)(x_\alpha(\xi)) = x_\alpha(X^{(\pm \omega, \alpha)}\cdot \xi),
\text{ for all } x_\alpha(\xi) \in \mathcal{X}_{\pm\omega}.
\end{equation}\end{dfn}
It is clear that the maps $\delta(\omega)$, $\delta(-\omega)$ are uniquely determined by~\eqref{eq:sigmadef}, so at most one $\omega$-pair may exist for any given $\omega$.
Moreover, $\delta(\omega), \delta(-\omega)$ always make the following diagram commute:
\begin{equation} \label{eq:sigma-diagram}
\xymatrix{ N_\omega \ar[r]_{\sigma(\omega)}\ar@{^{(}->}[d] \ar@/^1.5pc/[rr]^{\mathrm{id}} & N_{-\omega} \ar@{^{(}->}[d] \ar[r]_{\sigma(-\omega)} & N_\omega \ar@{^{(}->}[d] \\
\St(\Phi, A[X]) \ar[d] & \St(\Phi, A[X]) \ar[d] & \St(\Phi, A[X]) \ar[d] \\
\St(\Phi, A[X^{\pm 1}]) \ar@<-0.0ex>[r]_{\chi_{\omega, X}} \ar@/_1.5pc/[rr]^{\mathrm{id}} & \St(\Phi, A[X^{\pm 1}]) \ar@<-0.0ex>[r]_{\chi_{-\omega, X}} & \St(\Phi, A[X^{\pm 1}]).} \end{equation}
The question of existence of an $\omega$-pair is rather complicated and will be addressed in the sequel.
One of the necessary technical ingredients needed for this is the technique of another presentation, which we recall in detail in the following subsection.

\subsection{Presentations of affine Steinberg groups using weight elements}