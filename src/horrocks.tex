\begin{dfn}
    Let $k$ be commutative ring and $K$ be a functor $ \mathbf{Alg}_k \to \mathbf{Grp}$. $K$ is called \textit{locally acyclic (resp., locally acyclic for domains)}
    if for every commutative local ring (resp., domain) $R \in \mathbf{Alg}_k$ the following diagram whose arrows are induced by natural embeddings is a pullback square:
    \begin{equation}\label{eq:P1-square} \begin{tikzcd} K(R) \ar[r] \ar[d] & K(R[t]) \arrow{d} \\ K(R[t^{-1}]) \ar{r} & K(R[t, t^{-1}]). \end{tikzcd} \end{equation}
\end{dfn}

\begin{thm}[Horrocks theorem for $\K_2$]\label{thm:horrocks-k2}
Let $\Phi$ be a root system of simply-laced type of rank $\geq 3$ such that $\Phi \neq \rE_8$.
Then the functor $\K_{2}(\Phi, R)$ is (locally) acyclic for domains.
\end{thm}

\subsection{Main structure theorem} \label{subsec:structure-theorem-overview}
Let $A$ be a local ring with maximal ideal $M$ and residue field $k$.
Recall from~\cite[\S~4]{Su77} or~\cite[\S~VI.6]{Lam10} that for $r \geq 3$ the elementary linear group $\E_r(A[X, X\inv])$ admits the following decomposition:
\begin{equation}\label{eq:triple-decomposition}
\E_r(A[X^{\pm 1}]) = \E_r(A[X]) \cdot B(A[X^{\pm 1}]) \cdot \E_r(A[X^{\pm 1}], M[X^{\pm 1}]).
\end{equation}
Here $\E_r(-)$ denotes the same object as $\Esc(\rA_{r - 1}, -)$ before.
Recall that the relative elementary subgroup $\E_r(R, I)$ is defined as the kernel of canonical reduction homomorphism $\E_r(R) \to \E_r(R/I).$
Recall also that $B(R)$ denotes the Borel subgroup (i.\,e. the semidirect product of the unipotent radical $\UU(\Phi^+, R)$ and the group of diagonal matrices).

Decomposition~\eqref{eq:triple-decomposition} is a key tool in the study of the {$\K_1$-analogue of Serre's problem}.
We want to devise some analogue of it which would be suitable for study of Steinberg groups.

For a \textit{special} subset of $\Phi$ (i.\,e. a subset $S \subseteq \Phi$ such that $S \cap -S = \varnothing$)
we denote by $\UU(S, R)$ the subgroup of $\St(\Phi, R)$ generated by $x_\alpha(a)$, where $a \in R$, $\alpha \in S$.
Similarly, for an ideal $I \trianglelefteq R$ we denote by $\UU(S, I)$ the subgroup of $\St(\Phi, R, I)$ generated by $x_\alpha((0;m))C$, $m \in I$, $\alpha \in S$.
Notice that the restriction of $\pi$ to $\UU_r(S, R)$ is a monomorphism, so $\UU(S, R)$ is isomorphic to a unipotent subgroup of $\Gsc(\Phi, R)$.

Not let us introduce short notation for the Steinberg groups and their subgroups that will be relevant for us.
\begin{align*}
    G     =& \St(\Phi, A[X, X^{-1}]),\\
    G^+   =& \St(\Phi, A[X]),\\
    B     =& \UU(\Phi^+, A[X, X\inv]) \rtimes \StH(\Phi, A[X, X\inv]) \leq G,\\
    G_M   =& \St(\Phi, A[X, X^{-1}], M[X, X^{-1}]),\\
    U^+   =& \UU(\Phi^+, A[X]),\\
    G^+_M =& \St(\Phi, A[X], M[X]),\\
    B_M   =& \UU(\Phi^+, M[X, X^{-1}]) \times \{X, 1+M\} \leq G_M,\\
    U^+_M =& \UU(\Phi^+, M[X]).
\end{align*}
In the definition of $B_M$ we denote by $\{X, 1+M\}$ the image of the homomorphism $\{X, -\}_{r}$ from~\eqref{eq:relative-symbol}.

The above groups can be organized into the following commutative diagram:
\begin{equation} \label{eq:cube-Steinberg} \xymatrix{
    U^+_M \ar@{^{(}->}[rr] \ar@{^{(}->}[dd] \ar@{^{(}->}[rd] &                        & B_M \ar[dd]^(.3){g_B^M} \ar@{^{(}->}[rd]^{g^B_M} &           \\
    & G^+_M \ar[rr]^(.3){g^+_M} \ar^(.3){g^M_+}[dd] &                   & G_M \ar[dd]_{h_M} \\
    U^+ \ar@{^{(}->}[rr]^(.3){g_B^+} \ar@{^{(}->}[rd]_{g_+^B}          &                        & B \ar@{^{(}->}[rd]_{h_B}       &           \\
    & G^+ \ar[rr]_{h_+}              &                   & G.}\end{equation}
Here $g^M_+$ and $h_M$ are just renamed homomorphisms $\mu$ from~\eqref{eq:relative-Steinberg}.
Homomorphism $g^M_B$ is also induced by $\mu$.
Maps $g_M^+$ and $h^+$ are induced by the ring embedding $A[X] \to A[X, X\inv]$.
The other homomorphisms on the diagram are all obvious subgroup embeddings.

Set $V = G^+\times B \times G_M$, $W = U^+\times G^+_M \times B_M$.
Both homomorphisms $g_+^M$ and $h_M$ are crossed modules by~\cref{lem:rel-Steinberg-crossed-module}.
The fact that $(g^+_M, h_+)$ is a map of precrossed modules is obvious.
Thus, we find ourselves in the situation of~\cref{subsec:triples}.

The following lemma explains the strategy behind our proof of Horrocks theorem.
\begin{lemma}
    Suppose that $A$ is a domain.
    Suppose that $V/\sim_W$ admits an action of $G$ with the only property that
      for $g \in G_M$ one has \[ h_M(g) \cdot [(1, 1, 1)] = [(1, 1, g)]. \]
    Then the canonical homomorphism \[C(\Phi, A[X], M[X]) \to C(\Phi, A[X, X\inv], M[X, X\inv])\] is surjective.
\end{lemma}
\begin{proof}
  Since $A$ is a domain, $g^M_B$ is injective by~\cref{lem:symbols}.
    Thus, the back face of~\eqref{eq:cube-Steinberg} is pull-back.
  Notice that for any $g \in C(\Phi, A[X, X\inv], M[X, X\inv]) \subseteq G_M$ one has
    \[ [(1, 1, 1)] = h_M(g) \cdot [(1, 1, 1)] = [(1, 1, g)].\]
  It is clear now that $g$ lies in the image of $C(\Phi, A[X], M[X])$ under $g^+_M$ by~\cref{lem:one-one-z}.
\end{proof}