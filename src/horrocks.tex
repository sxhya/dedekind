\begin{dfn}
    Let $K$ be a functor from commutative rings to groups.
    Recall from~\cite{LSV2} that $K$ is called \textit{locally acyclic (resp., locally acylic for domains)} if for every commutative local ring (resp., domain) $A$ the following diagram whose arrows are induced by natural embeddings is a pullback square:
    \begin{equation}\label{eq:P1-square} \begin{tikzcd} K(A) \ar[r] \ar[d] & K(A[X]) \arrow{d} \\ K(A[X\inv]) \ar{r} & K(A[X, X\inv]). \end{tikzcd} \end{equation}
\end{dfn}

Now we can formulate the main result of this section.
\begin{thm}[Horrocks theorem for $\K_2$]\label{thm:horrocks-k2}
Let $\Phi$ be a root system of type $\rA_{\geq 4}$, $\rD_{\geq 5}$, $\rE_6$ or $\rE_7$.
Then the functor $\K_{2}(\Phi, -)$ is locally acyclic for domains.
\end{thm}
Horrocks theorem for $\K_2$ has been previously known in the linear and even orthogonal case $\Phi=\rA_{\geq 4},\rD_{\geq 7}$, see~\cite[Theorem~1]{LS20},\cite[Proposition~4.3]{Tu83}.
Notice also that the orthogonal version of Horrocks theorem for $\K_2$ was proved under the additional assumption that $2 \in A^\times$.
Our theorem generalizes both these results and also improves them in the following respects:
\begin{itemize}
    \item It applies to exceptional root systems ($\Phi = \rE_6, \rE_7$);
    \item In the orthogonal case it makes no assumptions on the invertibility of $2$ and also requires a weaker assumption on the rank of $\Phi$.
\end{itemize}

\subsection{Initial reductions} \label{subsec:structure-theorem-overview}
Significant progress towards proving~\cref{thm:horrocks-k2} has already been achieved in~\cite{LS20}.
In this subsection we reduce~\cref{thm:horrocks-k2} to a specific technical statement which will be addressed in the following sections.

The following lemma provides the initial key reduction in the proof of~\cref{thm:horrocks-k2}.
\begin{lemma} \label{lem:first-reduction}
Let $A$ be a local ring with maximal ideal $M$ and residue field $k$.
Let $\Phi$ be as in the statement of~\cref{thm:horrocks-k2}.
Suppose that the canonical homomorphism
\begin{equation} \label{eq:c-surj} C(\Phi, A[X], M[X]) \to C(\Phi, A[X, X\inv], M[X, X\inv]) \end{equation}
is surjective.
Then the square~\eqref{eq:P1-square} is pullback.
\end{lemma}
\begin{proof}
    Denote by $R$ the Laurent polynomial ring $A[X, X\inv]$ and by $B$ the subring $A[X\inv] + M[X] \subseteq R$.
    We also set $I \subseteq M[X, X\inv]$, which is clearly an ideal of both $B$ and $R$.
    Consider the following diagram with rows obtained from~\eqref{eq:relative-Steinberg}:
    \[\begin{tikzcd}
          C(\Phi, B, I) \arrow{r} \arrow[d, swap, twoheadrightarrow] & \St(\Phi, B, I) \arrow{r}{\mu_B} \arrow{d} & \St(\Phi, B) \arrow{r} \arrow{d} & \St(\Phi, k[X^{-1}]) \arrow[hookrightarrow]{d} \\
          C(\Phi, R, I) \arrow{r} & \St(\Phi, R, I) \arrow{r}{\mu_R} \arrow[ur, "t", dashrightarrow] & \St(\Phi, R) \arrow{r} & \St(\Phi, k[X, X^{-1}]).
    \end{tikzcd}\]
    The lifting $t$ is obtained from~\cite[Lemma~3.3]{LS20}.
    The right-hand side vertical arrow is injective by~\cite[Lemma~2.2]{LS20}.
    The left-hand side vertical arrow is surjective since the composite arrow
    $\xymatrix{ C(\Phi, A[X], M[X]) \ar[r] & C(\Phi, B, I) \ar[r] & C(\Phi, R, I) }$
    is surjective by lemma's assumption.
    It remains to repeat the diagram chasing argument of~\cite[Theorem~1]{LS20} to conclude that the homomorphism $\St(\Phi, B) \to \St(\Phi, A[X, X\inv])$ is injective.
    The assertion of the lemma will then follow from~\cite[Theorem~3]{LS20}.
\end{proof}
The proof of Horrocks theorem is, thus, reduced to showing that~\eqref{eq:c-surj} is surjective for every local domain $(A, M)$.
In order to formulate the second reduction we need to recollect some notation pertaining to subgroups of $\St(\Phi, R)$.

For a \textit{special} subset of $\Phi$ (i.\,e. a subset $S \subseteq \Phi$ such that $S \cap -S = \varnothing$)
we denote by $\UU(S, R)$ the subgroup of $\St(\Phi, R)$ generated by $x_\alpha(a)$, where $a \in R$, $\alpha \in S$.
Similarly, for an ideal $I \trianglelefteq R$ we denote by $\UU(S, I)$ the subgroup of $\St(\Phi, R, I)$ generated by $x_\alpha((0;m))C$, $m \in I$, $\alpha \in S$.
Notice that the restriction of $\pi$ to $\UU(S, R)$ is a monomorphism, so $\UU(S, R)$ is isomorphic to a unipotent subgroup of $\Gsc(\Phi, R)$.

Not let us introduce shorter notation for Steinberg groups and their subgroups that will be relevant for the proof of Horrocks theorem.
\begin{align*}
    G     =& \St(\Phi, A[X, X^{-1}]),\\
    G^+   =& \St(\Phi, A[X]),\\
    B     =& \UU(\Phi^+, A[X, X\inv]) \rtimes \StH(\Phi, A[X, X\inv]) \leq G,\\
    G_M   =& \St(\Phi, A[X, X^{-1}], M[X, X^{-1}]),\\
    U^+   =& \UU(\Phi^+, A[X]),\\
    G^+_M =& \St(\Phi, A[X], M[X]),\\
    B_M   =& \UU(\Phi^+, M[X, X^{-1}]) \times \{X, 1+M\} \leq G_M,\\
    U^+_M =& \UU(\Phi^+, M[X]).
\end{align*}
In the definition of $B_M$ we denote by $\{X, 1+M\}$ the image of the homomorphism $\{X, -\}_{r}$ from~\eqref{eq:relative-symbol}.

The above groups can be organized into the following commutative diagram:
\begin{equation} \label{eq:cube-Steinberg} \xymatrix{
    U^+_M \ar@{^{(}->}[rr] \ar@{^{(}->}[dd] \ar@{^{(}->}[rd] &                        & B_M \ar[dd]^(.3){g_B^M} \ar@{^{(}->}[rd]^{g^B_M} &           \\
    & G^+_M \ar[rr]^(.3){g^+_M} \ar^(.3){g^M_+}[dd] &                   & G_M \ar[dd]_{h_M} \\
    U^+ \ar@{^{(}->}[rr]^(.3){g_B^+} \ar@{^{(}->}[rd]_{g_+^B}          &                        & B \ar@{^{(}->}[rd]_{h_B}       &           \\
    & G^+ \ar[rr]_{h_+}              &                   & G.}\end{equation}
Here $g^M_+$ and $h_M$ are just renamed homomorphisms $\mu$ from~\eqref{eq:relative-Steinberg}.
Homomorphism $g^M_B$ is also induced by $\mu$.
Maps $g_M^+$ and $h^+$ are induced by the ring embedding $A[X] \to A[X, X\inv]$.
The other homomorphisms on the diagram are all obvious subgroup embeddings.

Set $V = G^+\times B \times G_M$, $W = U^+\times G^+_M \times B_M$.
Both homomorphisms $g_+^M$ and $h_M$ are crossed modules by~\cref{lem:rel-Steinberg-crossed-module}.
The fact that $(g^+_M, h_+)$ is a map of precrossed modules is obvious.
Thus, we find ourselves in the situation of~\cref{subsec:triples}.

The following lemma provides the second key reduction in the proof of Horrocks theorem.
\begin{lemma}
    Suppose that $A$ is a domain.
    Suppose that $V/\sim_W$ admits an action of $G$ with the only property that
      for $g \in G_M$ one has \[ h_M(g) \cdot [1, 1, 1] = [1, 1, g]. \]
    Then the canonical homomorphism~\eqref{eq:c-surj} is surjective.
\end{lemma}
\begin{proof}
  Since $A$ is a domain, $g^M_B$ is injective by~\cref{lem:symbols}.
    Thus, the back face of~\eqref{eq:cube-Steinberg} is pull-back.
  Notice that for any $g \in C(\Phi, A[X, X\inv], M[X, X\inv]) \subseteq G_M$ one has
    \[ [1, 1, 1] = h_M(g) \cdot [1, 1, 1] = [1, 1, g].\]
  It is clear now that $g$ lies in the image of $C(\Phi, A[X], M[X])$ under $g^+_M$ by~\cref{lem:one-one-z}.
\end{proof}

\subsection{Overview of Main structure theorem}

Let $A$ be a local ring with maximal ideal $M$ and residue field $k$.
Recall from~\cite[\S~4]{Su77} or~\cite[\S~VI.6]{Lam10} that for $r \geq 3$ the elementary linear group $\E_r(A[X, X\inv])$ admits the following decomposition:
\begin{equation}\label{eq:triple-decomposition}
\E_r(A[X^{\pm 1}]) = \E_r(A[X]) \cdot B(A[X^{\pm 1}]) \cdot \E_r(A[X^{\pm 1}], M[X^{\pm 1}]).
\end{equation}
Here $\E_r(-)$ denotes the same object as $\Esc(\rA_{r - 1}, -)$ before.
Recall that the relative elementary subgroup $\E_r(R, I)$ is defined as the kernel of canonical reduction homomorphism $\E_r(R) \to \E_r(R/I).$
Recall also that $B(R)$ denotes the Borel subgroup (i.\,e. the semidirect product of the unipotent radical $\UU(\Phi^+, R)$ and the group of diagonal matrices).

Decomposition~\eqref{eq:triple-decomposition} is a key tool in the study of the {$\K_1$-analogue of Serre's problem}.
We want to devise some analogue of it which would be suitable for study of Steinberg groups.

\subsection{Hermitian presentation of affine Steinberg groups}
$\Delta$, $\Sigma^+$, $\Sigma^-$.

By our assumption $\UU(\Sigma^+, R)$ is an abelian group.

Denote by $G$ the amalgamated product of $\St(\Phi, A[X])$ and the cyclic group $\mathbb{Z}$, whose generator we denote by $\sigma$ amalgamated over the following relations:
\begin{align}
    \sigma \cdot x_{\alpha}(Xf) \sigma^{-1} =& x_{\alpha} (f), & \alpha \in \Sigma^+, f \in A[X], \\
    \sigma \cdot x_{\beta}(f) \sigma^{-1}   =& x_{\beta} (Xf), & \beta \in \Sigma^-, f \in A[X], \\
    \sigma \cdot x_\gamma(f) \sigma^{-1}    =& x_\gamma(f), & \gamma \in \Delta, f \in A[X].
\end{align}
We denote by $i_+$ the canonical homomorphism $\St(\Phi, A[X]) \to G$.

The main result of this subsection is the following
\begin{prop}
    There exists an arrow $j$ making the diagram below commute:
    \[\begin{tikzcd}           & \St(\Phi, A[X, X\inv]) \arrow[rd, dashrightarrow, "j"] & \\
           \St(\Phi, A[X]) \arrow{ru}{h_+} \arrow{rr}{i_+} &                                & G.
    \end{tikzcd}\]
\end{prop}

To construct the arrow $j$ we will use an alternative presentation of $\St(\Phi, A[X, X\inv])$ with fewer generators from~\cite{LS20}.
To formulate it we need to introduce additional notation.

Recall that the Laurent polynomial ring $R = A[X, X\inv]$ can be reinterpreted as a $\ZZ$-graded ring in which $X^{-1}$ has degree $1$.
We denote by $R_d$ the degree $d$ component of $R$.
It is clear that $R_d$ is isomorphic to $A$ as additive group.

We call a generator $x_\alpha(r)$ of the group $\St(\Phi, R)$ \textit{homogeneous of degree $d$} if $r = aX^d \in R_d$ for some $a\in A$.
We denote by $\St^{\leq n}(\Phi, R)$ the group presented by all homogeneous generators of degree $\leq d$ and the following 4 families of homogeneous relations (in which $d, e \leq n$):
\begin{align}
    x_{\alpha}(r) x_{\alpha}(s)    &= x_{\alpha}(a + b), & \text{ for } r, s \in R_d; \tag{R$1_d$} \\
    [x_{\alpha}(r),\ x_{\beta}(s)] &= x_{\alpha + \beta}(N_{\alpha, \beta} \cdot rs), &\text{ for } \alpha + \beta \in \Phi,\ r \in R_d, s \in R_e; \tag{R$2_{d, e}$} \\
    [x_{\alpha}(r),\ x_{\beta}(s)] &= 1, &\text{ for }\alpha - \beta \in \Phi, r \in R_d, s \in R_e; \tag{R$3_{d, e}^\angle$} \\
    [x_{\alpha}(r),\ x_{\beta}(s)] &= 1, &\text{ for }\alpha \perp \beta, r \in R_d, s \in R_e. \tag{R$3_{d,e}^\perp$}
\end{align}
By a \textit{degree of a relation} we mean the maximum of degrees of generators appearing in the relation.
For example, the degree of R$3_{d, e}$ is $\max(d, e)$, while the degree of R$2_{d, e}$ is $\max(d, e, d+e).$

\begin{lemma} \label{lem:homog-presentation}
    Let $\Phi$ be a simply-laced root system of rank $\geq 3$. %TODO: Check if we may have such generality.
    The group $\St(\Phi, R)$ admits presentation by homogeneous generators and homogeneous relations of degree $\leq 1$.
    Moreover, we may omit from this presentation all relations of type $R3^\perp_{1, 1}$.
\end{lemma}
\begin{proof}
    This is precisely the assertion of~\cite[Proposition~5.3]{LS20} combined with~\cite[Lemma~5.2]{LS20} and~\cite[Remark~5.5]{LS20}.
\end{proof}

We now proceed with the construction of homomorphism $j$.
\begin{align}
    j(x_\alpha(a)) = & x_\alpha(a) & \text{ for } a \in R_d, $d \geq 0$ \\
    j(x_\alpha(a X\inv)) = & x_\alpha(a) & \text{ for }
\end{align}